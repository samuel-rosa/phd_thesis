\artigotrue
\chapter{DO MORE DETAILED COVARIATES DELIVER MORE ACCURATE SOIL MAPS?}
\chapternote{This chapter is based on A.~Samuel-Rosa, G.B.M.~Heuvelink, G.M.~Vasques, L.H.C.~Anjos. Do 
more detailed environmental covariates deliver more accurate soil maps? \emph{Geoderma}, v.243--244, 
p.214--227, 2015. Terms and expressions have been modified to match the standard terminology used throughout 
the thesis without compromising the content of the original text. Footnotes were added where a definition 
required correction or clarification.}
\shorttitle{Using More Detailed Covariates}
\label{chap:chap06}

% User defined commands
\def\elev{\texttt{ELEV}} % elevation
\def\slp{\texttt{SLP}}   % slope
\def\asp{\texttt{ASP}}   % aspect
\def\nor{\texttt{NOR}}   % northernness
\def\acc{\texttt{ACC}}   % flow accumulation
\def\twi{\texttt{TWI}}   % topographic wetness index
\def\spi{\texttt{SPI}}   % stream power index
\def\tpi{\texttt{TPI}}   % topographic position index
\def\ndvi{\texttt{NDVI}} %
\def\savi{\texttt{SAVI}} %
\def\sibcs{Brazilian System of Soil Classification}

% \def\portuguesekeys{Mapeamento Digital do Solo. Modelo Linear Misto. Informação Auxiliar. Seleção de 
% Variáveis. Acurácia do Modelo. Custo do Mapeamento do Solo}

% \begin{chapterabstract}{brazilian}{\portuguesekeys}
% Neste estudo nós avaliamos se investir em covariáveis espacialmente mais detalhadas aumenta a acurácia dos 
% mapas do solo. Nós usamos um estudo de caso no sul do Brasil para mapear o conteúdo de argila (CLAY), o 
% conteúdo de carbono orgânico (SOC), e capacidade de troca de cátions efetiva (ECEC) da camada superficial do 
% solo de uma área de \SI{\sim2000}{\hectare} localizada na borda do planalto da Bacia Sedimentar do Paraná. 
% Cinco covariáveis, cada uma com dois níveis de detalhe espacial, foram usadas: mapa areal-categórico de solo,
% modelos digitais de elevação (DEM), mapas geológicos, mapas de uso da terra, e imagens de satélite. Trinta e 
% dois modelos de regressão linear múltipla foram calibrados para cada propriedade do solo usando todas as 
% combinações de detalhe espacial das covariáveis. Para cada combinação, \textit{stepwise regression} foi 
% usada para selecionar as variáveis preditoras incorporadas no modelo. A avaliação dos modelos foi feita 
% usando o R-quadrado ajustado da regressão. O modelo de referência, calibrado com a versão menos detalhada de 
% cada covariável, e o modelo com o melhor desempenho, foram usados para calibrar dois modelos lineares mistos 
% para cada propriedade do solo. Parâmetros dos modelos foram estimados usando máxima verossimilhança 
% restrita. Predições espaciais foram realizadas usando o melhor preditor linear não-enviesado empírico. 
% Validação-cruzada foi usada para validar os modelos de regressão linear múltipla e dos modelos lineares 
% mistos de referência e com melhor desempenho. Os resultados mostram que para CLAY a acurácia da predição não 
% aumentou consideravelmente por usar covariáveis mais detalhadas. A quantidade de variância explicada 
% aumentou apenas \SI{\sim2}{\pp} (pontos percentuais), menos do que obtido pela inclusão do passo de 
% krigagem, que explicou \SI{4}{\pp}. Por outro lado, a predição de SOC e ECEC aumentou em \SI{\sim13}{\pp} 
% quando o modelo de referência foi substituído pelo modelo com melhor desempenho. Em geral, o aumento no 
% desempenho preditivo foi modesto e pode não sobrepor os custos adicionais do uso de covariáveis mais 
% detalhadas. Pode ser mais eficiente investir recursos adicionais na coleta de mais observações do solo, ou 
% no aumento do detalhe apenas da covariável que tem o efeito de aumento mais forte. Em nosso estudo, a última 
% funcionaria apenas para SOC e ECEC pelo investimento em um mapa de uso da terra mais detalhado e, 
% possivelmente, também em um mapa geológico e DEM mais detalhados.
% \end{chapterabstract}

\def\englishkeys{Digital Soil Mapping. Linear Mixed Model. Auxiliary Information. Variable Selection. Model
Accuracy. Soil Mapping Cost}
  
\begin{chapterabstract}{english}{\englishkeys}
In this study we evaluated whether investing in more spatially detailed covariates improves the accuracy of 
soil maps. We used a case study from Southern Brazil to map clay content (CLAY), organic carbon content 
(SOC), 
and effective cation exchange capacity (ECEC) of the topsoil for a \SI{\approx2000}{\hectare} area located on 
the edge of the plateau of the Paraná Sedimentary Basin. Five covariates, each with two levels of spatial 
detail were used: area-class soil maps, digital elevation models (DEM), geologic maps, land use maps, and 
satellite images. Thirty-two multiple linear regression models were calibrated for each soil property using 
all 
spatial detail combinations of the covariates. For each combination, stepwise regression was used to select 
predictor variables incorporated in the model. Model evaluation was done using the adjusted R-square of the 
regression. The baseline model, calibrated with the less detailed version of each covariate, and the best 
performing model were used to calibrate two linear mixed models for each soil property. Model parameters were 
estimated using restricted maximum likelihood. Spatial prediction was performed using the empirical best 
linear 
unbiased predictor. Validation of baseline and best performing linear multiple regression and linear mixed 
models was done using cross-validation. Results show that for CLAY the prediction accuracy did not 
considerably 
improve by using more detailed covariates. The amount of variance explained increased only \num{\sim2} 
percentage points (\si{\pp}), less than that obtained by including the kriging step, which explained 
\SI{4}{\pp}. On the other hand, prediction of SOC and ECEC improved by \SI{\sim13}{\pp} when the baseline 
model 
was replaced by the best performing model. Overall, the increase in prediction performance was modest and may 
not outweigh the extra costs of using more detailed covariates. It may be more efficient to spend extra 
resources on collecting more soil observations, or increasing the detail of only those covariates that have 
the 
strongest improvement effect. In our  case study, the latter would only work for SOC and ECEC, by investing in 
a more detailed land use map and possibly also a more detailed geologic map and DEM.
\end{chapterabstract}

\formatchapter

\section{INTRODUCTION}
\label{sec:chap06-intro}

Modern soil mapping relies on the use of statistical models to produce digital representations of spatial 
soil distribution using point soil observations and spatially exhaustive covariates \cite{McBratneyEtAl2003, 
ScullEtAl2003, Florinsky2012}. Three important weaknesses in the statistical soil distribution modelling 
approach can be pointed out. First, it requires sufficient and appropriately distributed point soil data 
within 
the area being mapped \cite{CarreEtAl2007a}. Second, the model structure explores only the empirical 
relationship among environmental conditions and soil properties, being less comprehensive than soil-landscape 
process models \cite{Grunwald2009}. Last, the covariates are only approximations of the true environmental 
conditions that helped shape the soil. They serve only as proxies (surrogates) of the current environmental 
conditions, which in many cases are different from the past conditions under which pedogenesis took place 
\cite{HeuvelinkEtAl2001}. In spite of these weaknesses, modern soil mapping techniques have proven very 
successful in the past decades in producing soil property maps that capture the main patterns of soil spatial 
variation \cite{MooreEtAl1993, McBratneyEtAl2000, Grunwald2009}.

More recently, there has been a growing interest in understanding how the characteristics of the covariates 
influence the success of soil mapping -- this study contributes to this effort. It is commonly accepted that 
the more resources are spent on the construction of a covariate and the more spatial information it has, the 
more accurately it describes the environmental conditions \cite{HupyEtAl2004, HenglEtAl2013a}. It is also 
generally believed that such \emph{more detailed} covariates will be more valuable for soil mapping and lead 
to 
more accurate soil property predictions \cite{CavazziEtAl2013, MaynardEtAl2014}. If these more detailed 
covariates convey more information and represent more adequately the environmental conditions -- the drivers 
of 
soil forming processes --, then it is fair to expect that they improve the accuracy of the resulting soil 
maps. 
However, some studies have shown the contrary \cite{ThompsonEtAl2001, EldeiryEtAl2008, KimEtAl2014}. For 
example, the window size at which DEM derivatives are calculated can be more important than the spatial 
resolution of the DEM \cite{Wood1996, ZhuEtAl2008, BehrensEtAl2010a}. The uncertainty about the added value of 
using more detailed covariates is of concern for those seeking to use resources efficiently, because using 
more 
detailed covariates generally increases soil mapping costs \cite{ShiEtAl2012}.

The objective of this study was to evaluate whether investing in more detailed covariates improves the 
accuracy of soil maps. The main difference of our study to previous ones is that we use a rigorous statistical 
approach to assess the added value of using five more detailed covariates simultaneously. We used a  case study 
in Brazil to compare the accuracy of maps of the clay content, organic carbon content and effective cation 
exchange capacity of the topsoil as obtained from regression kriging on the five covariates, whereby each 
covariate was evaluated on two levels of spatial detail.

\section{MATERIAL AND METHODS}
\label{sec:chap06-methods}

\subsection{Study Area and Soil Data}
\label{subsec:chap06-soil-data}

The study area constitutes a small catchment (\SI{\sim2000}{\hectare}) located on the southern edge of the 
plateau of the Paraná Sedimentary Basin, Rio Grande do Sul, Brazil (\autoref{fig:chap06-location}). The 
climate is classified as Cfa (K\"oppen -- subtropical humid without a dry season) with mean annual temperature 
of \SI{19.3}{\celsius}, and mean annual precipitation of \SI{1708}{\mm}, well distributed throughout the year 
\cite{Maluf2000}. Relief varies between plain (slope between \num{0} and \SI{3}{\percent}) and mountainous 
(slope between \num{45} and \SI{100}{\percent}), and elevations range between \num{140} and \SI{475}{\m}. 
Geology consists of basic, intermediate and acid igneous rocks (rhyolite-rhyodacite and andesite-basalt) of 
the Cretaceous period, consolidated sedimentary rocks (aeolian and fluvial sandstones) of the Triassic and 
Jurassic periods, and non-consolidated (fluvial and colluvial deposits) of the Quaternary period 
\cite{GasparettoEtAl1988, MacielFilho1990, Sartori2009}. Native semi-deciduous forests occupy more than 
half of the area, followed by native grassland used for animal husbandry, semi-deciduous shrubland, annual 
crop agriculture, forestry (Eucalyptus), urban areas, and artificial water bodies \cite{SamuelRosaEtAl2011a}.

\begin{figure}[!ht]
 \centering
 \begin{minipage}[b]{95mm}
  \subcaption{}
  \label{fig:chap06-brazil}
  \centering
  \includegraphics[width=90mm]{fig/chap06-FIG1a}
 \end{minipage}
 \begin{minipage}[b]{95mm}
  \subcaption{}
  \label{fig:chap06-points}
  \centering
  \includegraphics[width=90mm]{fig/chap06-FIG1b}
 \end{minipage}
 \caption[Location of the study area.]{Location of the study area in Santa Maria (a) and spatial distribution 
of the point soil observations and drainage network (b).}
 \label{fig:chap06-location}
\end{figure}

A dataset containing $n = 350$ point soil observations collected between \num{2004} and \num{2011} 
\cite{PedronEtAl2006b, SamuelRosaEtAl2011a, MiguelEtAl2012, Samuel-RosaEtAl2013} was used in this study 
(available at \url{https://github.com/samuel-rosa/dnos-sm-rs-general}). Sampling locations were selected 
purposively and by convenience \cite{Samuel-RosaEtAl2014b}. Three soil pits were opened within an area of 
\SI{\pm100}{\m\square} at most sampling locations to obtain composite samples of the topsoil for laboratory 
analysis. Soil was collected to a depth of \SI{20}{\cm} or less when soil depth was smaller than \SI{20}{\cm}. 
A few observations ($n = 10$) correspond to individual samples collected up to \SI{30}{\cm}. Sampling depth 
ranges from \num{2} to \SI{30}{\cm}, with a mean of \SI{17.3}{\cm}. We assumed that the vertical, horizontal 
and temporal support differences between soil samples is negligible for the purpose of this study.

Three soil properties (fine earth fraction, \SI{<2}{\mm}) were explored: clay content (CLAY, 
\si{\gram\per\kilo\gram}), organic carbon content (SOC, \si{\gram\per\kilo\gram}), and effective cation 
exchange capacity (ECEC, \si{\milli\mole\per\kilo\gram}). CLAY was determined by the pipette method. SOC was 
determined using wet digestion. ECEC was calculated as the sum of exchangeable bases plus exchangeable 
acidity. The soil properties selected were expected to present different patterns of spatial variation and 
correlation with the most dominant factors of soil formation \cite{Jenny1941} in the area: organisms 
(\textit{O}), relief (\textit{R}), and parent material (\textit{P}). CLAY was presumed to have a stronger 
relation with \textit{P}, while SOC was expected to be more correlated with \textit{O}. Because the soils of 
the study area were strongly eroded due to intense agriculture in the \num{20}th century, both CLAY and SOC 
were also expected to be closely related with \textit{R}. Finally, ECEC was expected to be strongly correlated 
with \textit{P} and \textit{O}, which is supported by its natural relationship with both CLAY and SOC.

\begin{figure}[!ht]
 \centering
 \begin{minipage}[b]{63mm}
  \subcaption{}
  \centering
  \includegraphics[width=63mm]{fig/chap06-FIG2a}
 \end{minipage}
 \begin{minipage}[b]{63mm}
  \subcaption{}
  \centering
  \includegraphics[width=63mm]{fig/chap06-FIG2d}
 \end{minipage}
 \begin{minipage}[b]{63mm}
  \subcaption{}
  \centering
  \includegraphics[width=63mm]{fig/chap06-FIG2b}
 \end{minipage}
 \begin{minipage}[b]{63mm}
  \subcaption{}
  \centering
  \includegraphics[width=63mm]{fig/chap06-FIG2e}
 \end{minipage}
 \begin{minipage}[b]{63mm}
  \subcaption{}
  \centering
  \includegraphics[width=63mm]{fig/chap06-FIG2c}
 \end{minipage}
 \begin{minipage}[b]{63mm}
  \subcaption{}
  \centering
  \includegraphics[width=63mm]{fig/chap06-FIG2f}
 \end{minipage}
 \caption[Summary statistics of CLAY, SOC, and ECEC.]{Histogram, empirical density function, and summary 
statistics of CLAY (a, b), SOC (c, d), and ECEC (e, f) in the original (left) and Box-Cox feature spaces 
(right).}
 \label{fig:chap06-soil-properties}
\end{figure}

Point soil data, here denoted by $Y(s)$, showed a positive skew (\autoref{fig:chap06-soil-properties}) and was 
normalized, $Y'(s)$, using the Box-Cox family of power transformations, where $Y'(s) = (Y(s)^{\lambda} - 1) / 
\lambda$, if $\lambda > 0$, and $Y'(s) = log(Y(s))$, if $\lambda = 0$ \cite{DiggleEtAl2007}. Lambda 
($\lambda$) values were selected empirically \cite{FoxEtAl2011}. Because the resulting distribution of the 
back-transform (see \autoref{subsec:chap06-validation}) has no expectation when $\lambda < 0$ 
\cite{RibeiroEtAl2001}, a logarithm transformation ($\lambda = 0$) was used when a negative $\lambda$ was 
estimated (SOC and ECEC).

\subsection{Covariates}
\label{subsec:chap06-sources}

Five freely available covariates were evaluated in this study, each with two levels of spatial detail: 
area-class soil maps (\texttt{soil}), geologic maps (\texttt{geo}), land use maps (\texttt{land}), digital 
elevation models (\texttt{dem}), and satellite images (\texttt{sat}). Each pair was composed of covariates 
that were produced separately from scratch using different data sources and/or production methods, thus 
demanding different amounts of resources (time, workforce, budget, technology, etc.). In this study, the level 
of spatial detail of a covariate is a function of the components of its production process such as the 
cartographic ratio (\texttt{soil}, \texttt{geo} and \texttt{land}), spatial sampling support (\texttt{sat}), 
number and diversity of data sources explored (\texttt{dem}), and quantity of spatial data used (all five). 
Thus, the reader should bear in mind that our definition of spatial detail is broader than spatial resolution 
or spatial scale. It should also not be confounded with spatial support \cite{WebsterEtAl2007} or thematic 
detail \cite{Rossiter2000}.

\def\footcovars{\footnote{In statistical terms, the terms \emph{covariate} and \emph{predictor variable} are 
synonymous, and the reason for the use given in this study is purely operational.}}

The covariates were transformed to predictor variables\footcovars{} that were used in the geostatistical 
modelling. Since the transformation is different for categorical and continuous covariates, the procedures are 
explained below for each type separately.

\subsubsection{Categorical predictor variables}
\label{subsubsec:chap06-categorical-covars}

Area-class soil maps, geologic maps and land use maps are categorical covariates (factors). Mapping units are 
the $k$ factor levels that are transformed to as many dummy (indicator, binary) variables as there are factor 
levels, before model calibration. Each dummy variable receives a value equal to one (\num{1}) when a given 
class is present, and zero (\num{0}) otherwise \cite{Everitt2006}. If the number of point soil observations 
falling inside the spatial domain of a mapping unit is too small to accurately estimate a regression 
coefficient (we used a threshold of $n = 15$ observations), the mapping unit is merged with a similar mapping 
unit prior to calculating dummy variables. The resulting generalized categorical covariate maps are shown in 
\autoref{fig:chap06-cat-covars}. The binary maps are the categorical predictor variables.

\noindent\textit{Soil maps}. The less detailed soil map (\soilOld) was published with a \scale{100000} and 
has five mapping units \cite{AzolinEtAl1988} (\autoref{fig:chap06-soil-old}). It was produced using existing 
soil maps and technical reports (\scale{750000}) \cite{Brasil1973}, aerial photographs (\scale{60000}), 
topographic maps (\scale{50000}), and sparse point soil observations along the road network. The more detailed 
soil map (\soilNew) was prepared with a \scale{25000} and has eight mapping units \cite{MiguelEtAl2012} 
(\autoref{fig:chap06-soil-new}). It was produced using high spatial resolution satellite images 
(\SI{65}{\cm}), existing soil maps and technical reports published with a \scale{50000} \cite{Poelking2007} 
and \num{1}:\num{25000} \cite{PedronEtAl2006b}, topographic maps (\scale{25000}), and descriptions from 
\num{\sim350} point soil observations. Five dummy predictor variables were derived from \soilOld{} and seven 
from \soilNew{} (\autoref{tab:chap06-soil-covars}).

\begin{figure}[!ht]
 \centering
 \begin{minipage}[b]{63mm}
  \subcaption{Cartographic scale: \num{1}:\num{100000}}
  \label{fig:chap06-soil-old}
  \centering
  \includegraphics[width=60mm]{fig/chap06-FIG3a}
 \end{minipage}
 \begin{minipage}[b]{63mm}
  \subcaption{Cartographic scale: \num{1}:\num{25000}}
  \label{fig:chap06-soil-new}
  \centering
  \includegraphics[width=60mm]{fig/chap06-FIG3d}
 \end{minipage}    
 \begin{minipage}[b]{63mm}
  \subcaption{Cartographic scale: \num{1}:\num{50000}}
  \label{fig:chap06-geo-old}
  \centering
  \includegraphics[width=60mm]{fig/chap06-FIG3b}
 \end{minipage}
 \begin{minipage}[b]{63mm}
  \subcaption{Cartographic scale: \num{1}:\num{25000}}
  \label{fig:chap06-geo-new}
  \centering
  \includegraphics[width=60mm]{fig/chap06-FIG3e}
 \end{minipage}
 \begin{minipage}[b]{63mm}
  \subcaption{Cartographic scale: \num{1}:\num{500000}}
  \label{fig:chap06-land-old}
  \centering
  \includegraphics[width=60mm]{fig/chap06-FIG3c}
 \end{minipage}
 \begin{minipage}[b]{63mm}
  \subcaption{Cartographic scale: \num{1}:\num{2000}}
  \label{fig:chap06-land-new}
  \centering
  \includegraphics[width=60mm]{fig/chap06-FIG3f}
 \end{minipage}
 \caption[Area-class soil maps, geologic maps, and land use maps compared in the study. ]{Area-class soil maps 
(a, b), geologic maps (c, d), and land use maps (e, f) compared in our study. The less and more detailed 
version are displayed at the left and right, respectively. Legend abbreviations and derived dummy variables are 
described in Tables \ref{tab:chap06-soil-covars}--\ref{tab:chap06-land-covars}.}
 \label{fig:chap06-cat-covars}
\end{figure}

\noindent\textit{Geologic maps}. The less detailed geologic map (\geoOld) was produced using topographic maps 
with \scale{50000} \cite{GasparettoEtAl1988} (\autoref{fig:chap06-geo-old}). The more detailed geologic map 
(\geoNew) was produced using topographic maps with \scale{25000}, and includes the location of overlaying 
Quaternary sedimentary deposits \cite{MacielFilho1990} (\autoref{fig:chap06-geo-new}). \geoNew{} did not cover 
a small part in the North of the study area, where \geoOld{} was used instead (this strategy was approved by 
experts on the local geology). The mapping unit of both geologic maps depicting the Caturrita Formation was 
used indirectly by deriving dummy predictor variables from all other individual mapping units. Three dummy 
predictor variables were derived from \geoOld{} and four from \geoNew{} (\autoref{tab:chap06-geology-covars}).

\noindent\textit{Land use maps}. The less detailed land use map (\landOld) was produced by manually digitizing 
land use data included in topographic maps with a \scale{25000} \cite{DSG1980, DSG1992, DSG1992a} 
(\autoref{fig:chap06-land-old}). The more detailed land use map (\landNew) was prepared (\scale{2000}) by 
manual digitization using \SI{65}{\cm} spatial resolution satellite images covering the years \num{2008} and 
\num{2009} \cite{SamuelRosaEtAl2011a} (\autoref{fig:chap06-land-new}). Mapping units depicting human 
settlements and water bodies ($n = 0$) were not masked out from the prediction grid and were merged with other 
mapping units to derive dummy predictor variables. Five dummy predictor variables were derived from \landNew{} 
and two from \landOld{} (\autoref{tab:chap06-land-covars}).

\subsubsection{Continuous predictor variables}
\label{subsubsec:chap06-continuous-covars}

\def\arcgis{\href{http://resources.arcgis.com/en/help/main/10.1/index.html}{ArcGIS}}

The less detailed DEM (\demOld) is the hole-filled SRTM DEM version~\num{4} \cite{JarvisEtAl2008} 
(\autoref{fig:chap06-dem-old}). The spatial sampling support of the SRTM DEM is \SI{1}{\arcsecond} 
(\SI{\sim30}{\m}), but elevation data were aggregated to \SI{3}{\arcsecond} (\SI{\sim90}{\m}) for public 
release in regions outside the United States \cite{ReuterEtAl2007}. The more detailed DEM (\demNew) was 
produced by interpolating contour lines with vertical spacing of \SI{10}{\m} along with data about the 
drainage network, lakes and peaks digitized from topographic maps with \scale{25000} 
(\autoref{fig:chap06-dem-new}). Interpolation to \SI{5}{\m} pixel size was performed using a hydrologically 
correct algorithm implemented in \arcgis{} software by ESRI \cite{Hutchinson1989}. Contour line artefacts were 
minimized using a seven by seven low-pass filter (\grass{r.neighbors}). The window size was chosen such that 
the smoothed DEM best matched the original contour map while also respecting the original drainage network 
pattern.

\input{chap/tab/chap06-TAB1.tex}

\ctable[
 caption  = {Description of the $p = 7$ dummy predictor variables derived from the two geologic maps.},
 cap      = {Predictor variables derived from geologic maps.},
 label    = tab:chap06-geology-covars,
 notespar,
 pos      = !ht,
 maxwidth = \textwidth,
 % doinside = \scriptsize\setstretch{1.1}
 doinside = \small
 ]{l p{0.85\textwidth} l}{
 \tnote[a]{Minimum Legible Delineation calculated following \citet{Rossiter2000}.}
 }{ \FL
 Code & Mapping unit(s) included and Description\tmark[a] \ML
 
 \multicolumn{2}{p{0.98\linewidth}}{Source: \citet{GasparettoEtAl1988}. Cartographic scale: 
 \num{1}:\num{50000}. Minimum Legible Delineation: \SI{10}{\hectare}.} \NN
 
 \texttt{GEO\_50a} & \textit{SG-I}. Inferior Sequence of the Serra Geral Formation. Composed mainly of basic 
 igneous rocks (tholeiitic basalt and andesite). It is likely to be related with high CLAY and ECEC. \NN
 \texttt{GEO\_50b} & \textit{SG-S}. Superior Sequence of the Serra Geral Formation. Composed mainly of acid 
 igneous rocks (granophyric rhyolite and rhyodacite). It is likely to be related with moderate to high CLAY 
 and ECEC. \NN
 \texttt{GEO\_50c} & \textit{BT}. Botucatu Formation. Composed mainly of aeolian sandstones. It is likely to 
 be related with low CLAY and ECEC. \NN
 Other & \textit{CT} depicts the Caturrita Formation, which is composed mainly of fluvial sandstones. \NN
 & \NN
 
 \multicolumn{2}{p{0.98\linewidth}}{Source: \citet{MacielFilho1990}. Cartographic scale: 
 \num{1}:\num{25000}. Minimum Legible Delineation: \SI{2.5}{\hectare}.} \NN
 
 \texttt{GEO\_25a} & \textit{SG-I}. Inferior Sequence of the Serra Geral Formation. \NN
 \texttt{GEO\_25b} & \textit{SG-S}. Superior Sequence of the Serra Geral Formation. \NN
 \texttt{GEO\_25c} & \textit{BT}. Botucatu Formation. \NN
 \texttt{GEO\_25d} & \textit{QD}. Quaternary deposits of fluvial, alluvial, and colluvial origin. It can help 
 explaining the low CLAY of soils supposedly derived from igneous rocks. \NN
 Other & \textit{CT} depicts the Caturrita Formation. \LL
 }


\input{chap/tab/chap06-TAB3.tex}

\begin{figure}[!ht]
 \centering
 \begin{minipage}[b]{63mm}
  \subcaption{Spatial resolution: \SI{90}{\m}}
  \label{fig:chap06-dem-old}
  \centering
  \includegraphics[width=60mm]{fig/chap06-FIG4a}
 \end{minipage}
 \begin{minipage}[b]{63mm}
  \subcaption{Spatial resolution: \SI{30}{\m}}
  \label{fig:chap06-sat-old}
  \centering
  \includegraphics[width=60mm]{fig/chap06-FIG4b}
 \end{minipage}
 \begin{minipage}[b]{63mm}
  \subcaption{Vertical spacing of contours: \SI{10}{\m}}
  \label{fig:chap06-dem-new}
  \centering
  \includegraphics[width=60mm]{fig/chap06-FIG4c}
 \end{minipage}
 \begin{minipage}[b]{63mm}
  \subcaption{Spatial resolution: \SI{5}{\m}}
  \label{fig:chap06-sat-new}
  \centering
  \includegraphics[width=60mm]{fig/chap06-FIG4d}
 \end{minipage}
 \caption[Digital elevation models and satellite images compared in the study.]{Digital elevation models (a, c) 
and satellite images, depicted using the normalized difference vegetation index (b, d), compared in our study. 
The less detailed version is displayed at the top, while the more detailed version is shown on the bottom.}
 \label{fig:chap06-con-covars}
\end{figure}

Eight DEM derivatives were calculated: elevation (\elev), slope (\slp), aspect (\asp), northernness (\nor), 
flow accumulation (\acc), topographic wetness index (\twi), stream power index (\spi), and topographic 
position index (\tpi). \slp{} and \asp{} were calculated using \grass{r.param.scale} with seven window sizes 
(sampling support, analysis scale): \num{3}, \num{7}, \num{15}, \num{31}, \num{63}, \num{127}, and \num{255}. 
\asp{} was scaled to the standard \num{0}--\ang{360} range and orientation, and was transformed to \nor{} 
using $\texttt{NOR} = abs(\ang{180} - \texttt{ASP})$. \twi{} and \spi{} were calculated using \slp{} 
calculated with different window sizes, and \acc{} calculated using \grass{r.watershed}. \tpi{} was calculated 
using \saga{ta\_morphometry} with the same seven window sizes. The combination of DEM derivatives (\elev, 
\slp, \nor, \twi, \spi, and \tpi) and window sizes yielded $p = 36$ continuous predictor variables from each 
DEM.

The less detailed satellite image was acquired by the Landsat-\num{5} Thematic Mapper on December \num{26}, 
\num{2010} (available at Instituto Nacional de Pesquisas Espaciais - Divisão de Geração de Imagens -- 
\inpedgi) (\autoref{fig:chap06-sat-old}). It has \SI{8}{\bit} radiometric resolution and \SI{\sim30}{\m} 
spatial resolution. Spectral bands were orthorectified (Geomatica OrthoEngine) and radiometrically corrected 
(\grass{i.landsat.toar}). The more detailed satellite image comes from the RapidEye constellation (available 
at Ministério do Meio Ambiente -- \mma) (\autoref{fig:chap06-sat-new}). It was acquired on November \num{16}, 
\num{2012}, has \SI{16}{\bit} radiometric resolution, \SI{6.5}{\m} spatial resolution, and was orthorectified 
to \SI{5}{\m} spatial resolution. Both images were atmospherically (6S atmospheric model 
\cite{VermoteEtAl1997}, \grass{i.atcorr}) and topographically corrected (\grass{i.topo.corr}). Derived 
predictor variables are the spectral bands (except the thermal band) and vegetation indices (normalized 
difference vegetation index - NDVI, and soil-adjusted vegetation index - SAVI). Eight continuous predictor 
variables were derived from the Landsat-5~TM image and nine from the RapidEye image.

\subsubsection{Additional processing}
\label{subsubsec:chap06-sources-processing}

Soil maps, geologic maps, land use maps, and satellite images were registered with the prediction grid 
(\SI{5}{\m} pixel size) using nearest neighbour resampling. \demOld{} was registered using cubic resampling 
\cite{Samuel-RosaEtAl2013c}. Systematic positional errors were corrected using affine transformation 
\cite{Samuel-RosaEtAl2014}.

\subsection{Linear Mixed Model of Spatial Variation}
\label{subsec:chap06-lmm}

We model each of the soil properties of interest as the outcome of a spatial stochastic process. The model is 
composed of fixed and random effects \cite{HeuvelinkEtAl2001, LarkEtAl2006}. We use the point soil 
observations and spatially exhaustive predictor variables to calibrate the model and predict the outcome of 
the spatial stochastic process at unobserved locations. This fixed effect (deterministic trend), 
$m(\textbf{s})$, describes that part of the spatial variation of the soil property that is explained by the 
covariates. We assume here that is a linear function of the predictor variables. The random effect (stochastic 
residuals, latent variables), $e(\textbf{s})$, describes that part of the spatial variation that cannot be 
explained by the covariates \cite{Cressie1993}. It is represented by a spatially correlated, Gaussian 
distributed random variable, that is assumed stationary in the mean and covariance. Thus, the linear mixed 
model of spatial variation that we employed is given by

\begin{equation}\label{eqn:chap06-lmm}
 Y'(\textbf{s}) = m(\textbf{s}) + e(\textbf{s}) 
                = \sum_{j=0}^{p} \beta_{j}\cdot X_{j}(\textbf{s}) + e(\textbf{s}),
\end{equation}

\noindent{where $Y'(\textbf{s})$ is the soil property after Box-Cox transformation, $m(\textbf{s})$ and 
$e(\textbf{s})$ are defined as above, $\beta_{j}$ are the regression model coefficients, and 
$X_{j}(\textbf{s})$ is the regression model matrix, with $j = 0, 1, 2, \ldots, p$, $p$ being the number of 
predictor variables. Variable $X_{0}(\textbf{s})$ is taken as unity so that $\beta_{0}$ is the intercept.}

\subsubsection{Model selection}

We calibrated $k = 2^5 = 32$ multiple linear regression models for each soil property (fitted using ordinary 
least squares, OLS) to model the deterministic trend for each combination of the five covariates (recall from 
\autoref{sec:chap06-intro} that each covariate is available at two levels of spatial detail, hence $2^5$ 
combinations). The number of predictor variables used to calibrate each model varied among combinations 
between $p = 52$ and $p = 62$, because more detailed covariates enabled the derivation of a larger number of 
predictor variables (except the DEM). Backward VIF (variance inflation factor) selection followed by stepwise 
AIC (Akaike's Information Criterion) selection were used to select predictor variables to enter the models 
\cite{Samuel-RosaEtAl2014c, VenablesEtAl2002}.

The $k = 32$ multiple linear regression models calibrated for each soil property were ranked using the ratio 
between the regression sum of squares and the total sum of squares. Because stepwise regression results in 
biased models \cite{Harrell2001a}, the ratio of sum of squares was adjusted (${R}^{2}_{adj}$) using the number 
of predictor variables initially offered to enter the model instead of the reduced number of predictor 
variables that entered the model. Next, the five covariates were ranked based on how their level of spatial 
detail related with the calibration of models with improved predictive performance. The relation between the 
level of spatial detail of the covariates and model performance was evaluated using a graphical output called 
\emph{model series plot} (\Rpackage{pedometrics}, \citet{Samuel-RosaEtAl2014c}). Pedological evaluation 
of predictor variables included in the models was omitted because this was beyond our objectives.

The multiple linear regression model calibrated using only the less detailed covariates, which we call the 
\emph{baseline} model, and the multiple linear regression model with the highest ${R}^{2}_{adj}$, which we 
call the \emph{best performing} model, were extended to linear mixed models of spatial variation 
(\autoref{eqn:chap06-lmm}) for each soil property. Estimation of the parameters of the linear mixed models was 
performed using residual (restricted, marginal) maximum likelihood (REML) \cite{RibeiroEtAl2001, 
LarkEtAl2004}. The spatial correlation function adopted was the exponential function (this is equivalent to 
the Matérn correlation function with smoothness parameter $\nu = 0.5$ \cite{Stein1999}).

\subsubsection{Model validation}
\label{subsec:chap06-validation}

Only the \emph{baseline} and \emph{best performing} multiple linear regression and linear mixed models 
calibrated for each soil property were validated. Model validation was performed using leave-one-out 
cross-validation (LOO-CV) \cite{BrusEtAl2011}. All model parameters were re-estimated at each LOO-CV run 
to reduce bias \cite{LaslettEtAl1987}. LOO-CV predicted values were back-transformed from the Box-Cox space 
to the original space of soil properties using stochastic simulation \cite{ChristensenEtAl2001}:

\begin{enumerate}[label=(\Roman*)]
 \item each predicted value and associated prediction error variance were used to simulate $n = \num{20000}$ 
 values from a Gaussian distribution;
 
 \item simulated values were back-transformed using $Y(s) = (Y'(s) \times \lambda + 1)^{1 / \lambda}$, if 
 $\lambda > 0$, and $Y(s) = exp(Y'(s))$, if $\lambda = 0$;
 
 \item the mean and variance of back-transformed simulated values were used as the predicted value and 
 prediction error variance in the original space of soil properties.
\end{enumerate}

Five error statistics were computed from the leave-one-out cross-validation results \cite{JanssenEtAl1995, 
KempenEtAl2010, BrusEtAl2011}. The mean error (\textit{ME}), which measures the prediction bias, the mean 
absolute error (\textit{MAE}) and the root mean squared error (\textit{RMSE}), which measure the prediction 
accuracy, the scaled root mean squared error (\textit{SRMSE}, also known as mean squared deviation ratio), 
which measures how well the prediction error variance matches the squared differences between predicted and 
observed soil property, where $\textit{SRMSE} > 1$ indicates under-estimation, while $\textit{SRMSE} < 1$ 
indicates over-estimation, and the amount of variance explained (\textit{AVE}, also known as coefficient of 
determination or ratio of scatter), which measures the fraction of the overall spread of observed values that 
is explained by the model. The AVE ranges from \num{0} to \num{100}, where $\textit{AVE} = 100$ is the optimal 
value.

\subsubsection{Spatial prediction}
\label{subsec:chap06-prediction}

Only the \emph{baseline} and \emph{best performing} linear mixed models calibrated for each soil property were 
used for spatial prediction. Spatial predictions at a fine grid of \num{\sim800000} point locations were made 
in the Box-Cox space using the best linear unbiased predictor (BLUP) with the empirical estimates of the 
random effects (EBLUP) \cite{LarkEtAl2006}. EBLUP with a fixed effect model is conceptually equivalent to 
kriging with external drift and regression kriging, and mathematically equivalent to kriging with external 
drift and universal kriging. Point predicted values and prediction error variances were back-transformed to the 
original soil property space using stochastic simulation as described above 
(\autoref{subsec:chap06-validation}).

\section{RESULTS}
\label{sec:chap06-results}

\subsection{Model Series Plots}

The model series plot is a graphical description of the relation between the prediction accuracy of multiple 
linear regression models and the covariates used to calibrate them (\autoref{fig:chap06-model-series}). The 
magnitude of improvement in prediction accuracy is depicted in the bottom panel with the ${R}^{2}_{adj}$. The 
top panel is interpreted both horizontally and vertically. In the vertical direction we identify which version 
of each covariate was used to calibrate a given model. The less and the more detailed versions are identified 
by the yellow (bright) and green (dark) colours, respectively. The \emph{baseline} model is identified by the 
column containing only yellow cells, while the column with only green cells represents the model calibrated 
using only the more detailed version of each covariate, which we call the \emph{most detailed} model. The 
first important results that we obtain from the model series plots is that a) the \emph{baseline} model is not 
the model with the lowest ${R}^{2}_{adj}$, which we call the \emph{poorest performing} model, and b) the 
\emph{most detailed} model is not the \emph{best performing} model.

\begin{figure}[!ht]
 \centering
 \begin{minipage}[b]{\textwidth}
  \subcaption{}
  \includegraphics[width=\textwidth]{fig/chap06-FIG5a}
 \end{minipage}
 \begin{minipage}[b]{\textwidth}
  \subcaption{}
  \includegraphics[width=\textwidth]{fig/chap06-FIG5b}
 \end{minipage}
 \begin{minipage}[b]{\textwidth}
  \subcaption{}
  \includegraphics[width=\textwidth]{fig/chap06-FIG5c}
 \end{minipage}
 \caption[Model series plots for CLAY, SOC, and ECEC.]{Model series plots for CLAY (a), SOC (b), and ECEC (c). 
The less and more detailed version of each covariate are identified with the yellow (bright) and green (dark) 
colours, respectively. Multiple linear regression models were ranked using their ${R}^{2}_{adj}$. Triangles 
show the mean ranking of the more detailed covariates (i.e. centre of green cells).}
 \label{fig:chap06-model-series}
\end{figure}

The row-wise analysis of the model series plots shows if a model calibrated with the more detailed version of 
a given covariate has a higher prediction accuracy. This information is retrieved by looking at the row-wise 
distribution of green cells -- these cells represent the $k = 16$ models calibrated using the more detailed 
version of a given covariate, irrespective of the version of the other covariates. The more concentrated the 
green cells are in the right half of the plot, the larger the relative benefit of using the more detailed 
version of that covariate. For example, the top row of the second model series plot shows the SOC models 
calibrated using the two versions of the land use map (\texttt{land}). All green cells are on the right half 
of the plot between rankings \num{1} and \num{16} (see the x axis). The four lower rows show that the green 
cells of the other four covariates are distributed along the entire ranking range (from \num{1} to \num{32}). 
This means that the relative benefit of calibrating a SOC model with a more detailed land use map is larger 
compared to that of using a more detailed version of the other covariates.

The centre of the row-wise distribution of the green cells for each covariate, calculated as the mean ranking, 
is represented by the triangles. The mean ranking quantifies the relative benefit of using a more detailed 
version of each covariate. For example, the mean ranking of the SOC models calibrated using the more detailed 
land use map is about \num{8} (top row), while the mean ranking of the models calibrated using the more 
detailed satellite image (\texttt{sat}) is close to \num{20} (bottom row). Using the more detailed DEM 
(\texttt{dem}) is almost as beneficial as using the more detailed geologic map (\texttt{geo}) -- the mean 
ranking of the SOC models calibrated using the more detailed version of these two covariates is about 
\num{15}--\num{16} (second and third rows). Using the more detailed version of the soil map (\texttt{soil}, 
fourth row) is not as beneficial as using \texttt{land}, \texttt{geo} or \texttt{dem}, but more beneficial 
than using \texttt{sat}. Because the covariates were ranked based on the mean rankings, the covariate 
displayed in the top row of each model series plot is the one which resulted in the largest improvement of the 
prediction accuracy when the more detailed version was used to calibrate the model -- for SOC this is the land 
use map.

For CLAY, calibrating the models with the more detailed soil map resulted in the largest improvement of the 
prediction accuracy relative to the other covariates. The DEM was the second most beneficial covariate (mean 
ranking of \num{15}), but the benefit of using its more detailed version was similar to that of using the more 
detailed version of any other covariate (mean rankings between \num{17} and \num{18}). Nine models had a 
poorer prediction performance than the baseline model, ranked \num{27}th, the poorest performing model being 
that calibrated with the more detailed land use map and satellite image. Despite these patterns, calibrating 
CLAY models with the more detailed version of any covariate resulted in a small improvement of the prediction 
accuracy, as evidenced by the small increases of the ${R}^{2}_{adj}$. The difference between the poorest and 
best performing models is less than \SI{3}{\pp} (percentage points). In comparison, for SOC, by simply 
using the more detailed land use map we already obtained a model ranked \num{9}th, an increase of \SI{8}{\pp} 
in ${R}^{2}_{adj}$ compared to the baseline model, ranked \num{24}th.

The same general pattern observed for SOC models was observed for ECEC models -- the more detailed land use 
map results in the largest improvement of the prediction accuracy. The main difference is that calibrating the 
models with the more detailed geologic map was slightly more beneficial for ECEC (mean ranking of \num{12}) 
than for SOC (mean ranking of \num{14}). The poorest performing ECEC model was that calibrated with the more 
detailed satellite image. Using only the more detailed land use map resulted in an improvement of \SI{6}{\pp} 
in ${R}^{2}_{adj}$ (model ranked \num{7}th), differing from the best performing model by only \SI{2}{\pp}. 
Using the more detailed version of all covariates except the soil map or satellite image resulted in increases 
of about \num{6} and \SI{7}{\pp} in ${R}^{2}_{adj}$, respectively. The baseline model was ranked as 
\num{28}th, which is a higher ranking than the models calibrated with all possible combinations of the more 
detailed satellite image and the more detailed soil map and/or DEM.

The patterns observed in the model series plots resulted from the change (increase or decrease) of the 
importance of each covariate on explaining the variance when the more detailed version was used 
(\autoref{tab:chap06-drop}). We used the \emph{baseline} and \emph{most detailed} models to quantify this 
change. Each model was refitted dropping one covariate at a time. The difference $\Delta$ between the 
${R}^{2}_{adj}$ of the model calibrated with all five $q$ covariates (${R}^{2}_{adj}{}_{q = 5}$) and the model 
calibrated without the $q$-th covariate ($R^{2}_{adj}{}_{q = 5 - 1}$) was calculated. The more positive 
$\Delta{R}^{2}_{adj}$ becomes, the more beneficial the more detailed version of the $q$-th covariate is for 
improving prediction accuracy. For CLAY, \texttt{dem} and \texttt{land} were the most important covariates in 
the \emph{baseline} model, while \texttt{geo} was the least important. The importance of \texttt{soil} and 
\texttt{geo} increased when their more detailed version was used (change of \SI{+0.013}{\pp} for both), while 
\texttt{sat}, \texttt{land} and \texttt{dem} became less important. For SOC and ECEC, \texttt{land} was not 
the most important covariate in the \emph{baseline} model. But it was the covariate whose importance had the 
largest positive shift when the more detailed version was used (\SI{+0.085}{\pp} for SOC and \SI{+0.045}{\pp}
for ECEC). \texttt{sat} became less important when the more detailed version was used -- see its low ranking 
in all model series plots. The increase of the importance of \texttt{geo} was larger for ECEC 
(\SI{+0.026}{\pp}) than for SOC (\SI{+0.013}{\pp}) -- see the difference in the mean ranking of \texttt{geo} 
in 
the SOC (\num{14}) and ECEC (\num{12}) model series plots.

\input{chap/tab/chap06-TAB4.tex}

\subsection{REML Fit of the Variogram Model}

\def\footnugget{\footnote{To be more precise, the small number of point observations separated by short 
distances reduces the ability of modelling the behaviour of the variogram near the origin as a whole.}}

The small improvement in the prediction accuracy of the CLAY linear mixed model calibrated with the more 
detailed covariates is evidenced by \autoref{fig:chap06-lmm}. The shape of the experimental variogram is very 
similar for both \emph{baseline} and \emph{best performing} linear mixed models, which is also true for SOC 
and ECEC. However, the sill variance had a very small reduction for CLAY compared to SOC and ECEC. The last 
two showed a more considerable improvement in prediction accuracy. It can also be seen that the number of 
point observations separated by short distances is very small, reducing the accuracy of the estimate of the 
nugget variance\footnugget{}. The result is that the estimated nugget variance changes rather erratically 
from the \emph{baseline} to the \emph{best performing} models, decreasing for CLAY and SOC, and increasing 
for ECEC.

\begin{figure}[!ht]
 \centering
 \begin{minipage}[b]{90mm}
  \subcaption{}
  \includegraphics[width=90mm]{fig/chap06-FIG6a} 
 \end{minipage}
 \begin{minipage}[b]{90mm}
  \subcaption{}
  \includegraphics[width=90mm]{fig/chap06-FIG6b}
 \end{minipage}
 \begin{minipage}[b]{90mm}
  \subcaption{}
  \includegraphics[width=90mm]{fig/chap06-FIG6c}
 \end{minipage}
 \caption[Linear mixed models for CLAY, SOC, and ECEC.]{Experimental variogram (dots) and REML fit of the 
linear mixed models (line) for CLAY (a), SOC (b), and ECEC (c). Left -- baseline model. Right -- best 
performing model.}
 \label{fig:chap06-lmm}
\end{figure}

\subsection{Validation}

The LOO-CV results indicate that the linear mixed models for CLAY are slightly positively biased, while 
those for SOC and ECEC are slightly negatively biased (\autoref{tab:chap06-cv-stats}). For both CLAY and ECEC, 
the \textit{MAE} shows that these models are more accurate than the multiple linear regression models, 
suggesting that the kriging step improves the prediction accuracy.

\ctable[
 caption = {Statistics$^a$ of the leave-one-out cross-validation of baseline and best performing multiple
 linear regression models (LM) and linear mixed models (LMM).},
 cap     = {Cross-validation of baseline and best performing models.},
 label   = tab:chap06-cv-stats,
 notespar,
 maxwidth = \textwidth,
 pos     = !th,
 % doinside = \scriptsize\setstretch{1.1}
 doinside = \small
 ]{llrrrrr}{
 \tnote[a]{Statistics: mean error (\textit{ME}), mean absolute error (\textit{MAE}), root mean squared error 
 (\textit{RMSE}), scaled root mean squared error (\textit{SRMSE}, unitless), and amount of variance explained 
 (\textit{AVE}, percent).}
 }{\FL
   \multicolumn{1}{l}{Model}&\multicolumn{1}{c}{Type}&\multicolumn{1}{c}{\textit{ME}}&\multicolumn{1}{c}{\textit{MAE}}&\multicolumn{1}{c}{\textit{RMSE}}&\multicolumn{1}{c}{\textit{SRMSE}}&\multicolumn{1}{c}{\textit{AVE}}\ML
   \multicolumn{7}{l}{CLAY (g kg$^{-1}$)}\NN
   ~~Baseline&LM&$ 1.31$&$52.1$&$ 72.1$&$0.89$&$56.8$\NN
   ~~&LMM&$ 0.94$&$48.5$&$ 68.8$&$1.03$&$60.7$\NN
   ~~Best performing&LM&$ 1.59$&$51.3$&$ 70.7$&$0.91$&$58.4$\NN
   ~~&LMM&$ 1.08$&$47.8$&$ 68.1$&$1.03$&$61.5$\ML
   \multicolumn{7}{l}{SOC (g kg$^{-1}$)}\NN
   ~~Baseline&LM&$-0.30$&$10.9$&$ 18.9$&$1.22$&$35.8$\NN
   ~~&LMM&$-0.39$&$11.0$&$ 19.4$&$1.43$&$32.5$\NN
   ~~Best performing&LM&$-0.20$&$10.1$&$ 16.9$&$0.91$&$49.0$\NN
   ~~&LMM&$-0.25$&$10.4$&$ 17.6$&$1.16$&$44.3$\ML
   \multicolumn{7}{l}{ECEC (mmol kg$^{-1}$)}\NN
   ~~Baseline&LM&$-0.88$&$70.6$&$112.4$&$0.97$&$22.3$\NN
   ~~&LMM&$-0.32$&$63.3$&$101.1$&$1.32$&$37.1$\NN
   ~~Best performing&LM&$-0.76$&$64.9$&$101.7$&$0.86$&$36.3$\NN
   ~~&LMM&$-0.29$&$62.6$&$ 97.9$&$1.09$&$41.1$\LL
}


Overall, all models had a moderate to poor prediction performance. The errors are, in absolute values, 
somewhat large, mainly for ECEC. The best \textit{AVE} are about \SI{60}{\percent} for CLAY, \SI{50}{\percent} 
for SOC, and \SI{40}{\percent} for ECEC. In general, the prediction error variance was under-estimated by the 
linear mixed models and over-estimated by the multiple regression models. The best estimates of the prediction 
error variance were obtained by both CLAY linear mixed models, and the ECEC baseline linear regression model.

For CLAY, the increase in the \textit{AVE} was larger when including a kriging step ($\Delta\textit{AVE} = 
\SI{3.9}{\pp}$) than when using more detailed covariates ($\Delta\textit{AVE} = \SI{1.6}{\pp}$). In the case 
of SOC, including a kriging step reduced the \textit{AVE} by \SI{3.2}{\pp}, and for ECEC, both strategies 
increased the \textit{AVE} (\autoref{tab:chap06-cv-stats}).

\subsection{Spatial Prediction}

Both \emph{baseline} and \emph{best performing} linear mixed models captured the same overall pattern of 
spatial variation of the soil properties (\autoref{fig:chap06-kriging}). The main difference is that the 
spatial patterns of the different covariates used to calibrate each model produced different features in the 
prediction maps. For example, the CLAY map produced by the best performing model 
(\autoref{fig:chap06-clay-best-pred}) displays abrupt changes in the predicted values in the north-north-east 
due to the use of the more detailed soil map. Strongly-marked features following the stream network obtained 
through the use of the more detailed DEM are also observed (\autoref{fig:chap06-clay-best-pred} and 
\autoref{fig:chap06-clay-best-var}).

SOC maps (\autoref{fig:chap06-soc-best-pred} and \autoref{fig:chap06-soc-best-var}) show peculiar features in 
the central part of the study area, where predictions reached values as high as \SI{507}{\gram\per\kilo\gram}, 
while the maximum value in the calibration data is \SI{163}{\gram\per\kilo\gram}. The extremely high predicted 
values resulted from the inclusion of the topographic position index derived from the more detailed DEM, using 
a window size of $15 \times 15$~pixels (\texttt{TPI\_10\_15}) to model the deterministic trend. 
\texttt{TPI\_10\_15} values in the point calibration data range from \num{-7} to \SI{6}{\m}, while in the 
central part of the study area they range from \num{12} to \SI{31}{\m}. Thus, feature-space extrapolation 
explains the extremely high predicted values for SOC. Abrupt changes in predicted SOC are also observed at 
locations with low to moderate SOC (\SIrange{40}{80}{\gram\per\kilo\gram}). This is caused by using the more 
detailed land use map.

Predicted ECEC (\autoref{fig:chap06-ecec-base-pred} and \autoref{fig:chap06-ecec-best-pred}) had a large 
dependency on land use and geologic maps. Several features observed in the prediction maps derive from these 
two covariates. The influence of land use is seen in the northern part, while in the western, central, and 
eastern parts the influence of both covariates create an irregular pattern in the spatial distribution of 
ECEC. It is also in these parts that the largest prediction error standard deviations occur, following the 
spatial pattern of the covariates.

\begin{figure}[!ht]
 \centering
 \begin{minipage}[b]{63mm}
  \subcaption{}
  \label{fig:chap06-clay-base-pred}
  \centering
  \includegraphics[width=63mm]{fig/chap06-FIG7a}
 \end{minipage}
 \begin{minipage}[b]{63mm}
  \subcaption{}
  \label{fig:chap06-clay-best-pred}
  \centering
  \includegraphics[width=63mm]{fig/chap06-FIG7d}
 \end{minipage}
 \begin{minipage}[b]{63mm}
  \subcaption{}
  \label{fig:chap06-soc-base-pred}
  \centering
  \includegraphics[width=63mm]{fig/chap06-FIG7b}
 \end{minipage}
 \begin{minipage}[b]{63mm}
  \subcaption{}
  \label{fig:chap06-soc-best-pred}
  \centering
  \includegraphics[width=63mm]{fig/chap06-FIG7e}
 \end{minipage}
 \begin{minipage}[b]{63mm}
  \subcaption{}
  \label{fig:chap06-ecec-base-pred}
  \centering
  \includegraphics[width=63mm]{fig/chap06-FIG7c}
 \end{minipage}
 \begin{minipage}[b]{63mm}
  \subcaption{}
  \label{fig:chap06-ecec-best-pred}
  \centering
  \includegraphics[width=63mm]{fig/chap06-FIG7f}
 \end{minipage}
 \caption[Predicted values for CLAY, SOC and ECEC.]{Predicted values for CLAY (\si{\gram\per\kilo\gram}) (a, 
b), SOC (\si{\gram\per\kilo\gram}) (c, d), and ECEC (\si{\milli\mole\per\kilo\gram}) (e, f) using the 
\emph{baseline} (left) and \emph{best performing} (right) linear mixed models.}
 \label{fig:chap06-kriging}
\end{figure}

\begin{figure}[!ht]
 \centering
 \begin{minipage}[b]{63mm}
  \subcaption{}
  \label{fig:chap06-clay-base-var}
  \centering
  \includegraphics[width=60mm]{fig/chap06-FIG8a}
 \end{minipage}
 \begin{minipage}[b]{63mm}
  \subcaption{}
  \label{fig:chap06-clay-best-var}
  \centering
  \includegraphics[width=60mm]{fig/chap06-FIG8d}
 \end{minipage}
 \begin{minipage}[b]{63mm}
  \subcaption{}
  \label{fig:chap06-soc-base-var}
  \centering
  \includegraphics[width=60mm]{fig/chap06-FIG8b}
 \end{minipage}
 \begin{minipage}[b]{63mm}
  \subcaption{}
  \label{fig:chap06-soc-best-var}
  \centering
  \includegraphics[width=60mm]{fig/chap06-FIG8e}
 \end{minipage}
 \begin{minipage}[b]{63mm}
  \subcaption{}
  \label{fig:chap06-ecec-base-var}
  \centering
  \includegraphics[width=60mm]{fig/chap06-FIG8c}
 \end{minipage}
 \begin{minipage}[b]{63mm}
  \subcaption{}
  \label{fig:chap06-ecec-best-var}
  \centering
  \includegraphics[width=60mm]{fig/chap06-FIG8f}
 \end{minipage}
 \caption[Prediction error standard deviations for CLAY , SOC and ECEC.]{Prediction error standard deviations 
for CLAY (\si{\gram\per\kilo\gram}) (a, b), SOC (\si{\gram\per\kilo\gram}) (c, d), and ECEC 
(\si{\milli\mole\per\kilo\gram}) (e, f) using the \emph{baseline} (left) and \emph{best performing} (right) 
linear mixed models.}
 \label{fig:chap06-kriging-variance}
\end{figure}

The smallest prediction error standard deviations occur at lower elevations, along the three main streams, and 
close to the water outlet in the southern part of the study area. These areas have the highest density of 
point soil observations used to calibrate the models, and the smallest values for all three soil properties. 
While the first determines the accuracy of the EBLUP, the second influences the final accuracy through the 
back-transformation of predicted values.

\section{DISCUSSION}

Our main goal was to evaluate whether investing in more spatially detailed covariates improves the accuracy of 
soil maps. We saw that calibrating the models with more detailed covariates generally has a small to moderate, 
but positive, impact on the predictions. The magnitude of this benefit depends on the magnitude of the 
increase of the spatial detail of the covariate, on the other covariates included in the model, and on the 
soil 
property. However, there seems to be a limit above which the increase of spatial detail has a negative impact 
on the predictions. In the next two subsections we interpret the results from a pedological perspective and 
assess whether the investment in more detailed covariates is worthwhile or if alternatives to improve 
prediction accuracy should be favoured.

\subsection{Spatio-Temporal Controls of Soil Properties}

CLAY was moderately well predicted using less detailed covariates, with small improvement when using the more 
detailed covariates. CLAY was expected to have a strong correlation with topography and parent material. This 
correlation was already considerable when the less detailed DEM and geologic map were used, and improved only 
marginally with the more detailed version. One sensible explanation is that the effective (actual rather than 
theoretical) spatial detail of the two geologic maps was similar, although they had a four-fold difference in 
the size of the minimum legible delineation (see \citet{HenglEtAl2006a} for a discussion on effective 
scale). For the DEM, many studies have already suggested that its resolution may be of secondary importance 
when calculating DEM derivatives for soil mapping \cite{ZhuEtAl2008, BehrensEtAl2010a, MillerEtAl2015}. The 
influence of land use on CLAY is currently small due to reduction of soil erosion in the first decade of the 
\num{21}st century \cite{MiguelEtAl2012, TenCatenEtAl2012b}. A moderate within-field spatial variation may 
exist due to past erosional processes \cite{MouraBueno2012}, but we lack evidence of how well this source of 
variation was captured in the present-time point soil data.

It is worthwhile to consider the influence of the more detailed soil map on predicting CLAY. Due to its 
production process, the more detailed soil map derives a large amount of spatial detail from the geologic map, 
land use map and DEM -- note that the second-best performing model for CLAY included the more detailed 
geologic map instead of the more detailed soil map (\autoref{fig:chap06-model-series}). However, most of the 
additional spatial detail included in the more detailed soil map was probably based on the spatial variation 
of soil texture, because this is a strongly marked soil feature in the area \cite{MiguelEtAl2012}. Soil 
texture 
is one of the most important soil properties used by soil surveyors in the field to identify mapping units 
\cite{Legros2006}. These findings help explain why in the end the more detailed soil map was the most 
beneficial for CLAY instead of the geologic map.

SOC and ECEC were considerably better predicted when more detailed covariates were used. Our expectation that 
SOC and ECEC would have a strong correlation with land use was confirmed by the fact that this covariate 
explained a large amount of the variance and was highly beneficial for improving the predictions. Although the 
available point soil data are limited to the \num{2004}--\num{2011} period, we believe that land use changes 
in the last \num{30}~years \cite{MiguelEtAl2012, TenCatenEtAl2012b} strongly affected SOC and ECEC. Thus, the 
more detailed land use map is likely to have considerably improved model performance because it is up-to-date 
and, possibly, because it has \num{40}~times more spatial detail than its less detailed version. Despite the 
fact that the two land use maps used in this study were from different time periods, which confounds the 
analysis, the results obtained indicate that a more detailed land use map improves the prediction of SOC. For 
example, the areas used for crop agriculture, which are well known for having lower SOC and ECEC 
\cite{Menezes2008, MouraBueno2012}, are not depicted in the less detailed land use map.

We expected SOC to have a stronger correlation with the DEM than with the geologic map due to its strong 
dependence on erosion, but we observed the contrary. This result may be partially explained by the fact that 
there is a strong relation between geology and topography in the study area \cite{Sartori2009}. Due to its 
production process \cite{MacielFilho1990}, the geologic maps can be interpreted as an aggregated version of a 
DEM. A second sensible explanation is that the effect of erosion on SOC is not that large because erosion was 
considerably reduced in the last decade \cite{MiguelEtAl2012, TenCatenEtAl2012b}. A last possible explanation, 
which integrates the previous two, is the existence of a spatial relation between SOC and CLAY, the last being 
strongly correlated with parent material. These relations help explain why the more detailed DEM was almost as 
beneficial as the more detailed geologic map for SOC predictions. In the case of ECEC, our expectation of a 
strong dependency on a more detailed geologic map for producing more accurate predictions was confirmed.

The observed benefit of the more detailed geologic map and DEM for making more accurate CLAY, SOC, and ECEC 
predictions suggests that these soil properties are spatially related in the study area. We also hypothesize 
that the complexity of current land use makes it difficult to achieve SOC and ECEC models with performances 
comparable to CLAY. One important source of variation in forested areas is its use for animal grazing 
\cite{SamuelRosaEtAl2011a}. This influences nutrient cycling and soil nutrient availability 
\cite{SchramaEtAl2013}. Current remote sensing technology is unable to capture the data needed to proxy the 
environmental conditions created by these processes.

\subsection{Using More Detailed Covariates}

More detailed covariates are usually expected to improve predictions in soil mapping \cite{CavazziEtAl2013, 
MaynardEtAl2014}. However, deciding whether to invest or not in more detailed covariates requires careful 
thinking and depends on case-specific elements. We generally saw improvement in the predictions in our study, 
but the improvement was not large and may not outweigh the costs. Also, the models calibrated with the more 
detailed versions of all covariates were not the best performing models. Using more detailed satellite images 
and land use maps degraded CLAY predictions. Although the more detailed soil map had the largest benefit for 
CLAY, it may be too costly and impractical since its production usually requires having available more 
detailed 
versions of all other covariates. For SOC and ECEC, simply using a more detailed land use map resulted in 
considerably more accurate predictions. However, the superior performance may not outweigh the extra costs 
because producing a more detailed land use map usually requires up-to-date field observations and satellite 
images. Thus, the decision to adopt a more detailed covariate for soil mapping will ultimately depend on a 
trade-off between the increased accuracy and the extra budget required. It may also depend on other potential 
applications of the covariates, but this is not our concern here.

One interesting observation is that if a less detailed covariate yields poor predictions, its more detailed 
version has the potential to produce larger improvement in model performance. However, this is only a 
potential, not a guarantee. For instance, \citet{EldeiryEtAl2008} were not able to increase the $R^2 = 
0.31$ of linear regression models of soil salinity by more than \num{0.07}~points using \num{7.5}~times more 
detailed satellite images. On the other hand, model performance is likely to be hardly improved using more 
detailed covariates if their less detailed version has already produced accurate predictions. This agrees with 
findings by \citet{ThompsonEtAl2001} and \citet{KimEtAl2014}.

We also observed that the predictions can be degraded when using the more detailed version of covariates. In 
our study, this happened with the satellite image (all three soil properties), land use map (CLAY) and soil 
map (SOC and ECEC). A (small) benefit was observed only when these covariates were used along with the more 
detailed version of other covariates. As pointed out above, such a small benefit may not outweigh the increase 
in mapping costs. The trade-off between reducing model performance and being beneficial seems to depend on how 
much more spatial detail a covariate will have and on its correlation with the soil property. For example, the 
land use map was strongly correlated with SOC and ECEC, but not with CLAY, and its more detailed version had 
\num{40}~times more spatial detail. It helped improve SOC and ECEC predictions, but degraded CLAY predictions, 
resulting in only a small improvement when used along with the more detailed satellite image and geologic map.

If the influence of a more detailed covariate depends on the increase of spatial detail, then the priority 
should be to improve the spatial detail of the most beneficial covariate. This requires solid subject area 
knowledge because empirical evidence from the \emph{baseline} model may be insufficient. The most beneficial 
covariate is not necessarily that which explained the largest part of the variance in the \emph{baseline} 
model (see \autoref{tab:chap06-drop}). This occurs because increasing the spatial detail reduces the 
correlation between the covariate and the soil property. And also because there is little room to improve a 
correlation that is already high in the \emph{baseline} model. \citet{CavazziEtAl2013} suggest that the 
more detailed covariate has an excess of detail, a \q{noise} that degrades the predictions. This could explain 
the results for \texttt{sat}: higher resolution images can resolve smaller objects (e.g. individual plants) 
whose spectral behaviours are highly variable, adding noise to the \texttt{sat}-soil property correlation; on 
the other hand, lower resolution images capture collections of objects, and thus their variation is smoothed 
out in the pixel, reducing noise.

According to information theory one should optimize (maximize) the correlation between the point soil data and 
the covariates. This was described elsewhere as matching the \q{phenomenon scale} (the spatial pattern of the 
soil property) with the \q{analysis scale} (the spatial pattern of the covariates) \cite{DunganEtAl2002, 
MillerEtAl2014}. Finding the \q{optimum} requires evaluating the strength of the correlation using covariates 
with different levels of spatial detail \cite{DragutEtAl2009, CavazziEtAl2013, MillerEtAl2015}. Our results 
show that this approach may be too costly and impractical. Since modern soil mapping techniques explore only 
the empirical relation among environmental conditions and soil properties \cite{Grunwald2009}, the \q{optimum}
is a \q{conditional optimum} -- conditional on the point soil data available. It does not necessarily mean 
that the most accurate predictions will be made, but only that there is a level of spatial detail at which the 
correlation between the covariate and the point soil data is at a maximum. We suggest that instead more 
comprehensive approaches should be used to explore the full potential of the available covariates (see 
\citet{BehrensEtAl2010a} and \citet{MillerEtAl2015} for examples).

Finally, one must still judge whether the potential improvement in predictions is sufficient given the extra 
costs involved with using more detailed covariates. If the extra budget is spent on deriving more detailed 
covariates, we suggest that it may be better to substantially improve the detail of a less influential
covariate than marginally increase the detail of the most influential covariate. However, other means to spend 
the extra budget should be considered. For instance, it may be more efficient to concentrate on obtaining more 
soil observations. These may focus on better capturing the short range spatial variation \cite{BrusEtAl2007a} 
or improving the representation of the feature space to avoid undesirable extrapolations 
\cite{MinasnyEtAl2006b}.

\section{CONCLUSIONS}

This study has shown that:

\begin{enumerate}[label = (\Roman*)]
 \item Using more detailed covariates results in only a modest increase in the prediction accuracy of linear 
 prediction models;
 
 \item A more detailed covariate has a greater potential to improve prediction accuracy when the soil property
 is poorly predicted by its less detailed version;
 
 \item The impact on prediction accuracy when using the more detailed version of a less important covariate 
 may depend on which other covariates are included in the model;
 
 \item Choosing whether or not to invest in more detailed covariates depends on the strength of the 
 relationship between the covariates and the soil property being modelled, and on the relative difference 
 between the less detailed and the more detailed versions of the covariates.
\end{enumerate}
