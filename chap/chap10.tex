\artigofalse
\chapter{GENERAL CONCLUSIONS}
\shorttitle{General Conclusions}
\label{chap:chap10}

This thesis has made a pedological contribution with the development of a comprehensive description of the 
soil-forming factors and processes that determine the spatio-temporal distribution of soil properties in the 
Santa Maria case study area. The conceptual model of pedogenesis, presented in \autoref{chap:chap03}, showed 
that the spatial distribution of soil properties is highly variable, even when under the same land use. At 
coarse spatial scales, this spatial variation is determined by the geological and geomorphological diversity of 
the area, while at fine spatial scales, past and current (poor) agricultural practices seem to play a major 
role. Along with the conceptual model of pedogenesis, \autoref{chap:chap04} and \autoref{chap:chap05} 
constitute a technical contribution of this thesis. These chapters provide the basis for soil spatial modelling 
exercises in the study area.

\autoref{chap:chap06} demonstrated that existing, freely available covariates are suitable for calibrating soil 
spatial models. It was shown that using more detailed covariates results in only a modest increase in the 
prediction accuracy of linear soil spatial models. The observed increase is comparable to the effect of 
incorporating spatial dependence in the soil spatial model, and may not outweigh the extra costs of using more 
detailed covariates. In general, a more detailed covariate has a greater potential to improve 
prediction accuracy when a soil property is poorly predicted by its less detailed version. However, the 
magnitude of the improvement may depend on which other covariates are included in the model. Choosing whether 
or not to invest in more detailed covariates depends on the strength of the relationship between the covariates 
and the soil property being modelled, and on the relative difference between the less, and more detailed 
versions of the covariates. It is likely better to substantially improve the detail of a less influential 
covariate than marginally increase the detail of the most influential covariate. However, one should always 
consider if more efficient means of increasing prediction accuracy exist (e.g. obtaining more soil 
observations).

\autoref{chap:chap07} showed that several factors influence how field soil spatial modellers decide upon where 
to place soil observation locations. These are of three types: conceptual, operational, and psychological. The 
first concerns the knowledge of the soil spatial modellers about soil-landscape relationships, and seems to be 
connected with the years of field experience. The second relates to the available resources (infrastructure, 
workforce, and budget) to make soil observations, as well as to access restrictions imposed by landowners and 
geographic barriers, for example. The third relates to how the soil modellers perceive their surrounding 
physical environment and how the course of their motivation shifts during the soil observation process. Point 
pattern analysis helped understanding that there is a trade-off between conceptual and operational factors, 
which determines how the motivation of field soil modellers shifts focus towards one or another immediate goal. 
Depending on the focal goal, the resulting sample configuration resembles a random (learning/verifying 
soil-landscape relationships -- means-focused motivation) or a regular (maximizing the number of observations 
and geographic coverage -- outcome-focused motivation) point pattern.

\autoref{chap:chap08} showed that the conditioned Latin hypercube sampling algorithm, a popular algorithm used 
to optimize spatial sample configurations for spatial trend estimation, can be considerably improved. Compared 
to the original CLHS, our proposed modifications resulted in a sampling algorithm with an improved numerical 
behaviour, but this does not necessarily translates into improved prediction accuracy. For instance, sample 
size has a larger influence on prediction accuracy than the sampling algorithm. However, aiming only at the 
association/correlation between covariates degrades prediction accuracy possibly because the coverage of the 
geographic space is poorer. As such, when optimizing a sample configuration for spatial trend estimation, it 
should suffice to aim only at reproducing the marginal distribution of the covariates. This should be done 
using only the non-empty marginal sampling strata.

\autoref{chap:chap09} showed how to optimize sample configurations for spatial trend and variogram estimation, 
and spatial interpolation in situations where we know very little about the soil spatial distribution. The only 
requirement is that one formulates a sound multi-objective optimization problem using robust versions of 
existing sampling algorithms. The resulting spatial sample should reproduce the marginal distribution of the 
covariates such that the spatial trend can be accurately estimated. It should also contain several small 
clusters scattered throughout the spatial domain to enable making an accurate estimate of the behaviour of the 
variogram, specially near the origin. Finally, it should cover the sampling region in the most uniform way such 
that the average prediction error variance is the least possible.

This thesis has also contributed with two packages for the software environment for statistical computing and 
graphics \texttt{R}. The first package, called \texttt{pedometrics} (\autoref{appen:pedometrics}), contains 
various functions for spatial exploratory data analysis and model calibration designed for the development of 
this thesis. The second package, called \texttt{spsann} (\autoref{appen:spsann}), contains functions to 
optimize sample configurations to identify and estimate the variogram and spatial trend, and make spatial 
predictions. The latter was developed as part of \autoref{chap:chap08} and \autoref{chap:chap09}. Both are 
freely available and can be obtained from The Comprehensive R Archive Network (\cran).

Overall, this thesis showed that the complex interplay between soil and covariate data can have a large 
influence on the accuracy of soil maps. A single, universal, cost-effective recipe for reducing uncertainty 
in soil spatial modelling seems out of range. The case studies suggested that solutions are case specific and 
primarily depend on the existing soil and covariate data. Obtaining more soil samples showed to be an efficient 
strategy provided the available resources allow extra sampling. Otherwise, deciding upon cost-effective ways of 
reducing uncertainty requires, first, that we explore the full potential of existing soil and covariate 
data using robust spatial modelling techniques. Such an exercise requires a comprehensive knowledge of the 
soil-landscape relationships, as well as a thorough documentation of the soil and covariate data so that their 
weaknesses and strengths can be easily identified. Then, the decision of whether to invest on improving the 
quality of soil or covariate or both data sources will depend upon the trade-off between the increased 
data/prediction quality and the amount of resources required.
