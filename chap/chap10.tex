\artigofalse
\chapter{GENERAL CONCLUSION}
\label{chap:conclusion}

\autoref{chap:chap05} has demonstrated that the existing, freely available covariates are suitable for 
calibrating soil spatial models. It was shown that using more detailed covariates results in only a modest 
increase in the prediction accuracy of linear soil spatial models. The observed increase is comparable to the 
effect of incorporating spatial dependence in the soil spatial model, and may not outweigh the extra costs of 
using more detailed covariates. In general, a more detailed covariate has a greater potential to improve 
prediction accuracy when a soil property is poorly predicted by its less detailed version. However, the 
magnitude of the improvement may depend on which other covariates are included in the model. Choosing whether 
or not to invest in more detailed covariates depends on the strength of the relationship between the 
covariates and the soil property being modelled, and on the relative difference between the less detailed and 
the more detailed versions of the covariates. It is likely better to substantially improve the detail of a 
less influential covariate than marginally increase the detail of the most influential covariate. However, one 
should always consider if more efficient means of increasing prediction accuracy exist (e.g. obtaining more 
soil observations).

\autoref{chap:chap06} showed that several factors influence how field soil spatial modellers decide upon where 
to place soil observation locations. These are of three types: conceptual, operational, and psychological. The 
first concerns the knowledge of the soil spatial modellers about soil-landscape relationships, and seems to be 
connected with the years of field experience. The second relates with the available resources (infrastructure, 
work force, and budget) to make soil observations, as well with access restrictions imposed, for example, by 
land owners and geographic barriers. The third relates with how the soil modellers perceive their surrounding 
physical environment and how the course of their motivation shifts during the soil observation process. Point 
pattern analysis helped understanding that there is a trade-off between conceptual and operational factors, 
which determines how the motivation of field soil modellers shifts towards one or another focal goal. 
Depending on the focal goal, the resulting point pattern resembles a random (testing hypotheses of 
soil-landscape relationships -- means-focused motivation) or a regular (maximizing the number of observations 
-- outcome-focused motivation) spatial sample.

   

