\setcounter{page}{1}
\artigofalse
\chapter{General Introduction}
\label{chap:introduction}


\section{Objectives and research questions}
\label{sec:thesis-objectives}

\begin{enumerate}
 \item Determine the suitability of freely available covariates to calibrate soil spatial models.

  \begin{enumerate}[label=(\alph*)]
   \item Does the use of more detailed covariates result in considerably more accurate soil maps?
   
   \item How does incorporation of spatial dependence in a soil spatial model compare to the gain in 
   prediction accuracy obtained with using more detailed covariates?
   
   \item Are the answers to these research questions consistent across soil properties?
  \end{enumerate}
\end{enumerate}




\section{Outline}
\label{sec:thesis-outline}

The thesis is composed of eight chapters where each of the above mentioned objectives are met 
(\autoref{sec:thesis-objectives}). Although there is a logical sequence in their presentation, all chapters 
were planned so that they could be read separately. This means that there is some overlap between them, i.e. 
repeated information. References to specific sections of other chapters using coloured (blue) hyperlinks are 
common.

\emph{Chapter I}

\autoref{chap:chap02} presents the conceptual model of pedogenesis (in Portuguese), which consists of a 
description of the study area that includes an explicit description of soil-forming factors (climate, geology, 
geomorphology, hydrology, land use, and vegetation) and processes that determine the soil spatio-temporal 
distribution. \autoref{chap:chap03} describes the soil data included in the \emph{Santa Maria dataset}, which 
was used to develop the case studies presented in this thesis. The Santa Maria dataset is composed of 
$n = 410$ soil observations compiled from studies carried out between \num{2004} and \num{2013}. These studies 
aimed at producing semi-detailed soil and land use maps, and modelling topsoil carbon stock and vulnerability 
to erosion. A comprehensive description of the covariate data included in the Santa Maria dataset, and their 
processing, is given in \autoref{chap:chap04}. The goal of these three chapters is to provide the bases for 
future soil spatial modelling exercises in the study area and to serve as examples for new soil spatial 
modelling studies developed elsewhere.

Based on an article published in the peer reviewed journal \geoderma, \autoref{chap:chap05} serves the purpose 
of meeting the first objective of the thesis and answering its respective research questions. There, the 
prediction performance of linear soil spatial models calibrated using covariates (area-class soil maps, land 
use maps, geological maps, digital elevation models, and orbital images) available in two levels of detail is 
evaluated. The influence of taking the spatial dependence of the residuals into account is assessed as well. 

\emph{Chapter VI} presents the results of the experiment devised to evaluate if improving a popular sampling 
algorithm results in more accurate spatial predictions. The comparison of five sampling algorithms on how they 
affect estimated model parameters and prediction accuracy is presented in \emph{Chapter VII}. The chapter also
introduces a new general purpose sampling algorithm. Both chapters contain only partial results, and will be 
the basis of manuscripts to be submitted to peer reviewed journals, both dealing with the second objective of 
this thesis and its respective research questions.

\autoref{chap:chap08}

The sequence of eight chapters is closed with a \emph{General Conclusion} where I highlight the main results 
of the research, contributions and merits of the study. Next, there are two appendices, both devoted to the 
description of the two packages for \texttt{R} developed to support the case studies: \texttt{pedometrics} 
(\autoref{apen:pedometrics}) and \texttt{spsann} (\autoref{apen:spsann}). The first includes miscellaneous 
functions that were put together for ease of use. The second was designed for the optimization of sample 
configurations using spatial simulated annealing. All literature references are presented under a unique 
\emph{Bibliographic References} chapter (\autoref{chap:references}) at the end of the thesis.
