\artigotrue
\chapter{The Santa Maria dataset -- soil data}
\label{chap:chap03}

\tocless\section{Introduction}

The Santa Maria dataset is a data set comprising soil data from $n = 410$ point soil observations 
made between 2004 and 2013 in the catchment of the DNOS/CORSAN reservoir, located in the southern 
border of the plateau of the Paraná Sedimentary Basin, in the city of Santa Maria, state of Rio 
Grande do Sul, Brazil. These point soil observations cover the northern sector of the catchment -- 
an area of \SI{\pm2000}{\hectare}, which corresponds to \SI{\pm60}{\percent} of the entire 
catchment. These soil data was produced during the development of research projects that aimed at 
producing semi-detailed soil and land use maps (\scale{25000}) \cite{Pedron2005, Miguel2010, 
SamuelRosaEtAl2011a, MiguelEtAl2012}, and predicting topsoil carbon stock and vulnerability to erosion 
\cite{Samuel-Rosa2009, MouraBueno2012, Miguel2013}.

This document (v.1) presents a description of the soil data contained in the Santa Maria dataset, 
including the procedures for soil sampling and description, as well as the analytical methods 
employed. The soil data is freely available as CSV files in a repository hosted in
\href{https://github.com/samuel-rosa/dnos-sm-rs-general/tree/master/data}{GitHub}. Files 
fieldData.csv and labData.csv contain the identification of each observation locations, their 
geographic coordinates (latlong, WGS1984), and field and laboratory data, respectively. Files 
fieldMetadata.csv and labMetadata.csv contain the metadata. Every soil property is identified with a 
code composed of three or four capital letters. For example, soil organic carbon is identified with 
\texttt{ORCA}. A column contains the number of laboratory replicates is identified with the code of 
the soil property followed by the letter \q{N}. The column containing the sample standard deviation 
is identified in the same manner, but using \q{SD}. For example, \texttt{ORCA\_N} and \texttt{ORCA\_SD}.

The reader should be aware that soil science evolved in Brazil following a somewhat different 
pathway than in the countries of the northern hemisphere due to the specific features of 
\emph{tropical soils}. Methods have been adapted along the years, possibly leading to nomenclature 
mismatches. The reader is invited to contribute to solve any problems in this document.

\tocless\section{Field sampling}

The Santa Maria dataset is composed of three subsets which are described in the next three 
sections. Together, these subsets yield a sampling density of \num{\pm0.18}~observations per 
hectare, with an average separation distance between two neighbouring points of \SI{180}{\metre}, 
minimum and maximum separation distances of \num{18} and \SI{328}{\metre}, \SI{95}{\percent} of 
neighbouring observations being separated by more than \SI{49}{\metre}.

\tocless\subsection{Subset I}

The first subset ($n = 340$, \autoref{fig:soil-data-subsets-I-III}) was produced by 
\citeonline{Samuel-Rosa2009}, \citeonline{SamuelRosaEtAl2011a}, \citeonline{MiguelEtAl2012}, 
\citeonline{Moura-BuenoEtAl2012}, and \citeonline{Samuel-RosaEtAl2013} as part of projects that aimed at 
producing semi-detailed soil and land use maps, and predicting topsoil carbon stock and vulnerability to 
erosion. The researchers faced several difficulties with a budget cut and shortage of workforce. They also had 
restricted access to several areas due to geographic barriers and prohibition of 
access by some landowners. These difficulties forced the researchers to reduce the originally aimed number of 
observations ($n = 500$) during the development of the project.

All observation locations were selected purposively or by convenience. Tacit knowledge was the main tool
to choose the observation locations, a process that was carried out in the office using Google 
Earth\textregistered{} imagery of the years of 2008 and 2009. The main goal of the researchers was 
to obtain a sample that they understood as being representative of the different landforms, land uses, and 
soil taxa present in the study area. They also wanted the observations to be spread throughout the entire 
study area.

At the observation locations, the researchers defined an area of \SI{\pm100}{\metre\squared} within which
they opened three soil pits. Soil samples were collected up to a depth of \SI{20}{\centi\metre}, the depth 
being measured with a ruler. The resulting sampling depth of Subset I varies from \num{2} to 
\SI{20}{\centi\metre}, with an average of \SI{17}{\centi\metre}. This variation of the vertical sampling 
support was not a problem for the researchers because their goal was to sample the \emph{topsoil}. The
\emph{topsoil} was defined as the topmost soil layer, with a depth equal or inferior to 
\SI{20}{\centi\metre}, being the soil layer most susceptible to degradation induced by poor agricultural 
practices and land use changes.

Soil samples from the three pits opened in each sampling area were used to produce a composite sample
which was used for laboratory analyses (see below). Subsurface soil features were observed with an auger 
in each pit, and the average (continuous variables) or most common (categorical variables) value recorded.
Note that soil sampling was done using an areal support -- an area of \SI{\pm100}{\metre\squared}. However, 
the shape and exact area of the sampling units are unknown, and georeferencing took place at point 
support. Thus, the use of this subset requires to make the assumption that it was obtained using a point
sampling support. The possible negative consequences of this assumption have not been explored so far.

\begin{figure}[!ht]
\centering
<<fig:soil-data-subsets-I-III, fig = TRUE, echo = FALSE>>=
cal_data <- read.table(
 "~/projects/dnos-sm-rs/dnos-sm-rs-general/data/labData.csv", sep = ";",
 header = TRUE, na.strings = "na")[1:350, c("longitude", "latitude")]
sp::coordinates(cal_data) <- ~ longitude + latitude
sp::proj4string(cal_data) <- sp::CRS("+init=epsg:4674") # sirgas2000
cal_data <- sp::spTransform(cal_data, sp::CRS("+init=epsg:32722")) # wgs1984utm22s
pts <- data.frame(slot(cal_data, "coords"))
pts[, "id"] <- c(rep("black", 340), rep("red", 10))
invisible(capture.output(
 spgrass6::initGRASS(
   gisBase = "/usr/lib/grass64/", gisDbase = path.expand("~/dbGRASS"),
   location = "dnos-sm-rs", mapset = "predictions", pid = Sys.getpid(),
   override = TRUE)
))
invisible(capture.output(pol <- spgrass6::readVECT6("buffer_BASIN_10", ignore.stderr = TRUE)))
invisible(capture.output(drain <- spgrass6::readVECT6("STREAM_10", ignore.stderr = TRUE)))
invisible(capture.output(drain <- rgeos::gIntersection(drain, pol, byid = TRUE)))
p <- sp::spplot(
 pol, zcol = "cat", col = "gray", fill = "lemonchiffon",
 scales = list(draw = TRUE),
 colorkey = FALSE, aspect = "iso",
 xlim = c(cal_data@bbox[1, 1] * 0.9998, cal_data@bbox[1, 2] * 1.0005),
 ylim = c(cal_data@bbox[2, 1] * 0.99999, cal_data@bbox[2, 2] * 1.00001),
 par.settings = list(
   fontsize = list(text = 12, points = 8),
   layout.widths = list(left.padding = 0, right.padding = 0),
   layout.heights = list(top.padding = 0, bottom.padding = 0)),
 panel = function(x, y, ...) {
   sp::panel.polygonsplot(x, y, ...)
   lattice::panel.points(
     x = pts[, "longitude"], y = pts[, "latitude"], pch = 20, col = pts[, "id"])
   lattice::panel.abline(
     v = seq(227000, 232000, 1000), h = seq(6712000, 6722000, 1000),
     col = "gray", lty = "dashed", lwd = 0.5)
   }) +
 latticeExtra::layer(sp::sp.lines(drain, col = "dodgerblue", lty = "dashed"))
print(p)
@
\caption{Spatial distribution of the point soil observations contained in \emph{Subset I} ($n = 340$, black) 
and \emph{Subset III} ($n = 10$, red) of the Santa Maria dataset. The drainage network is shown in the 
background to give an idea of how the locations of soil observations is related to terrain features.}
\label{fig:soil-data-subsets-I-III}
\end{figure}

Georeferencing was done in the field using a Global Navigation Satellite System (GNSS) receiver with a 
horizontal positional error of less than \SI{\pm8}{\metre} positioned approximately at the centre of the
sampling area. Sometimes, the horizontal positional error was larger than \SI{\pm8}{\metre} due the effects
of vegetation biomass, terrain features, and satellite configuration. In this cases, observation locations 
were georeferenced in the office using \SI{1}{\metre} spatial resolution Google Earth\textregistered{} 
imagery of the years of 2008 and 2009. The positional horizontal error of these images imagery is 
\SI{\pm6}{\metre} (see below).

Every observation was identified with a number in increasing order, following the order in which 
the observations were made (\num{001}--\num{340}). A total of \num{17}~field campaigns were carried out,
yielding an observation density of about \num{18}~observations per \si{\kilo\metre\squared}.

\tocless\subsection{Subset II}

The second subset ($n = 60$) was produced in the years of \num{2012} and \num{2013}, and was intended to 
constitute an independent dataset for validation purposes. Because of the many many access limitations 
(geographic barriers and prohibition by landowners) and shortage of workforce, budget, infrastructure and 
time faced in previous field campaigns, researchers chose to employ transect (cluster) sampling 
\cite{MiguelEtAl2012,Moura-BuenoEtAl2012,Samuel-RosaEtAl2013}. They started defining the population of 
transects using their knowledge of the study area, taking into account the factors that they thought 
determined the spatial distribution of soil properties. Each researcher (three) delineated $m = 60$ easily 
accessible, straight transects of \SI{400}{\metre} following the spatial gradients of environmental features 
(topography, geology, vegetation, land use, and soils). Accordingly, knowledge of existing roads, human 
settlements, water bodies, and other access limitations was used as well. The activity was carried out using 
Google Earth\textregistered{} imagery of the years of \num{2008} and \num{2009}.

% TODO: add a figure with all transects highlighting the selected ones.
% \caption{Three experts drawn 180 linear transects aligned in the direction of maximum expected 
% spatial variance of environmental features. They avoided locations where it was known that geographic 
% barriers or landowners would impede the access to collect soil samples. Using probability sampling, 12 
% transects were selected to collect validation observations (red).}

Twelve out of the $m = 180$ transects were randomly selected using as many iterations as necessary until 
there were no intersecting transects, and there was at least one transect in each of the three major 
geomorphological units of the study area (Planalto, Rebordo do Planalto, and Depressão Periférica). 
Finally, $n = 5$ observation locations, separated by equidistant intervals of \SI{100}{\metre}, were 
selected in each transect. Observation locations were named with a number in increasing order, 
following the order in which the observations were made, starting from \num{341} (\num{341}--\num{400}).

Observation locations were identified in the field using a GNSS receiver with a horizontal positional error
of less than \SI{\pm8}{\metre}. A soil pit was opened within a radius of \SI{2}{\metre} from the predefined 
observation location. Soil sampling and description was carried out using the same procedure used with 
\emph{Subset I}. Georeferencing was performed using a Differential Global Positioning System (DGPS) with 
a horizontal positional error of less than \SI{1}{\centi\metre}.

\tocless\subsection{Subset III}

The third subset ($n = 10$) contains data compiled from the studies of \citeonline{Pedron2005} and
\citeonline{Miguel2010}, specifically from the uppermost A horizon of modal soil profiles whose locations 
were purposively selected after a preliminary area-class soil map had been produced and/or the observations
included in \emph{Subset I} had been made. The researchers aimed at observation locations that they 
understood as being most representative of the soil mapping units depicted in their respective area-class 
soil maps. A single soil sample was taken from each of the described soil horizon and used for laboratory 
analysis (see below). The resulting depth of the uppermost A horizons varies from \num{12} to 
\SI{30}{\centi\metre}, with a mean of \SI{22.6}{\centi\metre}. Georeferencing was carried using a GNSS 
receiver with a horizontal positional error of less than \SI{\pm8}{\metre} positioned at the observation 
location. Data are identified in the \emph{Santa Maria database} using the same identification that was 
used in the studies from where they were compiled.

\tocless\section{Field description}

Several environmental features were described at the observation locations. The next sections 
present a description of how this was done.

\tocless\subsection{Land use}

Land use (\texttt{LAND}) was assessed in the years of \num{2008} and \num{2009} using data 
collected in the field and Google Earth\textregistered imagery. Five land uses were identified: 
native forest (primary or secondary), shrubland (abandoned areas with predominance of shrub-sized 
vegetation, known in Brazil as \emph{capoeira}), animal husbandry (grasslands), forestry 
(\textit{Eucalyptus} sp. and \textit{Pinus} sp.), and crop agriculture. Other land uses such as 
human settlements and water bodies were not observed/sampled.

\tocless\subsection{Geology}

Soil parent material (\texttt{PARENT}, sedimentary or igneous), geological formation (\texttt{GEO}),
and lithology (\texttt{LITHO}) was inferred from direct observation of soil properties and local 
environmental features in the field. Existing geological maps were used only when soil 
characteristics and environmental features were insufficient to reach an agreement about the most 
likely soil parent material, geological formation, and lithology.

\tocless\subsection{Soil classification}

The most likely soil classification (\texttt{TAXON}) was inferred in the field using direct 
observation of soil properties (\SI{20}{\centi\metre}-deep soil pits and auger holes down to the 
diagnostic subsurface horizon or bedrock) and local environmental features. Soil taxon was inferred 
only till the second categorical level of the Brazilian System of Soil Classification 
\cite{SantosEtAl2013a} because further levels require data that cannot be easily identified in the 
field.

% TODO: move this description of the soil taxa to the metadata file.
% RF - Neossolo Flúvico, RL - Neossolo Litólico, RR - Neossolo Regolítico, RQ - Neossolo 
% Quartzarênico, PV - Argissolo Vermelho, PVA - Argissolo Vermelho-Amarelo, PA - Argissolo 
% Amarelo, PBAC - Argissolo Bruno-Acinzentado, SX - Planossolo Háplico, CX - Cambissolo Háplico.

\tocless\subsection{Slope}

The slope gradient (\texttt{SLOPE}, \si{\percent}) was measured using a clinometer, the observer
and target being at a constant height above the ground. The distance between observer and target was 
between \SI{30}{\metre} (dense forests) and \SI{50}{\metre} (open fields).

\tocless\subsection{Drainage}

Soil drainage status (\texttt{DRAIN}) was inferred visually from soil features observed with an 
auger using the classification scheme proposed by \citeonline{SantosEtAl2013}.

\tocless\subsection{Coarse fragments and rock outcrops}

Presence of coarse fragments (\texttt{FRAG}) -- soil material of diameter \SI{>2}{\centi\metre} -- was 
described 
as a binary variable, that is, a value of \num{1} (one) was annotated when coarse fragments were present, and 
\num{0} (zero) otherwise. The approach was adopted to describe the presence of rock outcrops (\texttt{ROCK}). 
The 
quantity of gravel (\texttt{GRAVEL}, \si{percent}) was estimated visually.

The employed to describe the presence of coarse fragments and rock outcrops is not in line with the soil 
description guidelines currently used by most Brazilian soil scientists. The reason for its use is the limited 
importance given to these soil properties in the respective research projects.

\tocless\subsection{Canopy}

% TODO: add three figures as examples of each class.
Soil coverage with vegetation (\texttt{CANOPY}) was inferred visually in the field using three classes: 
low (\SI{<25}{\percent}), medium (\SIrange{25}{75}{\percent}), and high (\SI{>75}{\percent}).

\tocless\subsection{Additional information}

Additional information was recorded at each observation location during the field campaigns. They refer to 
peculiarities of each observation location and were not recorded in a systematic way.

\tocless\section{Laboratory analysis}

Soil samples were air dried, crushed and passed through a \SI{2}{\milli\metre}-sieve prior to laboratory 
analyses using the methods described in the next sections. One or more laboratory replicates were used to 
calculate analytical errors.

\tocless\subsection{Soil organic fraction}
\label{apen:soil-data-orca}

The soil organic carbon content (\texttt{ORCA}, \si{\gram\per\kilo\gram}) was determined using the method of
\citeonline{Mebius1960} modified by \citeonline{YeomansEtAl1988} as described by
\citeonline{ClaessenEtAl1997}.

Sample aliquots of \num{0.050} to \SI{0.500}{\gram} were placed in glass digestion tubes 
(\SI{80}{\milli\liter}). The amount of sample used varied according to the ORCA estimated by visual 
interpretation of soil colour. Every digestion tube received an aliquot of \SI{10}{\milli\liter} of 
sulfochromic solution\footnote{See more about the sulfochromic solution, or chromic acid, at
\url{http://en.wikipedia.org/wiki/Chromic_acid}} [\SI{0.067}{\mole\per\liter} potassium bichromate solution
(\ce{K2Cr2O7}) in the presence of concentrated sulphuric acid (\ce{H2SO4})] and a small reflux funnel 
to avoid loss of reagent during digestion. A digestion block with capacity for \num{40}~samples was used:
\num{36}~tubes with soil sample plus \num{3}~tubes with blank plus \num{1}~tube with \ce{H2SO4} and a
thermometer for temperature control. Digestion at \SI{150}{\celsius} last \SI{30}{\minute}. Three blanks 
were prepared and set aside at room temperature to estimate the loss of reagent due to heat in the digestion 
block. After digestion the tubes were set aside at room temperature to cool down. Next, the solution was 
transferred to Erlenmeyer flasks (\SI{250}{\milli\liter}) with \SI{60}{\milli\liter} of distilled water and
\SI{2}{\milli\liter} of concentrated orthophosphoric acid [\ce{H3PO4}] and \num{3}~drops of 
\SI{1}{\percent}~diphenylamine). The solution was titrated using \SI{0.1}{\mole\per\liter} ammonium ferrous
sulphate (\ce{FeSO4(NH4)2 * 6H2O}) until persistent green colour. The results were multiplied by \num{1.11}
to correct to the standard method (dry combustion).

The soil organic matter content of the observations compiled from \citeonline{Pedron2005} was determined 
using the method described by \citeonline{TedescoEtAl1995}. Sample aliquots of \SI{2.5}{\milli\liter}
were placed in Erlenmeyer flasks (\SI{50}{\milli\liter}). Every Erlenmeyer flask received an aliquot of 
\SI{15}{\milli\liter} of \SI{0.067}{\mole\per\liter} sulfochromic solution (\ce{Na2Cr2O7 + H2SO4}). The 
flasks were heated in a water bath at \num{75} to \SI{80}{\celsius} during \SI{30}{\minute} and shaken 
for \SI{5}{\minute}. A water aliquot of \SI{15}{\milli\liter} was added to the flask and let overnight 
(\num{15} to \SI{18}{\hour}). In the next day, an aliquot of \SI{3.0}{\milli\liter} was sampled to a 
small cup with \SI{3.0}{\milli\liter} of distilled water. The absorbency of the supernatant was measured 
at \SI{645}{\nano\metre}. The results were transformed to organic carbon content using the Van Bemmelen 
factor (\num{1.724}), the result assumed to be equivalent to soil organic carbon content measured using the 
standard method. The results are expressed using a volume-basis and were converted to a mass-basis using a
1:1 relation because the mass of the sample aliquot used in the analyses is unknown.

\tocless\subsection{Particle size analysis}
\label{apen:soil-data-clay}

Particle size analysis was performed using the pipette method, with the sand fraction 
(\texttt{SAND}, \SIrange{0.053}{2}{\milli\metre}, \si{\gram\per\kilo\gram}) determined by wet sieving, and the 
silt 
fraction (\texttt{SILT}, \SIrange{0.002}{0.053}{\milli\metre}, \si{\gram\per\kilo\gram}) calculated by 
difference. 
The analytical procedure is an adaptation of the method of the Soil Conservation Service of the United States
Department of Agriculture \cite{UnitedStates1972} made by the Soil Physics Laboratory of the Universidate
Federal de Santa Maria \footnote{As far as I know, a comprehensive description of this method has not been 
published so far, neither in Portugues nor in English. You can visit the homepage of the Soil Physics 
Laboratory 
of the Universidate Federal de Santa Maria at \url{https://coral.ufsm.br/fisicadosolo/} to get more information
about the method or contact their developers.} \cite{SuzukiEtAl2004,SuzukiEtAl2004a}.

First, a sample aliquot of \SI{20}{\gram} was placed in a \SI{100}{\milli\liter} glass container 
(height: \SI{10.5}{\centi\metre}; diameter: \SI{2.75}{\centi\metre}; weight: \SI{85}{\gram}). Two nylon
spheres with a diameter of \SI{1.71}{\centi\metre} and weighting \SI{3.04}{\gram} (density: 
\SI{1.11}{\gram\per\centi\metre\cubic}) were added to act as physical disaggregating agents. Then, an 
aliquot of \SI{10}{\milli\liter} of \SI{1}{\mole\per\liter} sodium hydroxide (\ce{NaOH}) solution was 
added to act as chemical dispersing agent along with \SI{40}{\milli\liter} of distilled water. The glass 
container was closed with a plastic cap, manually shaken for \SI{10}{\second}, and placed in a horizontal
mechanical shaker with capacity for \num{85}~samples. The suspension was left to stand overnight 
(\SI{10}{\hour}). In the next day the suspension was submitted to horizontal mechanical agitation during 
\SI{4}{\hour} at \si{120} cycles per minute \cite{SuzukiEtAl2004,SuzukiEtAl2004a}.

After horizontal agitation, the suspension was poured in a plastic graduated cylinder with capacity for 
\SI{1000}{\milli\liter} using a glass funnel and a metal sieve to hold the two nylon spheres. The 
suspension in the graduated cylinder was completed to \SI{1000}{\milli\liter} and homogenized using a
hand stirrer (\SI{30}{\second}). The suspension was allowed to stand until sedimentation was complete. 
The time needed was calculated using the Stokes law with the temperature measured in a graduated cylinder 
filled with distilled water.

%TODO provide a more detailed description of how CLAY was determined as well as of the oxidative
%treatment with H2O2.

The clay fraction (\texttt{CLAY}, \SI{<0.002}{\milli\metre}, \si{\gram\per\kilo\gram}) was determined by the 
pipette method. Soil samples with organic matter content \SI{>5}{\percent} were submitted to oxidative 
treatment with hydrogen peroxide (\ce{H2O2) prior to the analysis following the recommendations of
\citeonline{ClaessenEtAl1997}.

%The sand fraction was separated into five size classes:
%
%\begin{itemize}
%\item \SIrange{1.00}{2.00}{\milli\metre}: very coarse sand;
%\item \SIrange{0.50}{1.00}{\milli\metre}: coarse sand;
%\item \SIrange{0.25}{0.50}{\milli\metre}: median sand;
%\item \SIrange{0.106}{0.25}{\milli\metre}:fine sand;
%\item \SIrange{0.053}{0.106}{\milli\metre}: very fine sand.
%\end{itemize}

% The clay fraction (\textless0.002~mm) was initially determined by the pipette method without any 
% pretreatment. A 1~mol~L$^{-1}$ NaOH solution was used as the dispersing agent, with the addition of two 
% nylon spheres as disaggregating agent plus horizontal mechanical agitation during 4~hours 
% \cite{SuzukiEtAl2004}.

% A propor{\c{c}}{\~{a}}o da fra{\c{c}}{\~{a}}o argila dispersa em {\'{a}}gua foi determinada conforme 
% descrito acima para a fra{\c{c}}{\~{a}}o argila total. A diferen{\c{c}}a {\'{e}} que n{\~{a}}o foi usado o 
% agente dispersante (NaOH) e o agente desagregante (esferas de nylon) \cite{ClaessenEtAl1997}.

\tocless\subsection{Soil density}
\label{apen:soil-data-bude}

% TODO: Provide a more detailed description of how this is done.
The bulk soil density (\texttt{BUDE}, \si{\mega\gram\per\cubic\metre}) was determined using the core method 
with a 
metallic ring (height: \SI{3}{\centi\metre}; diameter: \SI{5}{\centi\metre}) as described by 
\citeonline{ClaessenEtAl1997}. The bulk soil density was not determined in the locations where the soil was 
very 
shallow or stony.

\tocless\subsection{Exchangeable bases and acidity}
\label{apen:soil-data-ecec}

The exchangeable calcium (\texttt{CALC}, \si{\milli\mole\per\kilo\gram}) and magnesium (\texttt{MAGN}, 
\si{\milli\mole\per\kilo\gram}) were determined by atomic absorption spectroscopy after extraction with 
\SI{1.0}{\mole\per\liter} \ce{KCl} solution using the method described by \citeonline{ClaessenEtAl1997}. 
The exchangeable sodium (\texttt{SODI}, \si{\milli\mole\per\kilo\gram}) and potassium (\texttt{POTA}, 
\si{\milli\mole\per\kilo\gram}) were extracted with a \SI{0.05}{\mole\per\liter} \ce{HCl} solution plus 
\SI{0.025}{\mole\per\liter} \ce{H2SO} (Mehlich-\num{1} solution). Both were quantified by means of flame 
atomic emission spectrometry using the method described by \citeonline{TedescoEtAl1995}.

The exchangeable acidity (\texttt{EXAC}, \si{\milli\mole\per\kilo\gram}), also known in Brazil as exchangeable 
aluminium, was extracted using the same \SI{1.0}{\mole\per\liter} \ce{KCl} solution used to extract the 
exchangeable calcium and magnesium. It was determined by titrimetry with \SI{0.025}{\mole\per\liter} 
\ce{NaOH} solution as described by \citeonline{ClaessenEtAl1997}.

% TODO: Include POAC in the database and improve the description of how it was determined.
% The potential acidity (POAC, \si{\milli\mole\per\kilo\gram}) was determined with \SI{1.0}{\mole\per\liter} 
% calcium acetate solution at pH~\num{7.0} and titrated with \SI{0.0606}{\mole\per\liter} \ce{NaOH} solution 
% as described by \citeonline{ClaessenEtAl1997}.

The effective cation exchange capacity (ECEC, \si{\milli\mole\per\kilo\gram}) was defined as the sum of 
exchangeable bases and exchangeable acidity, i.e. 

\begin{equation*}
 \texttt{ECEC} = \texttt{CALC} + \texttt{MAGN} + \texttt{POTA} + \texttt{SODI} + \texttt{EXAC}.
\end{equation*}


% TODO: Provide a more detailed description of how these are calculated and include the data in the database.
% The sum of exchangeable bases (BASES) is given by the sum of the exchangeable calcium, magnesium, sodium and 
% potassium. The effective cation exchange capacity (ECEC) is the given by the exchangeable acidity plus the 
sum of 
% exchangeable bases. The potential cation exchange capacity (CEC) is given by the potential acidity plus the 
sum of 
% exchangeable bases. Note that the standard method for determining exchangeable bases relies on the use of 
barium 
% chloride [BaCl$_2$].
% The base saturation (BASA) is given by the sum of exchangeable bases divided by the potential cation 
exchange 
% capacity. The saturation of the ECEC with exchangeable acidity, or the aluminum saturation (ALSA), is given 
by the 
% sum of exchangeable bases divided by the effective cation exchange capacity. The results are multiplied by 
100. 

\tocless\section{Acknowledgements}

Collaborated in the preparation of this document (in order of importance): Pablo Miguel (UFPel), 
Jean Michel Moura Bueno (UFSM), Ricardo Simão Diniz Dalmolin (UFSM), Andrisa Balbinot (UFSM), Lúcia 
Helena Cunha dos Anjos (UFRRJ), Gustavo de Mattos Vasques (Embrapa Soils), Gerard B. M. Heuvelink 
(ISRIC -- World Soil Information), and Ad van Oostrum (ISRIC -- World Soil Information).

% \begin{figure}[!ht]
% \centering
% <<echo = FALSE>>=
% options(useFancyQuotes = FALSE)
% tmp <- read.table(
%  '~/projects/dnos-sm-rs/dnos-sm-rs-general/data/labData.csv', sep = ";",
%  header = TRUE, na.strings = 'na')
% lattice::trellis.par.set(
%  fontsize = list(text = 16, points = 15), axis.line = list(lwd = 0.01),
%  layout.widths = list(left.padding = 0, right.padding = 0),
%  layout.heights = list(top.padding = 0, bottom.padding = 0))
% aa <- pedometrics::plotHD(tmp$CLAY, xlab = 'CLAY')
% bb <- pedometrics::plotHD(tmp$ORCA, xlab = 'ORCA')
% cc <- pedometrics::plotHD(tmp$ECEC, xlab = 'ECEC')
% dd <- pedometrics::plotHD(na.exclude(tmp$BUDE), xlab = "BUDE")
% @
% \begin{minipage}[b]{63mm}
% \subcaption{}
% \centering
% <<intro-clay, fig = TRUE, echo = FALSE>>=
% print(aa)
% @
% \end{minipage}
% \begin{minipage}[b]{63mm}
% \subcaption{}
% \centering
% <<intro-orca, fig = TRUE, echo = FALSE>>=
% print(bb)
% @
% \end{minipage}
% \begin{minipage}[b]{63mm}
% \subcaption{}
% \centering
% <<intro-ecec, fig = TRUE, echo = FALSE>>=
% print(cc)
% @
% \end{minipage}
% \begin{minipage}[b]{63mm}
% \subcaption{}
% \centering
% <<intro-bude, fig = TRUE, echo = FALSE>>=
% print(dd)
% @
% \end{minipage}
% \caption{The four soil properties explored in this thesis: (a) clay content, (b) organic carbon
% content, (c) effective cation exchange capacity, and (d) bulk density. Each panel shows the sample
% histogram and summary statistics of the soil properties in their original scale ($\lambda = 1$), as
% well as the theoretical probability density function so that we can assess how good is the fit of
% the normal distribution to the data -- a product of the \Rpackage{pedometrics}.}
% \label{fig:intro-soil-properties}
% \end{figure}
