\ctable[
 caption  = [Dummy predictor variables derived from land use maps.]{Description of the $p = 7$ dummy 
predictor variables derived from the two land use maps.},
 label    = tab:chap05-land-covars,
 notespar,
 pos      = !ht,
 % doinside = \scriptsize\setstretch{1.1}
 doinside = \small
 ]{l p{0.85\textwidth} l}{
 \tnote[a]{Minimum Legible Delineation calculated following \citeonline{Rossiter2000}.}
 }{ \FL
 Code & Mapping unit(s) included and Description\tmark[a] \ML
 \multicolumn{2}{l}{Source: \citeonline{DSG1980, DSG1992, DSG1992a}. Cartographic scale: \num{1}:\num{25000}.
 Minimum Legible Delineation: \SI{2.5}{\hectare}.} \NN
 \texttt{LU1980a} & \textit{FS}. Native forest, which is likely to have soil with higher fertility. \NN
 \texttt{LU1980b} & \textit{H}. Animal husbandry, which is likely to have soil fertility status lower than 
 native forests and is the second most important land use in the area. \NN
 Other & Plantation forests (\textit{PF}) and human settlements (\textit{S}). \NN
 & \NN
 \multicolumn{2}{l}{Source: \citeonline{SamuelRosaEtAl2011a}. Cartographic scale: \num{1}:\num{2000}. Minimum 
 Legible Delineation: \SI{100}{\square\m}.} \NN
 \texttt{LU2009a} & \textit{FS}. Native forest. \NN
 \texttt{LU2009b} & \textit{SS}. Shrubland, which is likely to have SOC and ECEC level above those found in 
 areas used with annual crop agriculture and animal husbandry, but lower than in native forests. \NN
 \texttt{LU2009c} & \textit{H}. Animal husbandry. \NN
 \texttt{LU2009d} & \textit{AA}. Annual crop agriculture, which is likely to have the lowest soil fertility 
 due to the usually poor management practices employed. \NN
 \texttt{LUdiff}  & Land use difference between 1980 and 2009. It can be useful to explain, for example, low 
 SOC in forest soil due to previous use with crop agriculture or animal husbandry. \NN
 Other & Plantation forests (\textit{PF}), human settlements (\textit{S}), and other land uses 
 (\textit{O}), comprising natural and artificial water bodies. \LL
 }
