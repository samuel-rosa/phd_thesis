\ctable[
  caption  = {Description of the $p = 12$ dummy predictor variables derived from the two soil maps.},
  cap      = {Predictor variables derived from soil maps.},
  label    = tab:chap06-soil-covars,
  notespar,
  pos      = !ht,
  maxwidth = \textwidth,
  % doinside = \scriptsize\setstretch{1.1},
  doinside = \small
  ]{l p{0.85\textwidth} l}{
  \tnote[a]{Soil classification according to the old Brazilian classification (only for 
  \citet{AzolinEtAl1988}), the current Brazilian classification, according to \citet{SantosEtAl2013a}, and the 
  international classification, following \citet{IUSSWorkingGroupWRB2007}.}
  
  \tnote[b]{Minimum Legible Delineation calculated following \citet{Rossiter2000}.}
  
  }{ \FL
  Code & Mapping unit(s) included and Description\tmark[a,b] \ML
  
  \multicolumn{2}{p{0.98\linewidth}}{Source: \citet{AzolinEtAl1988}. Cartographic scale: 
  \num{1}:\num{100000}. Minimum Legible Delineation: \SI{40}{\hectare}.} \NN
  
  \texttt{SOIL\_100b} & \textit{Re4}. Shallow soil with low to high base saturation covering mountainous 
  terrain (Solo Litólico Eutrófico/Distrófico relevo montanhoso; Neossolo Litólico Distrófico/Eutrófico; 
  Distric/Eutric Leptosol). \NN 
  \texttt{SOIL\_100c} & \textit{Re-C-Co}. Shallow soil with high base saturation located in strongly sloping 
  terrain (Solo Litólico Eutrófico relevo forte ondulado; Neossolo Litólico Eutrófico; Eutric Leptosol), low 
  weathered soil (Cambissolo Eutrófico; Cambissolo Háplico Eutrófico; Eutric Cambisol), and colluvial
  deposits. \NN
  \texttt{SOIL\_100d} & \textit{TBa-Rd}. Deep, well-structured, low base saturation soil (Terra Bruna 
  Estruturada álica; Nitossolo; Nitisol), and shallow soil (Solo Litólico; Neossolo Litólico; Leptosol). \NN
  \texttt{SOIL\_100e} & \textit{Rd1} and \textit{Re4}. \textit{Rd1} is composed mainly of shallow soil with 
  low to high base saturation (Solo Litólico Distrófico/Eutrófico; Neossolo Litólico Distrófico/Eutrófico; 
  Distric/Eutric Leptosol) located in slopping terrain. This dummy predictor variable is composed of shallow 
  soil in both sloping and mountainous terrain. \NN
  \texttt{SOIL\_100f} & \textit{TBa-Rd} and \textit{C1}. \textit{C1} is composed of low weathered soil 
  developed in lower landscape positions, close to drainage channels (Cambissolo Eutrófico; Cambissolo 
  Eutrófico; Eutric Cambisol). This dummy predictor variable includes the best soil mapping units for crop 
  agriculture among those identified in the soil survey. \LL
  }

  \ctable[
  caption  = {Description of the $p = 12$ dummy predictor variables derived from the two soil maps.},
  cap      = {Predictor variables derived from soil maps.},
  label    = tab:chap06-soil-covars,
  notespar,
  continued,
  pos      = !ht,
  maxwidth = \textwidth,
  % doinside = \scriptsize\setstretch{1.1},
  doinside = \small
  ]{l p{0.85\textwidth} l}{
  \tnote[a]{Soil classification according to the old Brazilian classification (only for 
  \citet{AzolinEtAl1988}), the current Brazilian classification, according to \citet{SantosEtAl2013a}, and the 
  international classification, following \citet{IUSSWorkingGroupWRB2007}.}
  
  \tnote[b]{Minimum Legible Delineation calculated following \citet{Rossiter2000}.}
  
  }{ \FL
  Code & Mapping unit(s) included and Description\tmark[a,b] \ML
 
  \multicolumn{2}{p{0.98\linewidth}}{Source: \citet{MiguelEtAl2012}. Cartographic scale: 
  \num{1}:\num{25000}. Minimum Legible Delineation: \SI{2.5}{\hectare}.} \NN
  
  \texttt{SOIL\_25a}  & \textit{PBAC}. Moderately deep soil derived from sedimentary rocks, with abrupt 
  textural change and low base saturation (Argissolo Bruno-Acinzentado; Alisol). \NN
  \texttt{SOIL\_25b}  & \textit{PV}. Deep soil derived from igneous rocks, with moderate textural gradient, 
  and low base saturation (Argissolo Vermelho; Acrisol). \NN
  \texttt{SOIL\_25c}  & \textit{C-R}. Low weathered soil (Cambissolo; Cambisol) and shallow soil with low to 
  high base saturation (Neossolo Litólico/Regolítico Eutrófico/Distrófico; Eutric/Distric Leptosol/Regosol). 
  \NN
  \texttt{SOIL\_25d}  & \textit{RL}. Shallow soil with low to high base saturation (Neossolo Litólico 
  Eutrófico/Distrófico; Eutric/Distric Leptosol). \NN
  \texttt{SOIL\_25h}  & \textit{PBAC}, \textit{PV} and \textit{SX}. \textit{SX} is composed of moderately deep 
  soil derived from sedimentary rocks, with abrupt textural change, low base saturation, and which is 
  saturated with water for long periods of the year (Planossolo Háplico; Planosol). This dummy predictor 
  variable includes the best soil mapping units for crop agriculture among those identified in the soil
  survey. \NN
  \texttt{SOIL\_25i}  & \textit{RL}, \textit{RL-RR} and \textit{RR}. This dummy predictor variable includes 
  all three mapping units composed mainly of shallow soil (Neossolo Litólico and Neossolo Regolítico; 
  Leptosol and Regosol). \NN
  \texttt{SOIL\_25j}  & \textit{PV}, \textit{RL}, \textit{RL-RR} and \textit{C-R}. This dummy predictor 
  variable includes all four mapping units composed mainly of soil derived from igneous rocks. \LL
  }
