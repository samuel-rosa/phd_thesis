\ctable[
 caption  = {Description of the $p = 7$ dummy predictor variables derived from the two geologic maps.},
 label    = tab:chap05-geology-covars,
 notespar,
 pos      = !ht,
 % doinside = \scriptsize\setstretch{1.1}
 doinside = \scriptsize
 ]{l p{0.85\textwidth} l}{
 \tnote[a]{Minimum Legible Delineation calculated following \citeonline{Rossiter2000}.}
 }{ \FL
 Code & Mapping unit(s) included and Description\tmark[a] \ML
 \multicolumn{2}{l}{Source: \citeonline{GasparettoEtAl1988}. Cartographic scale: \num{1}:\num{50000}. Minimum 
 Legible Delineation: \SI{10}{\hectare}.} \NN
 \texttt{GEO\_50a} & \textit{SG-I}. Inferior Sequence of the Serra Geral Formation. Composed mainly of basic 
 igneous rocks (tholeiitic basalt and andesite). It is likely to be related with high CLAY and ECEC. \NN
 \texttt{GEO\_50b} & \textit{SG-S}. Superior Sequence of the Serra Geral Formation. Composed mainly of acid 
 igneous rocks (granophyric rhyolite and rhyodacite). It is likely to be related with moderate to high CLAY 
 and ECEC. \NN
 \texttt{GEO\_50c} & \textit{BT}. Botucatu Formation. Composed mainly of aeolian sandstones. It is likely to 
 be related with low CLAY and ECEC. \NN
 Other & \textit{CT} depicts the Caturrita Formation, which is composed mainly of fluvial sandstones. \NN
 & \NN
 \multicolumn{2}{l}{Source: \citeonline{MacielFilho1990}. Cartographic scale: \num{1}:\num{25000}. Minimum 
 Legible Delineation: \SI{2.5}{\hectare}.} \NN
 \texttt{GEO\_25a} & \textit{SG-I}. Inferior Sequence of the Serra Geral Formation. \NN
 \texttt{GEO\_25b} & \textit{SG-S}. Superior Sequence of the Serra Geral Formation. \NN
 \texttt{GEO\_25c} & \textit{BT}. Botucatu Formation. \NN
 \texttt{GEO\_25d} & \textit{QD}. Quaternary deposits of fluvial, alluvial, and colluvial origin. It can help 
 explaining the low CLAY of soils supposedly derived from igneous rocks. \NN
 Other & \textit{CT} depicts the Caturrita Formation. \LL
 }
