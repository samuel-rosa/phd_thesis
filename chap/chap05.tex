\artigotrue
\chapter{THE SANTA MARIA DATASET. PART II -- COVARIATE DATA}
\chapternote{This chapter is based on the studies \textit{Identifying and correcting oblique striping in the 
Topodata digital elevation model}, presented at the XXXIV Brazilian Congress of Soil Science 
\cite{Samuel-RosaEtAl2013c}, and \textit{Evaluation of freely available ancillary data used for detailed soil 
mapping in Brazil}, presented at the EGU General Assembly 2014 \cite{Samuel-RosaEtAl2014}. Also collaborated 
in the preparation of this document: Pablo Miguel (UFPel), Jean Michel Moura Bueno (UFSM), Ricardo Simão Diniz 
Dalmolin (UFSM), Lúcia Helena Cunha dos Anjos (UFRRJ), Gustavo de Mattos Vasques (Embrapa Solos), and Gerard 
B. M. Heuvelink (ISRIC -- World Soil Information).}
\shorttitle{Covariate Data}
\label{chap:chap05}
%\SweaveUTF8



%\def\ptkeys{}
%
%\begin{chapterabstract}{brazilian}{\ptkeys}
% Este é o resumo em português.
%\end{chapterabstract}

\def\enkeys{Area-class soil maps. Digital elevation models. Geological maps. Land use maps. Satellite images}
  
\begin{chapterabstract}{english}{\enkeys}
The Santa Maria dataset comprises a list of more than 100 covariates covering the catchment of the reservoir of 
the \textit{Departamento Nacional de Obras de Saneamento}-\textit{Companhia Riograndense de Saneamento} 
(DNOS-CORSAN), located in the southern Brazilian state of Rio Grande do Sul. These covariates were derived from 
five freely available sources in two levels of spatial detail: area-class soil maps, digital elevation models, 
geological maps, land use maps, and satellite images. These covariate data are the outcome of projects that 
aimed at modelling various environmental features and were carried out as part of local (soil, geology, land 
use), regional (terrain, land use), and global (terrain, land use) mapping initiatives.
\end{chapterabstract}

\formatchapter

\section{INTRODUCTION}
\label{sec:chap05-intro}

The Santa Maria dataset is a data set comprising spatially exhaustive covariate data produced in the 1980s, 
1990s, and 2000s covering the catchment of the reservoir of the \textit{Departamento Nacional de Obras de 
Saneamento}-\textit{Companhia Riograndense de Saneamento} (DNOS-CORSAN), henceforth called \emph{DNOS 
catchment}, located in the southern border of the plateau of the Paraná Sedimentary Basin, in the city of 
Santa Maria, state of Rio Grande do Sul, Brazil. Some of these covariate data cover only part of the DNOS
catchment, mainly the northern sector, which has an area of about \SI{2000}{\hectare}, corresponding to about 
\SI{60}{\percent} of the entire catchment. These covariate data are the outcome of projects that aimed at 
modelling various environmental features and were carried out as part of local (soil, geology, land use), 
regional (terrain, land use), and global (terrain, land use) mapping initiatives.

Covariate data were harmonized to a reference grid of \SI{5}{\m} grid size. The coordinate reference 
system (CRS) is WGS1984 / UTM zone 22S, coded \href{http://spatialreference.org/ref/epsg/32722/}{\num{32722}} 
by the European Petroleum Survey Group (\href{http://www.epsg.org/}{EPSG}).

This document presents a description of the covariate data contained in the Santa Maria dataset, including the
procedures for their production, as well as the processing methods employed. The original sources of the 
covariate data are freely available at the producers databases or in public libraries.

\section{GROUND CONTROL POINTS}
\label{sec:chap05-gcp}

All covariates were validated prior to their use. Horizontal (positional) validation was performed using a set
of $n = 14$ validation points, here called ground control points (GCP), spread throughout and beyond the 
limits of the study area (\autoref{fig:chap05-field-gcps}). The location of the GCPs was defined based on 
the existence of easily identifiable geographical markers across the covariates, including road intersection, 
fence corners, and property entrances.

\begin{figure}[!ht]
\centering
\includegraphics[width = 0.90\textwidth]{fig/chap05-field-gcps}
\caption[Ground control points used for the positional validation of the covariates.]{Spatial distribution of 
the ground control points ($n = 14$, red dots) used for the horizontal positional validation of covariates 
included in the Santa Maria dataset.}
\label{fig:chap05-field-gcps}
\end{figure}

Positional validation was performed comparing the x- and y-coordinates of GCPs (observed value) with the 
coordinates of the respective geographical markers visually identified on the covariates (predicted value). 
The differences in the observed and predicted x- and y-coordinates were used to calculate the mean error (ME, 
\si{\m}), mean absolute error (MAE, \si{\m}), and mean squared error (MSE, \si{\m\square}) to evaluate if 
there were differences in the accuracy and precision between coordinates. The error vector (or module, the 
euclidean distance between two points) and its azimuth (the orientation of the error vector) were computed as 
well for every point. The mean of the error vector and its azimuth give the size and orientation of the 
systematic error present in the covariates, while the square root of the mean squared error vector (RMSE) is a 
measure of the uncertainty about the true position of the covariate in the geographic space.

The field location of the GCPs is as follows (in Portuguese):

\begin{description}
\item[GCP 01]
% Em Santa Maria, no lado direito da barragem de concreto do reservatório do DNOS-CORSAN, \SI{6}{\m} antes de 
% chegar a ponte sobre o vertedouro, a \SI{3}{\m} de distância do centro da estrada que desce da rodovia 
% federal BR158.
In Santa Maria, on the right side of the concrete dam of the DNOS-CORSAN reservoir, \SI{6}{\m} 
before reaching the bridge over the spillway, distant \SI{3}{\m} from the centre of the road that descends 
from the BR-158 federal highway.

\item[GCP 02]
% Em Itaara, na entrada da Estrada do Perau, Rua Gralha Azul, no centro do canteiro, entre o outdoor e a árvore 
% (\textit{Cedrella fissilis}), junto à rodovia federal BR158.
In Itaara, at the entrance of the Estrada do Perau, Rua Gralha Azul, in the centre of the roundabout, between 
the outdoor and the tree (\textit{Cedrella fissilis}), close to the BR-158 federal highway.

\item[GCP 03]
% Em Itaara, próximo ao equipamento estático de fiscalização eletrônica de velocidade, na rodovia federal 
% BR158, a \SI{520}{\m} do Museu de Ufologia, em frente à Fruteira da Esquina, do lado oposto da torre de 
% telefonia celular e da entrada para a mina de extração de brita DallaPasqua, localizada a \SI{4}{\km} do 
% local.
In Itaara, near the speed monitoring radar of the BR-158 federal highway, distant \SI{520}{\m} from the UFO 
Museum, opposite the Fruteira da Esquina, opposite to the cell phone tower and to the entrance of the gravel 
pit DallaPasqua, located \SI{4}{\km} away from the site.

\item[GCP 04]
% Em Santa Maria, na entrada da rua que dá acesso ao cemitério do Bairro Campestre do Menino Deus, no lado 
% direito da Estrada do Perau em direção à rodovia federal BR158, alinhado com a fachada das residências, a 
% \SI{2,2}{\m} de distância do muro frontal, a \SI{6,5}{\m} de distância do meio-fio da Estrada do Perau, a 
% \SI{50}{\m} de distãncia da ponte sobre o Rio Vacacaí-Mirim.
In Santa Maria, at the entrance of the street that leads to the cemetery of the Campestre do Menino Deus 
neighbourhood, on the right side of Perau Road going towards the federal highway BR-158, aligned with the 
façade of the residences, \SI{2.2}{\m} away from the front wall, \SI{6.5}{\m} away from the curb of the Perau 
Road, \SI{50}{\m} away from the bridge over the Vacacaí-Mirim River.

\item[GCP 05]
% Em Santa Maria, na entrada do Rancho do Amaral, junto à porteira, no lado direito, fora da estrada, distante 
% \SI{1}{\m} de uma palmeira e \SI{1}{\m} do muro de pedras.
In Santa Maria, at the entrance of the Rancho do Amaral, next to the gate, on the right size, off the road, 
\SI{1}{\m} away of a palm tree and \SI{1}{\m} away from the stone wall.
 
\item[GCP 06]
% Em Itaara, na Avenida Etelvina, na beira da estrada, a \SI{2,5}{\m} de distância do centro da estrada,
% alinhado com a cerca que separa a floresta nativa do pomar de \textit{Citrus}~sp.~localizado do outro lado 
% da estrada.
In Itaara, Etelvina Avenue, on the roadside, \SI{2.5}{\m} away from the road centre, aligned with the fence 
that separates the native forest from the \textit{Citrus}~sp.~orchard located across the road.

\item[GCP 07]
% Em Itaara, Localidade de Estação Pinhal, no lado esquerdo do acesso à mina de extração de brita DallaPasqua, 
% sob a rede de transmissão de eletricidade.
In Itaara, Estação Pinhal Locality, on the left side of the beginning of the road that gives access to the 
DallaPasqua gravel pit, under the electricity transmission network.

\item[GCP 08]
% Em Santa Maria, Distrito de Santo Antão, na entrada da estrada que dá acesso à Central de Tratamento de 
% Resíduos da Caturrita, em frente à Escola Municipal de Ensino Fundamental Intendente Manoel Ribas.
In Santa Maria, Santo Antão District, at the beginning of the road that gives access to the Caturrita Waste 
Treatment Centre, opposite the Municipal Elementary School Intendente Manoel Ribas.
 
\item[GCP 09]
% Em São Martinho da Serra, Localidade de Água Negra, na bifurcação da estrada que vem de Santa Maria e que dá 
% acesso à Localidade de Campinas, junto à parada de ônibus, no canteiro no meio da bifurcação, \SI{40}{\m} 
% distante do Piquete Laçador Jorge R.~da Silva, em frente ao Mercado do Ronaldo.
In São Martinho da Serra, Água Negra Locality, at the bifurcation of the road coming from Santa Maria and 
giving access to the Campinas Locality, close to the bus stop, at the roundabout in the middle of bifurcation, 
\SI{40}{\m} away from the Piquete Laçador Jorge R.~da Silva, in front of Ronaldo's Market.

\item[GCP 10]
% Em Santa Maria, na estrada em direção à São Martinho da Serra, depois da capela Santo Antão, no lado externo 
% de uma curva, próximo a duas pequenas árvores, alinhado com a cerca que marca a divisa entre duas 
% propriedades ocupadas com campo nativo. Em frente à propriedade com duas residências, uma delas com dois 
% andares e quatro pequenos lagos nos fundos.
In Santa Maria, on the road towards São Martinho da Serra, after Santo Antão Chapel, on the outside of a 
curve, near two small trees, aligned with the fence that marks the boundary between two properties occupied 
with native grass. In front of the property with two houses, one with two floors and four small lakes in the 
backyard.
 
\item[GCP 11]
% Em Itaara, entrada para a Localidade Rincão dos Minello, no início da estrada que dá acesso à Brita Pinhal, 
% ao lado da rodovia federal BR158, \SI{5}{\m} de distância à frente do corte no terreno expondo a rocha de 
% arenito da Formação Botucatu, \SI{15}{\m} distante do poste da linha de transmissão de eletricidade da 
% companhia AES.
In Itaara, at the entrance to the Rincão dos Minello Locality, at the beginning of the road that gives access 
to Brita Pinhal, next to the federal highway BR-158, \SI{5}{\m} away and in front of the road cut exposing the 
sandstone rocks of the Botucatu Formation, \SI{15}{\m} away from the post of the electricity transmission line 
of the AES company.

\item[GCP 12]
% Em Itaara, na rodovia federal BR158, em frente ao lago da SOCEPE, na entrada da cidade, próximo ao Bar e 
% Armazém Ricardo, deslocado em \SI{1}{\m} para dentro do passeio em relação ao alinhamento dos postes da rede 
% elétrica.
In Itaara, at the federal highway BR-158, in front of the SOCEPE Lake, in the city entrance, near the Ricardo's 
Bar and Warehouse, shifted \SI{1}{\m} into the sidewalk in relation to the alignment of the posts of the 
electrical network.

\item[GCP 13]
% Em Itaara, na Vila Etelvina, com uma vinha à montante, alinhado com a cerca que divide duas propriedades 
% localizadas no lado oposto da estrada, uma à esquerda coberta com floresta nativa/exótica e outra à direita 
% ocupada com lavoura de culturas anuais.
In Itaara, at Vila Etelvina, with a vineyard upstream, aligned with the fence that divides two properties 
located on the opposite side of the road, the one on the left covered with native/exotic woods and the other on 
the right occupied with field of annual crops.
 
\item[GCP 14]
% Em Itaara, na estrada que sobe para a propriedade do Sr. Antoninho Luccas, logo após o término da subida 
% íngreme com calçamento apenas nos trilhos, no final da floresta e início do campo nativo, alinhado (a 
% \SI{20}{\cm}) com a cerca dos dois lados da estrada. Locado \SI{1}{\m} distante do moirão da cerca do lado 
% esquerdo subindo, no interior da estrada.
In Itaara, in the road that goes towards the property of Mr. Antoninho Luccas, soon after the steep climb, 
where there is only pavement on the tracks, at the end of the forest and beginning of native grass, aligned 
with the fence on both sides of road, \SI{1}{\m} away from the left corner fence post.
\end{description}

Attribute validation of soil, geologic, and land use maps, and digital elevation models was done using a set of 
$n = 60$ 
validation points located along $m = 12$ linear transects. The procedures for obtaining soil, geologic, land 
use, and elevation data at these validation points is described in \autoref{chap:chap04}, 
\autoref{sec:chap04-subset-ii}. Such a validation exercise was carried out because these maps originally had no 
accompanying validation information.

% TODO: figure with GCPs used to orthorectify satellite images. Show the bounding box of the image and the 
% boundary 
% of the study area.
% \begin{figure}
%  \centering
%  \includegraphics[width=\textwidth]{fig/ortho-gcps}
%  \caption{Ground control points used to orthorectify the image produced by Landsat-5 Thematic 
% Mapper.}
%  \label{fig:chap05-ortho-gcps}
% \end{figure}

\section{AREA-CLASS SOIL MAPS}
\label{sec:chap05-soil-maps}

Two area-class soil maps are included in the Santa Maria dataset. The first of them (\soilOld{}) was published 
at a \scale{100000} \cite{AzolinEtAl1988}. Existing area-class soil maps and technical reports 
\cite{Brasil1973, Azolin1977, MacielEtAl1987a, MacielEtAl1987, AbraoEtAl1988}, and sparse field observations 
were used to elaborate the preliminary legend of the soil map. Aerial photographs (\scale{60000}) were used to 
produce the first draft of the soil map. Field checks of soil polygons (i.e. mapping units) was done along the 
road network (i.e. by convenience sampling). These observations were used to estimate the composition 
(occurrence and spatial distribution of soil taxa) of soil mapping units. They were also used to review the 
first draft of the soil map. The final version of \soilOld{} was prepared using topographic maps originally 
published at a \scale{50000} and resampled to a \scale{100000}. Soil classification followed the criteria 
adopted by the Brazilian soil science community at that time \cite{Brasil1973, CamargoEtAl1982, Carvalho1982, 
LemosEtAl1982, OlmosEtAl1982}. Identification of soil taxa was performed based on morphological features, 
analytical data compiled from existing technical reports, and analysis of soil samples collected from soil 
profiles observed along the road network. Description of each soil mapping unit includes the estimated area 
(\si{\hectare}) and the approximate taxonomic composition (\si{\percent}).

\begin{figure}[!ht]
\centering
\begin{minipage}[b]{0.45\textwidth}
\subcaption{}
\centering
\includegraphics[width = \textwidth]{fig/chap05-soil-old}
\end{minipage}
\begin{minipage}[b]{0.45\textwidth}
\subcaption{}
\centering
\includegraphics[width = \textwidth]{fig/chap05-soil-new}
\end{minipage} 
\caption[Area-class soil maps included in the Santa Maria dataset.]{Area-class soil maps (a) \soilOld{} and (b) 
\soilNew{} used to derive indicator covariates included in the Santa Maria dataset. Legend abbreviations and 
derived dummy variables are described in the text.}
\label{fig:chap05-soil-maps}
\end{figure}

The second area-class soil map (\soilNew) included in the Santa Maria dataset \cite{Miguel2010} was prepared 
at a \scale{25000}. Satellite images produced by Digital Globe\rr{} Quickbird satellite (\SI{2.4}{\m} spatial 
resolution), freely available for visualization in Google Earth\rr{}, were used to produce the first 
draft of the soil map. Existing area-class soil maps and technical reports \cite{Pedron2005, Poelking2007, 
Sturmer2008} were used to help defining the preliminary soil map legend. Point field observations (soil pits 
and boreholes) were made in more than \num{350} locations using a purposive sampling approach (see 
\autoref{chap:chap04} and 
\autoref{chap:chap07}). These observations helped to identify six modal (representative) soil profiles. Soil 
sampling and description of modal soil profiles, and laboratory analyses of soil samples, followed the standard 
protocols adopted in Brazil \cite{ClaessenEtAl1997, SantosEtAl2005}. Soil classification was done following the 
criteria of the Brazilian System of Soil Classification (SiBCS) \cite{SantosEtAl2006}. The final version of the 
map was prepared using satellite images freely available for visualization in \googleearth{} and 
manually-digitalized topographic sheets published at a \scale{25000} \cite{DSG1992a, DSG1992}. Description of 
soil mapping units includes only the most common soil taxon, followed by morphological and laboratory data of 
modal soil profiles.

The area-class soil maps went through different preprocessing routines. The original \soilOld{} is available 
only in analogical format, what required its digitalization. Georeferencing was carried out using the GDAL 
Georeferencer plug-in in QGIS \cite{GDAL2013, QGIS2013}. Intersections between all meridians and parallels (a 
total of nine) were used as control points to adjust a second order polynomial model. Resampling was performed 
using the cubic resampling method. Soil polygons and their attributes were also manually digitalized in QGIS. 
Because of the coarseness of the cartographic map scale, most geographical markers used to locate validation 
GCPs could not be identified and positional validation was performed using only four GCPs. Estimated error 
statistics suggest that there are large positional errors in all directions, with an RMSE of \SI{114}{\m} and 
a mean azimuth of \SI{128}{\degree} (\autoref{tab:chap05-soil-geo-val}).

\ctable[
  caption  = {Error statistics of the horizontal positional validation of \soilOld{} using $n = 4$ ground
  control points.},
  cap      = {Error statistics of the horizontal positional validation of \soilOld.},
  label    = tab:chap05-soil-geo-val,
  notespar,
  pos      = !ht,
  maxwidth = \textwidth,
  % doinside = \scriptsize\setstretch{1.1},
  doinside = \small
  ]{lrrrr}{
  }{                                                                          \FL
  Statistics                   & x-coord & y-coord & Error vector & Azimuth   \ML
  Mean, \si{\m}                & 30      & -36     & 105          & \ang{128} \NN
  Absolute mean, \si{\m}       & 58      & 64      & -            & -         \NN 
  Squared mean, \si{\m\square} & 7241    & 5712    & 12953        & -         \LL
  }

% \begin{figure}[!ht]
%   \centering
%   \includegraphics[width=0.45\textwidth]{azim-soil100}
%   \includegraphics[width=0.45\textwidth]{azim-soil25}
%   \caption{Histogram of the azimuth distribution of the validation of area-class soil maps \soilOld{} 
% and \soilNew{} in the attribute space. Azimuth values were estimated using four and XX
% ground control points located in easily identifiable geographical markers, respectively. 
% The graph was produced using R-package \textit{VecStatGraphs2D}.}
%   \label{fig:soil-azim}
% \end{figure}

The original \soilNew{} is available in digital format in the personal database of the author 
\cite{Miguel2010}. A topology check (Topology Checker plug-in in QGIS) identified that there were many gaps 
and overlaps between polygons. This required a topological edition prior to the use of \soilNew. There also 
was a mismatch between the boundary of \soilNew{} and the actual boundary of the study area as estimated using 
\demNew{} (\autoref{sec:chap05-dem}). This occurred because the database used to produce \soilNew{} 
included \googleearth{} imagery and topographic maps, which are data sources that differ considerably in their 
positional accuracy (\autoref{sec:chap05-dem} and \autoref{sec:chap05-land-use}). To avoid data 
losses, all boundary gaps were manually filled using the closest mapping unit. Boundaries of soil polygons 
were defined based on land use (\landNew{}, \autoref{sec:chap05-land-use}) and topographic data (contour 
lines, \autoref{sec:chap05-dem}) as it was done for the original map \cite{Miguel2010}. New delineations were 
checked and approved without modifications by the author of the original map. Because \soilNew{} includes very 
few geographical markers, its positional validation was not possible with the available GCPs. However, the 
RMSE is expected to vary between \SIrange{8}{114}{\m} across the study area as a result of the different 
errors present in the data sources used in its production.

Both \soilOld{} and \soilNew{} were cropped to the bounding box of the study area, and resampled to match the 
reference grid using the nearest neighbour resampling method to maintain data integrity. Each category was 
named with the code of the respective mapping unit in the original map. Prior to validation in the attribute 
space, class codes of \soilOld{} were changed to match soil taxa codes of the current Brazilian System of Soil 
Classification using a standard correlation table \cite{SantosEtAl2006}. The overall purities of both soil maps 
are not considerably different. A reason for this could be that validation was performed considering only the 
second level of the SiBCS -- it is likely that \soilNew{} would outperform \soilOld{} if validation data 
included soil taxa up to the fourth level of the SiBCS. The low overall purity of \soilOld{} and 
\soilNew{} (\num{31.67} and \SI{30.00}{\percent}, respectively) is likely due to several sources of error. 
First, the small number of modal soil profiles used to produce both maps that might have resulted in an 
optimistic view of the homogeneity of each mapping unit. Second, soil classes of \soilOld{} were translated to 
the more recent SiBCS using only a standard correlation table \cite{SantosEtAl2006} and expert knowledge 
because the survey report does not include analytical soil data. Last, soil classes at the $n = 60$ validation 
sites were inferred in the field using only morphological soil properties and the general concepts of the 
SiBCS.

Five ($p = 5$) covariates were derived from \soilOld{} as described below (including the soil classes according 
to the original and updated classification \cite{AzolinEtAl1988, SantosEtAl2013a}, and the international 
classification \cite{IUSSWorkingGroupWRB2007}):

\begin{description}
 \item[\texttt{SOIL\_100b}] Shallow soil (\textit{Re4}) with low to high base saturation covering mountainous 
 terrain (Solo Litólico Eutrófico/Distrófico relevo montanhoso; Neossolo Litólico Distrófico/Eutrófico;
 Distric/Eutric Leptosol).
  
 \item[\texttt{SOIL\_100c}] Association (\textit{Re-C-Co}) of shallow soil with high base saturation located in
steep terrain (Solo Litólico Eutrófico relevo forte ondulado; Neossolo Litólico Eutrófico; Eutric
 Leptosol), low weathered soil (Cambissolo Eutrófico; Cambissolo Háplico Eutrófico; Eutric Cambisol), and
 colluvial deposits.
  
 \item[\texttt{SOIL\_100d}] Association (\textit{TBa-Rd}) of deep, well-structured, low base saturation soil 
(Terra
 Bruna Estruturada álica; Nitossolo; Nitisol), and shallow soil (Solo Litólico; Neossolo Litólico; Leptosol).
  
 \item[\texttt{SOIL\_100e}] Shallow soil (\textit{Rd1} and \textit{Re4}) with low to high base
 saturation (Solo Litólico Distrófico/Eutrófico; Neossolo Litólico Distrófico/Eutrófico; Distric/Eutric
 Leptosol) located in undulating to mountainous terrain.
  
 \item[\texttt{SOIL\_100f}] This covariate includes the best soil mapping units for row crop agriculture among 
those 
 identified in the soil survey, that is \textit{TBa-Rd}, described above, and \textit{C1}, which is composed
 of low weathered soils developed in lower landscape positions, close to drainage channels (Cambissolo
 Eutrófico; Cambissolo Eutrófico; Eutric Cambisol).
\end{description}

Covariates derived from \soilNew{} are presented below. Mapping unit \textit{RY}, composed mainly of soils 
developed 
from fluvial deposits (Neossolo Flúvico; Fluvisol) does not appear due to the small area that it occupies.

\begin{description}
  \item[\texttt{SOIL\_25a}] Moderately deep soil (\textit{PBAC}) derived from sedimentary rocks, with abrupt 
textural
  change and low base saturation (Argissolo Bruno-Acinzentado; Alisol).
  
  \item[\texttt{SOIL\_25b}] Deep soil (\textit{PV}) derived from igneous rocks, with moderate textural 
gradient,
  and low base saturation (Argissolo Vermelho; Acrisol).
 
  \item[\texttt{SOIL\_25c}] Low weathered soil (Cambissolo Háplico; Cambisol) and shallow soil with low to high 
base
  saturation (Neossolo Litólico/Regolítico Eutrófico/Distrófico; Eutric/Distric Leptosol/Regosol) 
(\textit{C-R}).
 
  \item[\texttt{SOIL\_25d}] Shallow soil (\textit{RL}) with low to high base saturation (Neossolo Litólico 
  Eutrófico/Distrófico; Eutric/Distric Leptosol).
 
  \item[\texttt{SOIL\_25h}] This covariate includes the mapping units with the best soils for row crop 
agriculture
  among those identified in the soil survey (\textit{PBAC}, \textit{PV}, and \textit{SX}). \textit{PBAC} and 
\textit{PV} are as
  described above. \textit{SX} is composed of moderately deep soil derived from sedimentary rocks, with abrupt 
  textural change, low base saturation, and which are saturated with water for long periods of the year 
  (Planossolo Háplico; Planosol).
  
  \item[\texttt{SOIL\_25i}] This covariate includes all three mapping units (\textit{RL}, \textit{RL-RR}, and 
\textit{RR})
  composed mainly of shallow soils (Neossolo Litólico and Neossolo Regolítico; Leptosol and Regosol).
  
  \item[\texttt{SOIL\_25j}] This covariate includes all four mapping units (\textit{PV}, \textit{RL}, 
\textit{RL-RR}, and 
  \textit{C-R}) composed mainly of soil derived from igneous rocks.
\end{description}

\section{DIGITAL ELEVATION MODELS}
\label{sec:chap05-dem}

Two digital elevation models (DEMs) are included in the Santa Maria dataset as sources of terrain covariates. 
The first DEM (\demNew) is the result of the interpolation of the contour lines of the most recent topographic 
sheets produced by the Brazilian Army (\scale{25000}) that cover the study area \cite{DSG1980, DSG1992, 
DSG1992a}. Topographic sheets were digitalized and georeferenced using the GDAL Georeferencer plugin in QGIS. 
Intersections between all meridians and parallels (about \num{160} per topographic sheet) were used as control 
points to adjust a third order polynomial model. Resampling was performed using the cubic resampling method. 
All contour lines, peaks, lakes and rivers, and their respective attributes within a distance of 
\SI{1000}{\metre} from the boundary of the study area were manually digitized and stored in vector format. 
After digitalization, the original coordinate reference system (EPSG:31982 -- SIRGAS2000 / UTM zone 22S) of all 
vector files was transformed to WGS1984 / UTM zone 22S (EPSG:32722) using the \Rpackage{rgdal} 
\cite{BivandEtAl2013a}.

The horizontal positional validation of topographic maps was performed using the $n = 14$ GCPs. According to 
Brazilian legislation, high-quality map standards require that at least 90% of the GCPs have horizontal 
positional errors smaller than \SI{13}{\metre}, and an overall horizontal error (i.e. RMSE) smaller than 
\SI{8}{\metre} at the \scale{25000} \cite{Brasil1984}. Estimated validation statistics show that the overall 
observed horizontal error ($\text{RMSE} = \SI{65}{\m}$) is larger than those established by current regulations 
(\autoref{tab:chap05-topomap-geo-val}). The mean error vector is larger than \SI{60}{\metre} with an azimuth of 
\SI{63}{\degree}. Both x- and y-coordinates are positively biased, but the largest error occurs in the 
x-coordinate (\SI{50}{\metre}). Similar ME and MAE values suggests that there is a strong systematic positional 
error. An affine transformation was employed using the \Rpackage{vec2dtransf} \cite{Carrillo2012} to correct 
this systematic error. Model parameters were adjusted using the same set of GCPs used for the validation in the 
geographic space.

\ctable[
  caption  = {Error statistics of the horizontal validation of topographic maps (\scale{25000}) using $n = 14$ 
  ground control points.},
  cap      = {Error statistics of the horizontal validation of topographic maps.},
  label    = tab:chap05-topomap-geo-val,
  notespar,
  pos      = !ht,
  maxwidth = \textwidth,
  % doinside = \scriptsize\setstretch{1.1},
  doinside = \small
  ]{lrrrr}{
  }{                                                                          \FL
  Statistics                    & x-coord & y-coord & Error vector & Azimuth  \ML
  Mean, \si{\m}                 & 50      & 27      & 63           & \ang{63} \NN
  Absolute mean, \si{\m}        & 50      & 32      & -            & -        \NN
  Squared mean, \si{\m\squared} & 3088    & 1180    & 4268         & -        \LL
  }

% \begin{figure}[!ht]
%   \centering
%   \includegraphics[width=0.5\textwidth]{azim-car25}
%   \caption{Histogram of the azimuth distribution of the validation of topographic maps in the geographic 
% space. Azimuth values were estimated using 14 ground control points located in easily identifiable 
% geographical markers. 
% The graph was produced using R-package 
% \textit{VecStatGraphs2D}.}
%   \label{fig:topomap-azim}
% \end{figure}

Interpolation of the raster surface with \SI{5}{\metre} grid size was performed using the function 
\texttt{Topo to Raster} in ArcGIS\textregistered{} software by ESRI, which includes an interpolation method 
based on the ANUDEM program \cite{Hutchinson1989}. Vector files of contour lines (\texttt{multiline}), 
drainage network (\texttt{multiline}), lakes (\texttt{polygons}) and peaks (\texttt{points}) were used to 
generate a hydrologically sound DEM, that is, a DEM without spurious depressions and giving a precise 
representation of the hydrological data. Next, the interpolated DEM was imported into GRASS, where a 
neighbourhood average filter was used to remove stair-like artefacts. A window of $7 \times 7$ pixels was used 
because it removed a significant amount of the artefacts and did not affect the derived boundary of the study 
area (see more bellow).

The vertical datum of the DEM was transformed from the local datum to a global datum. The geoidal models 
MAPGEO2010 \cite{IBGE2010a} and EGM1996 \cite{LemoineEtAl1998} were used to calculate the geoidal undulation 
for the local and global datums, respectively. MAPGEO2010 is optimized to estimate geoidal undulations in the 
Brazilian territory, while EGM1996 is a gravitational model of the Earth and is used as the vertical datum for 
Shuttle Radar Topography Mission (SRTM) products. The following equation was used:

\begin{equation}\label{eqn:geoidal}
 h = H + N,
\end{equation}

\noindent where $h$ is the ellipsoidal height (height above the reference ellipsoid that approximates the 
surface of the planet), $H$ is the orthometric height (height above the imaginary surface called geoid and 
commonly referred to as mean sea level), and $N$ is the geoidal undulation. Ellipsoidal heights estimated by 
MAPGEO2010 are referenced to the world ellipsoid of 1980, while EGM1996 estimates ellipsoidal heights 
referenced to the world ellipsoid of 1984. Because the difference between both ellipsoids is of the order of 
millimetres, it can be assumed that both models estimate the same ellipsoidal height. Therefore, if 
$h_{\text{EGM1996}} = h_{\text{MAPGEO2010}}$, then orthometric heights referenced to the local vertical datum 
can be transformed to the global vertical datum using the following equation:

\begin{equation}
 H_{\text{EGM1996}} = H_{\text{MAPGEO2010}} + N_{\text{MAPGEO2010}} - N_{\text{EGM1996}}.
\end{equation}

\noindent The difference in the geoidal undulation estimated by both models is of about one meter in the 
entire study area. Thus, transforming the vertical datum was done adding one meter to the raster surface 
interpolated from contour lines, yielding the first DEM included in the Santa Maria dataset (\demNew{}).

The second DEM (\demOld{}) is the well known SRTM DEM ($\SI{3}{\arcsecond} \approx \SI{90}{\m}$ spatial 
resolution) produced by the National Aeronautics and Space Administration’s Jet Propulsion Laboratory in 
collaboration with the National Geospatial\-/Intelligence Agency \cite{RodriguezEtAl2006}. The SRTM DEM version 
used here is the \emph{sink\-/filled SRTM version \num{4}}, prepared by the Consultative Group for 
International Agricultural Research (\cgiar) using the same hydrologically correct interpolation method that 
was used above to produce \demNew{} \cite{ReuterEtAl2007, JarvisEtAl2008}. However, the only data source used 
was the SRTM DEM version 3 converted to point data.

The SRTM DEM was processed to match the reference grid using cubic resampling (\gdal{gdalwarp} and 
\grass{r.resamp.interp}). This resampling method was used because it is efficient in minimizing the 
double-oblique stripping present in SRTM products \cite{Samuel-RosaEtAl2013c}. Sinks produced during datum 
transformation were filled using the \grass{r.fill.dir}. Vertical datum transformation was not necessary 
because elevation values of the SRTM DEM already are referenced to the global geoid model EGM1996 
(orthometric heights). These DEMs were used to derive the terrain covariates used in the thesis. 

\begin{figure}[!ht]
\centering
\begin{minipage}[b]{0.45\textwidth}
\subcaption{}
\centering
\includegraphics[width = \textwidth]{fig/chap05-dem-old}
\end{minipage}
\begin{minipage}[b]{0.45\textwidth}
\subcaption{}
\centering
\includegraphics[width = \textwidth]{fig/chap05-dem-new}
\end{minipage} 
\caption[Digital elevation models included in the Santa Maria dataset.]{Digital elevation models (a) \demOld{} 
and (b) \demNew{} used to derive terrain covariates included in the Santa Maria dataset.}
\label{fig:chap05-dem}
\end{figure}

A third DEM was included with the sole purpose of supporting the orthorectification -- removal of the effects 
of the perspective of the sensor on the relative position of objects in the image -- and topographic correction 
-- removal of the effects of the topography on the reflectance values, i.e. on the illumination of the 
geomorphic surfaces -- of satellite images (\autoref{sec:chap05-sat-image}) \cite{Mather2004, 
Schowengerdt2007}. The third DEM (\topodata) was produced by the Brazilian National Institute for Space 
Research (\inpe) by refining the SRTM DEM version 1 (\q{unfinished}) to $\SI{1}{\arcsecond} \approx 
\SI{30}{\m}$ spatial resolution using ordinary kriging with a Gaussian spatial autocorrelation model 
\cite{ValerianoEtAl2012}. Eight tiles were mosaicked (\gdal{gdal\_translate}) and the CRS transformed from 
WGS1984 (EPSG:4326) to WGS1984 / UTM zone 22S (EPSG:32722) using cubic resampling (\gdal{gdalwarp}), and the 
sinks filled using \grass{r.fill.dir}. Before the atmospheric correction of satellite images, orthometric 
heights were converted to ellipsoidal heights using \autoref{eqn:geoidal}, the geoidal undulation calculated 
with the gravitational model EGM1996. This conversion was done because orbital satellites use the WGS1984 
ellipsoid as vertical datum. For the orthorectification of satellite images, the DEM was then processed using 
\grass{r.resamp.interp} with the bicubic resampling method to match the reference grid.

The three DEMs present similar vertical accuracy (\autoref{tab:chap05-dem-attr-val}). In the case of \demNew{}, 
which was derived from contour lines published at a \scale{25000}, the vertical positional accuracy does not 
meet the high-quality standards of current Brazilian legislation, which states that 90% of the GCPs should have 
vertical errors smaller than \SI{5}{\metre}, which is half of the distance between contour lines 
\cite{Brasil1984}. The corresponding RMSE should be less than \SI{3}{\metre}.

\ctable[
  caption  = {Error statistics of the vertical validation of \demOld, TOPODATA, and \demNew{} using $n = 60$
  validation points located along $m = 12$ linear transects.},
  cap      = {Error statistics of the vertical validation of \demOld, TOPODATA, and \demNew.},
  label    = tab:chap05-dem-attr-val,
  notespar,
  pos      = !ht,
  maxwidth = \textwidth,
  % doinside = \scriptsize\setstretch{1.1},
  doinside = \small
  ]{lrrrrrr}{
  }{                                                              \FL
  Statistics                   & \demOld{} & TOPODATA & \demNew{} \ML
  Mean, \si{\m}                & -15       & -17      & -16       \NN
  Absolute mean, \si{\m}       & 15        & 17       & 16        \NN
  Squared mean, \si{\m\square} & 350       & 361      & 374       \LL
  }

% Figure \ref{fig:cdf-elev} shows that estimated validation statistics have different cumulative 
% distribution functions (CDF). The estimates are more uniformly distributed along the interval of 
% values for \texttt{ELEV\_10} than for \texttt{ELEV\_90} and \texttt{ELEV\_30}. While 
% \texttt{ELEV\_10} has a 50\% probability that absolute errors are below 15 m, \texttt{ELEV\_90} has 
% a 70\% probability that absolute errors are below 15 m. This suggests that the accuracy of 
% \texttt{ELEV\_90} is very consistent across the study area, with a few extreme values, while the 
% accuracy of \texttt{ELEV\_10} have a stronger spatial variation. For \texttt{ELEV\_30}, the 
% interpolation method used to refine the original SRTM DEM to 30 m \cite{ValerianoEtAl2012} seems to 
% have produced a spatial redistribution of the errors.

% \begin{figure}[!ht]
%   \centering
%   \includegraphics[width=0.9\textwidth]{fig/cdf-ELEV-90} 
%   \includegraphics[width=0.9\textwidth]{fig/cdf-ELEV-30}
%   \includegraphics[width=0.9\textwidth]{fig/cdf-ELEV-10}
%   \caption{Cumulative distribution functions of mean error, mean absolute error, and squared error of 
% elevation values estimates by digital elevation models \texttt{ELEV\_90}, \texttt{ELEV\_30}, and 
% \texttt{ELEV\_10}.}
%   \label{fig:cdf-elev}
% \end{figure}

Despite all DEMs present a similar vertical accuracy, \demNew{} is considered the highest quality DEM in the 
Santa Maria dataset. Because it was produced using information about the drainage network and location of lakes 
and natural depressions, it is likely to provide a better hydrological representation of the study area. As 
such, \demNew{} was used to estimate the geographical limits (boundary) of the catchment that constitute the 
study area, for which GRASS modules \texttt{r.watershed} and \texttt{r.water.outlet} were employed. Because the 
overall deviation between the affine-corrected coordinates of topographic maps and target coordinates of GCPs 
is $\text{RMSE} = \SI{29.55}{\m}$ -- there still is uncertainty about the correct position of topographic maps 
-- a \SI{30}{\m} buffer was added to the geographical limits of the catchment. The water outlet point used to 
estimate the boundary is located on the bridge that crosses the main drainage channel (\ang{-29.65868}, 
\ang{-53.78969}).

Eight terrain attributes were derived from each of \demOld{} and \demNew{} to produce the covariate data 
included in the Santa Maria dataset, the first of them being elevation (\texttt{ELEV}). The others are slope, 
aspect, northernness, flow accumulation, topographic wetness index, stream power index, and topographic 
position index.

Slope (\texttt{SLP}) and aspect (\texttt{ASP}) were calculated using \grass{r.param.scale}. This module 
calculates terrain attributes by fitting a bivariate quadratic polynomial using least squares \cite{Wood1996}. 
It allows using different window sizes to fit the bivariate quadratic polynomial, thus including the effect of 
scale in the calculation of terrain attributes. Seven window sizes were used (3, 7, 15, 31, 63, 127, and 255) 
and the results for calculated slope can be seen in \autoref{fig:chap05-slope}. Larger window sizes result in 
a smoother version of the terrain attribute, while smaller windows sizes result in raster maps with more 
(small-scale) details. Several flat surfaces (slope of \ang{0}) were produced in the slope raster maps 
calculated using \texttt{ELEV\_90} as a result of resampling the original DEM from \num{90} to \SI{5}{\m}. A 
value of \ang{0.1} was added to the rasters to remove these flat surfaces.

\begin{figure}[!ht]
\centering
\includegraphics[width = 0.90\textwidth]{fig/chap05-slope}
\caption[Slope derived using windows of different sizes.]{Slope \texttt{SLP} raster surfaces derived from 
\demNew{} using windows of sizes (from left to right) $3 \times 3$, $31 \times 31$, $127 \times 127$, and $255 
\times 255$.}
\label{fig:chap05-slope}
\end{figure}

Aspect values had to be corrected before use because \grass{r.param.scale} stores aspect values in the range 
\SIrange{0}{+180}{\degree} from West to North to East, and \SIrange{0}{-180}{\degree} from West to South to 
East, when the standard procedure is to work with aspect values ranging from \SIrange{0}{360}{\degree} 
clockwise. This correction was done using

\begin{equation}
 \texttt{ASP}_{0} =
 \begin{cases}
  \texttt{ASP}_{GRASS} + \ang{360} & \text{if}\;\; \texttt{ASP}_{GRASS} < \ang{0}, \\
  \texttt{ASP}_{GRASS}             & \text{else},
 \end{cases}
\end{equation}

\noindent and

\begin{equation}
 \texttt{ASP} =
 \begin{cases}
  \texttt{ASP}_{0} + \ang{270} & \text{if}\;\; \texttt{ASP}_{0} < \ang{90}, \\
  \texttt{ASP}_{0} - \ang{90}  & \text{else}.
 \end{cases}
\end{equation}

\noindent A second correction of aspect values involved their linearization. This is necessary because aspect 
is a circular variable, that is, the beginning (\ang{0}) and end (\ang{360}) of the measurement scale have the 
same physical meaning. Aspect values were transformed to northernness (\texttt{NOR}), a measure of the degree 
of exposition of a given surface to the North, a linear variable, using the equation

\begin{equation}\label{eqn:NOR}
 \texttt{NOR} = abs(\ang{180} - \texttt{ASP}).
\end{equation}  

Flow accumulation (ACC), also known as catchment area or contributing area, was calculated using 
\grass{r.watershed}, the resulting raster surface being multiplied by the square of the cell size (i.e. by the 
cell area \SI{\approx25}{\metre\square}) to convert to areal units. This raster surface was used to calculate 
the topographic wetness index (\texttt{TWI}) and the stream power index (\texttt{SPI}) using

\begin{equation}\label{eqn:sACC}
 \text{sACC} = \dfrac{\text{ACC}}{5},
\end{equation}

\begin{equation}\label{eqn:TWI}
 \texttt{TWI} = log \dfrac{\text{sACC}}{tan(\texttt{SLP})},
\end{equation}

\noindent and

\begin{equation}\label{eqn:SPI}
 \texttt{SPI} = log(\text{sACC} \times tan(\texttt{SLP})),
\end{equation}

\noindent where sACC is the specific catchment area, \SI{5}{\m} is the cell size, and \texttt{SLP} is the 
slope 
raster surface calculated using seven different window sizes.

The topographic position index \texttt{TPI} was calculated with \saga{ta\_morphometry}. Different values of 
maximum radius were used to include the effect of scale, all of them related to the window sizes used to 
calculate previous terrain attributes. A minimum radius value of \SI{3}{\m} was used in all calculations.

A total of $1 \times \texttt{ELEV} + 7 \times (\texttt{NOR}, \texttt{SLP}, \texttt{TWI}, \texttt{SPI}, 
\texttt{TPI}) = 36$ covariates 
were defined using the terrain attributes derived from \demOld{} and \demNew{}.

\section{GEOLOGIC MAPS}
\label{sec:chap05-geo-maps}

Geologic data comes from the two most recent geologic maps (\geoOld{} and \geoNew{}) published at the 
\scales{50000}{25000} \cite{GasparettoEtAl1988, MacielFilho1990} (\autoref{fig:chap05-geo-maps}). Both of 
them were produced based on the most recent topographic sheets produced by the Brazilian Army at the 
\scales{50000}{25000} \cite{DSG1980, DSG1992, DSG1992a}. Alike topographic maps, geologic maps were 
available only in analogical format, and were hand digitized and georeferenced in QGIS. Intersections 
between all meridians and parallels ($n = 16$) were used as control points to adjust a second order polynomial 
model. Resampling was performed using the cubic resampling method. The original CRS 
(EPSG:31982 -- SIRGAS2000 / UTM zone 22S) was transformed to match the reference grid using the 
\Rpackage{rgdal} \cite{BivandEtAl2013a}.

\begin{figure}[!ht]
\centering
\begin{minipage}[b]{0.45\textwidth}
\subcaption{}
\centering
\includegraphics[width = \textwidth]{fig/chap05-geo-old}
\end{minipage}
\begin{minipage}[b]{0.45\textwidth}
\subcaption{}
\centering
\includegraphics[width = \textwidth]{fig/chap05-geo-new}
\end{minipage} 
\caption[Geologic maps included in the Santa Maria dataset.]{Geologic maps (a) \geoOld{} and (b) \geoNew{} used 
to derive indicator covariates included in the Santa Maria dataset. Legend abbreviations and derived dummy 
variables are described in the text.}
\label{fig:chap05-geo-maps}
\end{figure}

The positional validation of geologic maps was performed using $n = 8$ (\geoOld{}) and $n = 5$ (\geoNew{}) 
GCPs, respectively. Validation statistics show that the positional accuracy of neither geologic map meets the 
high-quality standards of the current Brazilian legislation (\autoref{tab:chap05-geology-geo-val}). Estimated 
RMSE are \num{147} and \SI{69}{\m} for \geoOld{} and \geoNew{}, respectively, when the maximum RMSE accepted 
are \num{25} and \SI{13}{\m}, respectively. For \geoOld{}, the lowest accuracy is found in the y-coordinate, 
while for \geoNew{}, the x-coordinate is the least accurate. Validation statistics suggest that there is a 
strong systematic error, which probably was propagated from the topographic maps used to produce the geologic 
maps. Therefore, the same strategy (affine transformation) used to remove the systematic positional error of 
the topographic maps was employed on geologic maps. Due to the lack of GCPs, model parameters were adjusted 
using the same set of GCPs used for the validation. The estimated uncertainty (RMSE) of the affine 
transformation is \num{86} and \SI{22}{\m} for \geoOld{} and \geoNew{}, respectively.

\ctable[
  caption  = {Error statistics of the horizontal validation of geologic maps \geoOld{} and \geoNew{} using 
  $n = 8$ and $n = 5$ ground control points.},
  cap      = {Error statistics of the horizontal validation of geologic maps.},
  label    = tab:chap05-geology-geo-val,
  notespar,
  pos      = !ht,
  maxwidth = \textwidth,
  % doinside = \scriptsize\setstretch{1.1},
  doinside = \small
  ]{lrrrr}{
  }{                                                                          \FL
  Statistics                   & x-coord & y-coord & Error vector & Azimuth   \ML
  \multicolumn{5}{l}{\geoOld{} ($n = 8$)}                                     \ML
  Mean, \si{\m}                & 10      & -102    & 140          & \ang{169} \NN
  Absolute mean, \si{\m}       & 43      & 125     & -            & -         \NN
  Squared mean, \si{\m\square} & 3431    & 18067   & 21498        & -         \ML
  \multicolumn{5}{l}{\geoNew{} ($n = 5$)}                                     \ML
  Mean, \si{\m}                & 51      & 29      & 67           & \ang{58}  \NN
  Absolute mean, \si{\m}       & 51      & 29      & -            & -         \NN
  Squared mean, \si{\m\square} & 3457    & 1312    & 4769         & -         \LL
  }

% \begin{figure}[ht]
%  \centering
%  \caption{Histogram of the azimuth distribution of the validation of geologic maps 
%  \texttt{GEO\_50} (left) and \texttt{GEO\_25} (right) in the geographic space. Azimuth values were 
%  estimated using, respectively, eight and five ground control points located in easily identifiable 
% geographical 
%  markers. The graph was produced using \Rpackage{VecStatGraphs2D}.}
%  \label{fig:chap05-geology-azim}
% \end{figure}

Three ($p = 3$) covariates were derived from \geoOld{}:

\begin{description}
 \item[\texttt{GEO\_50a}] Inferior Sequence of the Serra Geral Formation. Composed mainly by basic 
 igneous rocks (tholeiitic basalt and andesite). It is likely related to high soil clay content and effective 
cation exchange capacity;
 
 \item[\texttt{GEO\_50b}] Superior Sequence of the Serra Geral Formation. Composed mainly by acid 
 igneous rocks (granophyric rhyolite and rhyodacite). It is likely related to moderate to 
 high \texttt{CLAY} and \texttt{ECEC};
 
 \item[\texttt{GEO\_50c}] Botucatu Formation. Composed mainly by aeolian sandstones. It is likely  related to 
low \texttt{CLAY} and \texttt{ECEC};
\end{description}

Four ($p = 4$) covariates were derived from \geoNew{}, the first three of them having the same meaning 
of those derived from \geoOld{}:

\begin{description}
 \item[\texttt{GEO\_25a}] Inferior Sequence of the Serra Geral Formation;
 
 \item[\texttt{GEO\_25b}] Superior Sequence of the Serra Geral Formation;
 
 \item[\texttt{GEO\_25c}] Botucatu Formation;
 
 \item[\texttt{GEO\_25d}] Quaternary deposits of fluvial, alluvial, and colluvial origin. It can help 
explaining the low clay content in areas where the soil is believed to have developed from igneous rocks as 
depicted in less detailed geologic maps.
\end{description}

\section{LAND USE MAPS}
\label{sec:chap05-land-use}

The first land use map used to derive covariate data included in the Santa Maria dataset was produced by 
manually digitizing land use data of \num{1980} (\landOld{}) published in the most recent 
topographic map produced by the Brazilian Army (\scale{25000}) \cite{DSG1992a, DSG1992}. Most processing 
steps, including the correction of positional bias, are described in \autoref{sec:chap05-dem}, except for 
the use of the nearest neighbour resampling method to match \landOld{} to the reference grid.

\begin{figure}[!ht]
\centering
\begin{minipage}[b]{0.45\textwidth}
\subcaption{}
\centering
\includegraphics[width = \textwidth]{fig/chap05-land-old}
\end{minipage}
\begin{minipage}[b]{0.45\textwidth}
\subcaption{}
\centering
\includegraphics[width = \textwidth]{fig/chap05-land-new}
\end{minipage}
\begin{minipage}[b]{0.45\textwidth}
\subcaption{}
\centering
\includegraphics[width = \textwidth]{fig/chap05-land-diff}
\end{minipage}
\caption[Land use maps included in the Santa Maria dataset.]{Land use maps (a) \landOld{} and (b) \landNew{} 
used to derive indicator covariates included in the Santa Maria dataset such as (c) \texttt{LUdiff}, the land 
use change between the years of 1980 and 2009. Legend abbreviations and derived dummy variables are described 
in the text.}
\label{fig:chap05-land-use}
\end{figure}

The second land use map (\landNew{}) was prepared at a \scale{2000} using high resolution (\SI{2.4}{\m}) 
Quickbird satellite images of \num{2008} and \num{2009}, made publicly available in \googleearth{} 
\cite{SamuelRosaEtAl2011a}. Identification of land uses and delineation of mapping units were done manually, on 
the computer screen, without using any automated classification routine. Positional validation of 
\googleearth{} imagery revealed that they have only minor systematic positional errors 
(\autoref{tab:chap05-google-geo-val}). Despite this, the attribute validation of both land use maps, using $n = 
60$ validation points placed along $m = 12$ linear transects, showed that they have similar overall accuracy 
close to \SI{70.00}{\percent}.

\ctable[
  caption  = {Error statistics of the horizontal validation of \googleearth{} imagery using $n = 14$ ground 
  control points.},
  cap      = {Error statistics of the horizontal validation of \googleearth{} imagery.},
  label    = tab:chap05-google-geo-val,
  notespar,
  pos      = !ht,
  maxwidth = \textwidth,
  % doinside = \scriptsize\setstretch{1.1},
  doinside = \small
  ]{lrrrr}{
  }{                                                                          \FL
  Statistics                   & x-coord & y-coord & Error vector & Azimuth   \ML
  Mean, \si{\m}                & -1      & 3       & 6            & \ang{184} \NN
  Absolute mean, \si{\m}       & 3       & 5       & -            & -         \NN
  Squared mean, \si{\m\square} & 14      & 57      & 71           & -         \LL
  }

\landOld{} was used to derive $p = 2$ covariates defined as indicator variables, with plantation forests 
(\textit{PF}) and human settlements (\textit{S}) being grouped together due to their small importance in terms 
of 
covered area (\textit{PF}) and for not containing any soil observation (\textit{S}). These covariates are:

\begin{description}
 \item[\texttt{LU1980a}] Native forest (\textit{FS}), which is likely to have soils with higher fertility.
  
 \item[\texttt{LU1980b}] Animal husbandry (\textit{H}), the second most important land use in the study area, 
which is
 likely to have a soil fertility status lower than native forests.
\end{description}

Four ($p = 4$) indicator covariates were derived from \landNew{}, with plantation forests (\textit{PF}), human 
settlements (\textit{S}), and other land uses (\textit{O}), which comprise natural and artificial water bodies, 
being 
grouped together due to their small importance in terms of covered area (\textit{PF}) and for not containing 
any 
soil observation (\textit{S} and \textit{O}). These covariates are:

\begin{description}
 \item[\texttt{LU2009a}] Native forest (\textit{FS}), as described above.
 
 \item[\texttt{LU2009b}] Shrubland (\textit{SS}), which is likely to have a soil fertility level above those 
found in
 areas used with annual crop agriculture and animal husbandry, but lower than in native forests.
 
 \item[\texttt{LU2009c}] Animal husbandry (\textit{H}), as described above.
  
 \item[\texttt{LU2009d}] Annual crop agriculture (\textit{AA}), which is likely to present the lowest soil 
fertility 
 levels due to the usually poor management practices employed.
\end{description}

A seventh indicator covariate (\texttt{LUdiff}) was derived using data from both land use maps. It consists of 
the land use change between \num{1980} and \num{2009}, computed by checking if the land use has changed (1) or 
remained the same (0) in every grid cell after the 29-year period. \texttt{LUdiff} can be useful, for example, 
to explain the low soil organic carbon content in forest soils due to previous use with crop agriculture or 
animal husbandry.

\section{SATELLITE IMAGES}
\label{sec:chap05-sat-image}

Two sources of satellite images were used to produce covariate data included in the Santa Maria dataset. The 
first is the Thematic Mapper (TM) sensor aboard the longest-operating Earth observation satellite -- Landsat 5. 
The satellite image used was acquired on \num{26} December \num{2010} and is freely available in the database 
of the Division of Image Generation of the Brazilian National Institute for Space Research (\inpedgi). The 
image contains seven spectral bands (\autoref{tab:chap05-satellites}) (including a thermal band that is not 
included in the Santa Maria dataset), with \SI{8}{\bit} radiometric resolution (digital numbers from 
\numrange{0}{255}) and \SI{30}{\m} spatial resolution. The satellite image was orthorectified using 
Geomatica\textregistered{} OrthoEngine\textregistered{} with the Landsat rigorous model (Toutin's Model) 
\cite{Toutin2004, PCIGeomatics2007}. Due to the absence of good identifiable field GCPs, $n = 28$ GCPs were 
collected from \googleearth{} imagery, which have a high positional accuracy in the study area 
(\autoref{tab:chap05-google-geo-val}). GCPs were located 
at easily identifiable geographical markers such as road intersections and bridges, evenly distributed 
throughout the image, and covering a variety of elevations, following standard recommendations 
\cite{PCIGeomatics2007}. 
% TODO: add link to \autoref{fig:chap05-ortho-gcps}
The DEM used for orthorectification is TOPODATA after preprocessing as described in \autoref{sec:chap05-dem}. 
Resampling was done using the nearest neighbour method to avoid changing the digital numbers.

After orthorectification, all bands were imported into GRASS, where all other necessary corrections were 
performed. Radiometric correction (conversion from digital numbers to top-of-atmosphere reflectance) was done 
using \grass{i.landsat.toar}. Atmospheric correction (removal of the effects of the atmosphere on the 
reflectance values) was done with the 6S atmospheric model \cite{VermoteEtAl1997} as implemented in 
\grass{i.atcorr} using the tropical atmospheric model, the continental aerosols model, an image-based 
visibility estimate of \SI{20}{\km}, and a constant elevation of \SI{300}{\m}.

The second source of satellite images is RapidEye. Images are available through the Brazilian Ministry of the 
Environment \cite{Brasil2012}, which has a full coverage of the Brazilian territory for \num{2011} and 
\num{2012}. The satellite image used (tile number \num{2225403}) was acquired on \num{16} November \num{2012} 
(second coverage). It contains five spectral bands 
(\autoref{tab:chap05-satellites}), featuring among them the so-called red-edge band, located between the red 
and the near-infrared bands. This spectral band is the main feature distinguishing RapidEye images from most 
other sources of satellite images, considered to provide additional information about the vegetation 
\cite{WeicheltEtAl2013}. The satellite image has \SI{16}{\bit} radiometric resolution and \SI{6.5}{\m} spatial 
resolution, and was orthorectified at the source to \SI{5}{\m} spatial resolution using the sink-filled SRTM 
version \num{4} DEM \cite{RapidEye2013}.

\begin{figure}[!ht]
\centering
\begin{minipage}[b]{0.45\textwidth}
\subcaption{}
\centering
\includegraphics[width = \textwidth]{fig/chap05-sat-old}
\end{minipage}
\begin{minipage}[b]{0.45\textwidth}
\subcaption{}
\centering
\includegraphics[width = \textwidth]{fig/chap05-sat-new}
\end{minipage} 
\caption[Satellite images included in the Santa Maria dataset.]{Satellite images used to derive covariates such 
as (a) \texttt{NDVI\_30} and (b) \texttt{NDVI\_5b} included in the Santa Maria dataset.}
\label{fig:chap05-sat-image}
\end{figure}

\ctable[
  caption  = {Comparison between satellite images produced by Landsat 5 TM and RapidEye and the derived 
  covariates.},
  cap      = {Comparison between satellite images and derived covariates.},
  label    = tab:chap05-satellites,
  notespar,
  pos      = !ht,
  maxwidth = \textwidth,
  % doinside = \scriptsize\setstretch{1.1},
  doinside = \small
  ]{lllllll}{
  }{ \FL
  \multicolumn{3}{l}{Landsat 5 TM}                  && \multicolumn{3}{l}{RapidEye} \NN
  \cmidrule(r){1-3}\cmidrule(l){5-7}
  Band & Interval, \si{nm} & Covariate              && Band & Interval, \si{\nm} & Covariate \ML
  1 Visible       & 450--520   & \texttt{BLUE\_30}  && Blue & 440--510 &\texttt{BLUE\_5} \NN
  2 Visible       & 520--600   & \texttt{GREEN\_30} && Green & 520--590& \texttt{GREEN\_5} \NN
  3 Visible       & 630--690   & \texttt{RED\_30}   && Red & 630--685 & \texttt{RED\_5} \NN
  -               & -          & -                  && Red-edge & 690--730 & \texttt{EDGE\_5} \NN
  4 Near-infrared & 760--900   & \texttt{NIR\_30a}  && Near-infrared & 760--850 & \texttt{NIR\_5} \NN
  5 Near-infrared & 1550--1750 & \texttt{NIR\_30b}  && - & - & - \NN
  7 Mid-infrared  & 2080--2350 & \texttt{MIR\_30}   && - & - & - \LL
  }

The RapidEye image was atmospherically corrected using the 6S atmospheric model \cite{VermoteEtAl1997} 
employing the Fortran code developed by \citeonline{AntunesEtAl2013} -- \grass{i.atcorr} was not used because a 
\atcorrbug{} was found when trying to correct the RapidEye image -- assuming a tropical atmospheric model, the 
continental aerosols model, an image-based 
visibility estimate of \SI{20}{\km}, and a constant elevation of \SI{300}{\m}.

After atmospheric correction, Landsat 5 TM and RapidEye images were resampled using the nearest neighbour 
method to match the reference grid. Topographic correction was performed using \grass{i.topo.corr} with 
TOPODATA geometrically corrected to match the reference grid as described in \autoref{sec:chap05-dem}.

\ctable[
  caption  = {Error statistics of the horizontal validation of satellite images produced by Landsat 5 TM and 
  RapidEye using $n = 14$ ground control points.},
  cap      = {Error statistics of the horizontal validation of satellite images.},
  label    = tab:chap05-satellite-geo-val,
  notespar,
  pos      = !ht,
  maxwidth = \textwidth,
  % doinside = \scriptsize\setstretch{1.1},
  doinside = \small
  ]{lrrrr}{
  }{ \FL
  Statistics                   & x-coord & y-coord  & Error vector  & Azimuth   \ML
  \multicolumn{5}{l}{Landsat 5 TM}                                              \ML
  Mean, \si{\m}                & 31      & -11      & 45            & \ang{136} \NN
  Absolute mean, \si{\m}       & 33      & 25       & -             & -         \NN
  Squared mean, \si{\m\square} & 1494    & 1223     & 2717          & -         \ML
  \multicolumn{5}{l}{RapidEye}                                                  \ML
  Mean, \si{\m}                & -25     & -25      & 36            & \ang{226} \NN
  Absolute mean, \si{\m}       & 25      & 25       & -             & -         \NN
  Squared mean, \si{\m\square} & 680     & 708      & 1388          & -         \LL
  } 

Individual bands of both satellite images were defined as covariates, totalling $p = 6$ from Landsat 5 TM and 
$p = 5$ from RapidEye (Table \ref{tab:chap05-satellites}). Another $p = 6$ covariates ($p = 2$ from Landsat 5 
TM and $p = 4$ from RapidEye) were defined using two vegetation indices: the normalized difference vegetation 
index (NDVI) and the soil-adjusted vegetation index (SAVI), calculated as

\begin{equation}\label{eqn:ndvi}
 \text{NDVI} = \frac{\text{NIR} - \text{RED}}{\text{NIR} + \text{RED}}
\end{equation}

\noindent and 

\begin{equation}\label{eqn:savi}
  \text{SAVI} = (1.0 + 0.5) \times \frac{\text{NIR} - \text{RED}}{\text{NIR} + \text{RED} + 0.5},
\end{equation}

\noindent where $\text{NIR} = \texttt{NIR\_30a}$ and $\text{RED} = \texttt{RED\_30}$ for the Landsat 5 TM 
image, and $\text{NIR} = \texttt{NIR\_5}$ and $\text{RED} = \texttt{RED\_5}$ or $\text{RED} = \texttt{EDGE\_5}$ 
for RapidEye image.

\section{CONCLUSIONS}

This document presented a thorough description of the covariate data (area-class soil maps, digital elevation 
models, geological maps, land use maps, and satellite images) contained in the Santa Maria dataset. This 
description included the procedures employed in their production, as well as all processing methods employed. 
We expect the reader to have understood the features of the covariate data used in the thesis, and that this 
description will support future soil spatial modelling exercises in the catchment of the DNOS reservoir.

As a result of an ongoing collaborative effort, it is likely that, as new studies are developed, this 
documentation will be improved. For example, we already plan to include more figures with the indicator 
covariates derived from area-class soil maps, geological maps and land use maps. Other DEM derivatives might be 
included as well to better exemplify the effect of the window size on the spatial variation of calculated DEM 
derivatives. Finally, we plan to include descriptive statistics of soil properties as related to the different 
covariates as a means of improving our understanding of their empirical relationship.
