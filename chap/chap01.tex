\artigotrue
\chapter{Do more detailed environmental covariates deliver more accurate soil maps?}
\label{chap:chapter01}

% User defined commands
\def\geoNew{\texttt{GEO\_25}}
\def\geoOld{\texttt{GEO\_50}}
\def\soilNew{\texttt{SOIL\_25}}
\def\soilOld{\texttt{SOIL\_100}}
\def\demNew{\texttt{ELEV\_10}}
\def\demOld{\texttt{ELEV\_90}}
\def\landOld{\texttt{LU1980}}
\def\landNew{\texttt{LU2009}}
\def\elev{\texttt{ELEV}} % elevation
\def\slp{\texttt{SLP}}   % slope
\def\asp{\texttt{ASP}}   % aspect
\def\nor{\texttt{NOR}}   % northernness
\def\acc{\texttt{ACC}}   % flow accumulation
\def\twi{\texttt{TWI}}   % topographic wetness index
\def\spi{\texttt{SPI}}   % stream power index
\def\tpi{\texttt{TPI}}   % topographic position index
\def\ndvi{\texttt{NDVI}} %
\def\savi{\texttt{SAVI}} %
\def\sibcs{Brazilian System of Soil Classification}
\newcommand{\grass}[1]{GRASS GIS module \texttt{#1}} % GRASS GIS modules
\newcommand{\saga}[1]{SAGA GIS library \texttt{#1}}  % SAGA GIS libraries
%\newcommand{\scale}[1]{cartographic scale of 1:#1} % scale

\def\englishkeys{Digital Soil Mapping, Linear Mixed Model, Auxiliary 
  Information, Variable Selection, Model Accuracy, Soil Mapping Cost}
\begin{chapterabstract}{english}{\englishkeys}
In this study we evaluated whether investing in more spatially detailed 
environmental covariates improves the accuracy of digital soil maps. We used a 
case study from Southern Brazil to map clay content (CLAY), organic carbon 
content (SOC), and effective cation exchange capacity (ECEC) of the topsoil for
a $\sim$~2,000~ha area located on the edge of the plateau of the Paraná 
Sedimentary Basin. Five covariates, each with two levels of spatial detail were
used: area-class soil maps, digital elevation models (DEM), geologic maps, land
use maps, and satellite images. Thirty-two multiple linear regression models 
were calibrated for each soil property using all spatial detail combinations of
the covariates. For each combination, stepwise regression was used to select 
predictor variables incorporated in the model. Model evaluation was done using 
the adjusted R-square of the regression. The baseline model, calibrated with the
less detailed version of each covariate, and the best performing model were used
to calibrate two linear mixed models for each soil property. Model parameters 
were estimated using restricted maximum likelihood. Spatial prediction was 
performed using the empirical best linear unbiased predictor. Validation of 
baseline and best performing linear multiple regression and linear mixed models
was done using cross-validation. Results show that for CLAY the prediction 
accuracy did not considerably improve by using more detailed covariates. The 
amount of variance explained increased only $\sim$~2 percentage points (pp), 
less than that obtained by including the kriging step, which explained 4~pp. On
the other hand, prediction of SOC and ECEC improved by $\sim$~13~pp when the 
baseline model was replaced by the best performing model. Overall, the increase
in prediction performance was modest and may not outweigh the extra costs of 
using more detailed covariates. It may be more efficient to spend extra 
resources on collecting more soil observations, or increasing the detail of only
those covariates that have the strongest improvement effect. In our case study,
the latter would only work for SOC and ECEC, by investing in a more detailed 
land use map and possibly also a more detailed geologic map and DEM.
\end{chapterabstract}

\def\portuguesekeys{Mapeamento Digital do Solo, Modelo Linear Misto, Informação 
Auxiliar, Seleção de Variáveis, Acurácia do Modelo, Custo do Mapeamento do Solo}
\begin{chapterabstract}{brazilian}{\portuguesekeys}
Neste estudo nós avaliamos se investir em covariáveis ambientais espacialmente
mais detalhadas aumenta a acurácia dos mapas digitais do solo. Nós usamos um 
estudo de caso no sul do Brasil para mapear o conteúdo de argila (CLAY), o
conteúdo de carbono orgânico (SOC), e capacidade de troca de cátions efetiva 
(ECEC) da camada superficial do solo de uma área de $\sim$~2000~ha localizada
na borda do planalto da Bacia Sedimentar do Paraná. Cinco covariáveis, cada uma
com dois níveis de detalhe espacial, foram usadas: mapa areal-categórico de solo,
modelos digitais de elevação (DEM), mapas geológicos, mapas de uso da terra, e 
imagens de satélite. Trinta e dois modelos de regressão linear múltipla foram
calibrados para cada propriedade do solo usando todas as combinações de detalhe
espacial das covariáveis. Para cada combinação, stepwise regression foi usada 
para selecionar as variáveis preditoras incorporadas no modelo. A avaliação dos
modelos foi feita usando o R-quadrado ajustado da regressão. O modelo baseline, 
calibrado com a versão menos detalhada de cada covariável, e o modelo com o 
melhor desempenho, foram usados para calibrar dois modelos lineares mistos para
cada propriedade do solo. Parâmetros dos modelos foram estimados usando 
restricted maximum likelihood. Predições espaciais foram realizadas usando o 
empirical best linear unbiased predictor. Validação-cruzada foi usada para 
validar os modelos de regressão linear múltipla e dos modelos lineares mistos 
de linha de base e com melhor desempenho. Os resultados mostram que para CLAY a 
acurácia da predição não aumentou consideravelmente por usar covariáveis mais
detalhadas. A quantidade de variância explicada aumentos apenas $\sim$~2 pontos
percentuais (pp), menos do que obtido pela inclusão do passo de krigagem, que
explicou 4-pp. Por outro lado, a predição de SOC e ECEC aumentos em $\sim$~13~pp
quando o modelo de base foi substituído pelo modelo com melhor desempenho. Em 
geral, o aumento no desempenho preditivo foi modesto e pode não sobrepor os 
custos adicionais do uso de covariáveis mais detalhadas. Pode ser mais eficiente
investir recursos adicionais na coleta de mais observações do solo, ou no 
aumento do detalhe apenas da covariável que tem o efeito de aumento mais forte.
Em nosso estudo, a última funcionaria apenas para SOC e ECEC pelo investimento
em um mapa de uso da terra mais detalhado e, possivelmente, também em um mapa 
geológico e DEM mais detalhados.
\end{chapterabstract}

\formatchapter

\section{Introduction}
\label{sec:intro}

Digital soil mapping relies on the use of statistical models to produce digital
representations of spatial soil distribution using point soil observations and 
spatially exhaustive environmental covariates \citep{McBratneyEtAl2003, 
ScullEtAl2003, Florinsky2012}. Three important weaknesses in the statistical 
soil distribution modelling approach can be pointed out. First, it requires 
sufficient and appropriately distributed point soil data within the area being 
mapped \citep{CarreEtAl2007a}. Second, the model structure explores only the 
empirical relationship among environmental conditions and soil properties, being
less comprehensive than soil-landscape process models \citep{Grunwald2009}. 
Last, the covariates are only approximations of the true environmental 
conditions that helped shape the soil. They serve only as proxies (surrogates) 
of the current environmental conditions, which in many cases are different from
the past conditions under which pedogenesis took place \citep{HeuvelinkEtAl2001}.
In spite of these weaknesses, digital soil mapping has proven very successful 
in the past decades in producing soil property maps that capture the main 
patterns of soil spatial variation 
\citep{MooreEtAl1993, McBratneyEtAl2000, Grunwald2009}.

More recently, there has been a growing interest in understanding how the 
characteristics of the environmental covariates influence the success of digital
soil mapping -- this study contributes to this effort. It is commonly accepted 
that the more resources are spent on the construction of a covariate and the 
more spatial information it has, the more accurately it describes the 
environmental conditions \citep{HupyEtAl2004, HenglEtAl2013a}. It is also 
generally believed that such \textit{more detailed} covariates will be more 
valuable for digital soil mapping and lead to more accurate soil property 
predictions \citep{CavazziEtAl2013, MaynardEtAl2014}. If these more detailed 
covariates convey more information and represent more adequately the 
environmental conditions -- the drivers of soil forming processes --, then it 
is fair to expect that they improve the accuracy of the resulting soil maps. 
However, some studies have shown the contrary 
\citep{ThompsonEtAl2001, EldeiryEtAl2008, KimEtAl2014}. For example, the window
size at which DEM derivatives are calculated can be more important than the 
spatial resolution of the DEM \citep{Wood1996, ZhuEtAl2008, BehrensEtAl2010a}. 
The uncertainty about the added value of using more detailed covariates is of 
concern for those seeking to use resources efficiently, because using more 
detailed covariates generally increases soil mapping costs \citep{ShiEtAl2012}.

The objective of this study was to evaluate whether investing in more detailed 
environmental covariates improves the accuracy of digital soil maps. The main 
difference of our study to previous ones is that we use a rigorous statistical 
approach to assess the added value of using five more detailed covariates 
simultaneously. We used a case study in Brazil to compare the accuracy of 
digital maps of the clay content, organic carbon content and effective cation 
exchange capacity of the topsoil as obtained from regression kriging on the 
five covariates, whereby each covariate was evaluated on two levels of spatial 
detail.

\section{Material and Methods}
\label{sec:methods}

\subsection{Study area and soil data}
\label{subsec:soil-data}

The study area constitutes a small catchment ($\sim$~2,000~ha) located on the 
southern edge of the plateau of the Paraná Sedimentary Basin, Rio Grande do Sul,
Brazil (\autoref{fig:location}). The climate is classified as Cfa (K\"oppen - 
subtropical humid without a dry season) with mean annual temperature of 
19.3$^{\circ}$C, and mean annual precipitation of 1,708~mm, well distributed 
throughout the year \citep{Maluf2000}. Relief varies between plain (slope 
between 0 and 3\%) and mountainous (slope between 45 and 100\%), and elevations 
range between 140 and 475~m. Geology consists of basic, intermediate and acid 
igneous rocks (rhyolite-rhyodacite and andesite-basalt) of the Cretaceous period,
consolidated sedimentary rocks (aeolian and fluvial sandstones) of the Triassic 
and Jurassic periods, and non-consolidated (fluvial and colluvial deposits) of 
the Quaternary period \citep{GasparettoEtAl1988, MacielFilho1990, Sartori2009}. 
Native semi-deciduous forests occupy more than half of the area, followed by 
native grassland used for animal husbandry, semi-deciduous shrubland, annual 
crop agriculture, forestry (Eucalyptus), urban areas, and artificial water 
bodies \citep{SamuelRosaEtAl2011a}.

 \begin{figure}[!ht]
    \centering
    \begin{minipage}[b]{95mm}
      \subcaption{}
      \label{fig:brazil}
      \centering
      \includegraphics[width=90mm]{chap01FIG1a}
    \end{minipage}
    \begin{minipage}[b]{95mm}
      \subcaption{}
      \label{fig:points}
      \centering
      \includegraphics[width=90mm]{chap01FIG1b}
    \end{minipage}
  \caption{Location of the study area in Santa Maria (a) and spatial 
  distribution of the point soil observations and drainage network (b).}
  \label{fig:location}
 \end{figure}

A dataset containing $n=350$ point soil observations collected between 2004 and 2011 
\citep{PedronEtAl2006b, SamuelRosaEtAl2011a, MiguelEtAl2012, Samuel-RosaEtAl2013}
was used in this study (available at 
\url{https://github.com/samuel-rosa/dnos-sm-rs}). Sampling locations were 
selected purposively and by convenience \citep{Samuel-RosaEtAl2014b}. Three soil
pits were opened within an area of about 100~m$^2$ at most sampling locations to
obtain composite samples of the topsoil for laboratory analysis. Soil was 
collected to a depth of 20~cm or less when soil depth was smaller than 20~cm.
A few observations ($n=10$) correspond to individual samples collected up to 
30~cm. Sampling depth ranges from 2 to 30~cm, with a mean of 17.3~cm. We assumed
that the vertical, horizontal and temporal support differences between soil 
samples is negligible for the purpose of this study.

Three soil properties (fine earth fraction, $<2$~mm) were explored: clay content
(CLAY,~g~kg$^{-1}$), organic carbon content (SOC,~g~kg$^{-1}$), and effective 
cation exchange capacity (ECEC,~mmol~kg$^{-1}$). CLAY was determined by the 
pipette method. SOC was determined using wet digestion. ECEC was calculated as 
the sum of exchangeable bases plus exchangeable acidity. The soil properties 
selected were expected to present different patterns of spatial variation and 
correlation with the most dominant factors of soil formation \citep{Jenny1941} 
in the area: organisms (\textit{O}), relief (\textit{R}), and parent material 
(\textit{P}). CLAY was presumed to have a stronger relation with \textit{P}, 
while SOC was expected to be more correlated with \textit{O}. Because the soils 
of the study area were strongly eroded due to intense agriculture in the 20th 
century, both CLAY and SOC were also expected to be closely related with 
\textit{R}. Finally, ECEC was expected to be strongly correlated with \textit{P}
and \textit{O}, which is supported by its natural relationship with both CLAY 
and SOC.

 \begin{figure}[!ht]
   \centering
    \begin{minipage}[b]{63mm}
      \subcaption{}
      \centering
      \includegraphics[width=63mm]{chap01FIG2a}
    \end{minipage}
    \begin{minipage}[b]{63mm}
      \subcaption{}
      \centering
      \includegraphics[width=63mm]{chap01FIG2d}
    \end{minipage}
    \begin{minipage}[b]{63mm}
      \subcaption{}
      \centering
      \includegraphics[width=63mm]{chap01FIG2b}
    \end{minipage}
    \begin{minipage}[b]{63mm}
      \subcaption{}
      \centering
      \includegraphics[width=63mm]{chap01FIG2e}
    \end{minipage}
    \begin{minipage}[b]{63mm}
      \subcaption{}
      \centering
      \includegraphics[width=63mm]{chap01FIG2c}
    \end{minipage}
    \begin{minipage}[b]{63mm}
      \subcaption{}
      \centering
      \includegraphics[width=63mm]{chap01FIG2f}
    \end{minipage}
  \caption{Histogram, empirical density function, and summary statistics of CLAY
  (a, b), SOC (c, d), and ECEC (e, f) in the original (left) and Box-Cox 
  feature spaces (right).}
  \label{fig:soil-properties}
 \end{figure}

Point soil data, here denoted by $Z(s)$, showed a positive skew 
(\autoref{fig:soil-properties}) and was normalized, $Z'(s)$, using the Box-Cox 
family of power transformations, where $Z'(s) = (Z(s)^{\lambda} - 1) / \lambda$,
if $\lambda > 0$, and $Z'(s) = log(Z(s))$, if $\lambda = 0$ 
\citep{DiggleEtAl2007}. Lambda ($\lambda$) values were selected empirically 
\citep{FoxEtAl2011}. Because the resulting distribution of the back-transform 
(see \autoref{subsec:validation}) has no expectation when $\lambda<0$ 
\citep{RibeiroEtAl2001}, a logarithm transformation ($\lambda=0$) was used when 
a negative $\lambda$ was estimated (SOC and ECEC).

\subsection{Environmental covariates}
\label{subsec:sources}

Five freely available environmental covariates were evaluated in this study, 
each with two levels of spatial detail: area-class soil maps (\texttt{soil}), 
geologic maps (\texttt{geo}), land use maps (\texttt{land}), digital elevation 
models (\texttt{dem}), and satellite images (\texttt{sat}). Each pair was 
composed by covariates that were produced separately from scratch using 
different data sources and/or production methods, thus demanding different 
amounts of resources (time, workforce, budget, technology, etc.). In this study,
the level of spatial detail of an environmental covariate is a function of the 
components of its production process such as the cartographic ratio 
(\texttt{soil}, \texttt{geo} and \texttt{land}), spatial sampling support 
(\texttt{sat}), number and diversity of data sources explored (\texttt{dem}),
and quantity of spatial data used (all five). Thus, the reader should bear in
mind that our definition of spatial detail is broader than spatial resolution
or spatial scale. It should also not be confounded with spatial support 
\citep{WebsterEtAl2007} or thematic detail \citep{Rossiter2000}.

The environmental covariates were transformed to predictor variables that were 
used in the geostatistical modelling. Since the transformation is different for
categorical and continuous covariates, the procedures are explained below for 
each type separately.

\subsubsection*{Categorical predictor variables}
\label{subsubsec:categorical-covars}

Area-class soil maps, geologic maps and land use maps are categorical 
environmental covariates (factors). Mapping units are the $k$ factor levels that
are transformed to as many dummy (indicator, binary) variables as there are 
factor levels, before model calibration. Each dummy variable receives a value 
equal to one (1) when a given class is present, and zero (0) otherwise 
\citep{Everitt2006}. If the number of point soil observations falling inside 
the spatial domain of a mapping unit is too small to accurately estimate a 
regression coefficient (we used a threshold of $n=15$ observations), the mapping
unit is merged with a similar mapping unit prior to calculating dummy variables.
The resulting generalized categorical covariate maps are shown in 
\autoref{fig:cat-covars}. The binary maps are the categorical predictor variables.

\noindent\textit{Soil maps}. The less detailed soil map (\soilOld) was published with a 
\scale{100,000} and has five mapping units \citep{AzolinEtAl1988} 
(\autoref{fig:soil-old}). It was produced using existing soil maps and technical
reports (\scale{750,000}) \citep{Brasil1973}, aerial photographs 
(\scale{60,000}), topographic maps (\scale{50,000}), and sparse point soil 
observations along the road network. The more detailed soil map (\soilNew) was 
prepared with a \scale{25,000} and has eight mapping units \citep{MiguelEtAl2012}
(\autoref{fig:soil-new}). It was produced using high spatial resolution satellite
images (65~cm), existing soil maps and technical reports published with a 
\scale{50,000} \citep{Poelking2007} and 1:25,000 \citep{PedronEtAl2006b}, 
topographic maps (\scale{25,000}), and descriptions from $\sim$~350 point soil 
observations. Five dummy predictor variables were derived from \soilOld{} and 
seven from \soilNew{} (\autoref{tab:soil-covars}).

 \begin{figure}[!ht]
    \centering
    \begin{minipage}[b]{63mm}
       \subcaption{Cartographic scale: 1:100,000}
       \label{fig:soil-old}
       \centering
       \includegraphics[width=60mm]{chap01FIG3a}
    \end{minipage}
    \begin{minipage}[b]{63mm}
       \subcaption{Cartographic scale: 1:25,000}
       \label{fig:soil-new}
       \centering
       \includegraphics[width=60mm]{chap01FIG3d}
    \end{minipage}    
    \begin{minipage}[b]{63mm}
       \subcaption{Cartographic scale: 1:50,000}
       \label{fig:geo-old}
       \centering
       \includegraphics[width=60mm]{chap01FIG3b}
    \end{minipage}
    \begin{minipage}[b]{63mm}
       \subcaption{Cartographic scale: 1:25,000}
       \label{fig:geo-new}
       \centering
       \includegraphics[width=60mm]{chap01FIG3e}
    \end{minipage}
    \begin{minipage}[b]{63mm}
       \subcaption{Cartographic scale: 1:500,000}
       \label{fig:land-old}
       \centering
       \includegraphics[width=60mm]{chap01FIG3c}
    \end{minipage}
    \begin{minipage}[b]{63mm}
       \subcaption{Cartographic scale: 1:2,000}
       \label{fig:land-new}
       \centering
       \includegraphics[width=60mm]{chap01FIG3f}
    \end{minipage}
   \caption{Area-class soil maps (a, b), geologic maps (c, d), and land use 
   maps (e, f) compared in our study. The less detailed version is displayed 
   at the left, while the more detailed version is shown on the right. Legend 
   abbreviations and derived dummy variables are described in Tables 
   \ref{tab:soil-covars}--\ref{tab:land-covars}.}
  \label{fig:cat-covars}
\end{figure}

\noindent\textit{Geologic maps}. The less detailed geologic map (\geoOld) was produced 
using topographic maps with \scale{50,000} \citep{GasparettoEtAl1988} 
(\autoref{fig:geo-old}). The more detailed geologic map (\geoNew) was produced 
using topographic maps with \scale{25,000}, and includes the location of 
overlaying Quaternary sedimentary deposits \citep{MacielFilho1990} 
(\autoref{fig:geo-new}). \geoNew{} did not cover a small part in the North of 
the study area, where \geoOld{} was used instead (this strategy was approved by
experts on the local geology). The mapping unit of both geologic maps depicting
the Caturrita Formation was used indirectly by deriving dummy predictor 
variables from all other individual mapping units. Three dummy predictor 
variables were derived from \geoOld{} and four from \geoNew{} 
(\autoref{tab:geology-covars}).

\noindent\textit{Land use maps}. The less detailed land use map (\landOld) was 
produced by manually digitizing land use data included in topographic maps with a 
\scale{25,000} \citep{DSG1980, DSG1992, DSG1992a} (\autoref{fig:land-old}). 
The more detailed land use map (\landNew) was prepared (\scale{2,000}) by 
manual digitization using 65~cm spatial resolution satellite images covering 
the years 2008 and 2009 \citep{SamuelRosaEtAl2011a} (\autoref{fig:land-new}). 
Mapping units depicting human settlements and water bodies ($n=0$) were not 
masked out from the prediction grid and were merged with other mapping units to
derive dummy predictor variables. Five dummy predictor variables were derived 
from \landNew{} and two from \landOld{} (\autoref{tab:land-covars}).

\subsubsection*{Continuous predictor variables}
\label{subsubsec:continuous-covars}

The less detailed DEM (\demOld) is the hole-filled SRTM DEM version~4 
\citep{JarvisEtAl2008} (\autoref{fig:dem-old}). The spatial sampling support of 
the SRTM DEM is 1~arc-second ($\sim$~30~m), but elevation data were aggregated 
to 3~arc-seconds ($\sim$~90~m) for public release in regions outside the United 
States \citep{ReuterEtAl2007}. The more detailed DEM (\demNew) was produced by 
interpolating contour lines with vertical spacing of 10~m along with data about 
the drainage network, lakes and peaks digitized from topographic maps with 
\scale{25,000} (\autoref{fig:dem-new}). Interpolation to 5-m pixel size was 
performed using a hydrologically correct algorithm implemented in \href{http://resources.arcgis.com/en/help/main/10.1/index.html#/How_Topo_to_Raster_works/009z0000007m000000/}{ArcGIS\textregistered{}} software by ESRI \citep{Hutchinson1989}. Contour line 
artefacts were minimized using a seven by seven low-pass filter 
(\grass{r.neighbors}). The window size was chosen such that the smoothed DEM 
best matched the original contour map while also respecting the original 
drainage network pattern.

\ctable[
   caption  = {Description of the $p=12$ dummy predictor variables derived from the two soil maps.},
   label    = tab:soil-covars,
   notespar,
   pos      = !ht,
%    doinside = \scriptsize\setstretch{1.1},
   doinside = \scriptsize
   ]{l p{0.85\textwidth} l}{
   \tnote[a]{Soil classification according to the old Brazilian classification (only for \citet{AzolinEtAl1988}), the current Brazilian classification \citep{SantosEtAl2013a}, and the international classification \citep{IUSSWorkingGroupWRB2007}.}
   \tnote[b]{Minimum Legible Delineation calculated following \citet{Rossiter2000}.}
   }{                                                                                                                    \FL
   Code                & Mapping unit(s) included and Description\tmark[a,b]                                             \ML
   \multicolumn{2}{l}{Source: \citet{AzolinEtAl1988}. Cartographic scale: 1:100,000. Minimum Legible Delineation: 40~ha.} \NN
   \texttt{SOIL\_100b} & \textit{Re4}. Shallow soils with low to high base saturation covering mountainous terrain (Solo Litólico Eutrófico/Distrófico relevo montanhoso; Neossolo Litólico Distrófico/Eutrófico; Distric/Eutric Leptosol). \NN 
   \texttt{SOIL\_100c} & \textit{Re-C-Co}. Shallow soils with high base saturation located in strongly sloping terrain (Solo Litólico Eutrófico relevo forte ondulado; Neossolo Litólico Eutrófico; Eutric Leptosol), low weathered soils (Cambissolo Eutrófico; Cambissolo Háplico Eutrófico; Eutric Cambisol), and colluvial deposits. \NN
   \texttt{SOIL\_100d} & \textit{TBa-Rd}. Deep, well-structured, low base saturation soils (Terra Bruna Estruturada álica; Nitossolo; Nitisol), and shallow soils (Solo Litólico; Neossolo Litólico; Leptosol). \NN
   \texttt{SOIL\_100e} & \textit{Rd1} and \textit{Re4}. \textit{Rd1} is composed mainly by shallow soils with low to high base saturation (Solo Litólico Distrófico/Eutrófico; Neossolo Litólico Distrófico/Eutrófico; Distric/Eutric Leptosol) located in slopping terrain. This dummy predictor variable is composed by shallow soils in both sloping and mountainous terrain. \NN
   \texttt{SOIL\_100f} & \textit{TBa-Rd} and \textit{C1}. \textit{C1} is composed by low weathered soils developed in lower landscape positions, close to drainage channels (Cambissolo Eutrófico; Cambissolo Eutrófico; Eutric Cambisol). This dummy predictor variable includes the best soil mapping units for crop agriculture among those identified in the soil survey. \NN
   &                                                                                                                     \NN
   \multicolumn{2}{l}{Source: \citet{MiguelEtAl2012}. Cartographic scale: 1:25,000. Minimum Legible Delineation: 2.5~ha.} \NN
   \texttt{SOIL\_25a}  & \textit{PBAC}. Moderately deep soils derived from sedimentary rocks, with abrupt textural change and low base saturation (Argissolo Bruno-Acinzentado; Alisol). \NN
   \texttt{SOIL\_25b}  & \textit{PV}. Deep soils derived from igneous rocks, with moderate textural gradient, and low base saturation (Argissolo Vermelho; Acrisol). \NN
   \texttt{SOIL\_25c}  & \textit{C-R}. Low weathered soils (Cambissolo; Cambisol) and shallow soils with low to high base saturation (Neossolo Litólico/Regolítico Eutrófico/Distrófico; Eutric/Distric Leptosol/Regosol). \NN
   \texttt{SOIL\_25d}  & \textit{RL}. Shallow soils with low to high base saturation (Neossolo Litólico Eutrófico/Distrófico; Eutric/Distric Leptosol). \NN
   \texttt{SOIL\_25h}  & \textit{PBAC}, \textit{PV} and \textit{SX}. \textit{SX} is composed by moderately deep soils derived from sedimentary rocks, with abrupt textural change, low base saturation, and which are saturated with water for long periods of the year (Planossolo Háplico; Planosol). This dummy predictor variable includes the best soil mapping units for crop agriculture among those identified in the soil survey. \NN
   \texttt{SOIL\_25i}  & \textit{RL}, \textit{RL-RR} and \textit{RR}. This dummy predictor variable includes all three mapping units composed mainly by shallow soils (Neossolo Litólico and Neossolo Regolítico; Leptosol and Regosol). \NN
   \texttt{SOIL\_25j}  & \textit{PV}, \textit{RL}, \textit{RL-RR} and \textit{C-R}. This dummy predictor variable includes all four mapping units composed mainly by soils derived from igneous rocks. \LL
   }

\ctable[
   caption  = {Description of the $p=7$ dummy predictor variables derived from the two geologic maps.},
   label    = tab:geology-covars,
   notespar,
   pos      = !ht,
   %    doinside = \scriptsize\setstretch{1.1}
   doinside = \scriptsize
   ]{l p{0.85\textwidth} l}{
   \tnote[a]{Minimum Legible Delineation calculated following \citet{Rossiter2000}.}
   }{                                                                                                       \FL
   Code                & Mapping unit(s) included and Description\tmark[a]                          \ML
   \multicolumn{2}{l}{Source: \citet{GasparettoEtAl1988}. Cartographic scale: 1:50,000. Minimum Legible Delineation: 10~ha.} \NN
   \texttt{GEO\_50a}   & \textit{SG-I}. Inferior Sequence of the Serra Geral Formation. Composed mainly by basic igneous rocks (tholeiitic basalt and andesite). It is likely to be related with high CLAY and ECEC. \NN
   \texttt{GEO\_50b}   & \textit{SG-S}. Superior Sequence of the Serra Geral Formation. Composed mainly by acid igneous rocks (granophyric rhyolite and rhyodacite). It is likely to be related with moderate to high CLAY and ECEC. \NN
   \texttt{GEO\_50c}   & \textit{BT}. Botucatu Formation. Composed mainly by aeolian sandstones. It is likely to be related with low CLAY and ECEC. \NN
   Other               & \textit{CT} depicts the Caturrita Formation, which is composed mainly by fluvial sandstones. \NN
    & \NN
   \multicolumn{2}{l}{Source: \citet{MacielFilho1990}. Cartographic scale: 1:25,000. Minimum Legible Delineation: 2.5~ha.} \NN
   \texttt{GEO\_25a}   & \textit{SG-I}. Inferior Sequence of the Serra Geral Formation. \NN
   \texttt{GEO\_25b}   & \textit{SG-S}. Superior Sequence of the Serra Geral Formation. \NN
   \texttt{GEO\_25c}   & \textit{BT}. Botucatu Formation. \NN
   \texttt{GEO\_25d}   & \textit{QD}. Quaternary deposits of fluvial, alluvial, and colluvial origin. It can help explaining the low CLAY of soils supposedly derived from igneous rocks. \NN
   Other               & \textit{CT} depicts the Caturrita Formation. \LL
   }

\ctable[
   caption  = {Description of the $p=7$ dummy predictor variables derived from the two land use maps.},
   label    = tab:land-covars,
   notespar,
   pos      = !ht,
   %    doinside = \scriptsize\setstretch{1.1}
   doinside = \scriptsize
   ]{l p{0.85\textwidth} l}{
   \tnote[a]{Minimum Legible Delineation calculated following \citet{Rossiter2000}.}
   }{                                                                                                                                \FL
   Code             & Mapping unit(s) included and Description\tmark[a]                                                                       \ML
   \multicolumn{2}{l}{Source: \citet{DSG1980, DSG1992, DSG1992a}. Cartographic scale: 1:25,000. Minimum Legible Delineation: 2.5~ha.} \NN
   \texttt{LU1980a} & \textit{FS}. Native forest, which is likely to have soils with higher fertility.                              \NN
   \texttt{LU1980b} & \textit{H}. Animal husbandry, which is likely to have a soil fertility status lower than native forests and is the second most important land use in the area. \NN
   Other            & Plantation forests (\textit{PF}) and human settlements (\textit{S}).                                           \NN
    & \NN
   \multicolumn{2}{l}{Source: \citet{SamuelRosaEtAl2011a}. Cartographic scale: 1:2,000. Minimum Legible Delineation: 100~m$^2$.}      \NN
   \texttt{LU2009a} & \textit{FS}. Native forest.                                                                                    \NN
   \texttt{LU2009b} & \textit{SS}. Shrubland, which is likely to have SOC and ECEC level above those found in areas used with annual crop agriculture and animal husbandry, but lower than in native forests. \NN
   \texttt{LU2009c} & \textit{H}. Animal husbandry. \NN
   \texttt{LU2009d} & \textit{AA}. Annual crop agriculture, which is likely to have the lowest soil fertility due to the usually poor management practices employed. \NN
   \texttt{LUdiff}  & Land use difference between 1980 and 2009. It can be useful to explain, for example, low SOC in forest soils due to previous use with crop agriculture or animal husbandry. \NN
   Other            & Plantation forests (\textit{PF}), human settlements (\textit{S}), and other land uses (\textit{O}), comprising natural and artificial water bodies. \LL
   }


\begin{figure}[!ht]
  \centering
    \begin{minipage}[b]{63mm}
      \subcaption{Spatial resolution: 90~m}
      \label{fig:dem-old}
      \centering
      \includegraphics[width=60mm]{chap01FIG4a}
    \end{minipage}
    \begin{minipage}[b]{63mm}
      \subcaption{Spatial resolution: 30~m}
      \label{fig:sat-old}
      \centering
      \includegraphics[width=60mm]{chap01FIG4b}
    \end{minipage}
    \begin{minipage}[b]{63mm}
      \subcaption{Vertical spacing of contours: 10~m}
      \label{fig:dem-new}
      \centering
      \includegraphics[width=60mm]{chap01FIG4c}
    \end{minipage}
    \begin{minipage}[b]{63mm}
      \subcaption{Spatial resolution: 5~m}
      \label{fig:sat-new}
      \centering
      \includegraphics[width=60mm]{chap01FIG4d}
    \end{minipage}
  \caption{Digital elevation models (a, c) and satellite images, depicted using 
  the normalized difference vegetation index (b, d), compared in our study. The
  less detailed version is displayed at the top, while the more detailed version
  is shown on the bottom.}
  \label{fig:con-covars}
\end{figure}

Eight DEM derivatives were calculated: elevation (\elev), slope (\slp), aspect 
(\asp), northernness (\nor), flow accumulation (\acc), topographic wetness index
(\twi), stream power index (\spi), and topographic position index (\tpi). \slp{}
and \asp{} were calculated using \grass{r.param.scale} with seven window sizes 
(sampling support, analysis scale): 3, 7, 15, 31, 63, 127, and 255. \asp{} was 
scaled to the standard 0-360$^\circ$ range and orientation, and was transformed
to \nor{} using $\texttt{NOR} = abs(180^\circ - \texttt{ASP})$. \twi{} and \spi{}
were calculated using \slp{} calculated with different window sizes, and \acc{} 
calculated using \grass{r.watershed}. \tpi{} was calculated using 
\saga{ta\_morphometry} with the same seven window sizes. The combination of DEM
derivatives (\elev, \slp, \nor, \twi, \spi, and \tpi) and window sizes yielded 
$p=36$ continuous predictor variables from each DEM.

The less detailed satellite image was acquired by the Landsat-5 Thematic Mapper
on December 26, 2010 (available at Instituto Nacional de Pesquisas Espaciais - 
Divisão de Geração de Imagens -- \href{http://www.dgi.inpe.br/CDSR/}{INPE-DGI})
(\autoref{fig:sat-old}). It has 8~bits radiometric resolution and $\sim$~30~m 
spatial resolution. Spectral bands were orthorectified (Geomatica\textregistered{}
OrthoEngine\textregistered{}) and radiometrically corrected (\grass{i.landsat.toar}).
The more detailed satellite image comes from the RapidEye constellation 
(available at Ministério do Meio Ambiente -- 
\href{http://geocatalogo.ibama.gov.br/}{MMA}) (\autoref{fig:sat-new}). It was 
acquired on November 16, 2012, has 16~bits radiometric resolution, 6.5~m spatial
resolution, and was orthorectified to 5~m spatial resolution. Both images were 
atmospherically (6S atmospheric model \citep{VermoteEtAl1997}, \grass{i.atcorr})
and topographically corrected (\grass{i.topo.corr}). Derived predictor variables
are the spectral bands (except the thermal band) and vegetation indices 
(normalized difference vegetation index - NDVI, and soil-adjusted vegetation 
index - SAVI). Eight continuous predictor variables were derived from the 
Landsat-5~TM image and nine from the RapidEye image.

\subsubsection*{Additional processing}
\label{subsubsec:sources-processing}

Soil maps, geologic maps, land use maps, and satellite images were registered 
with the prediction grid (5-m pixel size) using nearest neighbour resampling. 
\demOld{} was registered using cubic resampling \citep{Samuel-RosaEtAl2013c}. 
Systematic positional errors were corrected using affine transformation 
\citep{Samuel-RosaEtAl2014}.

\subsection{Linear mixed model of spatial variation}
\label{subsec:lmm}

We model each of the soil properties of interest as the outcome of a spatial 
stochastic process. The model is composed of fixed and random effects 
\citep{HeuvelinkEtAl2001, LarkEtAl2006}. We use the point soil observations and 
spatially exhaustive predictor variables to calibrate the model and predict the
outcome of the spatial stochastic process at unobserved locations. This fixed 
effect (deterministic trend), $\mu(\textbf{s})$, describes that part of the 
spatial variation of the soil property that is explained by the covariates. We 
assume here that is a linear function of the predictor variables. The random 
effect (stochastic residuals, latent variables), $\varepsilon(\textbf{s})$, 
describes that part of the spatial variation that cannot be explained by the 
covariates \citep{Cressie1993}. It is represented by a spatially correlated, 
Gaussian distributed random variable, that is assumed stationary in the mean 
and covariance. Thus, the linear mixed model of spatial variation that we 
employed is given by

\begin{equation}\label{eq:lmm}
 Z'(\textbf{s}) = \mu(\textbf{s}) + \varepsilon(\textbf{s}) = \sum_{j=0}^{p} 
 \beta_{j}\cdot X_{j}(\textbf{s}) + \varepsilon(\textbf{s}),
\end{equation}

\noindent{where $Z'(\textbf{s})$ is the soil property after Box-Cox 
transformation, $\mu(\textbf{s})$ and $\varepsilon(\textbf{s})$ are defined as 
above, $\beta_{j}$ are the regression model coefficients, and $X_{j}(\textbf{s})$
is the regression model matrix, with $j=0, 1, 2, \ldots, p$, $p$ being the number
of predictor variables. Variable $X_{0}(\textbf{s})$ is taken as unity so that 
$\beta_{0}$ is the intercept.}

\subsubsection*{Model selection}

We calibrated $m=2^5=32$ multiple linear regression models for each soil property
(fitted using ordinary least squares, OLS) to model the deterministic trend for 
each combination of the five covariates (recall from \autoref{sec:intro} that 
each covariate is available at two levels of spatial detail, hence $2^5$ 
combinations). The number of predictor variables used to calibrate each model 
varied among combinations between $p=52$ and $p=62$, because more detailed 
covariates enabled the derivation of a larger number of predictor variables 
(except the DEM). Backward VIF (variance inflation factor) selection followed 
by stepwise AIC (Akaike's Information Criterion) selection were used to select 
predictor variables to enter the models \citep{Samuel-RosaEtAl2014c, VenablesEtAl2002}.

The $m=32$ multiple linear regression models calibrated for each soil property 
were ranked using the ratio between the regression sum of squares and the total 
sum of squares. Because stepwise regression results in biased models 
\citep{Harrell2001a}, the ratio of sum of squares was adjusted (${R}^{2}_{adj}$)
using the number of predictor variables initially offered to enter the model 
instead of the reduced number of predictor variables that entered the model. 
Next, the five environmental covariates were ranked based on how their level of 
spatial detail related with the calibration of models with improved predictive 
performance. The relation between the level of spatial detail of the covariates 
and model performance was evaluated using a graphical output called model series
plot (\Rpackage{pedometrics}, \citet{Samuel-RosaEtAl2014c}). Pedological 
evaluation of predictor variables included in the models was omitted because 
this was beyond our objectives.

The multiple linear regression model calibrated using only the less detailed 
environmental covariates, which we call the \textit{baseline} model, and the 
multiple linear regression model with the highest ${R}^{2}_{adj}$, which we call
the \textit{best performing} model, were extended to linear mixed models of 
spatial variation (\autoref{eq:lmm}) for each soil property. Estimation of the 
parameters of the linear mixed models was performed using residual (restricted,
marginal) maximum likelihood (REML) \citep{RibeiroEtAl2001, LarkEtAl2004}. The 
spatial correlation function adopted was the exponential function (this is 
equivalent to the Matérn correlation function with smoothness parameter 
$\nu=0.5$ \citep{Stein1999}).

\subsubsection*{Model validation}
\label{subsec:validation}

Only the \textit{baseline} and \textit{best performing} multiple linear 
regression and linear mixed models calibrated for each soil property were 
validated. Model validation was performed using leave-one-out cross-validation 
(LOO\-/CV) \citep{BrusEtAl2011}. All model parameters were re-estimated at each
LOO\-/CV run to reduce bias \citep{LaslettEtAl1987}. LOO\-/CV predicted values 
were back-transformed from the Box-Cox space to the original space of soil 
properties using stochastic simulation \citep{ChristensenEtAl2001}:

\begin{enumerate}
 \item each predicted value and associated prediction error variance were used 
 to simulate $n = 20,000$ values from a Gaussian distribution;
 \item simulated values were back-transformed using 
 $Z(s) = (Z'(s) \times \lambda + 1)^{1 / \lambda}$, if $\lambda > 0$, and 
 $Z(s) = exp(Z'(s))$, if $\lambda = 0$;
 \item the mean and variance of back-transformed simulated values were used as 
 the predicted value and prediction error variance in the original space of 
 soil properties.
\end{enumerate}

Five error statistics were computed from the LOO\-/CV results 
\citep{JanssenEtAl1995, KempenEtAl2010, BrusEtAl2011}. The mean error 
(\textit{ME}), which measures the prediction bias, the mean absolute error 
(\textit{MAE}) and the root mean squared error (\textit{RMSE}), which measure 
the prediction accuracy, the scaled root mean squared error (\textit{SRMSE}, 
also known as mean squared deviation ratio), which measures how well the 
prediction error variance matches the squared differences between predicted and 
observed soil property, where $\textit{SRMSE}>1$ indicates under-estimation, 
while $\textit{SRMSE}<1$ indicates over-estimation, and the amount of variance 
explained (\textit{AVE}, also known as coefficient of determination or ratio of
scatter), which measures the fraction of the overall spread of observed values 
that is explained by the model. The AVE ranges from 0 to 100, where 
$\textit{AVE}=100$ is the optimal value.

\subsubsection*{Spatial prediction}
\label{subsec:prediction}

Only the \textit{baseline} and \textit{best performing} linear mixed models 
calibrated for each soil property were used for spatial prediction. Spatial 
predictions at a fine grid of $\sim$~800,000 point locations were made in the 
Box-Cox space using the best linear unbiased predictor (BLUP) with the 
empirical estimates of the random effects (EBLUP) \citep{LarkEtAl2006}. EBLUP 
with a fixed effect model is conceptually equivalent to kriging with external 
drift and regression kriging, and mathematically equivalent to kriging with 
external drift and universal kriging. Point predicted values and prediction 
error variances were back-transformed to the original soil property space using
stochastic simulation as described above (\autoref{subsec:validation}).

\section{Results}
\label{sec:results}

\subsection{Model series plots}

The model series plot is a graphical description of the relation between the 
prediction accuracy of multiple linear regression models and the environmental 
covariates used to calibrate them (\autoref{fig:model-series}). The magnitude 
of improvement in prediction accuracy is depicted in the bottom panel by the 
${R}^{2}_{adj}$. The top panel is interpreted both horizontally and vertically.
In the vertical direction we identify which version of each covariate was used 
to calibrate a given model. The less and the more detailed versions are 
identified by the yellow (bright) and green (dark) colours, respectively. The 
\textit{baseline} model is identified by the column containing only yellow 
cells, while the column with only green cells represents the model calibrated 
using only the more detailed version of each covariate, which we call the 
\textit{most detailed} model. The first important results that we obtain from 
the model series plots is that a) the \textit{baseline} model is not the model
with the lowest ${R}^{2}_{adj}$, which we call the \textit{poorest performing} 
model, and b) the \textit{most detailed} model is not the \textit{best 
performing} model.

\begin{figure}[!ht]
  \centering
    \begin{minipage}[b]{\textwidth}
      \subcaption{}
      \includegraphics[width=\textwidth]{chap01FIG5a}
    \end{minipage}
    \begin{minipage}[b]{\textwidth}
      \subcaption{}
      \includegraphics[width=\textwidth]{chap01FIG5b}
    \end{minipage}
    \begin{minipage}[b]{\textwidth}
      \subcaption{}
      \includegraphics[width=\textwidth]{chap01FIG5c}
    \end{minipage}
  \caption{Model series plots for CLAY (a), SOC (b), and ECEC (c). The less and 
  more detailed version of each environmental covariate are identified by the 
  yellow (bright) and green (dark) colours, respectively. Multiple linear 
  regression models were ranked by their ${R}^{2}_{adj}$. Triangles show the 
  mean ranking of the more detailed covariates (i.e. centre of green cells).}
  \label{fig:model-series}
\end{figure}

The row-wise analysis of the model series plots shows if a model calibrated with
the more detailed version of a given environmental covariate has a higher 
prediction accuracy. This information is retrieved by looking at the row-wise 
distribution of green cells -- these cells represent the $m=16$ models 
calibrated using the more detailed version of a given covariate, irrespective 
of the version of the other covariates. The more concentrated the green cells 
are in the right half of the plot, the larger the relative benefit of using the
more detailed version of that environmental covariate. For example, the top row
of the second model series plot shows the SOC models calibrated using the two 
versions of the land use map (\texttt{land}). All green cells are on the right
half of the plot between rankings 1 and 16 (see the x axis). The four lower rows
show that the green cells of the other four covariates are distributed along the
entire ranking range (from 1 to 32). This means that the relative benefit of 
calibrating a SOC model with a more detailed land use map is larger compared to 
that of using a more detailed version of the other covariates.

The centre of the row-wise distribution of the green cells for each 
environmental covariate, calculated as the mean ranking, is represented by the 
triangles. The mean ranking quantifies the relative benefit of using a more 
detailed version of each covariate. For example, the mean ranking of the SOC 
models calibrated using the more detailed land use map is about 8 (top row), 
while the mean ranking of the models calibrated using the more detailed 
satellite image (\texttt{sat}) is close to 20 (bottom row). Using the more 
detailed DEM (\texttt{dem}) is almost as beneficial as using the more detailed 
geologic map (\texttt{geo}) -- the mean ranking of the SOC models calibrated 
using the more detailed version of these two covariates is about 15-16 (second 
and third rows). Using the more detailed version of the soil map (\texttt{soil},
fourth row) is not as beneficial as using \texttt{land}, \texttt{geo} or 
\texttt{dem}, but more beneficial than using \texttt{sat}. Because the 
covariates were ranked based on the mean rankings, the covariate displayed in 
the top row of each model series plot is the one which resulted in the largest
improvement of the prediction accuracy when the more detailed version was used 
to calibrate the model -- for SOC this is the land use map.

For CLAY, calibrating the models with the more detailed soil map resulted in 
the largest improvement of the prediction accuracy relative to the other 
environmental covariates. The DEM was the second most beneficial covariate (mean
ranking of 15), but the benefit of using its more detailed version was similar 
to that of using the more detailed version of any other covariate (mean rankings
between 17 and 18). Nine models had a poorer prediction performance than the 
baseline model, ranked 27th, the poorest performing model being that calibrated 
with the more detailed land use map and satellite image. Despite these patterns,
calibrating CLAY models with the more detailed version of any covariate resulted
in a small improvement of the prediction accuracy, as evidenced by the small 
increases of the ${R}^{2}_{adj}$. The difference between the poorest and best
performing models is less than 3~percentage points (pp). In comparison, for 
SOC, by simply using the more detailed land use map we already obtained a model
ranked 9th, an increase of 8~pp in ${R}^{2}_{adj}$ compared to the baseline 
model, ranked 24th.

The same general pattern observed for SOC models was observed for ECEC models -- 
the more detailed land use map results in the largest improvement of the 
prediction accuracy. The main difference is that calibrating the models with 
the more detailed geologic map was slightly more beneficial for ECEC (mean 
ranking of 12) than for SOC (mean ranking of 14). The poorest performing ECEC 
model was that calibrated with the more detailed satellite image. Using only 
the more detailed land use map resulted in an improvement of 6~pp in 
${R}^{2}_{adj}$ (model ranked 7th), differing from the best performing model by 
only 2~pp. Using the more detailed version of all environmental covariates 
except the soil map or satellite image resulted in increases of about 6 and 
7~pp in ${R}^{2}_{adj}$, respectively. The baseline model was ranked as 28th, 
which is a higher ranking than the models calibrated with all possible 
combinations of the more detailed satellite image and the more detailed
soil map and/or DEM.

The patterns observed in the model series plots resulted from the change 
(increase or decrease) of the importance of each environmental covariate on 
explaining the variance when the more detailed version was used 
(\autoref{tab:drop}). We used the \textit{baseline} and \textit{most detailed} 
models to quantify this change. Each model was refitted dropping one covariate 
at a time. The difference $\Delta$ between the ${R}^{2}_{adj}$ of the model 
calibrated with all five $q$ covariates (${R}^{2}_{adj}{}_{q=5}$) and the model 
calibrated without the $q$-th covariate ($R^{2}_{adj}{}_{q=5-1}$) was calculated.
The more positive $\Delta{R}^{2}_{adj}$ becomes, the more beneficial the more 
detailed version of the $q$-th covariate is for improving prediction accuracy.
For CLAY, \texttt{dem} and \texttt{land} were the most important covariates in 
the baseline model, while \texttt{geo} was the least important. The importance 
of \texttt{soil} and \texttt{geo} increased when their more detailed version 
was used (change of $+0.013$~pp for both), while \texttt{sat}, \texttt{land} 
and \texttt{dem} became less important. For SOC and ECEC, \texttt{land} was not 
the most important covariate in the baseline model. But it was the covariate 
whose importance had the largest positive shift when the more detailed version 
was used ($+0.085$~pp for SOC and $+0.045$~pp for ECEC). \texttt{sat} became 
less important when the more detailed version was used -- see its low ranking
in all model series plots. The increase of the importance of \texttt{geo} was 
larger for ECEC ($+0.026$) than for SOC ($+0.013$) -- see the difference in the 
mean ranking of \texttt{geo} in the SOC (14) and ECEC (12) model series plots.

\ctable[
 caption  = {The importance of each covariate$^a$ ($\Delta{R}^{2}_{adj}{}^b$) in the models calibrated with 
 their less and more spatially detailed version.},
 label    = tab:chap05-drop,
 pos      = !h,
 % doinside = \scriptsize\setstretch{1.1}
 doinside = \scriptsize
 ]{lrrcrrcrr}{
 \tnote[a]{Covariate: \texttt{soil} - soil map, \texttt{land} - land use map, \texttt{geo} - geologic map, 
 \texttt{sat} - satellite image, and \texttt{dem} - digital elevation model.}
 \tnote[b]{$\Delta{R}^{2}_{adj} = {R}^{2}_{adj}{}_{q=5} - R^{2}_{adj}{}_{q=5-1}$, where $q$ is the number of 
 covariates included in the model. Negative values result from adjusting the $R^{2}$ using the number of 
 predictor variables initially offered to enter the model instead of the reduced number of predictor variables 
 that entered the model.}
 }{\FL
   \multicolumn{1}{l}{Covariate}&\multicolumn{2}{c}{CLAY}&\multicolumn{1}{c}{}&\multicolumn{2}{c}{SOC}&\multicolumn{1}{c}{}&\multicolumn{2}{c}{ECEC}\NN
   \cline{2-3} \cline{5-6} \cline{8-9}
   \multicolumn{1}{l}{}&\multicolumn{1}{c}{Less}&\multicolumn{1}{c}{More}&\multicolumn{1}{c}{}&\multicolumn{1}{c}{Less}&\multicolumn{1}{c}{More}&\multicolumn{1}{c}{}&\multicolumn{1}{c}{Less}&\multicolumn{1}{c}{More}\ML
   \texttt{soil} &$-0.009$&$ 0.004$&&$-0.006$&$-0.008$&&$ 0.011$&$-0.003$\NN
   \texttt{land} &$ 0.003$&$-0.002$&&$ 0.003$&$ 0.088$&&$-0.004$&$ 0.041$\NN
   \texttt{geo}  &$-0.019$&$-0.006$&&$-0.005$&$ 0.008$&&$ 0.007$&$ 0.033$\NN
   \texttt{sat}  &$-0.010$&$-0.016$&&$ 0.018$&$-0.014$&&$ 0.011$&$-0.029$\NN
   \texttt{dem}  &$ 0.030$&$ 0.001$&&$-0.009$&$ 0.016$&&$-0.035$&$-0.041$\LL
}


\subsection{REML fit of the variogram model}

The small improvement in the prediction accuracy of the CLAY linear mixed model 
calibrated with the more detailed environmental covariates is evidenced by 
\autoref{fig:lmm}. The shape of the experimental variogram is very similar 
for both baseline and best performing linear mixed models, which is also true 
for SOC and ECEC. However, the sill variance had a very small reduction for 
CLAY compared to SOC and ECEC. The last two showed a more considerable 
improvement in prediction accuracy. It can also be seen that the number of 
point observations separated by short distances is very small, reducing the 
accuracy of the estimate of the nugget variance. The result is that the 
estimated nugget variance changes rather erratically from the baseline to the 
best performing models, decreasing for CLAY and SOC, and increasing for ECEC.

\begin{figure}[!ht]
  \begin{center}
    \begin{minipage}[b]{90mm}
      \subcaption{}
      \includegraphics[width=90mm]{chap01FIG6a} 
    \end{minipage}
    \begin{minipage}[b]{90mm}
      \subcaption{}
      \includegraphics[width=90mm]{chap01FIG6b}
    \end{minipage}
    \begin{minipage}[b]{90mm}
      \subcaption{}
      \includegraphics[width=90mm]{chap01FIG6c}
    \end{minipage}
    \caption{Experimental variogram (dots) and REML fit of the linear mixed 
    models (line) for CLAY (a), SOC (b), and ECEC (c). Left -- baseline model.
    Right -- best performing model.}
    \label{fig:lmm}
  \end{center}
\end{figure}

\subsection{Validation}

The LOO\-/CV results indicate that the linear mixed models for CLAY are slightly
positively biased, while those for SOC and ECEC are slightly negatively biased 
(\autoref{tab:cv-stats}). For both CLAY and ECEC, the \textit{MAE} shows that 
these models are more accurate than the multiple linear regression models, 
suggesting that the kriging step improves the prediction accuracy.

\input{chap/tab/chap01TAB5.tex}

Overall, all models had a moderate to poor prediction performance. The errors 
are, in absolute values, somewhat large, mainly for ECEC. The best \textit{AVE}
are about 60\% for CLAY, 50\% for SOC and 40\% for ECEC. In general, the 
prediction error variance was under-estimated by the linear mixed models and 
over-estimated by the multiple regression models. The best estimates of the 
prediction error variance were obtained by both CLAY linear mixed models, and 
the ECEC baseline linear regression model.

For CLAY, the increase in the \textit{AVE} was larger when including a kriging
step ($\Delta\textit{AVE}=3.9$~pp) than when using more detailed environmental
covariates ($\Delta\textit{AVE}=1.6$~pp). In the case of SOC, including a 
kriging step reduced the \textit{AVE} by 3.2~pp, and for ECEC, both strategies 
increased the \textit{AVE} (\autoref{tab:cv-stats}).

\subsection{Spatial prediction}

Both baseline and best performing linear mixed models captured the same overall
pattern of spatial variation of the soil properties (\autoref{fig:kriging}). 
The main difference is that the spatial patterns of the different environmental
covariates used to calibrate each model produced different features in the 
prediction maps. For example, the CLAY map produced by the best performing 
model (\autoref{fig:clay-best-pred}) displays abrupt changes in the predicted
values in the north-northeast due to the use of the more detailed soil map. 
Strongly-marked features following the stream network obtained through the use 
of the more detailed DEM are also observed (Figures \ref{fig:clay-best-pred} 
and \ref{fig:clay-best-var}).

SOC maps (Figures \ref{fig:soc-best-pred} and \ref{fig:soc-best-var}) show 
peculiar features in the central part of the study area, where predictions 
reached values as high as 507~g~kg$^{-1}$, while the maximum value in the 
calibration data is 163~g~kg$^{-1}$. The extremely high predicted values 
resulted from the inclusion of the topographic position index derived from 
the more detailed DEM, using a window size of 15~x~15~pixels 
(\texttt{TPI\_10\_15}) to model the deterministic trend. \texttt{TPI\_10\_15} 
values in the point calibration data range from -7 to 6~m, while in the central
part of the study area they range from 12 to 31~m. Thus, feature-space 
extrapolation explains the extremely high predicted values for SOC. Abrupt 
changes in predicted SOC are also observed at locations with low to moderate 
SOC (40 to 80~g~kg$^{-1}$). This is caused by using the more detailed land use 
map.

Predicted ECEC (Figures \ref{fig:ecec-base-pred} and \ref{fig:ecec-best-pred}) 
had a large dependency on land use and geologic maps. Several features observed
in the prediction maps derive from these two covariates. The influence of land 
use is seen in the northern part, while in the western, central, and eastern 
parts the influence of both covariates create an irregular pattern in the 
spatial distribution of ECEC. It is also in these parts that the largest 
prediction error standard deviations occur, following the spatial pattern of 
the covariates.

 \begin{figure}[!ht]
    \centering
    \begin{minipage}[b]{63mm}
      \subcaption{}
      \label{fig:clay-base-pred}
      \centering
      \includegraphics[width=63mm]{chap01FIG7a}
    \end{minipage}
    \begin{minipage}[b]{63mm}
      \subcaption{}
      \label{fig:clay-best-pred}
      \centering
      \includegraphics[width=63mm]{chap01FIG7d}
    \end{minipage}
    \begin{minipage}[b]{63mm}
      \subcaption{}
      \label{fig:soc-base-pred}
      \centering
      \includegraphics[width=63mm]{chap01FIG7b}
    \end{minipage}
    \begin{minipage}[b]{63mm}
      \subcaption{}
      \label{fig:soc-best-pred}
      \centering
      \includegraphics[width=63mm]{chap01FIG7e}
    \end{minipage}
    \begin{minipage}[b]{63mm}
      \subcaption{}
      \label{fig:ecec-base-pred}
      \centering
      \includegraphics[width=63mm]{chap01FIG7c}
    \end{minipage}
    \begin{minipage}[b]{63mm}
      \subcaption{}
      \label{fig:ecec-best-pred}
      \centering
      \includegraphics[width=63mm]{chap01FIG7f}
    \end{minipage}
  \caption{Predicted values for CLAY (g~kg$^{-1}$) (a, b), SOC (g~kg$^{-1}$) 
  (c, d), and ECEC (mmol~kg$^{-1}$) (e, f) using the baseline (left) and best 
  performing (right) linear mixed models.}
  \label{fig:kriging}
\end{figure}

\begin{figure}[!ht]
\centering
    \begin{minipage}[b]{63mm}
      \subcaption{}
      \label{fig:clay-base-var}
      \centering
      \includegraphics[width=60mm]{chap01FIG8a}
    \end{minipage}
    \begin{minipage}[b]{63mm}
      \subcaption{}
      \label{fig:clay-best-var}
      \centering
      \includegraphics[width=60mm]{chap01FIG8d}
    \end{minipage}
    \begin{minipage}[b]{63mm}
      \subcaption{}
      \label{fig:soc-base-var}
      \centering
      \includegraphics[width=60mm]{chap01FIG8b}
    \end{minipage}
    \begin{minipage}[b]{63mm}
      \subcaption{}
      \label{fig:soc-best-var}
      \centering
      \includegraphics[width=60mm]{chap01FIG8e}
    \end{minipage}
    \begin{minipage}[b]{63mm}
      \subcaption{}
      \label{fig:ecec-base-var}
      \centering
      \includegraphics[width=60mm]{chap01FIG8c}
    \end{minipage}
    \begin{minipage}[b]{63mm}
      \subcaption{}
      \label{fig:ecec-best-var}
      \centering
      \includegraphics[width=60mm]{chap01FIG8f}
    \end{minipage}
  \caption{Prediction error standard deviations for CLAY (g~kg$^{-1}$) (a, b), 
  SOC (g~kg$^{-1}$) (c, d), and ECEC (mmol~kg$^{-1}$) (e, f) using the baseline 
  (left) and best performing (right) linear mixed models.}
  \label{fig:kriging-variance}
\end{figure}

The smallest prediction error standard deviations occur at lower elevations, 
along the three main streams, and close to the water outlet in the southern 
part of the study area. These areas have the highest density of point soil 
observations used to calibrate the models, and the smallest values for all 
three soil properties. While the first determines the accuracy of the EBLUP, 
the second influences the final accuracy through the back-transformation of 
predicted values.

\section{Discussion}

Our main goal was to evaluate whether investing in more spatially detailed 
environmental covariates improves the accuracy of soil maps. We saw that 
calibrating the models with more detailed covariates generally has a small to 
moderate, but positive, impact on the predictions. The magnitude of this 
benefit depends on the magnitude of the increase of the spatial detail of the 
covariate, on the other covariates included in the model, and on the soil 
property. However, there seems to be a limit above which the increase of spatial
detail has a negative impact on the predictions. In the next two subsections we
interpret the results from a pedological perspective and assess whether the 
investment in more detailed covariates is worthwhile or if alternatives to 
improve prediction accuracy should be favoured.

\subsection{Spatio-temporal controls of soil properties}

CLAY was moderately well predicted using less detailed environmental covariates,
with small improvement when using the more detailed covariates. CLAY was 
expected to have a strong correlation with topography and parent material. 
This correlation was already considerable when the less detailed DEM and 
geologic map were used, and improved only marginally with the more detailed 
version. One sensible explanation is that the effective (actual rather than 
theoretical) spatial detail of the two geologic maps was similar, although they
had a four-fold difference in the size of the minimum legible delineation (see 
\citet{HenglEtAl2006a} for a discussion on effective scale). For the DEM, many 
studies have already suggested that its resolution may be of secondary 
importance when calculating DEM derivatives for digital soil mapping 
\citep{ZhuEtAl2008, BehrensEtAl2010a, MillerEtAl2015}. The influence of land 
use on CLAY is currently small due to reduction of soil erosion in the first 
decade of the 21st century \citep{MiguelEtAl2012, TenCatenEtAl2012b}. A 
moderate within-field spatial variation may exist due to past erosional 
processes \citep{MouraBueno2012}, but we lack evidence of how well this source 
of variation was captured in the present-time point soil data.

It is worthwhile to consider the influence of the more detailed soil map on 
predicting CLAY. Due to its production process, the more detailed soil map 
derives a large amount of spatial detail from the geologic map, land use map 
and DEM -- note that the second-best performing model for CLAY included the 
more detailed geologic map instead of the more detailed soil map 
(\autoref{fig:model-series}). However, most of the additional spatial detail 
included in the more detailed soil map was probably based on the spatial 
variation of soil texture, because this is a strongly marked soil feature in 
the area \citep{MiguelEtAl2012}. Soil texture is one of the most important soil 
properties used by soil surveyors in the field to identify mapping units 
\citep{Legros2006}. These findings help explain why in the end the more 
detailed soil map was the most beneficial for CLAY instead of the geologic map.

SOC and ECEC were considerably better predicted when more detailed environmental
covariates were used. Our expectation that SOC and ECEC would have a strong 
correlation with land use was confirmed by the fact that this covariate 
explained a large amount of the variance and was highly beneficial for improving
the predictions. Although the available point soil data are limited to the 
2004-2011 period, we believe that land use changes in the last 30~years 
\citep{MiguelEtAl2012, TenCatenEtAl2012b} strongly affected SOC and ECEC. Thus, 
the more detailed land use map is likely to have considerably improved model 
performance because it is up-to-date and, possibly, because it has 40 times 
more spatial detail than its less detailed version. Despite the fact that the 
two land use maps used in this study were from different time periods, which 
confounds the analysis, the results obtained indicate that a more detailed land 
use map improves the prediction of SOC. For example, the areas used for crop 
agriculture, which are well know for having lower SOC and ECEC 
\citep{Menezes2008, MouraBueno2012}, are not depicted in the less detailed land 
use map.

We expected SOC to have a stronger correlation with the DEM than with the 
geologic map due to its strong dependence on erosion, but we observed the 
contrary. This result my be partially explained by the fact that there is a 
strong relation between geology and topography in the study area 
\citep{Sartori2009}. Due to its production process \citep{MacielFilho1990}, 
the geologic maps can be interpreted as an aggregated version of a DEM. A 
second sensible explanation is that the effect of erosion on SOC is not that 
large because erosion was considerably reduced in the last decade 
\citep{MiguelEtAl2012, TenCatenEtAl2012b}. A last possible explanation, which 
integrates the previous two, is the existence of a spatial relation between SOC
and CLAY, the last being strongly correlated with parent material. These 
relations help explain why the more detailed DEM was almost as beneficial as 
the more detailed geologic map for SOC predictions. In the case of ECEC, our 
expectation of a strong dependency on a more detailed geologic map for 
producing more accurate predictions was confirmed.

The observed benefit of the more detailed geologic map and DEM for making more 
accurate CLAY, SOC and ECEC predictions suggests that these soil properties are
spatially related in the study area. We also hypothesize that the complexity of
current land use makes it difficult to achieve SOC and ECEC models with 
performances comparable to CLAY. One important source of variation in 
forested areas is its use for animal grazing \citep{SamuelRosaEtAl2011a}. 
This influences nutrient cycling and soil nutrient availability 
\citep{SchramaEtAl2013}. Current remote sensing technology is unable to 
capture the data needed to proxy the environmental conditions created by
these processes.

\subsection{Using more detailed environmental covariates}

More detailed covariates are usually expected to improve predictions in digital 
soil mapping \citep{CavazziEtAl2013, MaynardEtAl2014}. However, deciding whether
to invest or not in more detailed covariates requires careful thinking and 
depends on case-specific elements. We generally saw improvement in the 
predictions in our study, but the improvement was not large and may not outweigh
the costs. Also, the models calibrated with the more detailed versions of all 
covariates were not the best performing models. Using more detailed satellite 
images and land use maps degraded CLAY predictions. Although the more detailed
soil map had the largest benefit for CLAY, it may be too costly and impractical 
since its production usually requires having available more detailed versions 
of all other covariates. For SOC and ECEC, simply using a more detailed land 
use map resulted in considerably more accurate predictions. However, the 
superior performance may not outweigh the extra costs because producing a more 
detailed land use map usually requires up-to-date field observations and 
satellite images. Thus, the decision to adopt a more detailed covariate for 
digital soil mapping will ultimately depend on a trade-off between the increased
accuracy and the extra budget required. It may also depend on other potential 
applications of the covariates, but this is not our concern here.

One interesting observation is that if a less detailed covariate yields poor 
predictions, its more detailed version has the potential to produce larger 
improvement in model performance. However, this is only a potential, not a 
guarantee. For instance, \citet{EldeiryEtAl2008} were not able to increase the 
$R^2 = 0.31$ of linear regression models of soil salinity by more than 0.07 
points using 7.5 times more detailed satellite images. On the other hand, model 
performance is likely to be hardly improved using more detailed covariates if 
their less detailed version has already produced accurate predictions. This 
agrees with findings by \citet{ThompsonEtAl2001} and \citet{KimEtAl2014}.

We also observed that the predictions can be degraded when using the more 
detailed version of covariates. In our study, this happened with the satellite 
image (all three soil properties), land use map (CLAY) and soil map (SOC and 
ECEC). A (small) benefit was observed only when these covariates were used 
along with the more detailed version of other covariates. As pointed out above,
such a small benefit may not outweigh the increase in mapping costs. The 
trade-off between reducing model performance and being beneficial seems to 
depend on how much more spatial detail a covariate will have and on its 
correlation with the soil property. For example, the land use map was strongly 
correlated with SOC and ECEC, but not with CLAY, and its more detailed version 
had 40~times more spatial detail. It helped improve SOC and ECEC predictions, 
but degraded CLAY predictions, resulting in only a small improvement when used
along with the more detailed satellite image and geologic map.

If the influence of a more detailed covariate depends on the increase of 
spatial detail, then the priority should be to improve the spatial detail of 
the most beneficial covariate. This requires solid subject area knowledge 
because empirical evidence from the baseline model may be insufficient. The 
most beneficial covariate is not necessarily that which explained the largest 
part of the variance in the baseline model (see \autoref{tab:drop}). This occurs
because increasing the spatial detail reduces the correlation between the 
covariate and the soil property. And also because there is little room to 
improve a correlation that is already high in the baseline model. 
\citet{CavazziEtAl2013} suggest that the more detailed covariate has an excess 
of detail, a ``noise'' that degrades the predictions. This could explain the 
results for \texttt{sat}: higher resolution images can resolve smaller objects
(e.g. individual plants) whose spectral behaviours are highly variable, adding
noise to the \texttt{sat}-soil property correlation; on the other hand, lower 
resolution images capture collections of objects, and thus their variation is 
smoothed out in the pixel, reducing noise.

According to information theory one should optimize (maximize) the correlation 
between the point soil data and the covariates. This was described elsewhere as 
matching the ``phenomenon scale'' (the spatial pattern of the soil property) 
with the ``analysis scale'' (the spatial pattern of the covariates) 
\citep{DunganEtAl2002, MillerEtAl2014}. Finding the ``optimum'' requires 
evaluating the strength of the correlation using covariates with different 
levels of spatial detail \citep{DragutEtAl2009, CavazziEtAl2013, MillerEtAl2015}.
Our results show that this approach may be too costly and impractical. Since 
digital soil mapping explores only the empirical relationship among 
environmental conditions and soil properties \citep{Grunwald2009}, the 
``optimum'' is a ``conditional optimum'' -- conditional on the point soil data 
available. It does not necessarily mean that the most accurate predictions will
be made, but only that there is a level of spatial detail at which the 
correlation between the covariate and the point soil data is at a maximum. We 
suggest that instead more comprehensive approaches should be used to explore 
the full potential of the available covariates (see \citet{BehrensEtAl2010a} and 
\citet{MillerEtAl2015} for examples).

Finally, one must still judge whether the potential improvement in predictions 
is sufficient given the extra costs involved with using more detailed covariates.
If the extra budget is spent on deriving more detailed covariates, we suggest 
that it may be better to substantially improve the detail of a less influential
covariate than marginally increase the detail of the most influential covariate.
However, other means to spend the extra budget should be considered. For 
instance, it may be more efficient to concentrate on obtaining more soil 
observations. These may focus on better capturing the short range spatial
variation \citep{BrusEtAl2007a} or improving the representation of the feature 
space to avoid undesirable extrapolations \citep{MinasnyEtAl2006b}.

\section{Conclusions}

This study has shown that:

\begin{enumerate}
 \item Using more detailed environmental covariates results in only a modest 
 increase in the prediction accuracy of linear prediction models;
 
 \item A more detailed covariate has a greater potential to improve prediction
 accuracy when the soil property is poorly predicted by its less detailed version;
 
 \item The impact on prediction accuracy when using the more detailed version 
 of a less important covariate may depend on which other covariates are included
 in the model;
 
 \item Choosing whether or not to invest in more detailed covariates depends on 
 the strength of the relationship between the covariates and the soil property 
 being modelled, and on the relative difference between the less detailed and 
 the more detailed versions of the covariates.
\end{enumerate}

\section*{Acknowledgements}

We are grateful to Dr.~Bradley A.~Miller, from the Leibniz Centre for 
Agricultural Landscape Research, M\"uncheberg, Germany, for his comments during 
the revision of the manuscript. The following colleagues collaborated in 
different phases of data collection and/or processing: Dr.~Ricardo Simão Diniz 
Dalmolin, Dr.~Edgardo Ramos Medeiros and Jean Michel Moura Bueno (UFSM), 
Dr.~Pablo Miguel (UFPel), Dr.~Mauro Antonio Homem Antunes, Fábio Paes Leme 
Ferreira, and Anastácia Perci Campos de Almeida (UFRRJ), and Luis Fernando 
Chimelo Ruiz (UFRGS). We are grateful to Dr.~Bas Kempen, Dr.~Tom Hengl, and 
Marcos Angelini (ISRIC) for their helpful comments during the conception of the
study and data analysis. We also thank the development teams and module/package 
authors of the many free and open source software (FOSS) and operational system 
(OS) that were used to develop our study. The first author was supported by the
CAPES Foundation, Ministry of Education of Brazil (Process BEX 11677/13-9). The 
last author was supported by the CNPq foundation, Ministry of Science and 
Technology of Brazil. The use of the RapidEye images is endorsed by the 
Corporate Commitment Agreement signed between the Federal Rural University of 
Rio de Janeiro and the Brazilian Ministry of the Environment.

\section*{Note}
This chapter is based on A.~Samuel-Rosa, G.B.M.~Heuvelink, G.M.~Vasques, 
L.H.C.~Anjos. Do more detailed environmental covariates deliver more accurate 
soil maps? \textit{Geoderma}, v.243--244, p.214--227, 2015.

% \section*{REFERENCES}
% \bibliography{biblio.bib}
