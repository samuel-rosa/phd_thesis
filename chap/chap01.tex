\setcounter{page}{1}
\artigofalse
\chapter{General Introduction}
\label{chap:introduction}

As soil scientists, we recognize the importance of soil...

Modern soil spatial modelling is based on exploring the empirical relationship among environmental conditions 
and soil properties.

Modern soil mapping relies on the use of statistical models to produce digital representations of spatial 
soil distribution using point soil observations and spatially exhaustive covariates \cite{McBratneyEtAl2003, 
ScullEtAl2003, Florinsky2012}. Three important weaknesses in the statistical soil distribution modelling 
approach can be pointed out. First, it requires sufficient and appropriately distributed point soil data 
within 
the area being mapped \cite{CarreEtAl2007a}. Second, the model structure explores only the empirical 
relationship among environmental conditions and soil properties, being less comprehensive than soil-landscape 
process models \cite{Grunwald2009}. Last, the covariates are only approximations of the true environmental 
conditions that helped shape the soil. They serve only as proxies (surrogates) of the current environmental 
conditions, which in many cases are different from the past conditions under which pedogenesis took place 
\cite{HeuvelinkEtAl2001}. In spite of these weaknesses, modern soil mapping techniques have proven very 
successful in the past decades in producing soil property maps that capture the main patterns of soil spatial 
variation \cite{MooreEtAl1993, McBratneyEtAl2000, Grunwald2009}.


Soil spatial models, like any other model, are nothing more than a gross simplification of reality. Unless we 
observe the soil everywhere -- which would destroy the soil and render the observations useless --, no matter 
how large the volume of data is, or how comprehensive our background knowledge, it will never be possible to 
construct a model that explains the entire complexity of the soil. Thus, the outcome of a soil spatial model, 
i.e. a soil map, will always deviate from the \q{truth} -- this deviation from the \q{truth} is what we call 
\emph{error}. What a soil map conveys is what we expect the soil to be, not our \emph{certainty} about it.

Because soil spatial modellers aim at using the available resources to produce the most accurate 
representation of the soil, a sensible research programme is to investigate the main causes for soil maps being 
more or less \emph{uncertain}. There are many sources of uncertainty in soil spatial modelling.



\section{Objectives and research questions}
\label{sec:thesis-objectives}

\begin{enumerate}
 \item Determine the suitability of freely available covariates to calibrate soil spatial models.

  \begin{enumerate}[label=(\alph*)]
   \item Does the use of more detailed covariates result in considerably more accurate soil maps?
   
   \item How does incorporation of spatial dependence in a soil spatial model compare to the gain in 
   prediction accuracy obtained with using more detailed covariates?
   
   \item Are the answers to these research questions consistent across soil properties?
  \end{enumerate}
 
 \item Identify the factors that determine how field soil spatial modellers select soil observation locations.
  
  \begin{enumerate}[label=(\alph*)]
   \item Do the main criteria employed for deciding upon the location of soil observations have a pedological 
   origin?
   
   \item Can point pattern analysis help understanding the purposive sampling strategy traditionally employed
   by field soil spatial modellers?
   
   
   
   \item ?
  \end{enumerate}
 
\end{enumerate}




\section{Outline}
\label{sec:thesis-outline}

The thesis is composed of eight chapters where each of the above mentioned objectives are met 
(\autoref{sec:thesis-objectives}). Although there is a logical sequence in their presentation, all chapters 
were planned so that they could be read separately. This means that there is some overlap between them, i.e. 
repeated information. References to specific sections of other chapters using coloured (blue) hyperlinks are 
common.

\emph{Chapter I}

\autoref{chap:chap02} presents the conceptual model of pedogenesis (in Portuguese), which consists of a 
description of the study area that includes an explicit description of soil-forming factors (climate, geology, 
geomorphology, hydrology, land use, and vegetation) and processes that determine the soil spatio-temporal 
distribution. \autoref{chap:chap03} describes the soil data included in the \emph{Santa Maria dataset}, which 
was used to develop the case studies presented in this thesis. The Santa Maria dataset is composed of 
$n = 410$ soil observations compiled from studies carried out between \num{2004} and \num{2013}. These studies 
aimed at producing semi-detailed soil and land use maps, and modelling topsoil carbon stock and vulnerability 
to erosion. A comprehensive description of the covariate data included in the Santa Maria dataset, and their 
processing, is given in \autoref{chap:chap04}. The goal of these three chapters is to provide the bases for 
future soil spatial modelling exercises in the study area and to serve as examples for new soil spatial 
modelling studies developed elsewhere.

Based on an article published in the peer reviewed journal \geoderma, \autoref{chap:chap05} serves the purpose 
of meeting the first objective of the thesis and answering its respective research questions. There, the 
prediction performance of linear soil spatial models calibrated using covariates (area-class soil maps, land 
use maps, geological maps, digital elevation models, and orbital images) available in two levels of detail is 
evaluated. The influence of taking the spatial dependence of the residuals into account is assessed as well. 

\autoref{chap:chap06} presents an approach that aims at helping to understand the purposive sampling strategy 
traditionally employed by field soil modellers, i.e. free survey. This is important because most soil 
spatial modelling projects rely on legacy data, i.e. soil data produced many years ago, whose observation 
locations were purposively selected by soil spatial modellers using poorly formalized tacit rules. The chapter 
shows that the location of soil observations were strongly determined by subjective elements unrelated to the 
local soil-landscape relationship. Understanding the reasons behind the location of free survey soil 
observations can help soil spatial modellers designing more efficient data-driven sampling strategies.

The results of the experiment devised to evaluate if improving a popular sampling algorithm results in more 
accurate spatial predictions is presented in \autoref{chap:chap07}. The comparison of five sampling algorithms 
on how they affect estimated model parameters and prediction accuracy is presented in \autoref{chap:chap08}. 
The chapter also introduces a new general purpose sampling algorithm. Both chapters contain only partial 
results, and will be the basis of manuscripts to be submitted to peer reviewed journals, both dealing with the 
second objective of this thesis and its respective research questions.

The sequence of eight chapters is closed with a \emph{General Conclusion} where I highlight the main results 
of the research, contributions and merits of the study. Next, there are two appendices, both devoted to the 
description of the two packages for \texttt{R} developed to support the case studies: \texttt{pedometrics} 
(\autoref{apen:pedometrics}) and \texttt{spsann} (\autoref{apen:spsann}). The first includes miscellaneous 
functions that were put together for ease of use. The second was designed for the optimization of sample 
configurations using spatial simulated annealing. All literature references are presented under a unique 
\emph{Bibliographic References} chapter (\autoref{chap:references}) at the end of the thesis.
