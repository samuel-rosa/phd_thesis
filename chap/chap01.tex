\setcounter{page}{1}
\artigofalse
\chapter{GENERAL INTRODUCTION}
\label{chap:introduction}

Modern soil spatial modelling is based on using statistical models to explore the empirical relationship among 
environmental conditions and soil properties. These soil spatial models, like any other model, are nothing 
more than a gross simplification of reality. Unless we observe the soil everywhere -- which would destroy the 
soil and render the observations useless --, no matter how large the volume of data is, or how comprehensive 
our background knowledge, it will never be possible to construct a model that explains the entire complexity 
of the soil. Thus, the outcome of a soil spatial model, i.e. a soil map, will always deviate from the 
\q{truth} -- this deviation from the \q{truth} is what we call \emph{error}. What a soil map conveys is what 
we expect the soil to be, not our \emph{certainty} about it.

Because soil spatial modellers aim at using the available resources to produce the most accurate 
representation of the soil, a sensible research programme is to investigate the main causes for soil maps 
being more or less \emph{uncertain}. There are many sources of uncertainty in soil spatial modelling, such as 
the errors that result from using a poor statistical model and from making interpolations and extrapolations 
to predict soil properties at unvisited locations. Another important source of uncertainty is the data used to 
assess the empirical relationship among environmental conditions and soil properties: covariate and soil data.

The general objective of this thesis is to evaluate the soil and covariate data as source of uncertainty in 
soil spatial modelling. This general objective can be divided into specific objectives and their respective 
research questions:

\begin{enumerate}
 \item Determine the suitability of freely available covariates to calibrate soil spatial models.
  \begin{enumerate}[label=(\alph*)]
   \item Does the use of more detailed covariates result in considerably more accurate soil maps?
   \item How does incorporation of spatial dependence in a soil spatial model compare to the gain in 
   prediction accuracy obtained with using more detailed covariates?
   \item Are the answers to these research questions consistent across soil properties?
  \end{enumerate}
 
 \item Identify the factors that determine how field soil spatial modellers select soil observation locations.
  \begin{enumerate}[label=(\alph*)]
   \item Do the factors play the same role along the course of the soil observation process?
   \item Do the main criteria employed for deciding upon the location of soil observations have a pedological 
   origin?
   \item Can point pattern analysis help understanding the purposive sampling strategy traditionally employed
   by field soil spatial modellers?
  \end{enumerate}
 
 \item Identify appropriate calibration sample sizes and designs for soil spatial modelling.
  \begin{enumerate}[label=(\alph*)]
   \item Can theoretical and algorithmic improvements on existing spatial sample optimization algorithms 
   improve the performance of soil models?
   \item How suboptimal is it to use a sample configuration that was optimized to a different purpose than it 
   is going to be used for?
   \item Is the predictive performance of a soil-mapping model estimated using a sample configuration 
   optimized using heuristics poorer than that of another soil-mapping model whose parameters were estimated 
   using a sample configuration optimized using an \textit{a priori} knowledge of the model?
   \item Is it possible to obtain a sample configuration that is efficient in identifying and estimating i) 
   the spatial trend and ii) the variogram model, and iii) making spatial predictions?
   \item How does the sample configuration affect the estimated model parameters and thus the conclusions that 
   can be drawn under the light of the existing conceptual model of pedogenesis?
   \item Are the answers to the research questions above consistent across sample sizes and soil properties?
  \end{enumerate}
\end{enumerate}

The thesis is composed of eight chapters where each of the above mentioned objectives are met. Although there 
is a logical sequence in their presentation, all chapters were planned so that they could be read separately. 
This means that there is some overlap between them, i.e. repeated information. References to specific sections 
of other chapters using coloured (blue) hyperlinks are common.

\autoref{chap:chap02} is a commented review of the literature on soil spatial modelling and its main sources 
of uncertainty. The review starts with a discussion about the efforts made by soil spatial modellers to 
raise awareness about the importance of soil spatial information. These efforts seem to have fuelled a global 
scientific demand for up-to-date, high resolution soil spatial information. The chapter continues with a 
description of soil spatial modelling along human history, suggesting that the goal of producing soil maps 
remains more or less the same since the Neolithic Revolution (ca.~\num{10000}~years). The chapter closes with 
the main sources of uncertainty.

\autoref{chap:chap03} presents the conceptual model of pedogenesis (in Portuguese), which consists of a 
description of the study area that includes an explicit description of soil-forming factors (climate, geology, 
geomorphology, hydrology, land use, and vegetation) and processes that determine the soil spatio-temporal 
distribution. \autoref{chap:chap04} describes the soil data included in the \emph{Santa Maria dataset}, which 
was used to develop the case studies presented in this thesis. The Santa Maria dataset is composed of 
$n = 410$ soil observations compiled from studies carried out between \num{2004} and \num{2013}. These studies 
aimed at producing semi-detailed soil and land use maps, and modelling topsoil carbon stock and vulnerability 
to erosion. A comprehensive description of the covariate data included in the Santa Maria dataset, and their 
processing, is given in \autoref{chap:chap05}. The goal of these three chapters is to provide the bases for 
future soil spatial modelling exercises in the study area and to serve as examples for new soil spatial 
modelling studies developed elsewhere.

Based on an article published in the peer reviewed journal \geoderma, \autoref{chap:chap06} serves the purpose 
of meeting the first objective of the thesis and answering its respective research questions. There, the 
prediction performance of linear soil spatial models calibrated using covariates (area-class soil maps, land 
use maps, geological maps, digital elevation models, and orbital images) available in two levels of detail is 
evaluated. The influence of taking the spatial dependence of the residuals into account is assessed as well. 

\autoref{chap:chap07} presents an approach that aims at helping to understand the purposive sampling strategy 
traditionally employed by field soil modellers, i.e. free survey. This is important because most soil 
spatial modelling projects rely on legacy data, i.e. soil data produced many years ago, whose observation 
locations were purposively selected by soil spatial modellers using poorly formalized tacit rules. The 
chapter, designed to answer the research questions of the second objective of the thesis, shows that the 
location of soil observations were strongly determined by subjective elements unrelated to the local 
soil-landscape relationship. Understanding the reasons behind the location of free survey soil observations 
can help soil spatial modellers designing more efficient data-driven sampling strategies.

The results of the experiment devised to evaluate if improving a popular sampling algorithm results in more 
accurate spatial predictions is presented in \autoref{chap:chap08}. The comparison of five sampling algorithms 
on how they affect estimated model parameters and prediction accuracy is presented in \autoref{chap:chap09}. 
The chapter also introduces a new general purpose sampling algorithm. Both chapters contain only partial 
results, and will be the basis of manuscripts to be submitted to peer reviewed journals, both dealing with the 
second objective of this thesis and its respective research questions.

The sequence of eight chapters is closed with a \emph{General Conclusion} where I highlight the main results 
of the research, contributions and merits of the study. Next, there are two appendices, both devoted to the 
description of the two packages for \texttt{R} developed to support the case studies: \texttt{pedometrics} 
(\autoref{apen:pedometrics}) and \texttt{spsann} (\autoref{apen:spsann}). The first includes miscellaneous 
functions that were put together for ease of use. The second was designed for the optimization of sample 
configurations using spatial simulated annealing. All literature references are presented under a unique 
\emph{Bibliographic References} chapter (\autoref{chap:references}) at the end of the thesis.
