\artigofalse
\chapter{INTRODUÇÃO GERAL}
\shorttitle{Introdução Geral}
\label{appen:introduction-pt}

A modelagem espacial do solo moderna é baseada na utilização de modelos estatísticos para explorar a relação 
empírica entre as condições ambientais e as propriedades do solo. Estes modelos espaciais do solo, como 
qualquer outro modelo, não são nada mais do que uma simplificação da realidade. A menos que observemos o solo 
em todos os lugares -- o que destruiria o solo e tornaria as observações inúteis --, não importa quão grande 
seja o volume de dados, ou quão abrangente o nosso conhecimento for, \emph{nunca} será possível construir um 
modelo que explique toda a complexidade do solo. Assim, o resultado de um modelo espacial do solo, ou seja, um 
mapa do solo, \emph{sempre} desviará da \q{verdade} -- esse desvio da \q{verdade} é o que chamamos de 
\emph{error}. O que um mapa do solo transmite é o que esperamos que o solo seja, reconhecendo que há 
\emph{incerteza} sobre ele.

Dado que modeladores espaciais do solo procuram utilizar os recursos disponíveis para produzir a representação 
mais precisa do solo, um programa de pesquisa sensível é investigar as principais causas para os mapas do solo 
serem mais ou menos \emph{incertos}. Existem muitas fontes de incerteza na modelagem espacial do solo, tais 
como os erros que resultam da utilização de um modelo estatístico deficiente ou de fazer interpolações e 
extrapolações para predizer as propriedades do solo em locais não visitados. Outra importante fonte de 
incerteza é os dados utilizados para avaliar a relação empírica entre as condições ambientais e as 
propriedades do solo: dados de covariáveis e solo.

O objetivo geral dessa tese é avaliar importantes fontes de incerteza na modelagem espacial do solo com ênfase 
em dados de solo e de covariáveis. Este objetivo geral pode ser dividido em objetivos específicos e suas 
respectivas questões de pesquisa:

\begin{enumerate}[label=(\Roman*)]
\item Determinar a aptidão de covariáveis disponíveis gratuitamente para calibrar modelos espaciais do solo.

\begin{enumerate}[label=(\alph*)]
\item Será que o uso de covariáveis mais detalhadas resulta em mapas consideravelmente mais precisos do solo?

\item Como é que a incorporação de dependência espacial em um modelo espacial do solo se compara ao ganho na 
acurácia de predição obtido pelo uso de covariáveis mais detalhadas?

\item As respostas a essas questões de pesquisa são consistentes entre propriedades do solo?
\end{enumerate}
 
\item Identificar os fatores que determinam como modeladores espaciais de campo do solo selecionam os locais 
de observação do solo.

\begin{enumerate}[label=(\alph*)]

\item Quais fatores são considerados para decidir sobre a localização das observações do solo? Eles têm uma 
origem pedológica?

\item Será que os fatores desempenham o mesmo papel ao longo do curso do processo de observação do solo?

\item Pode a análise de padrão pontual ajudar a compreender a estratégia de amostragem intencional 
tradicionalmente empregada por modeladores espaciais de campo do solo?
\end{enumerate}

\newpage

\item Identificar tamanhos e delineamentos amostrais de calibração apropriados para a modelagem espacial do 
solo.
\begin{enumerate}[label=(\alph*)]
\item Pode o algoritmo de amostragem hipercubo Latino condicionado ser melhorado? Será que essa melhoria 
resulta em pedições espaciais do solo mais acuradas?

\item Quais são os algoritmos de amostragem teoricamente mais sólidos para estimativa da espacial tendência, 
estimativa do variograma, e predição espacial quando sabemos muito pouco sobre a variação espacial do solo?

\item Podem esses algoritmos de amostragem ser usados para construir um algoritmo genérico de amostragem 
intencional?

\end{enumerate}
\end{enumerate}


A tese é composta por oito capítulos, onde cada um dos objetivos mencionados acima são cumpridos. Embora haja 
uma sequência lógica na sua apresentação, todos os capítulos foram planejadas para que pudessem ser lidos 
separadamente. Isto significa que existe uma certa sobreposição entre eles, isto é, informações repetidas. As 
referências a seções específicas de outros capítulos utilizando hiperlinks coloridos (azul) são comuns.

\autoref{chap:chap02} é uma revisão comentada da literatura sobre modelagem espacial do solo e suas principais 
fontes de incerteza. A revisão começa com uma discussão sobre os esforços envidados pelos modeladores espaciais 
do solo para aumentar a conscientização sobre a importância da informação espacial do solo. Esses esforços 
parecem ter impulsionado uma demanda científica global por informação espacial do solo atualizada e em alta 
resolução. O capítulo continua com uma descrição da modelagem espacial do solo ao longo da história humana, 
sugerindo que o objetivo de produzir mapas do solo permanece mais ou menos o mesmo desde a Revolução Neolítica 
(ca.~\num{10000}~anos). O capítulo termina com as principais fontes de incerteza.

\autoref{chap:chap04} descreve os dados do solo incluídos no \emph{conjunto de dados de Santa Maria}, que foi 
usado para desenvolver os estudos de caso apresentados nesta tese. O conjunto de dados de Santa Maria é 
composto de $n = 410$ observações do solo compiladas de estudos realizados entre 2004 e 2013. Esses estudos 
visavam a geração de mapas semi-detalhados do solo e uso da terra, e a modelagem do estoque de carbono na 
camada superficial do solo e da vulnerabilidade à erosão. Uma descrição detalhada dos dados de covariáveis 
incluídos no conjunto de dados de Santa Maria, e seu processamento, é dada no \autoref{chap:chap05}. 
\autoref{chap:chap03} apresenta o modelo conceitual de pedogênese (em português), que consiste numa descrição 
da área de estudo que inclui uma descrição explícita dos fatores de formação do solo (clima, geologia, 
geomorfologia, hidrologia, uso da terra e vegetação) e processos que determinam a distribuição espaço-temporal 
do solo. Além de descrever os dados utilizados na tese, o objetivo desses capítulos, juntamente com o modelo 
conceitual de pedogênese, é fornecer a base para futuros exercícios de modelagem espacial do solo na área de 
estudo, e servir como exemplo para novas estudos de modelagem espacial do solo desenvolvidos em outros lugares.

Baseado em um artigo publicado na revista avaliada por pares \geoderma, \autoref{chap:chap06} serve o 
propósito de atingir o primeiro objetivo da tese e responder a suas respectivas questões de pesquisa. O 
desempenho preditivo de modelos lineares espaciais do solo calibrados utilizando covariáveis (mapas do solo 
área de classe, mapas de uso da terra, mapas geológicos, modelos digitais de elevação, e imagens de satélite) 
disponíveis em dois níveis de detalhe é avaliado. A influência de levar em conta a dependência espacial dos 
resíduos também é avaliada.

\autoref{chap:chap07} apresenta uma abordagem que visa ajudar a compreender a estratégia de amostragem 
intencional tradicionalmente empregada por modeladores de campo do solo, ou seja, caminhamento livre. Isso é 
importante porque muitos projetos de modelagem espacial do solo dependem de dados legados, ou seja, dados do 
solo produzidos previamente e disponibilizados (publicamente ou não), cujos locais de observação foram 
selecionados intencionalmente por modeladores espaciais do solo usando regras tácitas mal documentadas. O 
capítulo foi concebido para responder às questões de pesquisa do segundo objetivo da tese. Análise de padrão 
pontual é utilizada para caracterizar a configuração espacial da amostra, enquanto teorias emprestadas da 
Psicologia são usados para elaborar sobre os fatores subjetivos envolvidos na seleção de locais de observação 
do solo.

O objetivo 3 e suas questões de pesquisa são abordados no \autoref{chap:chap08} e \autoref{chap:chap09}. No 
\autoref{chap:chap08}, três algoritmos de amostragem aperfeiçoados são comparados com o algoritmo original de 
amostragem hipercubo Latino condicionado em como eles afetam a cobertura geográfica, os parâmetros estimados 
do modelo e a acurácia preditiva. A influência do tamanho da amostra também é discutida. \autoref{chap:chap09} 
apresenta as estratégias de amostragem mais eficientes para estimativa da tendência espacial, estimativa do 
variograma, e predição espacial quando sabemos muito pouco sobre a variação espacial do solo. O capítulo 
termina com um novo algoritmo genérico de amostragem que visa os três objetivos conjuntamente.

A sequência de oito capítulos é encerrada com \hyperref[chap:chap10]{Conclusões Gerais} onde destaco os 
principais resultados da pesquisa e contribuições do estudo. Em seguida, há dois apêndices, ambos dedicados à 
descrição dos dois pacotes para \texttt{R} desenvolvidos para apoiar a tese: \texttt{spsann} 
(\autoref{appen:spsann}) e \texttt{pedometria} (\autoref{appen:pedometrics}). O primeiro foi projetado para a 
otimização de configurações amostrais usando o recozimento simulado espacial. O segundo inclui funções 
auxiliares que foram colocadas juntas para facilidade de uso. Todas as referências da literatura são 
apresentadas sob uma lista única de \hyperref[chap:references]{Referências Bibliográficas} no final da tese.
