% ATTENTION: lines starting with “%”  will not be included in the document!

\artigofalse
\chapter{The Santa Maria dataset -- covariate data}
\label{apen:covar-data}
\usepackage[utf8]{inputenc}
\SweaveOpts{prefix.string = fig/apene}

<<R-settings, echo = FALSE>>=
 
 # Start GRASS
 invisible(capture.output(
   spgrass6::initGRASS(
     gisBase = "/usr/lib/grass64/", gisDbase = path.expand("~/dbGRASS"),
     location = "dnos-sm-rs", mapset = "predictions", pid = Sys.getpid(), override = TRUE)))
 
 # Load polygon with the boundary of the study area
 invisible(capture.output(pol <- spgrass6::readVECT6("buffer_BASIN_10", ignore.stderr = TRUE)))
 
 # Load calibration points
 pts <- read.table(
  "~/projects/dnos-sm-rs/dnos-sm-rs-general/data/labData.csv", sep = ";",
  header = TRUE, na.strings = "na")[, c("longitude", "latitude")]
 sp::coordinates(pts) <- ~ longitude + latitude
 sp::proj4string(pts) <- sp::CRS("+init=epsg:4674") # sirgas2000
 pts <- sp::spTransform(pts, sp::CRS("+init=epsg:32722")) # wgs1984utm22s
 #pts <- data.frame(slot(pts, "coords"))
 
 # Trellis settings
 trellis.par.set(fontsize = list(text = 12, points = 8),
      layout.widths = list(left.padding = 0, right.padding = 0),
      layout.heights = list(top.padding = 0, bottom.padding = 0))
@

\tocless\section{Introduction}
\label{sec:covar-data-intro}

The Santa Maria dataset is a data set comprising spatially exhaustive covariate data produced in the 1980's, 
1990's, and 2000's covering the catchment of the DNOS/CORSAN reservoir, located in the southern border of the 
plateau of the Paraná Sedimentary Basin, in the city of Santa Maria, state of Rio Grande do Sul, Brazil. Some 
of these covariate data cover only part of the catchment, mainly the northern sector, which has an area of 
\SI{\pm2000}{\hectare}, corresponding to \SI{\pm60}{\percent} of the entire catchment. These covariate data 
are the outcome of projects that aimed at modelling various environmental features and were carried out as 
part of local (soil, geology, land use), regional (terrain, land use), and global (terrain, land use) 
initiatives.

Covariate data was standardized to A reference grid of \SI{5}{\m} grid size. The coordinate reference 
system (CRS) is WGS1984 / UTM zone 22S, coded \href{http://spatialreference.org/ref/epsg/32722/}{\num{32722}} 
by the European Petroleum Survey Group (\href{http://www.epsg.org/}{EPSG}).

This document presents a description of the covariate data contained in the Santa Maria dataset, including the 
procedures for their production, as well as the processing methods employed. The original sources of the 
covariate data are freely available at the producers databases or in public libraries.

\tocless\section{Ground control points}
\label{sec:covar-data-gcp}

All covariates were validated prior to their use. Horizontal (positional) validation was performed using a set 
of $n = 14$ validation points, here called ground control points (GCP), spread throughout and beyond the limits 
of the study area (\autoref{fig:covar-data-field-gcps}). The location of the GCPs was defined based on the 
existence of easily identifiable geographical markers across the covariates, including road intersection, fence 
corners, and property entrances.

\begin{figure}[!ht]
 \centering
 <<covar-data-field-gcps, fig = TRUE, echo = FALSE>>=
  
  # Load ground control points collected in the field
  gcps <- read.table(
    "~/projects/dnos-sm-rs/dnos-sm-rs-general/data/gcpData.csv", sep = "\t",
    header = TRUE, na.strings = "na")
  gcps <- gcps[which(gcps$source == "field"), c("x", "y")]
  sp::coordinates(gcps) <- ~ x + y
  sp::proj4string(gcps) <- sp::CRS("+init=epsg:32722") # wgs1984utm22s
  
  # Prepare plot
  p <- sp::spplot(
    pol, zcol = "cat", col = "gray", fill = "lemonchiffon",
    scales = list(draw = TRUE), colorkey = FALSE, aspect = "iso",
    xlim = c(gcps@bbox[1, 1] * 0.998, gcps@bbox[1, 2] * 1.005),
    ylim = c(gcps@bbox[2, 1] * 0.9999, gcps@bbox[2, 2] * 1.0001),
    par.settings = list(
      fontsize = list(text = 12, points = 8),
      layout.widths = list(left.padding = 0, right.padding = 0),
      layout.heights = list(top.padding = 0, bottom.padding = 0)),
    panel = function(x, y, ...) {
      sp::panel.polygonsplot(x, y, ...)
      lattice::panel.points(x = gcps@coords[, "x"], y = gcps@coords[, "y"], pch = 20, col = "red")
      lattice::panel.abline(
        v = seq(222000, 233000, 1000), h = seq(6712000, 6725000, 1000),
        col = "gray", lty = "dashed", lwd = 0.5)
    })
  
  # Plot and clean workspace
  print(p)
  rm(gcps, p)
  gc()
 @
 \caption{Spatial distribution of the ground control points ($n = 14$, red) used for the positional validation 
 of covariates included in the Santa Maria dataset.}
 \label{fig:covar-data-field-gcps}
\end{figure}

Positional validation was performed comparing the x- and y-coordinates of GCPs (observed value) with the 
coordinates of the respective geographical markers visually identified on the covariates (predicted value). The 
differences in the observed and predicted x- and y-coordinates were used to calculate the mean error (ME, 
\si{\m}), mean absolute error (MAE, \si{\m}), and mean squared error (MSE, \si{\m\square}) to evaluate if 
there were differences in the accuracy and precision between coordinates. The error vector (or module, the 
euclidean distance between two points) and its azimuth (the orientation of the error vector) were computed as 
well for every point. The mean of the error vector and its azimuth give the size and orientation of the 
systematic error present in the covariates, while the square root of the mean squared error vector (RMSE) is a 
measure of the uncertainty about the true position of the covariate in the geographic space.

The field location of the GCPs is as follows (in Portuguese):

\begin{description}
 \item [GCP 01] Em Santa Maria, no lado direito da barragem de concreto do reservatório do DNOS/CORSAN, 
 \SI{6}{\m} antes de chegar a ponte sobre o vertedouro, a \SI{3}{\m} de distância do centro da estrada que 
 desce da rodovia federal BR158.
 
 \item [GCP 02] Na entrada da Estrada do Perau, Rua Gralha Azul, em Itaara, no centro do canteiro, entre o 
 outdoor e a árvore (\it{Cedrella fissilis}), junto a rodovia federal BR158.
 
 \item [GCP 03] Na entrada de Itaara, próximo ao equipamento estático de fiscalização eletrônica de 
 velocidade, na rodovia federal BR158, a \SI{520}{\m} do Museu de Ufologia, em frente à Fruteira da 
 Esquina. Do outro lado da rodovia há uma torre, possivelmente de telefonia celular, e entrada para a mina de 
 extração de brita DallaPasqua, localizada a \SI{4}{\km} do local.
 
 \item [GCP 04] Em Santa Maria, na entrada da estrada de acesso ao cemitério do Bairro Campestre do Menino 
 Deus, do lado direito da Estrada do Perau (subindo), alinhado (\SI{\pm{50}{\cm}) com a frente das casas, a
 \SI{2,2}{\m} do muro, a \SI{6,5}{\m} do meio-fio da Esrada do Perau, a \SI{50}{\m} da ponte sobre o Rio
 Vacacaí-Mirim.
 
 \item [GCP 05] Na entrada do Rancho do Amaral, junto a porteira, no lado direito, fora da estrada, distante 
 \SI{1}{\m} de uma palmeira e \SI{1}{\m} do muro de pedras.
 
 \item [GCP 06] Em Itaara, na Avenida Etelvina, na beira da estrada, \SI{2,5}{\m} do centro da estrada,
 alinhado (\SI{\pm20}{\cm}) com a cerca que separa a floresta nativa do pomar de \it{Citrus}~sp.~localizado 
 do outro lado da estrada.há{
 
 \item [GCP 07] Em Itaara, no lado esquerdo da entrada da estrada que da acesso à mina de extração de brita
 DallaPasqua. Localidade de Estação Pinhal. Obs.: receptor de GNSS instalado sob a rede de transmissão de
 eletricidade.
 
 \item [GCP 08] No 10 Distrito de Santo Antão, na entrada da estrada para a Central de Tratamento de Resíduos
 da Caturita, em frente à Escola Municipal de Ensino Fundamental Intendente Manoel Ribas.
 
 \item [GCP 09] Em São Martinho da Serra, na localidade de Água Negra, na bifurcação da estrada que vem de
 Santa Maria e que dá acesso à localidade de Campinas, junto a parada de ônibus, no canteiro no meio da
 bifurcação, a \SI{40}{\m} distante do Piquete Laçador Jorge R.~da Silva, em frente ao Mercado do Ronaldo.
 
 \item [GCP 10] Na estrada de Santa Maria para São Martinho da Serra, em uma curva, no lado externo, próximo a
 duas pequenas árvores, alinhado (\SI{\pm1}{\m}) com cerca que marca a divisa entre duas propriedades com campo
 nativo. Em frente a propriedade com duas casas, uma delas com dois andares, e quatro pequenos lagos nos
 fundos. Depois da capela Santo Antão.
 
 \item [GCP 11] Em Itaara, na entrada da estrada que dá acesso à Brita Pinhal, junto à rodovia federal BR185,
 ao lado (distante \SI{5}{\m}) do corte no terreno expondo a rocha de arenito da Formação Botucatu,
 \SI{\pm15}{\m} distante do poste da linha de transmissão de eletricidade da companhia AES, na entrada para a
 localidade de Rincão dos Minello.
 
 \item [GCP 12] Em Itaara, em frente ao lago da SOCEPE, na entrada da cidade, junto a rodovia federal BR158,
 próximo ao Bar e Armazém Ricardo, deslocado em \SI{1}{\m} para dentro do passeio em relação ao alinhamento dos 
 postes da rede elétrica.
 
 \item [GCP 13] Em Itaara, na Vila Etelvina, alinhado com a cerca (\SI{\pm30}{\cm}) que divide duas terras, à 
 esquerda coberta com floresta nativa/exótica, à direita ocupada com lavoura de culturas anuais. O ponto está 
 locado no lado oposto. Plantação de videiras logo acima, no divisor de águas.
 
 \item [GCP 14] Em Itaara, na estrada que sobe para a propriedade do Sr. Antoninho Luccas, logo após o término 
 da subida íngreme com calçamento apenas nos trilhos, no final da floresta e início do campo nativo, alinhado 
 (\SI{\pm20}{\cm}) com a cerca dos dois lados da estrada. Locado \SI{1}{\m} distante do moirão da cerca do lado 
 esquerdo subindo, no interior da estrada.
\end{description}

Attribute validation of soil, geological, and land use maps, and DEMs was done using a set of $n = 60$ 
validation points located along $m = 12$ linear transects. The procedures for obtaining soil, geological, land 
use, and elevation data at these validation points is described in \autoref{apen:soil-data}. Such a validation 
exercise was carried out because these maps originally have no accompanying validation information.

% TODO: figure with GCPs used to ortorectify orbital images. Show the bounding box of the image and the 
% boundary 
% of the study area.
% \begin{figure}
%  \centering
%  \includegraphics[width=\textwidth]{fig/ortho-gcps}
%  \caption{Ground control points used to orthorectify orbital the image produced by Landsat-5 Thematic 
% Mapper.}
%  \label{fig:covar-data-ortho-gcps}
% \end{figure}

\tocless\section{Area-class soil maps}
\label{sec:covar-data-soil-maps}

Two area-class soil maps are included in the Santa Maria dataset. The first of them (\soilOld{}) was published 
at a \scale{100000} \cite{AzolinEtAl1988}. Existing area-class soil maps and technical reports 
\cite{Brasil1973, Azolin1977, MacielEtAl1987a, MacielEtAl1987, AbraoEtAl1988}, and sparse field observations 
were used to elaborate the preliminary legend of the soil map. Aerial photographs (\scale{60000}) were used to 
produce the first draft of the soil map. Field checks of soil polygons was done along the road network (i.e. 
convenience sampling). These observations were used to estimate the composition (occurrence and spatial 
distribution of soil taxa) of soil mapping units. They were also used to review the first draft of the soil 
map. The final version of \soilOld{} was prepared using topographic maps originally published at a 
\scale{50000} and resampled to a \scale{100000}. Soil classification followed the criteria adopted by the 
Brazilian soil science community at that time \cite{Brasil1973, CamargoEtAl1982, Carvalho1982, LemosEtAl1982, 
OlmosEtAl1982}. Identification of soil taxa was performed based on morphological features, analytical data 
compiled from existing technical reports, and analysis of soil samples collected from soil profiles observed 
along the road network. Description of each soil mapping unit includes the estimated area (\si{\hectare}) and 
the approximate taxonomic composition (\si{\percent}).

%TODO: complete R code!!!
\begin{figure}[!ht]
 \centering
 \begin{minipage}[b]{63mm}
  \subcaption{}
  \centering
  <<fig = TRUE, echo = FALSE>>=
   
   # Load less detailed soil map
   
  @
 \end{minipage}
 
 \begin{minipage}[b]{63mm}
  \subcaption{}
  \centering
  <<fig = TRUE, echo = FALSE>>=
   
   # Load more detailed soil map
   
  @
 \end{minipage} 
 \caption{Area-class soil maps (a) \soilOld{} and (b) \soilNew{} used to derive indicator covariates included 
 in the Santa Maria dataset.}
\label{fig:covar-data-soil-maps}
\end{figure}

The second area-class soil map (\soilNew{}) included in the Santa Maria dataset \cite{Miguel2010} was prepared 
at a \scale{25000}. Orbital images produced by Digital Globe\textregistered{} (Quick Bird satellite), freely 
available for visualization in Google Earth\textregistered{}, were used to produce the first draft of the soil 
map. Existing area-class soil maps and technical reports \cite{Pedron2005, Poelking2007, Sturmer2008} were used 
to help defining the preliminary soil map legend. Punctual field observations (auger holes) were made in more 
than \num{350} locations using a purposive sampling approach. These observations helped to identify six 
modal (representative) soil profiles. Soil sampling and description of modal soil profiles, and laboratory 
analyses of soil samples, followed the standard protocols adopted in Brazil \cite{ClaessenEtAl1997, 
SantosEtAl2005}. Soil classification was done following the criteria of the Brazilian System of Soil 
Classification (SiBCS) \cite{SantosEtAl2006}. The final version of the map was prepared using orbital images 
freely available for visualization in Google Earth\textregistered{} and manually-digitalized topographic maps 
published at a \scale{25000} \cite{DSG1992a, DSG1992}. Description of soil mapping units includes only the 
most common soil taxon, followed by morphological and laboratory data of modal soil profiles.

Both area-class soil maps went through different preprocessing routines. The original \soilOld{} is available 
only in the analogical format, what required its digitalization. Georeferencing was carried out using the GDAL 
Georeferencer plug-in in QGIS \cite{GDAL2013, QGIS2013}. Intersections between all meridians and parallels (a 
total of nine) were used as control points to adjust a second order polynomial model. Resampling was performed 
using the cubic resampling method. Soil polygons and their attributes were also manually digitalized in QGIS. 
Because of the coarseness on the cartographic map scale, most geographical markers used to locate validation 
GCPs could not be identified and positional validation was performed using only four GCPs. Estimated error 
statistics suggest that there are large positional errors in all directions, with an RMSE of \SI{114}{\m} and 
a mean azimuth of \SI{128}{\degree} (\autoref{tab:covar-data-soil-geo-val}).

\begin{table}[ht]
 \caption{Error statistics of the horizontal validation of \soilOld{} using $n = 4$ GCPs.}
 \label{tab:covar-data-soil-geo-val}
 \centering
 {\small
 \begin{tabular}{lrrrr}
  \hline
  Statistics                   & x-coord & y-coord & Error vector & Azimuth   \\
  \hline
  Mean, \si{\m}                & 30      & -36     & 105          & \ang{128} \\ 
  Absolute mean, \si{\m}       & 58      & 64      & -            & -         \\ 
  Squared mean, \si{\m\square} & 7241    & 5712    & 12953        & -         \\ 
  \hline
 \end{tabular}}
\end{table}

% \begin{figure}[!ht]
%   \centering
%   \includegraphics[width=0.45\textwidth]{azim-soil100}
%   \includegraphics[width=0.45\textwidth]{azim-soil25}
%   \caption{Histogram of the azimuth distribution of the validation of area-class soil maps \soilOld{} 
% and \soilNew{} in the attribute space. Azimuth values were estimated using, respectively, four and
% GCPs located in easily identifiable geographical markers. Estimates were corrected to the size of the 
% population. The graph was produced using R-package \textit{VecStatGraphs2D}.}
%   \label{fig:soil-azim}
% \end{figure}

The original \soilNew{} is available in digital format in the personal database of the author 
\cite{Miguel2010}. A topology check (Topology Checker plug-in in QGIS) identified that there were many gaps 
and overlaps between polygons. This required a topological edition prior to the use of \soilNew{}. There also 
was a mismatch between the boundary of \soilNew{} and the actual boundary of the study area as estimated using 
\demNew{} (\autoref{sec:covar-data-dem}). This occurred because the database used to produce \soilNew{} 
included Google Earth imagery and topographic maps, which are data sources that differ 
considerably in their positional accuracy (\autoref{sec:covar-data-dem} and \autoref{sec:covar-data-land-use}). 
To avoid data losses, all boundary gaps were manually filled using the closest mapping unit. Boundaries of soil 
polygons were defined based on land use (\landNew{}, \autoref{sec:covar-data-land-use}) and topographic data 
(contour lines, \autoref{sec:covar-data-dem}) as it was done for the original map \cite{Miguel2010}. New 
delineations were checked and approved without modifications by the author of the original map. Because 
\soilNew{} includes very few geographical markers, its positional validation was not possible with the 
available GCPs. However, the RMSE is expected to vary between \SIrange{8}{114}{\m} across the study area as a 
result of the different errors present in the data sources used in its production.

Both \soilOld{} and \soilNew{} were cropped to the bounding box of the study area, and geometrically corrected 
to match the reference grid. Registration and geocoding was performed using the nearest neighbour resampling 
method to maintain data integrity. Each category was named with the code of the respective mapping unit in the 
original map. Prior to validation in the attribute space, class codes of \soilOld{} were changed to match soil 
taxa codes of the current Brazilian System of Soil Classification using a standard correlation table 
\cite{SantosEtAl2006}. The overall purity of both soil maps is not considerably different 
(\autoref{tab:covar-data-soil-attr-val}). A reason for this could be that validation was performed considering 
only the second level of the SiBCS -- it is likely that \soilNew{} would outperform \soilOld{} if validation 
data included soil taxa up to the fourth level of the SiBCS. The low overall purity of \soilOld{} and 
\soilNew{} (\num{31.67} and \SI{30.00}{\percent}, respectively) is likely due to several sources of error. 
First, the small number of modal soil profiles used to produce both maps which might resulted in an optimistic 
view of the homogeneity of each mapping unit. Second, soil taxa of \soilOld{} was translated to the current 
SiBCS using only a standard correlation table \cite{SantosEtAl2006} and expert knowledge because the survey 
report does not include analytical soil data. Last, soil taxa at validation points was inferred in the field 
using only morphological soil properties and the concepts of the SiBCS.

Five ($p = 5$) covariates were derived from \soilOld{} as described bellow (including the soil taxa according 
to the original and updated classification \cite{AzolinEtAl1988, SantosEtAl2013a}, and the international 
classification \cite{IUSSWorkingGroupWRB2007}):

\begin{description}
 \item[\tt{SOIL\_100b}] Shallow soil (\it{Re4}) with low to high base saturation covering mountainous 
 terrain (Solo Litólico Eutrófico/Distrófico relevo montanhoso; Neossolo Litólico Distrófico/Eutrófico;
 Distric/Eutric Leptosol).
  
 \item[\tt{SOIL\_100c}] Association (\it{Re-C-Co}) of shallow soil with high base saturation located in
 strongly sloping terrain (Solo Litólico Eutrófico relevo forte ondulado; Neossolo Litólico Eutrófico; Eutric
 Leptosol), low weathered soil (Cambissolo Eutrófico; Cambissolo Háplico Eutrófico; Eutric Cambisol), and
 colluvial deposits.
  
 \item[\tt{SOIL\_100d}] Association (\it{TBa-Rd}) of deep, well-structured, low base saturation soil (Terra
 Bruna Estruturada álica; Nitossolo; Nitisol), and shallow soil (Solo Litólico; Neossolo Litólico; Leptosol).
  
 \item[\tt{SOIL\_100e}] Composed of shallow soil (\it{Rd1} and \textit{Re4}) with low to high base
 saturation (Solo Litólico Distrófico/Eutrófico; Neossolo Litólico Distrófico/Eutrófico; Distric/Eutric
 Leptosol) located in slopping and mountainous terrain.
  
 \item[\tt{SOIL\_100f}] This covariate includes the best soil mapping units for crop agriculture among those 
 identified in the soil survey, such as \it{TBa-Rd}, already described above, and \it{C1}, which is composed
 of low weathered soils developed in lower landscape positions, close to drainage channels (Cambissolo
 Eutrófico; Cambissolo Eutrófico; Eutric Cambisol).
\end{description}

Covariates derived from \soilNew{} are the following (mapping unit \it{RY}, composed mainly of soils developed 
from fluvial deposits (Neossolo Flúvico; Fluvisol) does not appear due to the small area that it occupies):

\begin{description}
  \item[\tt{SOIL\_25a}] Moderately deep soil (\it{PBAC}) derived from sedimentary rocks, with abrupt textural
  change and low base saturation (Argissolo Bruno-Acinzentado; Alisol).
  
  \item[\tt{SOIL\_25b}] Deep soil (\it{PV}) derived from igneous rocks, with moderate textural gradient,
  and low base saturation (Argissolo Vermelho; Acrisol).
 
  \item[\tt{SOIL\_25c}] Low weathered soil (Cambissolo; Cambisol) and shallow soil with low to high base
  saturation (Neossolo Litólico/Regolítico Eutrófico/Distrófico; Eutric/Distric Leptosol/Regosol) (\it{C-R}).
 
  \item[\tt{SOIL\_25d}] Shallow soil (\it{RL}) with low to high base saturation (Neossolo Litólico 
  Eutrófico/Distrófico; Eutric/Distric Leptosol).
 
  \item[\tt{SOIL\_25h}] This covariate includes the mapping units with the best soil for crop agriculture
  among those identified in the soil survey (\it{PBAC}, \it{PV}, and \it{SX}). \it{PBAC} and \it{PV} are as
  described above. \textit{SX} is composed of moderately deep soil derived from sedimentary rocks, with abrupt 
  textural change, low base saturation, and which are saturated with water for long periods of the year 
  (Planossolo Háplico; Planosol).
  
  \item[\tt{SOIL\_25i}] This covariate includes all three mapping units (\it{RL}, \it{RL-RR}, and \it{RR})
  composed mainly of shallow soil (Neossolo Litólico and Neossolo Regolítico; Leptosol and Regosol).
  
  \item[\tt{SOIL\_25j}] This covariate includes all four mapping units (\it{PV}, \it{RL}, \it{RL-RR}, and 
  \it{C-R}) composed mainly of soil derived from igneous rocks.
\end{description}

\tocless\section{Digital elevation models}
\label{sec:covar-data-dem}

Two DEMs are include in the Santa Maria dataset as sources of covariates. The first DEM (\demNew{}) is the 
result of the interpolation of the contour lines of the most recent topographic maps produced by the Brazilian 
Army (\scale{25000}) that cover the study \cite{DSG1980, DSG1992, DSG1992a}. Topographic maps were digitalized
and georeferenced using the GDAL Georeferencer plug-in in QGIS. Intersections between all meridians and 
parallels (about \num{160} per topographic map) were used as control points to adjust a third order polynomial 
model. Resampling was performed using the cubic resampling method. All contour lines, peaks, lakes and rivers, 
and their respective attributes within a distance of \SI{\pm1000}{\metre} from the boundary of the study area 
were manually digitalized and stored in the vector format. After digitalization, the original coordinate 
reference system (EPSG:31982 -- SIRGAS2000 / UTM zone 22S) of all vector files was transformed to WGS1984 / 
UTM zone 22S (EPSG:32722) using the \Rpackage{rgdal} \cite{BivandEtAl2013a}.

The positional validation of topographic maps was performed using $n = 14$ GCPs. According to Brazilian 
legislation, the positional accuracy of these topographic maps is expected to be of, at least, \SI{15}{\metre} 
\cite{Brasil1984}. Estimated validation statistics show that the observed error ($\text{RMSE} = \SI{65}{\m}$) 
is larger than established by current regulations (\autoref{tab:covar-data-topomap-geo-val}). The mean error 
vector is larger than \SI{60}{\metre} with an azimuth of \SI{63}{\degree}. Both x- and y-coordinates are 
positively biased, but the largest error occurs in the x-coordinate (\SI{50}{\metre}). Similar mean and mean 
absolute errors suggest that there is a strong systematic positional error. An affine transformation was 
employed using the \Rpackage{vec2dtransf} \cite{Carrillo2012} to correct this systematic error. Model 
parameters were adjusted using the same set of GCPs used for the validation in the geographic space.

\begin{table}[ht]
 \caption{Error statistics of the horizontal validation of topographic maps (\scale{25000}) using $n = 14$ 
 GCPs.}
 \label{tab:tab:covar-data-topomap-geo-val}
 \centering
 {\small
 \begin{tabular}{lrrrr}
  \hline
  Statistics                    & x-coord & y-coord & Error vector & Azimuth  \\
  \hline
  Mean, \si{\m}                 & 50      & 27      & 63           & \ang{63} \\ 
  Absolute mean, \si{\m}        & 50      & 32      & -            & -        \\ 
  Squared mean, \si{\m\squared} & 3088    & 1180    & 4268         & -        \\ 
  \hline
 \end{tabular}}
\end{table}

% \begin{figure}[!ht]
%   \centering
%   \includegraphics[width=0.5\textwidth]{azim-car25}
%   \caption{Histogram of the azimuth distribution of the validation of topographic maps in the attribute 
% space. Azimuth values were estimated using 14 GCPs located in easily identifiable geographical markers. 
% Estimates were corrected to the size of the population. The graph was produced using R-package 
% \textit{VecStatGraphs2D}.}
%   \label{fig:topomap-azim}
% \end{figure}

Interpolation of the raster surface with \SI{5}{\metre} grid size was performed using the function \tt{Topo 
to Raster} in ArcGIS\textregistered{} software by ESRI, which includes an interpolation method based on the 
ANUDEM program developed by \citeonline{Hutchinson1989}. Vector files of contour lines (\tt{multiline}), 
drainage network (\tt{multiline}), lakes (\tt{polygons}) and peaks (\tt{points}) were used to generate a 
hydrologically correct DEM, that is, a DEM without spurious depressions and giving a precise representation 
of the hydrological data. Next, the interpolated DEM was imported into GRASS, where a neighbourhood average 
filter was used to remove stair-like artefacts. A window of $7 \times 7$ pixels was used because it removed a 
significant amount of the artefacts and did not affect the derived boundary of the study area (see more 
bellow).

The vertical datum of the DEM was transformed from the local datum to a global datum. The geoidal models 
MAPGEO2010 \cite{IBGE2010a} and EGM1996 \cite{LemoineEtAl1998} were used to calculate the geoidal undulation 
for the local and global datums, respectively. MAPGEO2010 is optimized to estimate geoidal undulations in the 
Brazilian territory, while EGM1996 is a gravitational model of the Earth and is used as the vertical datum for 
SRTM products. The following equation was used:

\begin{equation}
 \label{eq:geoidal}
 h = H + N,
\end{equation}

\noindent where $h$ is the ellipsoidal height (height above the reference ellipsoid that approximates the 
surface of the planet), $H$ is the orthometric height (height above the imaginary surface called geoid and 
commonly referred as mean sea level), and $N$ is the geoidal undulation. Ellipsoidal heights estimated by 
MAPGEO2010 are referenced to the world ellipsoid of 1980, while EGM1996 estimates ellipsoidal heights 
referenced to the world ellipsoid of 1984. Because the difference between both ellipsoids is of the order of 
millimetres, it can be assumed that both models estimate the same ellipsoidal height. Therefore, if 
$h_{\text{EGM1996}} = h_{\text{MAPGEO2010}}$, then orthometric heights referenced to the local vertical datum 
can be transformed to the global vertical datum using the following equation:

\begin{equation}
 H_{\text{EGM1996}} = H_{\text{MAPGEO2010}} + N_{\text{MAPGEO2010}} - N_{\text{EGM1996}}.
\end{equation}

\noindent The difference in the geoidal undulation estimated by both models is of about one meter in the 
entire study area. Thus, transforming the vertical datum was done adding one meter to the raster surface 
interpolated from contour lines, yielding the first DEM included in the Santa Maria dataset (\demNew{}).

The second DEM (\demOld{}) is the well known SRTM DEM ($\SI{3}{\arcsecond} \approx \SI{\pm90}{\m}$ spatial 
resolution) produced by NASA’s Jet Propulsion Laboratory in collaboration with the National 
Geospatial\-/Intelligence Agency \cite{RodriguezEtAl2006}. The SRTM DEM version used here is the 
\emph{hole\-/filled SRTM version \num{4}}, prepared by the Consultative Group for International Agricultural 
Research (\cgiar) using the same hydrologically correct interpolation method that was used above to produce 
\demNew{} \cite{ReuterEtAl2007, JarvisEtAl2008}. However, the only data source used was the original SRTM DEM 
converted to point data.

The SRTM DEM was processed to match the reference grid using cubic resampling (\gdal{gdalwarp} and 
\grass{r.resamp.interp}). This resampling method was used because it is efficient in minimizing the 
double-oblique stripping present in SRTM products \cite{Samuel-RosaEtAl2013c}. Sinks produced during datum 
transformation were filled using the \grass{r.fill.dir}. Vertical datum transformation was not necessary 
because elevation values of the SRTM DEM already are referenced to the global geoidal model EGM1996 
(orthometric heights).

%TODO: complete R code!!!
\begin{figure}[!ht]
 \centering
 \begin{minipage}[b]{63mm}
  \subcaption{}
  \centering
  <<fig = TRUE, echo = FALSE>>=
   
   # Load less detailed DEM
   
  @
 \end{minipage}
 
 \begin{minipage}[b]{63mm}
  \subcaption{}
  \centering
  <<fig = TRUE, echo = FALSE>>=
   
   # Load more detailed DEM
   
  @
 \end{minipage} 
 \caption{Digital elevation models (a) \demOld{} and (b) \demNew{} used to derive covariates included 
 in the Santa Maria dataset.}
\label{fig:covar-data-dem}
\end{figure}

A third DEM was used to produce the covariate data included in the Santa Maria dataset. However, instead of 
calculating DEM derivatives, it was used in the orthorectification and topographic correction of orbital 
images (\autoref{sec:covar-data-sat-image}). The third DEM (\topodata) was produced by the Brazilian National 
Institute for Space Research (\inpe) by refining the original SRTM DEM to $\SI{1}{\arcsecond} \approx 
\SI{\pm30}{\m}$ spatial resolution using ordinary kriging with a Gaussian spatial autocorrelation model 
\cite{ValerianoEtAl2012}. Eight tiles were mosaicked (\gdal{gdal\_translate}) and the CRS transformed from 
WGS1984 (EPSG:4326) to WGS1984 / UTM zone 22S (EPSG:32722) using cubic resampling (\gdal{gdalwarp}), the
sinks filled using \grass{r.fill.dir}. Before the atmospheric correction of orbital images, orthometric 
heights were converted to ellipsoidal heights using \autoref{eqn:geoidal}, the geoidal undulation calculated 
with the gravitational model EGM1996. This conversion was done because because orbital satellites use the 
WGS1984 ellipsoid as vertical datum. For the topographic correction of orbital images, TOPODATA was then 
processed using \grass{r.resamp.interp} with the bicubic resampling method to match the reference grid.

The three DEMs present similar vertical accuracy (\autoref{tab:covar-data-dem-attr-val}). In the case of 
\demNew{}, which was derived from contour lines published at a \scale{25000}, the estimated accuracy does not 
meet current Brazilian legislation, which states that the accuracy should be of, at least, \SI{5}{\metre}, i.e.
\num{1/2} of the distance between contour lines \cite{Brasil1984}.

\begin{table}[ht]
 \caption{Error statistics of the vertical validation of \demOld{}, TOPODATA, and \demNew{} using $n = 60$
 validation points located along $m = 12$ linear transects.}
 \label{tab:covar-data-dem-attr-val}
 \centering
 {\small
 \begin{tabular}{lrrrrrr}
  \hline
  Statistics                   & \demOld{} & TOPODATA & \demNew{} \\
  \hline
  Mean, \si{\m}                & -15       & -17      & -16       \\ 
  Absolute mean, \si{\m}       & 15        & 17       & 16        \\ 
  Squared mean, \si{\m\square} & 350       & 361      & 374       \\ 
  \hline
 \end{tabular}}
\end{table}

% Figure \ref{fig:cdf-elev} shows that estimated validation statistics have different cumulative 
% distribution functions (CDF). The estimates are more uniformly distributed along the interval of 
% values for \tt{ELEV\_10} than for \tt{ELEV\_90} and \tt{ELEV\_30}. While 
% \tt{ELEV\_10} has a 50\% probability that absolute errors are bellow 15 m, \tt{ELEV\_90} has 
% a 70\% probability that absolute errors are bellow 15 m. This suggests that the accuracy of 
% \tt{ELEV\_90} is very consistent across the study area, with a few extreme values, while the 
% accuracy of \tt{ELEV\_10} have a stronger spatial variation. For \tt{ELEV\_30}, the 
% interpolation method used to refine the original SRTM DEM to 30 m \cite{ValerianoEtAl2012} seems to 
% have produced a spatial redistribution of the errors.

% \begin{figure}[!ht]
%   \centering
%   \includegraphics[width=0.9\textwidth]{fig/cdf-ELEV-90} 
%   \includegraphics[width=0.9\textwidth]{fig/cdf-ELEV-30}
%   \includegraphics[width=0.9\textwidth]{fig/cdf-ELEV-10}
%   \caption{Cumulative distribution functions of mean error, mean absolute error, and squared error of 
% elevation values estimates by digital elevation models \tt{ELEV\_90}, \tt{ELEV\_30}, and 
% \tt{ELEV\_10}.}
%   \label{fig:cdf-elev}
% \end{figure}

Despite all DEMs present a similar vertical accuracy, \demNew{} is considered the highest quality DEM in the 
Santa Maria dataset. Because it was produced using information about the drainage network and location of lakes
and natural depressions, it is likely to provide a better hydrological representation of the study area. As 
such, \demNew{} was used to estimate the geographical limits (boundary) of the catchment that constitute the 
study area (\autoref{fig:covar-data-elev-maps}), for which GRASS modules \tt{r.watershed} and 
\tt{r.water.outlet} were employed. Because the overall deviation between the affine-corrected coordinates of 
topographic maps and target coordinates of GCPs is $\text{RMSE} = \SI{29.55}{\m}$ -- there still is an 
uncertainty about the correct position of topographic maps -- a \SI{30}{\m} buffer was added to the estimate 
of the geographical limits of the catchment. The water outlet point used to estimate the boundary is located 
on the bridge that crosses the main drainage channel (\ang{-29.65868}, \ang{-53.78969}).

Eight terrain attributes were derived from each of \demOld{} and \demNew{} to produce the covariate data 
included in the Santa Maria dataset, the first of them being elevation (\tt{ELEV}). The others are slope, 
aspect, northernness, flow accumulation, topographic wetness index, stream power index, and topographic 
position index.

Slope (\tt{SLP}) and aspect (\tt{ASP}) were calculated using \grass{r.param.scale}. This module calculates 
terrain attributes fitting a bivariate quadratic polynomial using least squares \cite{Wood1996}. It allows 
using different window sizes to fit the bivariate quadratic polynomial, thus including the effect of scale in 
the calculation of terrain attributes. Seven window sizes were used (3, 7, 15, 31, 63, 127, and 255) and the 
results for calculated slope can be seen in \autoref{fig:covar-data-slope}. Larger window sizes result in a 
smoothed version of the terrain attribute, while smaller windows sizes result in raster maps with more 
(small-scale) details. Several flat surfaces (slope of \ang{0}) were produced in the slope raster maps 
calculated using \tt{ELEV\_90} as a result of resampling the original DEM from \num{90} to \SI{5}{\m}. A 
value of \ang{0.1} was added to the rasters to remove these flat surfaces.

\begin{figure}[!ht]
 \centering
 <<covar-data-slope, fig = TRUE, echo = FALSE>>=
  
  # Load slope raster surfaces from GRASS
  invisible(capture.output(slp <- spgrass6::readRAST6("SLP_10_3", ignore.stderr = TRUE)))
  colnames(slp@data) <- "SLP_10_3"
  slp@data[, "SLP_10_31"] <- invisible(
    capture.output(spgrass6::readRAST6("SLP_10_31", ignore.stderr = TRUE)@data[, 1]))
  slp@data[, "SLP_10_255"] <- invisible(
    capture.output(spgrass6::readRAST6("SLP_10_255", ignore.stderr = TRUE)@data[, 1]))
  
  # Prepare plots
  slp <- sp::spplot(
    slp, scales = list(draw = TRUE), aspect = "iso", col = topo.colors(64),
    par.settings = list(
      fontsize = list(text = 12, points = 8),
      layout.widths = list(left.padding = 0, right.padding = 0),
      layout.heights = list(top.padding = 0, bottom.padding = 0))
    }) + latticeExtra::layer(sp::sp.points(pts, col = "red", pch = 20))
  
  # Plot and clean workspace
  print(slp)
  rm(slp)
  gc()
@
 \caption{Slope \tt{SLP}} raster surfaces derived from \demNew{} using windows of sizes (from left to right) 
 $3 \times 3$, $31 times 31$, and $255 \times 255$.}
\label{fig:covar-data-slope}
\end{figure}

Aspect values had to be corrected before use because \grass{r.param.scale} stores aspect values in the range 
\SIrange{0}{+180}{\degree} from West to North to East, and \SIrange{0}{-180}{\degree} from West to South to 
East, when the standard procedure is to work with aspect values ranging from \SIrange{0}{360}{\degree} 
clockwise. This correction was done using

\begin{equation}
 \tt{ASP}_{0} =
 \begin{cases}
  \tt{ASP}_{GRASS} + \ang{360} & \text{if}\;\; \tt{ASP}_{GRASS} < \ang{0}, \\
  \tt{ASP}_{GRASS}             & \text{else},
 \end{cases}
\end{equation}

\noindent and

\begin{equation}
 \tt{ASP} =
 \begin{cases}
  \tt{ASP}_{0} + \ang{270} & \text{if}\;\; \tt{ASP}_{0} < \ang{90}, \\
  \tt{ASP}_{0} - \ang{90}  & \text{else}.
 \end{cases}
\end{equation}

\noindent A second correction of aspect values involved their linearization. This is necessary because aspect 
is a circular variable, that is, the begging (\ang{0}) and end (\ang{360}) of the measurement scale have the 
same physical meaning. Aspect values were transformed to northernness (\tt{NOR}), a measure of the degree 
of exposition of a given surface to the North, a linear variable, using the equation

\begin{equation}
 \tt{NOR} = abs(\ang{180} - \tt{ASP}).
\end{equation}\label{eq:NOR}  

Flow accumulation (ACC), also known as catchment area and contributing area, was calculated using 
\grass{r.watershed}, the resulting raster surface being multiplied by the square of the cell size. This raster 
surface was used to calculate the topographic wetness index (\tt{TWI}) and the stream power index (\tt{SPI}) 
using

\begin{equation}
 \text{sACC} = \dfrac{\text{ACC}}{5},
\end{equation}\label{eq:sACC}

\begin{equation}
 \tt{TWI} = log \dfrac{\text{sACC}}{tan(\tt{SLP})},
\end{equation}\label{eq:TWI}

\noindent and

\begin{equation}
 \tt{SPI} = log(\text{sACC} \times tan(\tt{SLP})),
\end{equation}\label{eq:SPI}

\noindent where sACC is the specific catchment area, \SI{5}{\m} is the cell size, and \tt{SLP} is the slope 
raster surface calculated using seven different window sizes.

The topographic position index \tt{TPI} was calculated with \saga{ta\_morphometry}. Different values of 
maximum radius were used to include the effect of scale, all of them related to the window sizes used to 
calculate previous terrain attributes. A minimum radius value of \SI{3}{\m} was used in all calculations.

A total of $1 \times \tt{ELEV} + 7 \times (\tt{NOR}, \tt{SLP}, \tt{TWI}, \tt{SPI}, \tt{TPI}) = 36$ covariates 
were defined using the terrain attributes derived from \demOld{} and \demNew{}.

\tocless\section{Geological maps}
\label{sec:covar-data-geo-maps}

Geological data comes from the two most recent geological maps (\geoOld{} and \geoNew{}) published in the 
\scales{50000}{25000} \cite{GasparettoEtAl1988, MacielFilho1990} (\autoref{fig:covar-data-geo-maps}). Both of 
them were produced based on the most recent topographic maps produced by the Brazilian Army at the 
\scales{50000}{25000} \cite{DSG1980, DSG1992, DSG1992a}. Alike topographic maps, geological maps were available 
only in the analogical format, and were hand digitalized and georeferenced in QGIS. Intersections between all 
meridians and parallels ($n = 16$) were used as control points to adjust a second order polynomial model. 
Resampling was performed using the cubic resampling method. The original coordinate reference system 
(EPSG:31982 -- SIRGAS2000 / UTM zone 22S) was transformed to match the reference grid using the 
\Rpackage{rgdal} \cite{BivandEtAl2013a}.

\begin{figure}[!ht]
 \centering
 \begin{minipage}[b]{63mm}
  \subcaption{}
  \centering
  <<fig = TRUE, echo = FALSE>>=
   
   # Load less detailed geological map
   map <- spgrass6::readRAST6("recGEO_50")
   
   # Prepare data
   map@bbox <- pts@bbox
   map@data[, 1] <- as.factor(map@data[, 1])
   levels(map@data[, 1])
   map@data[, 1] <- revalue(
     map@data[, 1], c("1" = "SG-I",  # Serra Geral (inferior)
                      "2" = "BT",    # Botucatu
                      "3" = "CT",    # Caturrita
                      "4" = "SG-S")) # Serra Geral (superior)
   
   # Prepare plot
   map <- spplot(
     map, main = "", col.regions = topo.colors(4), asp = 1) +
     latticeExtra::layer(sp::sp.points(pts, col = "red", pch = 20))
   names(map$legend) <- "inside"
   map$legend$inside$x <- 0.795
   map$legend$inside$y <- 0.875
   map$legend$inside$args$key$space <- "left"
   
   # Plot
   plot(map)
  @
 \end{minipage}
 
 \begin{minipage}[b]{63mm}
  \subcaption{}
  \centering
  <<fig = TRUE, echo = FALSE>>=
   
   # Load more detailed geological map
   map <- spgrass6::readRAST6("predGEO_25")
   map@bbox <- pts@bbox
   map@data[, 1] <- as.factor(map@data[, 1])
   map@data[, 1] <- revalue(
     map@data[, 1], c("1" = "SG-I",  # Serra Geral (inferior)
                      "2" = "BT",    # Botucatu
                      "3" = "CT",    # Caturrita
                      "4" = "SG-S")) # Serra Geral (superior)
   map <- as(map, "SpatialPixelsDataFrame")
   system("r.to.vect --o in=GEO_25d out=poly feature=area")
   poly <- spgrass6::readVECT6("poly")
   poly@bbox <- pts@bbox
   map <- spplot(
     map, main = "", col.regions = topo.colors(4), asp = 1, 
     sp.layout = list("sp.polygons", poly, first = FALSE, 
     col = ifelse(poly$value == 0, "transparent", " black"), 
     fill = ifelse(poly$value == 0, "transparent", "lightgray"), alpha = 0.5))
   names(map$legend) <- "inside"
   map$legend$inside$x <- 0.795
   map$legend$inside$y <- 0.875
   map$legend$inside$args$key$space <- "left"
   map$legend$inside$args$key$at <- c(map$legend$inside$args$key$at, 5.5)
   map$legend$inside$args$key$col <- c(map$legend$inside$args$key$col, "lightgray")
   map$legend$inside$args$key$labels$at <- c(map$legend$inside$args$key$labels$at, 5)
   map$legend$inside$args$key$labels$labels <- c(map$legend$inside$args$key$labels$labels, "QD")
   
   # Plot and clean workspace
   plot(map)
   rm(map, poly)
   gc()
  @
 \end{minipage} 
 \caption{Geological maps (a) \geoOld{} and (b) \geoNew{} used to derive indicator covariates included in the 
 Santa Maria dataset.}
\label{fig:covar-data-geo-maps}
\end{figure}

The positional validation of geological maps was performed using $n = 8$ (\geoOld{}) and $n = 5$ (\geoNew{}) 
GCPs. Validation statistics show that the positional accuracy of both geological maps does not meet the current
regulations of the Brazilian legislation (\autoref{tab:covar-data-geology-geo-val}). Estimated RMSE are 
\num{147} and \SI{69}{\m} for \geoOld{} and \geoNew{}, respectively, when the maximum RMSE accepted is 
\num{30} and \SI{15}{\m}. For \geoOld{}, the lowest accuracy is found in the y-coordinate, while for 
\geoNew{}, the x-coordinate is the least accurate. Validation statistics suggest that there is a strong 
systematic error, which probably was propagated from the topographic maps used to produce the geological maps.
Therefore, the same strategy (affine transformation) used to remove the systematic positional error of the 
topographic maps was employed on geological maps. Due to the lack of GCPs, model parameters were adjusted 
using the same set of GCPs used for the validation. The estimated uncertainty (RMSE) of the affine 
transformation is \num{86} and \SI{22}{\m} for \geoOld{} and \geoNew{}, respectively. Curiously enough, the
attribute validation indicates that \geoOld{} is more accurate (\SI{83.33}{\percent} than \geoNew{} 
\SI{76.67}{\percent}.

\begin{table}[ht]
 \caption{Error statistics of the horizontal validation of geological maps \geoOld{} and \geoNew{} using 
 $n = 8$ and $n = 5$ GCPs.}
 \label{tab:covar-data-geology-geo-val}
 \centering
 {\small
 \begin{tabular}{lrrrr}
  \hline
  Statistics                   & x-coord & y-coord & Error vector & Azimuth   \\
  \hline
  \multicolumn{5}{l}{\geoOld{} ($n = 8$)}                                     \\
  \hline
  Mean, \si{\m}                & 10      & -102    & 140          & \ang{169} \\ 
  Absolute mean, \si{\m}       & 43      & 125     & -            & -         \\ 
  Squared mean, \si{\m\square} & 3431    & 18067   & 21498        & -         \\
  \hline
  \multicolumn{5}{l}{\geoNew{} ($n = 5$)}                                     \\
  \hline
  Mean, \si{\m}                & 51      & 29      & 67           & \ang{58}  \\ 
  Absolute mean, \si{\m}       & 51      & 29      & -            & -         \\ 
  Squared mean, \si{\m\square} & 3457    & 1312    & 4769         & -         \\
  \hline
 \end{tabular}}
\end{table}

% \begin{figure}[ht]
%  \centering
%  \caption{Histogram of the azimuth distribution of the validation of geological maps 
%  \tt{GEO\_50} (left) and \tt{GEO\_25} (right) in the attribute space. Azimuth values were 
%  estimated using, respectively, eight and five GCPs located in easily identifiable geographical 
%  markers. The graph was produced using \Rpackage{VecStatGraphs2D}.}
%  \label{fig:covar-data-geology-azim}
% \end{figure}

Three ($p = 3$) covariates were derived from \geoOld{}:

\begin{description}
 \item[\tt{GEO\_50a}] Inferior Sequence of the Serra Geral Formation. Composed mainly by basic 
 igneous rocks (tholeiitic basalt and andesite). It is likely to be related with high \tt{CLAY} and 
 \tt{ECEC};
 
 \item[\tt{GEO\_50b}] Superior Sequence of the Serra Geral Formation. Composed mainly by acid 
 igneous rocks (granophyric rhyolite and rhyodacite). It is likely to be related with moderate to 
 high \tt{CLAY} and \tt{ECEC};
 
 \item[\tt{GEO\_50c}] Botucatu Formation. Composed mainly by aeolian sandstones. It is likely 
 to be related with low \tt{CLAY} and \tt{ECEC};
\end{description}

Four ($p = 4$) covariates were derived from \geoNew{}, the first three of them having the same meaning 
of those derived from \geoOld{}:

\begin{description}
 \item[\tt{GEO\_25a}] Inferior Sequence of the Serra Geral Formation;
 
 \item[\tt{GEO\_25b}] Superior Sequence of the Serra Geral Formation;
 
 \item[\tt{GEO\_25c}] Botucatu Formation;
 
 \item[\tt{GEO\_25d}] Quaternary deposits of fluvial, alluvial, and colluvial 
 origin. It can help explaining the low \tt{CLAY} of soils supposedly derived from igneous rocks.
\end{description}

\tocless\section{Land use maps}
\label{sec:covar-data-land-use}

The first land use map used to derive covariate data included in the Santa Maria dataset was produced by 
manually digitizing land use data of the year of \num{1980} (\landOld{}) published in the most recent 
topographic map produced by the Brazilian Army (\scale{25000}) \cite{DSG1992a, DSG1992}. Most processing steps, 
including the correction of positional bias, are described in \autoref{sec:covar-data-dem}, except for the use 
of the nearest neighbour resampling method to match \landOld{} to the reference grid.

\begin{figure}[!ht]
 \centering
 \begin{minipage}[b]{63mm}
 \subcaption{}
 \centering
 <<covar-data-land-use-1, fig = TRUE, echo = FALSE>>=
 
  # Load less detailed land use map
  invisible(capture.output(land <- spgrass6::readRAST6("LU1980", ignore.stderr = TRUE)))
  
  # Prepare plot
  land <- sp::spplot(
    land, scales = list(draw = TRUE), aspect = "iso", col = topo.colors(64)) + 
    latticeExtra::layer(sp::sp.points(pts, col = "red", pch = 20))
  
  # Plot
  print(land)
 @
 \end{minipage}
 \begin{minipage}[b]{63mm}
 \subcaption{}
 \centering
 <<covar-data-land-use-2, fig = TRUE, echo = FALSE>>=
  
  # Load more detailed land use map
  invisible(capture.output(land <- spgrass6::readRAST6("LU2009", ignore.stderr = TRUE)))
  
  # Prepare plot
   land <- sp::spplot(
    land, scales = list(draw = TRUE), aspect = "iso", col = topo.colors(64),
    par.settings = list(
      fontsize = list(text = 12, points = 8),
      layout.widths = list(left.padding = 0, right.padding = 0),
      layout.heights = list(top.padding = 0, bottom.padding = 0))
    }) + latticeExtra::layer(sp::sp.polygons(pol, col = "red", lty = "dashed"))
  
  # Plot
  print(land)
 @
 \end{minipage}
 \begin{minipage}[b]{63mm}
 \subcaption{}
 \centering
 <<covar-data-land-use-3, fig = TRUE, echo = FALSE>>=
  
  # Load land use change raster surface
  invisible(capture.output(land <- spgrass6::readRAST6("LUdiff", ignore.stderr = TRUE)))
  
  # Prepare plot
  land <- sp::spplot(
    land, scales = list(draw = TRUE), aspect = "iso", col = topo.colors(64),
    par.settings = list(
      fontsize = list(text = 12, points = 8),
      layout.widths = list(left.padding = 0, right.padding = 0),
      layout.heights = list(top.padding = 0, bottom.padding = 0))
    }) + latticeExtra::layer(sp::sp.polygons(pol, col = "red", lty = "dashed"))
  
  # Plot and clean workspace
  print(land)
  rm(land)
 @
 \end{minipage}
  \caption{Land use maps (a) \landOld{} and (b) \landNew{} used to derive indicator covariates included in 
  the Santa Maria dataset such as (c) \tt{LUdiff}, the land use difference between the years of 1980 and 
  2009.}
 \label{fig:covar-data-land-use}
\end{figure}

The second land use map (\landNew{}) was prepared at a \scale{2000} using high resolution (\SI{2.4}{\m}) 
orbital images, produced by QuickBird satellite launched in 2001, of the years of \num{2008} and \num{2009}, 
made publicly available in Google Earth \cite{SamuelRosaEtAl2011a}. Identification of land uses and delineation 
of mapping units was done manually, on the computer screen, without using any automated classification routine. 
Positional validation of Google Earth imagery revealed that they have only minor systematic positional errors 
(\autoref{tab:covar-data-google-geo-val}). Despite of this, the attribute validation of both land use maps, 
using $n = 60$ validation points placed along $m = 12 linear transects$, showed that they have similar 
overall accuracy, \tt{LU1980} (\SI{70.00}{\percent}) slightly outperforming \tt{LU2009} (\SI{66.67}{\percent}).

\begin{table}[ht]
 \caption{Error statistics of the horizontal validation of Google Earth\textregistered imagery using $n = 14$ 
 GCPs.}
 \label{tab:covar-data-google-geo-val}
 \centering
 {\small
 \begin{tabular}{lrrrr}
  \hline
  Statistics                   & x-coord & y-coord & Error vector & Azimuth   \\
  \hline
  Mean, \si{\m}                & -1      & 3       & 6            & \ang{184} \\ 
  Absolute mean, \si{\m}       & 3       & 5       & -            & -         \\ 
  Squared mean, \si{\m\square} & 14      & 57      & 71           & -         \\ 
  \hline
 \end{tabular}}
\end{table}

\landOld{} was used to derive $p = 2$ covariates defined as indicator variables, with plantation forests 
(\it{PF}) and human settlements (\it{S}) being grouped together due to their small importance in terms of 
covered area (\it{PF}) and for not containing any soil observation (\it{S}). These covariates are:

\begin{description}
 \item[\tt{LU1980a}] Native forest (\it{FS}), which is likely to have soils with higher fertility.
  
 \item[\tt{LU1980b}] Animal husbandry (\it{H}), the second most important land use in the study area, which is
 likely to have a soil fertility status lower than native forests.
\end{description}

Four ($p = 4$) indicator covariates were derived from \landNew{}, with plantation forests (\it{PF}), human 
settlements (\it{S}), and other land uses (\it{O}), which comprise natural and artificial water bodies, being 
grouped together due to their small importance in terms of covered area (\it{PF}) and for not containing any 
soil observation (\it{S} and \it{O}). These covariates are:

\begin{description}
 \item[\tt{LU2009a}] Native forest (\it{FS}), as described above.
 
 \item[\tt{LU2009b}] Shrubland (\it{SS}), which is likely to have a soil fertility level above those found in
 areas used with annual crop agriculture and animal husbandry, but lower than in native forests.
 
 \item[\ttt{LU2009c}] Animal husbandry (\it{H}), as described above.
  
 \item[\tt{LU2009d}] Annual crop agriculture (\it{AA}), which is likely to present the lowest soil fertility 
 levels due to the usually poor management practices employed.
\end{description}

A seventh indicator covariate (\tt{LUdiff}) was derived using data from both land use maps. It consists of the 
land use difference between \num{1980} and \num{2009}, computed by checking if the land use has changed (1) or 
remained the same (0) in every grid cell after the 29-year period. \tt{LUdiff} can be useful, for example, to 
explain the low \tt{ORCA} in forest soils due to previous use with crop agriculture or animal husbandry.

\tocless\section{Orbital images}
\label{sec:covar-data-sat-image}

Two sources of orbital images were used to produce covariate data included in the Santa Maria dataset. The 
first is the longest-operating Earth observation satellite Landsat-5 Thematic Mapper, launched on \num{1} March 
\num{1984}. The orbital image used was acquired on \num{26} December \num{2010} and is available at the 
database of the Division of Image Generation of the National Institute for Space Research (\inpedgi). The 
image contains seven spectral bands (\autoref{tab:covar-data-satellites}) (including a thermal band that is not
included in the Santa Maria dataset), with \SI{8}{\bit} radiometric resolution (digital numbers from 
\numrange{0}{255}) and \SI{\pm30}{\m} spatial resolution. The orbital image was orthorectified using 
Geomatica\textregistered{} OrthoEngine\textregistered{} with the Landsat rigorous model (Toutin's Model). Due 
to the absence of field GCPs, $n = 28$ GCPs were collected from Google Earth\textregistered{} imagery, which 
have a high positional accuracy in the study area (\autoref{tab:covar-data-google-geo-val}). GCPs were located 
at easily identifiable geographical markers such as road intersection and bridges, evenly distributed 
throughout the image, and covering a variety of elevations, following standard recommendations 
\cite{PCIGeomatics2007} (\autoref{fig:covar-data-ortho-gcps}). The DEM used for orthorectification is TOPODATA 
after preprocessing as described in \autoref{sec:covar-data-dem}. Resampling was done using the nearest 
neighbour method to avoid changing the digital numbers.

After orthorectification, all bands were imported into GRASS, where all other necessary corrections were 
performed. Radiometric correction (conversion from digital numbers to top-of-atmosphere reflectance) was done
using \grass{i.landsat.toar}. Atmospheric correction (removal of the effects of the atmosphere on the 
reflectance values) was done with the 6S atmospheric model \cite{VermoteEtAl1997} as implemented in 
\grass{i.atcorr} using the tropical atmospheric model, the continental aerosols model, an image-based 
visibility estimate of \SI{20}{\km}, and a constant elevation of \SI{300}{\m}.

The second source of orbital images is the RapidEye constellation of five satellites, launched in August 
\num{2008}. Images are available through the Brazilian Ministry of the Environment \cite{Brasil2012}, who 
has a full coverage of the Brazilian territory with images from the RapidEye satellite constellation for the 
years of \num{2011} and \num{2012}. The orbital image used (tile number \num{2225403}) was acquired on \num{16} 
November \num{2012} (second coverage). It contains five spectral bands (\autoref{tab:covar-data-satellites}), 
featuring among them the so-called red-edge band, located between the red and the near-infrared bands. This 
spectral band is the main feature distinguishing RapidEye images from most other sources of orbital images, 
considered to provide additional information about the vegetation \cite{WeicheltEtAl2013}. The orbital image 
has \SI{16}{\bit} radiometric resolution and \SI{6.5}{\m} spatial resolution, and was orthorrectified in the 
source to \SI{5}{\m} spatial resolution using the hole-filled SRTM version \num{4} \cite{RapidEye2013}.

\begin{figure}[!ht]
 \centering
 \begin{minipage}[b]{63mm}
  \subcaption{}
  \centering
  <<fig = TRUE, echo = FALSE>>=
   
   # Load less detailed orbital image
   map <- spgrass6::readRAST6("NDVI_30")
   map <- sp::spplot(
    map, scales = list(draw = TRUE), aspect = "iso", main = "", col.regions = ndvi$colors, at = ndvi$breaks
    }) + latticeExtra::layer(sp::sp.points(pts, col = "red", pch = 20))
  
   # Plot
   print(map)
  @
 \end{minipage}
 
 \begin{minipage}[b]{63mm}
  \subcaption{}
  \centering
  <<fig = TRUE, echo = FALSE>>=
   
   # Load more detailed orbital image
   map <- spgrass6::readRAST6("NDVI_5b")
   
  @
 \end{minipage} 
 \caption{Orbital images used to derive covariates such as (a) \tt{NDVI_30} and (b) \tt{NDVI_5b} included in 
 the Santa Maria dataset.}
\label{fig:covar-data-sat-image}
\end{figure}

\begin{table}[ht]
 \caption{Comparison between satellite images produced by Landsat 5 TM and RapidEye constellation and 
 the derived covariates.}
 \label{tab:covar-data-satellites}
 \centering
 {\small
 \begin{tabular}{llllll}
  \hline
  \multicolumn{3}{l}{Landsat 5 TM}                       & \multicolumn{3}{l}{RapidEye}                     \\
  Band            & Interval, \si{nm}    & Covariate     & Band         & Interval, \si{\nm} & Covariate    \\
  \hline
  1 Visible       &\numrange{450}{520}   &\tt{BLUE\_30}  &Blue          &\numrange{440}{510} &\tt{BLUE\_5}  \\
  2 Visible       &\numrange{520}{600}   &\tt{GREEN\_30} &Green         &\numrange{520}{590} &\tt{GREEN\_5} \\
  3 Visible       &\numrange{630}{690}   &\tt{RED\_30}   &Red           &\numrange{630}{685} &\tt{RED\_5}   \\
  -               &-                     & -             &Red-edge      &\numrange{690}{730} &\tt{EDGE\_5}  \\
  4 Near-infrared &\numrange{760}{900}   &\tt{NIR\_30a}  &Near-infrared &\numrange{760}{850} &\tt{NIR\_5}   \\
  5 Near-infrared &\numrange{1550}{1750} &\tt{NIR\_30b}  & -            & -                  & -            \\
  7 Mid-infrared  &\numrange{2080}{2350} &\tt{MIR\_30}   & -            & -                  & -            \\
  \hline
 \end{tabular}}
\end{table}

RapidEye constellation image was atmospherically corrected using the 6S atmospheric model 
\cite{VermoteEtAl1997} employing the Fortran code developed by \citeonline{AntunesEtAl2013} -- \grass{i.atcorr} 
was not used because a \atcorrbug{} was found when trying to correct images from the RapidEye constellation -- 
assuming a tropical atmospheric model, the continental aerosols model, an image-based visibility estimate of 
\SI{20}{\km}, and a constant elevation of \SI{300}{\m}.

After atmospheric correction, both Landsat 5 TM and RapidEye constellation images were resampled using the 
nearest neighbour method to match the reference grid. Topographic correction (removal of the effects of the 
topography -- illumination -- on the reflectance values) was performed using \grass{i.topo.corr} with TOPODATA 
geometrically corrected to match the reference grid as described in \autoref{sec:covar-data-dem}.

\begin{table}[ht]
 \caption{Error statistics of the horizontal validation of orbital images produced by Landsat 5 TM and
 RapidEye constellation using $n = 14$ GCPs.}
 \label{tab:covar-data-satellite-geo-val}
 \centering
 {\small
 \begin{tabular}{lrrrr}
  \hline
  Statistics                   & x-coord & y-coord  & Error vector  & Azimuth   \\
  \hline
  \multicolumn{5}{l}{Landsat 5 TM}                                              \\
  \hline
  Mean, \si{\m}                & 31      & -11      & 45            & \ang{136} \\ 
  Absolute mean, \si{\m}       & 33      & 25       & -             & -         \\ 
  Squared mean, \si{\m\square} & 1494    & 1223     & 2717          & -         \\ 
  \hline
  \multicolumn{5}{l}{RapidEye}                                                  \\
  \hline
  Mean, \si{\m}                & -25     & -25      & 36            & \ang{226} \\ 
  Absolute mean, \si{\m}       & 25      & 25       & -             & -         \\ 
  Squared mean, \si{\m\square} & 680     & 708      & 1388          & -         \\ 
  \hline
 \end{tabular}}
\end{table}

Individual bands of both orbital images were defined as covariates, totalling $p = 6$ from Landsat 5 TM and 
$p = 5$ from RapidEye constellation (Table \ref{tab:covar-data-satellites}). Another $p = 6$ covariates ($p = 
2$ from Landsat 5 TM and $p = 4$ from RapidEye constellation) were defined using two vegetation indexes: the 
normalized difference vegetation index (NDVI) and the soil-adjusted vegetation index (SAVI), calculated as

\begin{equation}
 \text{NDVI} = \frac{\text{NIR} - \text{RED}}{\text{NIR} + \text{RED}}
\end{equation}\label{eq:ndvi}

\noindent and 

\begin{equation}
  \text{SAVI} = (1.0 + 0.5) \times \frac{\text{NIR} - \text{RED}}{\text{NIR} + \text{RED} + 0.5},
\end{equation}\label{eq:savi}

\noindent where $\text{NIR} = \tt{NIR\_30a}$ and $\text{RED} = \tt{RED\_30}$ for Landsat 5 TM image, and 
$\text{NIR} = \tt{NIR\_5}$ and $\text{RED} = \tt{RED\_5}$ or $\text{RED} = \tt{EDGE\_5}$ for RapidEye 
constellation image.
