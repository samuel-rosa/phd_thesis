Area-class soil map
Most existing soil maps use the area-class model of representation. It 
constitutes a \emph{discrete model of spatial variation} that divides the area being mapped into 
internally more homogeneous polygons sharing sharp and well-defined boundaries. Each polygon, 
understood as a mapping unit, receives a class name that is presented in the map legend. 
Statistically speaking, the main goal of the area-class model of representation, or equivalently 
the discrete model of spatial variation, is to minimize the within-class variance (and maximize 
the between-class variance) of the soil property being mapped.
