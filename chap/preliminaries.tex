


\begin{description}
 \item[Area-class soil map] Most existing soil maps use the area-class model of representation. It 
 constitutes a \emph{discrete model of spatial variation} that divides the area being mapped into 
 internally more homogeneous polygons sharing sharp and well-defined boundaries. Each polygon, 
 understood as a mapping unit, receives a class name that is presented in the map legend. 
 Statistically speaking, the main goal of the area-class model of representation, or equivalently 
 the discrete model of spatial variation, is to minimize the within-class variance (and maximize 
 the between-class variance) of the soil property being mapped.
 
 \item[Covariate] Covariates are used in quantitative spatial soil modelling because they are 
 proxies of soil-forming factors. For example, they provide information about topography, 
 vegetation, land use, geology, soil parent material, climate, soil itself and other 
 intimately-associated surface conditions.
 
 \item[Spatial detail] The level of spatial detail of covariate data is defined based on the data 
 sources and/or production methods, which demand different amounts of resources (time, workforce, 
 budget, technology, and so on). Specifically, the level of spatial detail of a covariate is a 
 function of the components of its production process such as the cartographic ratio, spatial 
 sampling support, number and diversity of data sources explored, and quantity of spatial data used. 
 This definition of spatial detail is broader than that of spatial resolution or spatial scale, and 
 should not be confounded with spatial support \cite{WebsterEtAl2007} or thematic detail 
 \cite{Rossiter2000}.
\end{description}