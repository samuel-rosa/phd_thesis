\artigofalse
\chapter{CONCLUSÃO GERAL}
\shorttitle{Conclusão Geral}
\label{appen:conclusion-pt}

Esta tese fez uma contribuição pedológica com o desenvolvimento de uma descrição abrangente dos fatores e 
processos de formação do solo que determinam a distribuição espaço-temporal das propriedades do solo na área 
de estudo de caso de Santa Maria. O modelo conceitual de pedogênese, apresentado em \autoref{chap:chap03}, 
mostrou que a distribuição espacial das propriedades do solo é muito variável, mesmo quando sob o mesmo uso da 
terra. Em escalas espaciais grosseiras, essa variação espacial é determinada pela diversidade geológica e 
geomorfológica da área, enquanto que em escalas espaciais finas, as (pobres) práticas agrícolas passadas e 
atuais parecem desempenhar um papel preponderante. Juntamente com o modelo conceitual de pedogênese, 
\autoref{chap:chap04} e \autoref{chap:chap05} constituem uma contribuição técnica dessa tese. Esses capítulos 
fornecem a base para exercícios de modelagem espacial do solo na área de estudo.

\autoref{chap:chap06} demonstrou que as covariáveis existentes, gratuitamente disponíveis, são adequados para 
calibrar modelos espaciais de solo. Foi demonstrado que a utilização de covariáveis mais detalhadas resulta em 
apenas um pequeno aumento na acurácia da predição de modelos lineares espaciais do solo. O aumento observado é 
comparável ao efeito da incorporação de dependência espacial no modelo espacial do solo, e pode não compensar 
os custos adicionais de usar covariáveis mais detalhados. Em geral, uma covariável mais detalhada tem um maior 
potencial de melhorar a acurácia da predição quando uma propriedade do solo é pobremente predita pela sua 
versão menos detalhada. No entanto, a magnitude da melhoria pode depender de outras covariáveis que estão 
incluídas no modelo. Escolher se deve-se ou não investir em covariáveis mais detalhados depende da força da 
relação entre as covariáveis e a propriedade do solo que está sendo modelada, e da diferença relativa entre as 
versões menos e mais detalhadas das covariáveis. É provavelmente melhor aumentar substancialmente o detalhe de 
uma covariável menos influente do que marginalmente aumentar o detalhe da covariável mais influente. No 
entanto, deve-se sempre considerar se meios mais eficientes de aumentar a acurácia da predição existem (por 
exemplo, a obtenção de mais observações do solo).

\autoref{chap:chap07} mostrou que vários fatores influenciam o modo como modeladores espaciais de campo 
do solo decidem onde colocar locais de observação do solo. Estes são de três tipos: conceituais, operacionais 
e psicológicos. O primeiro diz respeito ao conhecimento dos modeladores espacial do solo sobre as relações 
solo-paisagem, e parece estar relacionado com os anos de experiência de campo. A segunda refere-se aos 
recursos disponíveis (infra-estrutura, força de trabalho e do orçamento) para fazer observações do solo, bem 
como à restrições ao acesso impostas por proprietários de terras e as barreiras geográficas, por exemplo. A 
terceira refere-se à forma como os modeladores do solo percebem seu ambiente físico circundante e como o curso 
de sua motivação muda durante o processo de observação do solo. Análise do padrão pontual ajudou a compreender 
que existe um balanço entre os fatores conceituais e operacionais, que determina como a motivação dos 
modeladores de campo do solo muda o foco para um ou outro objetivo imediato. Dependendo do objetivo focal, a 
configuração amostral resultante se assemelha a um padrão pontual aleatório (aprendizagem/verificação das 
relações solo-paisagem -- motivação focada nos meios) ou regular (maximizar o número de observações e a 
cobertura geográfica -- motivação focada no resultado).

\autoref{chap:chap08} mostrou que o algoritmo de amostragem no hipercubo latino condicionado, um algoritmo 
popular usado para otimizar configurações amostrais espaciais para a estimativa da tendência espacial, pode 
ser consideravelmente aperfeiçoado. Em comparação com o CLHS original, as modificações propostas resultaram em 
um algoritmo de amostragem com um comportamento numérico superior, mas isso não se traduz necessariamente em 
maior acurácia da predição. Por exemplo, o tamanho da amostra tem uma influência maior na acurácia da predição 
do que o algoritmo de amostragem. No entanto, ter em vista a associação/correlação entre covariáveis degrada 
a acurácia da predição, possivelmente porque a cobertura do espaço geográfico é mais pobre. Como tal, ao 
otimizar uma configuração amostral para a estimativa de tendência espacial, deveria ser suficiente visar 
apenas a reprodução da distribuição marginal das covariáveis. Isso deve ser feito usando apenas os estratos 
marginais de amostragem não vazios.

\autoref{chap:chap09} mostrou como otimizar configurações amostrais para estimativa da tendência espacial e do 
variograma, e interpolação espacial em situações em que sabemos muito pouco sobre a distribuição espacial do 
solo. A única exigência é que seja formulado um sólido problema de otimização multi-objetivo usando versões 
robustas de algoritmos de amostragem existentes. A amostra espacial resultante deve reproduzir a distribuição 
marginal das covariáveis de modo que a tendência espacial possa ser estimada com acurácia. Ela também deve 
conter vários pequenos aglomerados dispersos por todo o domínio espacial para permitir fazer uma estimativa 
acurada do comportamento do variograma, especialmente próximo da origem. Finalmente, ela deve cobrir a região 
de amostragem da forma mais uniforme de tal modo que a média da variância do erro de predição é a menor 
possível.

Essa tese também contribuiu com dois pacotes para o ambiente de computação estatística e gráficos \texttt{R}. 
O primeiro pacote, chamado \texttt{pedometrics} (\autoref{appen:pedometrics}), contém várias funções para a 
análise exploratória de dados espaciais e calibração de modelos projetadas para o desenvolvimento dessa tese. 
O segundo pacote, chamado \texttt{spsann} (\autoref{appen:spsann}), contém funções para otimizar configurações 
amostrais para identificar e estimar o variograma e a tendência espacial, e fazer predições espaciais. O 
último foi desenvolvido como parte do \autoref{chap:chap08} e \autoref{chap:chap09}. Ambos estão disponíveis 
gratuitamente e podem ser obtidos no The Comprehensive R Archive Network (\cran).

No geral, essa tese mostrou que o complexa interação entre o solo e os dados covariáveis pode ter uma grande 
influência sobre a acurácia dos mapas do solo. Uma receita única, universal, de baixo custo para reduzir a 
incerteza na modelagem espacial do solo parece fora de alcance. Os estudos de caso sugeriram que soluções são 
específicas para cada caso e dependem principalmente dos dados do solo e covariáveis existentes. A obtenção 
mais amostras do solo mostrou ser uma estratégia eficiente, desde que os recursos disponíveis permitam a 
amostragem extra. Caso contrário, decidir sobre formas rentáveis de reduzir a incerteza requer, em primeiro 
lugar, que exploremos todo o potencial dos dados do solo e covariáveis existentes usando técnicas robustas de 
modelagem espaciais. Tal exercício exige um conhecimento abrangente das relações solo-paisagem, bem como uma 
minuciosa documentação dos dados do solo e covariáveis para que os seus pontos fracos e fortes podem ser 
facilmente identificados. Então, a decisão sobre investir na melhoria da qualidade dos dados do solo ou 
covariáveis ou ambos dependerá do balanço entre o aumento da qualidade dos dados/predição e a quantidade de 
recursos necessários.
