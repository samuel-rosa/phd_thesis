\artigotrue
\chapter{MODELO CONCEITUAL DE PEDOGÊNESE}
\label{chap:chap03}
%\SweaveUTF8


\def\ptkeys{Província Geológica do Paraná. Bacia do DNOS. Rebordo do Planalto. Fatores de formação do solo. 
Pedogênese}

\begin{chapterabstract}{brazilian}{\ptkeys}
O presente documento apresenta o modelo conceitual de pedogênese -- descrição explícita dos fatores e 
processos de formação do solo que determinam as características do solo e o seu padrão de distribuição 
espaço-temporal -- da bacia de captação do reservatório do DNOS/CORSAN (Departamento Nacional de Obras de 
Saneamento/Companhia Riograndense de Saneamento), localizada no sul do Brasil. O clima é 
subtropical úmido sem estação seca definida. O relevo é plano a montanhoso (variação de altitude entre 139 e 
\SI{475}{\m}), com vales encaixados que influenciam a precipitação e o fluxo radiativo nas diferentes 
superfícies geomórficas. A geologia é constituída pela sequência de três formações geológicas: rochas 
sedimentares 
(arenito fluvial), seguidas de rochas ígneas (basaltos-andesitos toleíticos e vitrófilos, riólitos-riodacitos 
granofíricos) intercaladas por rochas sedimentares (arenito eólico). Depósitos do Quaternário aparecem nas 
partes mais baixas. A geomorfologia atual é resultado dos processos erosivos do Terciário e Quaternário. A 
dissecação atual é fraca devido ao clima que favorece a instalação e permanência de vegetação exuberante. Três 
unidades morfoestruturais são identificadas: no topo, o Planalto, com relevo suave-ondulado a ondulado, 
seguido pelo Rebordo do Planalto, com ampla variação altimétrica, declividade acentuada e escarpas abruptas; na 
base, a Depressão Periférica, com formas agradacionais de planície fluvial. Nas partes altas, a rede de 
drenagem apresenta padrão bem definido, geralmente retangular, determinada pelas falhas e/ou fraturas. Já nas 
áreas mais baixas, devido aos processos de deposição sedimentar e erosão fluvial, sua configuração é sinuosa. 
Ali 
encontram-se um lençol freático próximo da superfície do solo e cursos de água perenes. O uso da terra para 
produção agrossilvopastoril foi intenso em tempos pretéritos e resultou em forte degradação do solo. O 
abandono 
de muitas áreas degradadas permitiu a regeneração da vegetação natural, resultando na atual ocupação com 
florestas e vegetação secundária de \SI{\pm60}{\percent}. Em geral, o solo é pouco profundo devido ao 
predomínio de condições de forte declividade. É comum encontrar solo raso mesmo em áreas de maior estabilidade 
como fruto da degradação pelo uso agrícola. O solo é mais profundo no Planalto, nos terraços do Rebordo, nas 
coxilhas (colinas) de relevo suave-ondulado a ondulado, e nas planícies aluviais. A textura é mais fina e 
homogênea ao 
longo do perfil quando desenvolvido a partir de rochas vulcânicas. As características do solo nas planícies 
aluviais são determinadas pela presença constante de lençol freático próximo da superfície.
\end{chapterabstract}

% \def\enkeys{Paraná Geological Province, DNOS Catchment, Plateau Border, Soil formation factors, Pedogenesis}
%   
% \begin{chapterabstract}{english}{\enkeys}
% This document presents the conceptual model of pedogenesis -- an explicit description of soil-forming 
% factors and processes that determine the spatio-temporal distribution of soil properties  -- of the 
% catchment of the DNOS/CORSAN reservoir, located in southern Brazil. Climate is subtropical humid without a 
% dry season. Relief varies between plain and mountainous, with enclosed valleys (elevation ranging between 
% \num{139} and \SI{475}{\metre} above sea level) that determine rainfall volume and radiative flux on 
% different surfaces. The geology is composed of a sequence of three geological formations: consolidated 
% sedimentary rocks (fluvial sandstone), followed by basic and acid igneous rocks (andesite-basalt and 
% rhyolite-rhyodacite), interlayered with consolidated sedimentary rocks (aeolian sandstone). Unconsolidated 
% colluvial deposits of the Quaternary period occur in the lower portions of the landscape. Current 
% geomorphology is a result of erosive processes of the Tertiary and Quaternary. Landscape dissection is weak 
% due to the current climate that favours the installation and maintenance of an exuberant vegetation. There 
% are three morphostructural units: at the top, the \textit{Planalto} (Plateau), with gently-rolling to 
% sloping relief, followed by the \textit{Rebordo do Planalto} (Plateau Border), with wide altimetric 
% variation, steep slopes and abrupt cliffs; at the bottom, the \textit{Depressão Periférica} (Peripheral 
% Depression), composed of aggradational fluvial plains. In higher altitudes, the drainage network has a well 
% defined pattern, generally rectangular, determined by the faults and/or fractures. In the lower areas, its 
% configuration is sinuous due to sediment deposition and fluvial erosion, with the presence of water table 
% close to the surface and perennial water streams. Land use for agrosilvopastoral production was intense in 
% past times, resulting in severe soil degradation. Recent abandonment of many degraded areas allowed the 
% regeneration of natural vegetation, resulting in \SI{\pm60}{\percent} of the area being now occupied with 
% forest and secondary vegetation. The soil is predominantly shallow due to the dominance of steep slopes. 
% Even in gently-sloping terrain it is common to find shallow soils as a result of soil degradation, Deeper 
% soil can be found in the Planalto, in the terraces of the Rebordo, and in the small hills with 
% gently-rolling slopes and alluvial plains. Soil texture is finer and more homogeneous throughout the soil 
% profile in soil developed from igneous rocks. Soil features in the alluvial plains are determined by the 
% constant presence of the water table close to the surface.
% \end{chapterabstract}

\formatchapter

\section{APRESENTAÇÃO}
\label{sec:chap03-apresentacao}

\titlenote{Colaboraram na preparação deste documento: Pablo Miguel (UFPel), Jean Michel Moura Bueno (UFSM), 
Ricardo Simão Diniz Dalmolin (UFSM), Andrisa Balbinot (UFSM), Lúcia Helena Cunha dos Anjos (UFRRJ), Gustavo de 
Mattos Vasques (Embrapa Soils), e Gerard B. M. Heuvelink (ISRIC -- World Soil Information).}

A modelagem espacial do solo inicia com a definição de um \emph{modelo conceitual de pedogênese}. Um modelo 
conceitual de pedogênese constitui uma representação verbal da realidade sob estudo que inclui a descrição 
explícita dos fatores e processos de formação do solo que determinam as características do solo e o seu padrão 
de distribuição espaço-temporal. Isso requer a reunião de toda a informação ambiental disponível e aplicação 
dos conceitos de relação solo-paisagem, desenvolvimento do solo em catenas, ou outro modelo teórico de 
explicação da variação espacial do solo.

O presente documento apresenta o modelo conceitual de pedogênese da bacia de captação do reservatório do 
DNOS/CORSAN (Departamento Nacional de Obras de Saneamento/Companhia Riograndense de Saneamento), localizada na 
divisa entre os municípios de \itaara{} (ao norte) e \santamaria{} (ao sul), na porção sul da \baciaparana{},
estado do Rio Grande do Sul, Brasil (\autoref{fig:chap03-location}). A bacia de captação do reservatório do 
DNOS/CORSAN corresponde à cabeceira da bacia hidrográfica do \riovacacaimirim{}, tributário do \riojacui{} e, 
consequentemente, do \rioguaiba{} e da \lagoadospatos{}. A bacia de captação do reservatório do DNOS/CORSAN
cobre uma área de \SI{\pm29}{\square\kilo\metre} e alimenta um reservatório com volume máximo 
de \SI{\pm3800000}{\cubic\metre} em uma área inundada de \SI{0,74}{\square\kilo\metre}. Este reservatório 
contribui com até \SI{30}{\percent} do abastecimento de água da cidade de Santa Maria \cite{Dias2003, 
DillEtAl2004, Miguel2010}.

\begin{figure}[!ht]
\centering
\begin{minipage}[b]{95mm}
\subcaption{}
