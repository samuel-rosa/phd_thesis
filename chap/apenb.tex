\artigofalse
\chapter{Modelo Conceitual de Pedogênese}
\label{apen:pedogenesis}

\tocless\section{Apresentação}

A construção de modelos de mapeamento do solo inicia com a definição de um \textit{modelo conceitual
de pedogênese}. Um modelo conceitual de pedogênese constitui uma representação verbal da realidade 
sob estudo que inclui a descrição explícita dos fatores e processos de formação do solo que 
determinam as características do solo e o seu padrão de distribuição espacial. Isso requer a reunião
de toda a informação ambiental disponível e aplicação dos conceitos de relação solo paisagem, 
desenvolvimento do solo em catenas, ou outro modelo teórico de explicação da variação espacial do 
solo.

O presente documento apresenta o modelo conceitual de pedogênese da Área de Estudos em Pedologia 
Quantitativa de Santa Maria (AEPQ-SM), criada no ano de 2008 pelo grupo de pesquisa 
\href{dgp.cnpq.br/dgp/espelhogrupo/9373361709890764}{Gênese, composição e comportamento dos solos do
RS}], coordenado pelos pesquisadores \href{http://lattes.cnpq.br/3735884911693854}{Ricardo Simão 
Diniz Dalmolin} e \href{http://lattes.cnpq.br/6868334304493274}{Fabrício de Araújo Pedron}, e 
sediado no Departamento de Solos da Universidade Federal de Santa Maria 
(\href{http://site.ufsm.br/}{UFSM}). Sua elaboração teve como motivação inicial o desenvolvimento do
projeto de doutorado intitulado \textit{Contribuição à construção de modelos de predição de 
propriedades do solo}. Os resultados obtidos com esse projeto contribuíram para a expansão desse 
documento. A presente versão não é definitiva, tendo como objetivo servir de suporte ao 
desenvolvimento de novos estudos em pedologia quantitativa. Dado que o conhecimento sobre a AEPQ-SM 
continua sendo construído, novas versões do modelo conceitual de pedogênese deverão ser construídas 
por esses estudos.

\tocless\section{Localização}

A AEPQ-SM faz parte da bacia de captação do reservatório do DNOS/CORSAN (Departamento Nacional de 
Obras de Saneamento/Companhia Riograndense de Saneamento), localizada na divisa entre os municípios 
de \href{http://pt.wikipedia.org/wiki/Itaara}{Itaara} (ao norte) e 
\href{http://pt.wikipedia.org/wiki/Santa_Maria_\%28Rio_Grande_do_Sul\%29}{Santa Maria} (ao sul), na 
porção sul da \href{http://pt.wikipedia.org/wiki/Bacia_do_Paran\%C3\%A1}{Bacia Sedimentar do Paraná},
estado do Rio Grande do Sul, Brasil (\autoref{fig:location}).

\begin{figure}[ht]
  \centering
  \includegraphics[width=0.45\textwidth]{figures/location.pdf}
  \caption{Localização da Área de Estudos de Pedologia Quantitativa de Santa 
  Maria (AEPQ-SM), estado do Rio Grande do Sul, Brasil.}
  \label{fig:location}
\end{figure}

A bacia de captação do reservatório do DNOS/CORSAN corresponde à cabeceira da bacia hidrográfica do 
\href{http://pt.wikipedia.org/wiki/Rio_Vacaca\%C3\%AD-Mirim}{rio Vacacaí-Mirim}, tributário do 
\href{http://pt.wikipedia.org/wiki/Rio_Jacu\%C3\%AD}{rio Jacuí} e, consequentemente, do 
\href{http://pt.wikipedia.org/wiki/Lago_Gua\%C3\%ADba}{rio Guaíba} e da 
\href{http://pt.wikipedia.org/wiki/Lagoa_dos_Patos}{Lagoa dos Patos}. A bacia de captação cobre uma 
área de aproximadamente \SI{29}{\square\kilo\metre} e alimenta um reservatório com volume máximo 
de aproximadamente \SI{3800000}{\cubic\metre} em uma área inundada de \SI{0,74}{\square\kilo\metre}.
Esse reservatório contribui com até \SI{30}{\percent} do abastecimento de água da cidade de Santa 
Maria \citep{Dias2003, DillEtAl2004, Miguel2010}.

Apesar de cobrir apenas \SI{60}{\percent} da bacia de captação do reservatório do DNOS/CORSAN, o 
que corresponde à uma área de \SI{\pm18}\square\kilo\metre}, a area de estudo engloba a maior parte 
da variação presente naquela. As limitações operacionais do grupo de pesquisa na época impuseram a 
necessidade de selecionar uma sub-área da da bacia de captação do reservatório do DNOS/CORSAN 
(\autoref{fig:aepsm}).

\begin{figure}[ht]
  \centering
  \includegraphics[width=0.45\textwidth]{figures/aepsm.pdf}
  \caption{A Área de Estudos de Pedologia Quantitativa de Santa Maria (AEPQ-SM) faz parte da bacia 
  de captação do reservatório do DNOS/CORSAN.}
  \label{fig:aepsm}
\end{figure}

\tocless\section{Clima}

O clima local é classificado como \href{http://pt.wikipedia.org/wiki/Clima_subtropical_\%C3\%BAmido}{Cfa}
(subtropical úmido sem estação seca definida), com temperatura média anual de \SI{19,1}{\celsius}. 
As temperaturas podem alcançar \SI{>40}{\celsius}, no verão, e valores negativos no inverno 
\citep{HeldweinEtAl2009}. A precipitação média anual é de \SI{1708}{\milli\metre} bem distribuídos 
ao longo do ano \citep{Maluf2000}. Predominam os ventos do quadrante leste (frio, úmido e de 
intensidade fraca a moderada), oeste (frio, seco e de intensidade fraca a moderada) e norte (quente,
seco e de intensidade moderada a forte) \citep{HeldweinEtAl2009}.

O padrão predominante das chuvas é o avançado, caracterizado por ter seu pico de maior intensidade 
no início da precipitação \citep{MehlEtAl2001}. As chuvas de maior intensidade ocorrem nos meses do 
final da primavera, verão e início do outono \citep{MouraBueno2012}. Como resultado desse padrão, as
chuvas de inverno são as menos erosivas, mesmo que o conteúdo de água do solo permaneça elevado 
durante todo o período. O padrão de precipitação também é condicionado pelo relevo. Observações 
feitas em três locais da AEPQ-SM durante o ano de 2011, marcado por forte estiagem, mostram variação
na lâmina total precipitada entre \num{1317} e \SI{1411}{\milli\metre} \citep{MouraBueno2012}. 
Assim, o relevo plano a montanhoso, com vales encaixados, parece condicionar a formação de 
diferentes regiões microclimáticas, refletindo no volume e intensidade das chuvas 
\citep{MouraBueno2012}.

O relevo também deve condicionar o fluxo radiativo que atinge as diferentes superfícies. Apesar de 
não haver estudos que demonstrem a efetividade desse fenômeno na área, é reconhecido que grande 
parte da superfície em terrenos de topografia complexa é influenciada pelo efeito de sombreamento, 
sobretudo nas primeiras horas da manhã e no final da tarde \citep{OliphantEtAl2003}. Além disso, a 
declividade do terreno possui forte influência sobre o ângulo de interceptação da radiação solar 
pelas superfícies \citep{Birkeland1999}. Como consequência, deve ocorrer variações na temperatura e 
conteúdo de água no solo nas diferentes superfícies. Os meses de inverno são marcados por ainda 
menor disponibilidade de radiação solar devido à alta frequência de nevoeiros, sobretudo nas partes 
mais baixas, com valores normais de insolação de \SI{5,1}{\hour\per\day} \citep{HeldweinEtAl2009}. 
Além disso, devido à variação de altitude entre \num{139} e \SI{475}{\metre}, deve ocorrer diferença
na temperatura da ordem de aproximadamente \SI{4}{\celsius} entre a parte mais baixa e a parte mais 
alta da AEPQ-SM \citep{HeldweinEtAl2009}.

\tocless\section{Geologia}

A geologia da AEPQ-SM é bastante complexa, sendo constituída por três formações geológicas, além de 
depósitos coluvionares e de aluvião do Quaternário. A literatura sobre o tema é vasta 
\citep{Bortoluzzi1974, Brasil1980, GasparettoEtAl1988, MacielFilho1990, PieriniEtAl2002, 
MarquesEtAl2005, Milani2005, Pinto2005, CPRM2007, Pedron2007, Sartori2009, NascimentoEtAl2010, 
WerlangEtAl2010, Machado2012, PedronEtAl2012}, e uma revisão da mesma é apresentada aqui.

% \begin{figure}[h]
%   \centering
%   \includegraphics[height=7cm]{figures/geo1988}
%   \caption{Mapa da geologia da AEPQ-SM publicado na escala 1:50.000 \citep{GasparettoEtAl1988}.\\Legenda: SG-S - Sequência Superior da Formação Serra Geral, CT - Formação Caturrita, BT - Formação Botucatu, SG-I - Sequência Inferior da Formação Serra Geral.}
%   \label{fig:geo1988}
% \end{figure}

Na base da sequência estratigráfica, em elevações abaixo de \SI{\pm200}{\metre}, está a 
\href{http://pt.wikipedia.org/wiki/Forma\%C3\%A7\%C3\%A3o_Caturrita}{Formação Caturrita}, 
constituída por material sedimentar depositado em ambiente fluvial no Triássico Superior. Sua 
composição é diversa, apresentando seixos de siltito argiloso vermelho na base, seguido por arenito 
avermelhado de granulometria fina à média, composição quartzosa e matriz argilosa, podendo ainda 
conter considerável teor de feldspato, sobreposto por siltito e folhelho também avermelhados. Em 
geral, a granulometria do arenito é mais grosseira e menos argilosa na base da deposição. Dado a sua
origem fluvial, a Formação Caturrita apresenta marcada estratificação cruzada acanalada e tabular. 
A origem fluvial também resulta em significativa variação espacial na granulometria do arenito, 
identificada pelo contraste entre áreas de maior cimentação e coesão, com outras de maior 
condutividade hidráulica. Imediatamente acima da Formação Caturrita encontra-se, ora a Formação 
Botucatu, ora a Sequência Inferior da Formação Serra Geral.

% \begin{figure}[h]
%   \centering
%   \includegraphics[height=7cm]{figures/geo1990}
%   \caption{Mapa da geologia da AEPQ-SM publicado na escala 1:25.000 \citep{MacielFilho1990}.\\Legenda: QD - Depósitos do Quaternário, SG-S - Sequência Superior da Formação Serra Geral, CT - Formação Caturrita, BT - Formação Botucatu, SG-I - Sequência Inferior da Formação Serra Geral.}
%   \label{fig:geo1990}
% \end{figure}

Em elevações entre \num{\pm200} e \SI{\pm350}{\metre} está a Sequência Inferior da 
\href{http://pt.wikipedia.org/wiki/Forma\%C3\%A7\%C3\%A3o_Serra_Geral}{Formação Serra Geral} 
(basaltos-andesitos tholeíticos). As rochas básicas são de coloração cinza-escura e são constituídas
por plagioclásio cálcico clinopiroxênio, magnetita e material intersticial de quartzo e material 
desvitrificado. Em elevações superiores a \SI{\pm350}{\metre} está a Sequência Superior da Formação 
Serra Geral (vitrófilos, riólitos-riodacitos granofíricos). As rochas ácidas apresentam cor 
cinza-clara, estrutura microcristalina e são constituídas por cristais e plagioclásio, 
clinopiroxênios, hornblenda uralítica e magnetita. A origem desse material remonta o Cretáceo, 
quando sucessivos derrames de lavas de origem vulcânica fissural ocorreram durante aproximadamente 
10 milhões de anos em toda a Bacia do Paraná. Esses eventos ocorreram ao mesmo tempo em que iniciava
a separação das plataformas continentais que hoje constituem a América do Sul e África, marcando o 
final da existência do supercontinente Pangea.

O arenito eólico constituinte da 
\href{http://pt.wikipedia.org/wiki/Forma\%C3\%A7\%C3\%A3o_Botucatu}{Formação Botucatu} é encontrado 
tanto assentado sobre a Formação Caturrita, como no interior da Formação Serra Geral (arenito 
\textit{intertrap}). Trata-se de um arenito quartzoso de granulometria fina à média, contendo 
feldspato alterado e cimentado por sílica ou por óxido de ferro, que lhe confere a coloração 
rosa-avermelhada. Sua deposição teve início no Cretáceo Inferior, período em que a Bacia do Paraná 
estava sob influência de clima desértico. Essa condição climática continuou durante todo o período 
em que ocorreram as dezenas de eventos de vulcanismo fissural, fazendo com que os mesmos fossem 
sucedidos por deposições eólicas de duração variável. Como a duração e a quantidade de material 
depositado pelos eventos de vulcanismo fissural era variável, assim como o intervalo de tempo entre 
cada novo evento e a intensidade das deposições de sedimentos eólicos, a espessura das camadas do 
arenito eólico e das rochas vulcânicas é bastante variável. Além disso, devido ao diversos eventos 
de subsidência que ocorreram no eixo central da Bacia do Paraná, com consequente soerguimento de 
suas bordas, as camadas dessas rochas possuem diferentes inclinações ao longo de sua faixa de 
exposição, sendo caracteristicamente ondulada e com suave tendência de inclinação para sudoeste.

As deposições do Quaternário são constituídas por depósitos coluvionares e de aluvião. Em elevações 
entre \num{\pm200} e \SI{\pm300}{\metre} encontram-se depósitos coluviais de material proveniente de
uma ou ambas as Formações Serra Geral (fragmentos de tamanho variado) e Botucatu. Em elevações 
abaixo de \SI{\pm200}{\metre} são mais comuns os depósitos coluviais de uma ou ambas as Formações 
Botucatu e Caturrita. Esses depósitos ocorrem de maneira descontínua nas encostas. Próximo aos 
cursos de água na porção mais baixa da bacia e no entorno do reservatório, encontram-se depósitos 
fluviais recentes, geralmente formados por fragmentos arredondados (seixo) de tamanho variável e/ou 
sedimento arenoso. Em pequenas áreas abaciadas e mal drenadas, os sedimentos apresentam 
granulometria mais fina.

\tocless\section{Geomorfologia}

A AEPQ-SM está situada na porção sul da Bacia Sedimentar do Paraná. Assim, as geoformas atuais são 
resultado dos processos erosivos que ocorreram durante o Terciário e o Quaternário 
\citep{Sartori2009}, após as últimas deposições de lavas vulcânicas e de sedimentos eólicos. Durante
esse período, a esculturação da paisagem foi determinada pelas alternações entre climas úmidos, 
semi-áridos e áridos \citep{Sartori2009}. Atualmente, o clima subtropical úmido favorece a 
instalação e permanência de uma vegetação mais eficiente na redução do processo de dissecação da 
paisagem \citep{Sartori2009, NascimentoEtAl2010}. Isso permite que as superfícies geomórficas 
atinjam maior estabilidade e maturidade.

% \begin{figure}[h]
%   \centering
%   \includegraphics[height=7cm]{figures/dem90}
%   \caption{Modelo digital de elevação da AEPQ-SM com resolução espacial de 90~m (SRTM DEM versão 4) \citep{JarvisEtAl2008}.}
%   \label{fig:dem90}
% \end{figure}

As unidades morfoestruturais que abrangem a área são o Planalto e o Rebordo do Planalto da Bacia do 
Paraná \citep{NascimentoEtAl2010}. O Planalto é marcado por relevo suave-ondulado a ondulado, com 
formas denudacionais de superfícies planas com topos convexos (fluxo hídrico divergente). Nesses 
locais, as vertentes assumem a forma convexa levemente ondulada \citep{NascimentoEtAl2010}, muitas 
vezes encaixadas em falhas e/ou fraturas. Já o Rebordo do Planalto é caracterizado pela ampla 
variação altimétrica, declividade acentuada e escarpas abruptas, apresentando formas denudacionais 
com topos convexos (fluxo hídrico divergente), aguçados e em formas de escarpas. Nesses locais, as 
vertentes assumem forma retilínea com grande desnível \citep{NascimentoEtAl2010}, muitas vezes 
interrompidas por degraus ou patamares, na maioria das vezes encaixadas em falhas e ou fraturas. 
Esses patamares são resultado da ação diferencial dos processos denudacionais sobre a paisagem, 
geralmente condicionados pela resistência do material de origem, seja ela química/mineralógica 
(rochas vulcânicas básicas vs. ácidas), seja ela física/granulométrica (rochas vulcânicas vs. 
sedimentares), ou estrutural (padrão de diaclasamento vertical vs. horizontal das rochas vulcânicas)
\citep{Holtz2003, Pedron2007, StreckEtAl2008}. Entretanto, em algumas situações, os patamares são 
formados a partir de eventos pluviométricos de elevada intensidade e/ou duração que causam 
movimentos de massa, formando depósitos coluviais mesmo nas partes altas do Rebordo do Planalto 
\citep{PinheiroEtAl2004, PaisaniEtAl2010}. Nesses casos, os patamares possuem menor dimensão e maior
declividade. Em outras situações, os patamares são inteiramente recobertos por colúvios provenientes
de áreas à montante, sobretudo dos arenitos da Formação Botucatu, formando vertentes alongadas e com
menor inclinação, que chegam até a margem dos cursos de água. Assim, as encostas que apresentam 
patamares constituem, de modo geral, vertentes mais curtas e inclinadas, muitas vezes escarpadas.

% \begin{figure}[h]
%   \centering
%   \includegraphics[height=7cm]{figures/dem10}
%   \caption{Modelo digital de elevação da AEPQ-SM com interpolado de curvas de nível com 10~m de equidistância \citep{DSG1980, DSG1992, DSG1992a}.}
%   \label{fig:dem10}
% \end{figure}

Apesar da AEPQ-SM ser abrangida pelas unidades morfoestruturais do Planalto e do Rebordo do Planalto
da Bacia do Paraná \citep{NascimentoEtAl2010}, as características geomorfológicas das porções mais 
baixas da paisagem são similares àquelas encontradas na Depressão Periférica. Essas áreas são 
marcadas pelo acúmulo de sedimentos provenientes do Planalto e do Rebordo, formando planícies 
aluviais que se intercalam entre as coxilhas (denominação regional de colinas). Essas áreas são 
caracterizadas pela presença de formas agradacionais de planície fluvial e formas denudacionais de 
topos convexos e de superfícies planas. As últimas correspondem às coxilhas de algumas dezenas de 
metros de altitude, geralmente formadas sobre o substrato da Formação Caturrita 
\citep{GasparettoEtAl1988}, geralmente assentadas na base do Rebordo do Planalto. Essas coxilhas 
constituem divisores de água de pequena amplitude que auxiliam na subdivisão da área em pequenas 
sub-bacias \citep{Marins2004, Sartori2009}. Nesses locais as vertentes costumam ser alongadas e 
assumem a forma predominantemente côncava devido aos processos de deposição sedimentar e erosão 
fluvial, apesar dos mesmos serem muito fracos sob a atual condição climática 
\citep{NascimentoEtAl2010, WerlangEtAl2010}.

A grande heterogeneidade geomorfológica da AEPQ-SM se traduz em uma grande heterogeneidade textural 
do relevo, ou rugosidade, resultante da ação climática ao longo do tempo geológico 
\citep{NascimentoEtAl2010}. Entretanto, em menor escala, a ação antrópica também atua sobre a 
configuração geomorfológica da área. O principal efeito se dá através da erosão da camada 
superficial do solo, cultivado intensivamente sem o uso de práticas conservacionistas ao longo de 
inúmeras décadas \citep{Menezes2008, Sturmer2008, Miguel2010, SamuelRosaEtAl2011a}. Além disso, a 
abertura de caminhos para acesso às áreas de produção nos patamares do Rebordo do Planalto e nos 
topos de morros proporcionou a formação de canais de concentração e escoamento dos fluxos hídricos, 
levando à formação inicial de voçorocas. Obras de maior expressividade, como ferrovias, ruas e 
rodovias pavimentadas, conjuntos habitacionais e construções isoladas, as quais envolvem operações 
de terraplanagem e aterramento, também resultaram em modificações localizadas na geomorfologia da 
área. Por fim, a construção do reservatório possibilitou a formação de uma área de agradação da 
paisagem bastante estável em seu entorno, uma vez que o sedimento removido do Planalto e do 
Rebordo do Planalto não são mais transportados a jusante.

\tocless\section{Hidrografia}

A hidrografia da AEPQ-SM é condicionada pelas condições geomorfológicas, geológicas, pedológicas e 
climáticas, ao mesmo tempo que exerce forte influência sobre a modelagem da paisagem 
\citep{NascimentoEtAl2010}. Como a AEPQ-SM é abrangida pelas unidades morfoestruturais do Planalto 
e do Rebordo do Planalto da Bacia do Paraná, a drenagem apresenta um padrão bem definido, geralmente
retangular, determinado pelas falhas e/ou fraturas \citep{Bortoluzzi1974, GasparettoEtAl1988, 
NascimentoEtAl2010}. A própria formação da bacia de captação do reservatório do DNOS/CORSAN se deve 
a existência de falhas e/ou fraturas \citep{GasparettoEtAl1988}. A principal e maior delas 
localiza-se no eixo central da bacia, onde atualmente está localizado o leito do rio Vacacaí-Mirim. 
Quanto aos tributários do rio Vacacaí-Mirim, a maioria possui o leito assentado sobre outras falhas 
e/ou fraturas de menor dimensão perpendiculares àquela do eixo central da bacia.

Nas áreas mais baixas, cujas características geomorfológicas se assemelham à Depressão Periférica, 
a drenagem apresenta configuração serpenteada, resultado dos processos de deposição sedimentar e 
erosão fluvial \citep{PaivaEtAl2001, SutiliEtAl2009}. Como o relevo é plano a suave-ondulado, e as 
vertentes longas e predominantemente côncavas, o lençol freático fica próximo da superfície do solo.
A variação das condições meteorológicas ao longo do ano fazem com que o lençol freático apresente 
flutuação significativa, mantendo o conteúdo de água do solo elevado durante os meses mais frios 
(menor evapotranspiração) \citep{HeldweinEtAl2009}. Isso também favorece a ocorrência de inundações,
sobretudo nas proximidades dos cursos de água e do reservatório \citep{Goldani2006}.

Muitos cursos de água localizados no Planalto, no Rebordo do Planalto e nas coxilhas assentadas em 
sua base, são sazonais. Em geral, esses cursos de água estão em atividade apenas nos meses mais 
frios do ano, quando a disponibilidade de água no ambiente é maior, ou durante os eventos de 
precipitação de forte intensidade, que ocorrem nos meses de verão \citep{HeldweinEtAl2009, 
MouraBueno2012}. Os cursos de água do Rebordo do Planalto costumam apresentar leito raso e 
pedregoso, muitas vezes assentado sobre rochas da Formação Serra Geral \citep{SutiliEtAl2009}. Já 
os cursos de água localizados nos patamares e coxilhas costumam ser rasos, se assentados sobre as 
rochas da Formação Caturrita, ou profundos, se assentados sobre as rochas da Formação Botucatu ou 
depósitos coluvionares, formando voçorocas. Segundo relatos de alguns dos moradores mais antigos, 
muitas nascentes e pequenos cursos de água já perderam totalmente sua atividade, sobretudo quando 
localizadas no interior ou à jusante de áreas de uso antrópico.

Dado que a declividade e o desnível entre a parte mais baixa e a parte mais alta da AEPQ-SM são 
acentuados, as cheias costumam apresentar velocidade e vazão bastante grandes \citep{PaivaEtAl2001, 
SutiliEtAl2009}. Em média, nos \SI{7}{\kilo\metre} de extensão do rio Vacacaí-Mirim, da nascente até
o reservatório, a declividade média é de \SI{0,03}{\metre\per\metre}. Isso representa um desnível de
\SI{\pm210}{\metre}, o que resulta em um tempo de concentração da bacia estimado em \SI{3}{\hours} 
\citep{PaivaEtAl2001}. Essas características causam erosão severa nas margens dos cursos de água nas
áreas mais baixas (depósitos aluviais), sobretudo nos raios externos das curvas, onde a velocidade 
da água é maior \citep{SutiliEtAl2009}. Em alguns trechos, os cursos de água chegam a ter sua 
largura duplicada em menos de uma década, resultando no aumento da sinuosidade e do nível de fundo 
\citep{PaivaEtAl2001}. Esses eventos comprometem áreas de produção, bem como a estrutura de 
residências e vias públicas localizadas nas margens dos cursos de água. Grande parte do material 
removido das margens dos cursos de água é transportado para dentro do reservatório, que já perdeu 
mais de \SI{30}{\percent} de sua capacidade inicial de armazenamento de água \citep{DillEtAl2004}.

\tocless\section{Uso da terra}

A AEPQ-SM já foi intensamente ocupada em tempos pretéritos. Isso é evidenciado pela grande 
quantidade, extensão e boa distribuição da rede viária em toda a sua área. Além disso, a área é 
cortada por uma estrada férrea e avizinha uma rodovia federal, ambas muito movimentadas. Entretanto,
nas últimas décadas, muitos caminhos internos das propriedades rurais foram desativados, fruto do 
abandono de áreas de produção, a maioria delas localizada nos topos dos morros, patamares do Rebordo
ou no fundo de vales, onde a manutenção dos caminhos é muito onerosa, ou ainda, distantes da sede 
das propriedades. No passado, essas estradas internas possibilitaram o trânsito de pessoas e o 
transporte da produção agrosilvopastoril, a qual era comercializada às margens da estrada férrea e 
nas áreas urbanas do município. Atualmente, o tráfego está concentrado nas estradas principais e 
secundárias, enquanto alguns poucos caminhos internos são utilizados apenas na condução de rebanhos 
ou no acesso a pequenas áreas de produção agrícola. A maioria dos caminhos internos inativos está 
localizada no interior de áreas de floresta natural regenerada, o que dificulta a sua identificação 
pelo uso de imagens aéreas ou orbitais. Em alguns casos, esses caminhos são utilizados em atividades
de turismo ecológico, especialmente para caminhadas a pé, ou atividades esportivas com motocicletas
(trilhas).

% \begin{figure}[h]
%   \centering
%   \includegraphics[height=7cm]{figures/land1980}
%   \caption{Mapa do uso da terra da AEPQ-SM publicado na escala 1:25.000 \citep{DSG1980, DSG1992, DSG1992a}.\\Legenda de acordo com o guia para descrição do solo da FAO \citep{FAO2006}: S - Assentamentos humanos, PF - Silvicultura, H - Pecuária, FS - Floresta nativa.}
%   \label{fig:land1980}
% \end{figure}

O abandono de muitas áreas de produção possibilitou a regeneração da vegetação natural em grande 
parte da AEPQ-SM. Atualmente, a área ocupada por florestas e vegetação secundária (capoeira) 
representa aproximadamente \SI{60}{\percent} da área total \citep{SamuelRosaEtAl2011a}. Isso mostra 
que, assim como toda a região do Rebordo do Planalto da Bacia do Paraná, o uso da terra foi 
desintensificado nas últimas décadas \citep{SEMA/UFSM2001, DillEtAl2004, Poelking2007, Miguel2010, 
SamuelRosaEtAl2011a, Dullius2012, tenCatenEtAl2012}. As áreas florestadas são encontradas nos mais 
diversos estádios de desenvolvimento, com os estratos intermediário e superior concentrando a maior 
parte dos indivíduos. As florestas originais, e secundárias em estágio avançado de desenvolvimento, 
são encontradas apenas em áreas de difícil acesso. Em geral, as áreas florestadas predominam nas 
regiões de maior declividade, com solo raso e pedregoso. Tais condições pedológicas são resultado, 
na maioria dos casos, das condições geológicas e geomorfológicas, mas também podem derivar da 
degradação causada pelo intenso uso da terra durante inúmeras décadas \citep{SamuelRosaEtAl2011a}. 
A ocorrência de fragmentos ao longo dos cursos de água, estradas e próximo de edificações é comum. 
Nas formações mais jovens, é comum encontrar fragmentos de carvão e sinais de uso agrícola no 
passado. As áreas sob vegetação secundária (capoeira) também ocorrem em toda a bacia de captação, 
predominantemente em locais de difícil acesso e acentuada declividade, geralmente interligados por 
caminhos internos, indicando sua utilização agrícola no passado \citep{SamuelRosaEtAl2011a}. Sua 
composição florística varia entre as áreas, predominando espécies de porte herbáceo e arbustivo. O 
solo dessas áreas costuma ser menos pedregoso que nas áreas florestadas, mas apresentam profundidade
semelhante \citep{SamuelRosaEtAl2011a}, indicando que as atividades agrícolas foram encerradas antes
do solo atingir seu nível máximo de degradação. Entretanto, a reserva de nutrientes do solo, assim 
como a sua fertilidade física, foram esgotadas à tal ponto que a vegetação secundária ainda não foi 
capaz de produzir melhorias significativas quando comparado com as condições originais 
\citep{Menezes2008, Zalamena2008}.

% \begin{figure}[h]
%   \centering
%   \includegraphics[height=7cm]{figures/land2009}
%   \caption{Mapa do uso da terra da AEPQ-SM publicado na escala 1:2.000 \citep{SamuelRosaEtAl2011a}.\\Legenda de acordo com o guia para descrição do solo da FAO \citep{FAO2006}: O - Outros usos (água), S - Assentamentos humanos, PF - Silvicultura, AA - Agricultura anual, H - Pecuária, SS - Vegetação secundária (capoeira), FS - Floresta nativa.}
%   \label{fig:land2009}
% \end{figure}

Apesar do abandono de muitas áreas de produção agrícola nas últimas décadas, sobretudo aquelas com 
maior dificuldade de acesso, os caminhos internos utilizados para o escoamento da produção, mesmo 
depois de inativos, continuam associados ao processo de degradação da terra. Isso ocorre porque, na 
grande maioria dos casos, todo o fluxo da água de escorrimento superficial proveniente dos eventos 
de precipitação é concentrado nesses caminhos internos, onde, geralmente, quantidade significativa 
de resíduos vegetais está depositada. Esse resíduo vegetal, mais a fração mineral do solo carregada 
pela enxurrada, costuma chegar com facilidade aos cursos de água. No caso dos caminhos internos 
ainda utilizados na condução de rebanhos, a produção de sedimentos minerais tende a ser ainda maior,
haja vista o impacto mecânico do pisoteio animal sobre a desagregação do solo. Além disso, muitas 
áreas florestadas e sob vegetação secundária ainda são utilizadas para o pastoreio de bovinos e 
equinos \citep{SamuelRosaEtAl2011a}. Isso causa a degradação da camada superficial do solo, 
sobretudo pela sua compactação, dificultando a infiltração da água dos eventos de precipitação, o 
que resulta na perda da serapilheira \citep{ScheneiderEtAl1978}. Além disso, a própria regeneração 
natural é prejudicada, uma vez que plantas em estádio inicial de desenvolvimento e raízes 
superficiais são destruídas \citep{ScheneiderEtAl1978, HackEtAl2005}. O resultado desse processo é 
a perda da qualidade do solo e, consequentemente, da produtividade da floresta \citep{KonigEtAl2002}.

% \begin{figure}[h]
%   \centering
%   \includegraphics[height=7cm]{figures/ndvi30}
%   \caption{Mapa do índice de vegetação por diferença normalizada (Landsat~5~TM) da AEPQ-SM em dezembro de 2010.}
%   \label{fig:ndvi30}
% \end{figure}

Dado que a AEPQ-SM também possui áreas com relevo plano a suave-ondulado e solo profundo, 
\SI{\pm30}{\percent} de sua superfície ainda é utilizada com atividades agrosilvipastoris ao longo 
de todo o ano, principalmente a pecuária extensiva \citep{SamuelRosaEtAl2011a}. As áreas de produção
animal extensiva predominam nas porções norte, sul e centro-oeste, ao longo dos cursos de água e 
estradas. O solo é mais profundo e menos pedregoso do que em áreas de floresta natural e vegetação 
secundária. Além disso, são encontrados fragmentos de carvão e formações de micro relevo devido ao 
uso de implementos agrícolas para o revolvimento do solo na maior parte dos locais, evidenciando seu
uso agrícola no passado. Essas áreas podem ser divididas entre aquelas de campo sujo (pastagens 
naturais mal manejadas, com predomínio de vegetação de porte herbáceo-arbustivo) e campo limpo 
(pastagens naturais e perenes bem manejadas) \citep{SamuelRosaEtAl2011a}. Em geral, as áreas de 
campo sujo estão próximas a áreas de vegetação secundária (capoeira), com relevo mais declivoso e 
solo mais pedregoso e raso do que aquelas de campo limpo, indicando que podem vir a ser abandonadas 
dentro de pouco tempo. Assim como as áreas florestadas e sob vegetação secundária, a dinâmica de uso
dessas áreas ao longo do tempo é bastante complexa, dificultando o estabelecimento de relações 
diretas com a maior parte das características do solo. Entretanto, sob o ponto de vista da reserva 
de nutrientes e matéria orgânica, o solo pode ser considerado pobre, uma vez que a exploração 
pecuária é totalmente extrativista e extensiva.

% \begin{figure}[h]
%   \centering
%   \includegraphics[height=7cm]{figures/ndvi5}
%   \caption{Mapa do índice de vegetação por diferença normalizada (RapidEye) da AEPQ-SM em novembro de 2012.}
%   \label{fig:ndvi5}
% \end{figure}

As atividades agrosilvopastoris que ocupam menor extensão territorial da AEPQ-SM são a agricultura e
a silvicultura. As áreas de lavoura estão dispersas em toda a área, geralmente localizadas em 
terrenos de pequena declividade e com solo medianamente profundo. Entretanto, algumas áreas de 
produção ocorrem em declives superiores a \SI{50}{\percent} e com solo raso e pedregoso 
\citep{SamuelRosaEtAl2011a}. Em qualquer das situações, as condições de degradação do solo costumam 
ser bastante avançadas, exceto por algumas áreas de produção olerícola profissional. Assim como as 
áreas de agricultura, as florestas exóticas (eucalipto) são implantadas em áreas de menor 
declividade e com solo mais profundo, geralmente onde o acesso com máquinas é melhor, sobretudo pela
necessidade de escoamento da produção. A área ocupada por essa atividade possui tendência de 
crescimento, haja vista os novos plantios existentes e o relato de alguns moradores 
\citep{SamuelRosaEtAl2011a}. Em geral, os novos plantios são implantados em áreas de produção 
agropecuária, seja pelo elevado nível de degradação do solo já atingido, seja pela redução da força 
de trabalho das famílias devido ao êxodo rural, ou pela maior lucratividade dessa atividade.

Por fim, as obras de engenharia e assentamentos urbanos são aquelas que ocupam a menor parte da 
AEPQ-SM. No que diz respeito à malha viária, a maior concentração ocorre na porção sul, junto ao 
maior assentamento urbano, localizado no entorno do reservatório. Entretanto, diversas construções 
são encontradas ao longo das estradas que cortam a área, muitas das quais são pertencentes a 
moradores do centro da cidade de Santa Maria e são utilizadas apenas como sítios de final de semana.
O número de sítios de final de semana aumentou significativamente ao longo da última década 
\citep{Goldani2006}, muitos dos quais construídos em locais inapropriados, como margens dos corpos 
de água e áreas de forte declividade. Esse processo de urbanização desordenada, que exigiu a 
realização de obras de corte e aterramento dos terrenos, é um importante contribuinte da carga de 
sedimentos recebida pelo reservatório anualmente \citep{PaivaEtAl2001, DillEtAl2004}. Soma-se a isso
os resíduos domésticos e cloacais despejados nos cursos de água devido à falta de coleta e 
tratamento \citep{Goldani2006}. Quanto às demais obras de engenharia, destacam-se os reservatórios 
de água, a maioria deles de pequena extensão, utilizados para a dessedentação animal. Como os cursos
de água de maior volume estão localizados na porção sul da área, a maior parte dos reservatórios de 
água está na porção norte \citep{SamuelRosaEtAl2011a}. Além disso, a menor permeabilidade do solo e 
do substrato rochoso, bem como da condição topográfica, favorecem essa característica.

\tocless\section{Pedologia}

O solo da área de estudo possui características com forte dependência do material de origem 
\citep{NascimentoEtAl2010} que, conforme descrito acima, é um importante condicionante da 
geomorfologia e da hidrologia. Nas superfícies geomórficas mais estáveis, como no topo do Planalto 
(rochas vulcânicas), nos terraços do Rebordo (rochas vulcânicas e sedimentares) e nas coxilhas de 
relevo suave-ondulado a ondulado (rochas sedimentares), as condições ambientais proporcionam maior 
desenvolvimento do solo em profundidade \citep{Moser1990}. Já nas áreas de relevo mais acidentado 
do Rebordo, onde a taxa de formação do solo deve ser semelhante a taxa de remoção, as condições 
ambientais são pouco favoráveis ao desenvolvimento do solo \citep{Moser1990, DalmolinEtAl2006a, 
Sturmer2008, SamuelRosaEtAl2011a}. Entretanto, além da condição geomorfológica que impede ou limita 
o desenvolvimento do solo nesses locais, a resistência do material parental ao intemperismo também 
é um importante condicionante \citep{Pedron2007}. Já nas planícies aluviais, as características do 
solo são fortemente influenciadas pelo hidromorfismo e deposição sedimentar \citep{Moser1990, 
Miguel2010}.

% \begin{figure}[h]
%   \centering
%   \includegraphics[height=7cm]{figures/solo1989}
%   \caption{Mapa do solo da AEPQ-SM publicado na escala 1:100.000 \citep{AzolinEtAl1988}.\\Legenda de acordo com o sistema brasileiro de taxonomia do solo \citep{SoaresEtAl2005}: TBa-Rd - Terra Bruna Estruturada álica, Re4 - Solo Litólico Eutrófico/Distrófico relevo montanhoso, Re-C-Co - Solo Litólico Eutrófico relevo forte ondulado, Rd1 - Solo Litólico Distrófico/Eutrófico, C1 - Cambissolo Eutrófico.}
%   \label{fig:solo1989}
% \end{figure}

Como a AEPQ-SM possui a maior parte de sua superfície em condições de forte declividade, o solo é, 
predominantemente, pouco desenvolvido, com profundidade inferior à \SI{50}{\centi\metre} até o 
contato lítico e comum ocorrência de pedregosidade e rochosidade \citep{Miguel2010}. Assim, 
predominam as áreas de solo classificado como Neossolo Litólico Distro-Úmbrico típico, Cambissolo 
Háplico Ta Eutrófico típico, Neossolo Litólico Eutro-Úmbrico típico e Neossolo Regolítico 
Distro-Úmbrico típico. Em muitas áreas de maior estabilidade (topos de morros, patamares do Rebordo 
do Planalto e coxilhas), onde as condições para o desenvolvimento pedogenético são mais favoráveis 
\citep{Moser1990}, o solo também apresenta características que indicam seu fraco desenvolvimento 
\citep{MouraBueno2012}. Entretanto, nessas áreas, o solo sofreu severa degradação ao longo das 
inúmeras décadas de cultivo intensivo sem uso de práticas conservacionistas, fazendo com que parte 
significativa de sua camada superficial fosse perdida \citep{SamuelRosaEtAl2011a}. Nos patamares 
constituídos inicialmente por colúvios sedimentares (arenito Botucatu) e vulcânicos (fragmentos de 
tamanhos variáveis), atualmente, devido à forte erosão à que foi submetido, o solo apresenta 
superfície recoberta por fragmentos rochosos que limitam seu uso para atividades agrosilvopastoris 
\citep{MouraBueno2012}. Dado que a ocorrência de ambientes com solo pouco desenvolvido não é função 
restrita da geomorfologia e do material de origem, mas também do uso antrópico, não é possível 
estabelecer uma relação direta e unívoca da textura do solo com os níveis taxonômicos mais altos. 
Essa diferenciação pode ser realizada mais adequadamente a partir da identificação do material 
originário. Assim, os arenitos conferem textura arenosa ao solo, sobretudo aqueles da Formação 
Botucatu, enquanto as rochas vulcânicas conferem textura média ao solo, sobretudo os 
basaltos-andesitos tholeíticos.

% \begin{figure}[h]
%   \centering
%   \includegraphics[height=7cm]{figures/solo2010}
%   \caption{Mapa do solo da AEPQ-SM publicado na escala 1:25.000 \citep{MiguelEtAl2012}.\\Legenda de acordo com o sistema brasileiro de taxonomia do solo \citep{SantosEtAl2006}: PBAC - Argissolo Bruno-Acinzentado, PV - Argissolo Vermelho, C-R - Cambissolo-Neossolo, RY - Neossolo Flúvico, RL - Neossolo Litólico, RL-RR - Neossolo Litólico-Neossolo Regolítico, RR - Neossolo Regolítico, SX - Planossolo Háplico.}
%   \label{fig:solo2010}
% \end{figure}

As áreas com maior desenvolvimento pedogenético em profundidade ocorrem na paisagem menos declivosa 
do Planalto (rochas vulcânicas), algumas coxilhas (rochas sedimentares) e depósitos aluviais, mas 
com menor expressão territorial \citep{Miguel2010}. Quando em áreas de material de origem vulcânica,
sobretudo basaltos-andesitos tholeíticos, encontra-se solo classificado como Argissolo Vermelho 
Alítico típico. Trata-se de solo com horizonte superficial de textura arenosa a média, sobrejacente 
a um horizonte subsuperficial de textura média a argilosa \citep{Miguel2010}. Nas áreas do Planalto 
em que ocorrem riólitos-riodacitos granofíricos, o solo costuma apresentar menor desenvolvimento, 
predominantemente classificado como Neossolo Litólico e Cambissolo Háplico. Em pequenas manchas, o 
solo é classificado como Argissolo Vermelho-Amarelo. O menor desenvolvimento do solo originário de 
riólitos-riodacitos granofíricos deve-se, exatamente, a maior resistência da rocha ao intemperismo, 
uma vez que possui maior teor de sílica, sobretudo na forma de grandes de cristais de quartzo 
cristalizados a baixa temperatura (\SI{<600}{\celsius}) \citep{Pedron2007}.

Quando em áreas de material de origem sedimentar (Formação Caturrita), o solo costuma ser 
classificado como Argissolo Bruno-Acinzentado Alítico abrúptico. Trata-se de solo com horizonte 
superficial de textura arenosa que transiciona de maneira abrupta para um horizonte subsuperficial 
de textura argilosa \citep{Miguel2010}. Essa descontinuidade textural pode ser devida a fatores bem 
distintos. O primeiro deles estaria relacionado às características do próprio material de origem 
que, por ter sido formado em ambiente fluvial durante um período de mudança climática no Triássico 
\citep{PieriniEtAl2002}, apresenta camadas deposicionais com granulometria diferenciada. Assim, a 
presença de camadas de siltitos e folhelhos pode ter contribuído para a formação da descontinuidade 
textural. Outra hipótese trata da contribuição dos arenitos da Formação Botucatu e 
\textit{intertrap} na Formação Serra Geral. Dado que esse material está localizado em posições 
superiores na paisagem e são bastante suscetíveis à erosão, o mesmo pode ter contribuído para a 
formação do horizonte superficial arenoso do solo. Em ambos os casos, a hipótese formulada refere-se
à ocorrência de descontinuidade litológica, com a diferença de que no primeiro caso as litologias 
pertencem à mesma formação geológica. Por fim, a descontinuidade textural observada pode ser 
resultado dos processos pedogenéticos, sobretudo a perda (erosão lateral seletiva) e translocação 
(argiluviação) das partículas mais finas. Até o presente momento, essa é a hipótese mais aceita.

Nas áreas deprimidas da paisagem do Planalto, formando pequenas bacias de acumulação, ou nas áreas 
planas ao longo dos cursos de água da Depressão Periférica, o solo é classificado como Planossolo 
Háplico Alítico típico \citep{Miguel2010}. Trata-se de solo com horizonte superficial de textura 
média sobre horizonte subsuperficial de textura média a argilosa, podendo ou não apresentar 
horizonte intermediário eluvial. Ainda mais próximo dos cursos de água, o solo é classificado como 
Neossolo Flúvico Tb Eutrófico fragmentário \citep{Miguel2010}. Como o material de origem é diverso 
e possui arranjamento espacial discordante, a textura do solo é variável, mas sempre arenosa ou 
média, nunca argilosa, mesmo quando presente em áreas do Planalto ou do Rebordo. Por fim, com 
expressão territorial ainda menor, o solo é classificado como Neossolo Quartzarênico Órtico típico 
em alguns patamares do Rebordo, sejam eles de origem estrutural ou da deposição de colúvios do 
arenito da Formação Botucatu.

Apesar das características do solo da área apontarem para sua fragilidade, sobretudo sua 
granulometria, a geomorfologia possui papel preponderante sobre o potencial de perda de solo por 
erosão laminar \citep{Miguel2010}. Contudo, as áreas de maior declividade estão atualmente ocupadas 
por densa cobertura florestal \citep{SamuelRosaEtAl2011a}, reduzindo a fragilidade do solo. 
Estimativas indicam que apenas uma pequena fração da área deve apresentar perdas de solo por erosão 
laminar acima de valores toleráveis \citep{Miguel2010}. Algumas observações até mesmo indicam que 
essas estimativas estão acima da perda real de solo por erosão laminar na bacia \citep{Branco1998, 
MouraBueno2012}. Isso permite afirmar que os processos de degradação do solo na AEPQ-SM são de 
pequena intensidade e bastante pontuais (áreas urbanizadas). A provável redução da dinâmica de 
alteração das características da paisagem num futuro próximo deverá contribuir para a construção de 
um entendimento mais acurado da relação entre o solo e os demais componentes ambientais.

