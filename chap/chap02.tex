\artigotrue
\chapter{MODELO CONCEITUAL DE PEDOGÊNESE}
\label{chap:chap02}
%\SweaveUTF8


\def\ptkeys{Província Geológica do Paraná, Bacia do DNOS, Rebordo do Planalto, Fatores de formação do solo, 
Pedogênese}

\begin{chapterabstract}{brazilian}{\ptkeys}
 Este é o resumo em português.
\end{chapterabstract}

\def\enkeys{Paraná Geological Province, DNOS Catchment, Plateau Border, Soil formation factors, Pedogenesis}
  
\begin{chapterabstract}{english}{\enkeys}
 This is the English abstract.
\end{chapterabstract}

\formatchapter

\section{APRESENTAÇÃO}
\label{sec:chap02-apresentacao}

\titlenote{Colaboraram na preparação deste documento: Pablo Miguel (UFPel), Jean Michel Moura Bueno (UFSM), 
Ricardo Simão Diniz Dalmolin (UFSM), Andrisa Balbinot (UFSM), Lúcia Helena Cunha dos Anjos (UFRRJ), Gustavo de 
Mattos Vasques (Embrapa Soils), e Gerard B. M. Heuvelink (ISRIC -- World Soil Information).}

A modelagem espacial do solo inicia com a definição de um \emph{modelo conceitual de pedogênese}. Um modelo 
conceitual de pedogênese constitui uma representação verbal da realidade sob estudo que inclui a descrição 
explícita dos fatores e processos de formação do solo que determinam as características do solo e o seu padrão 
de distribuição espaço-temporal. Isso requer a reunião de toda a informação ambiental disponível e aplicação 
dos conceitos de relação solo paisagem, desenvolvimento do solo em catenas, ou outro modelo teórico de 
explicação da variação espacial do solo.

O presente documento apresenta o modelo conceitual de pedogênese da bacia de captação do reservatório do 
DNOS/CORSAN (Departamento Nacional de Obras de Saneamento/Companhia Riograndense de Saneamento), localizada na 
divisa entre os municípios de \itaara{} (ao norte) e \santamaria{} (ao sul), na porção sul da \baciaparana{},
estado do Rio Grande do Sul, Brasil (\autoref{fig:chap02-location}). A bacia de captação do reservatório do 
DNOS/CORSAN corresponde à cabeceira da bacia hidrográfica do \riovacacaimirim{}, tributário do \riojacui{} e, 
consequentemente, do \rioguaiba{} e da \lagoadospatos{}. A bacia de captação do reservatório do DNOS/CORSAN
cobre uma área de \SI{\pm29}{\square\kilo\metre} e alimenta um reservatório com volume máximo 
de \SI{\pm3800000}{\cubic\metre} em uma área inundada de \SI{0,74}{\square\kilo\metre}. Este reservatório 
contribui com até \SI{30}{\percent} do abastecimento de água da cidade de Santa Maria \cite{Dias2003, 
DillEtAl2004, Miguel2010}.

\begin{figure}[!ht]
\centering
\begin{minipage}[b]{95mm}
\subcaption{}
