%% Tipo de documento e a classe a ser usada para sua formatação.
\documentclass[tese, header]{UFRuralRJ}

%%==============================================================================
%% Pacotes - língua, codificação e fonte
%%==============================================================================

\usepackage[english]{babel}
\usepackage[T1]{fontenc} %% Conjunto de caracteres correto
\usepackage[utf8]{inputenc} %% Para acentuação correta
\usepackage{calligra}

%%==============================================================================
%% Pacotes - formatação de equações, números, elementos químicos
%%==============================================================================

\usepackage{amsmath,latexsym,amssymb}
\usepackage[range-phrase = --, binary-units = true]{siunitx} %% Sistema Internacional de Unidades
\DeclareSIUnit\pp{pp} % pencentual point

% Access bold symbols in maths mode
\usepackage{bm}

% Elementos químicos
\usepackage[version=4]{mhchem}

%%==============================================================================
%% Pacotes - formatação de figuras
%%==============================================================================

%% Importar figuras corretamente
\usepackage{graphicx}

%% Diretório onde estão as figuras dos capítulos
\graphicspath{{chap/}}

% Im­proved in­ter­face for float­ing ob­jects
\usepackage{float}
\usepackage{wrapfig}

\usepackage{Sweave}

% Draw figures
\usepackage{tikz}
\usetikzlibrary{calc}

%%==============================================================================
%% Pacotes - formatação de hyperlinks
%%==============================================================================
%% Opção 'hidelinks' disponível no pacote 'hyperref' a partir da versão 
%% 2011-02-05  6.82a. 'hidelinks' retira os retângulos do entorno das palavras
%% com links.

\usepackage[%hidelinks%, 
            bookmarksopen=true,linktoc=page,colorlinks=true,
            linkcolor=blue, citecolor=blue, filecolor=magenta, urlcolor=blue,
            % linkcolor=black, citecolor=black, filecolor=black, urlcolor=black,
            pdftitle={Analysis of Sources of Uncertainty in Soil Spatial Modelling},
            pdfauthor={Alessandro Samuel-Rosa},
            pdfsubject={Tese de Doutorado},
            pdfkeywords={Pedometria, Modelos, Incerteza}
            ]{hyperref}
\usepackage[hyphenbreaks]{breakurl} % lidar com url longa

%% Mudar o nome padrão das seções, subseções, e subsubseções mostrados quando usa-se \autoref{label}. Para 
%% todos os três casos, o padrão passa a ser <Seção>, sempre com a inicial maiúscula.
%% https://www.tug.org/applications/hyperref/manual.html
\addto\extrasenglish{%
  \def\subsubsectionautorefname{Section}%
  \def\subsectionautorefname{Section}%
  \def\sectionautorefname{Section}%
  \def\chapterautorefname{Chapter}%
} 

%%TODO: Margens conforme MDT UFSM 7ª edição. Corrigir no arquivo UFRuralRJ.cls 
%%      para funcionar a opção twoside *PENDENTE*
%\usepackage[inner = 30mm, outer = 20mm, top = 30mm, bottom = 20mm]{geometry}

%% Se o pacote 'hyperref' acima foi carregado, a linha abaixo corrige um bug na 
%% hora de montar o sumário da lista de figuras e tabelas. Comente a linha se o
%% pacote 'hyperref' não foi carregado.
\input{macros/bugcaption}

%%=============================================================================================================
%% Pacotes - formatação da bibliografia de acordo com as normas da ABNT e da UFRRJ
%%=============================================================================================================

% IMPORTANTE: O pacote 'abntex2cite' precisa, obrigatoriamente, ser carregado depois do pacote 'hyperref'
\usepackage[alf, abnt-and-type=&, abnt-etal-cite=2, abnt-etal-list = 0]{abntex2cite}

%%==============================================================================
%% Pacotes - formatação de verbatim
%%==============================================================================
%% O ambiente verbatim é o ambiente onde são inseridos exemplos de código fonte.
%% Está opção adiciona cor de fundo ao ambiente verbatim.

\let\oldv\verbatim
\let\oldendv\endverbatim
\def\verbatim{\par\setbox0\vbox\bgroup\oldv}
\def\endverbatim{\oldendv\egroup\fboxsep0pt\noindent\colorbox[gray]{0.95}{\usebox0}\par}

%%==============================================================================
%% Packages - other
%%==============================================================================

% Include PDF documents in LaTeX
\usepackage{pdfpages}

% Place selected parts of a document in landscape
\usepackage{lscape}

% Publication quality tables in LaTeX
\usepackage{booktabs}

% Flexible typesetting of table and figure floats using key/value directives
\usepackage{ctable}
\usepackage{multirow}

% Customising captions in floating environments
\usepackage[font=small, labelfont=bf, compatibility=false]{caption}
\captionsetup{format=hang}

% Support for sub-captions
\usepackage[skip=0pt,position=top,singlelinecheck=off,justification=raggedright, 
            font+=footnotesize]{subcaption}

% A range of dash commands for compound words
\usepackage[shortcuts]{extdash}

% Crossing out sentences (\sout{})
\usepackage[normalem]{ulem}

% Con­trol lay­out of item­ize, enu­mer­ate, de­scrip­tion
\usepackage{enumitem}

%%==============================================================================
%% User-defined macros
%%==============================================================================

\newcommand{\Rpackage}[1]{\texttt{R}-package \texttt{#1}} % reference to R-packages
\newcommand{\refsec}[1]{\hyperref[sec:#1]{Section \ref{sec:#1}}} % link to a section in the document
\newcommand{\reffig}[1]{\hyperref[fig:#1]{Figure \ref{sec:#1}}} % link to a figure in the document
\newcommand{\scale}[1]{cartographic scale of 1:\num{#1}} % scale
\newcommand{\scales}[2]{cartographic scales of 1:\num{#1} and 1:\num{#2}} % scales
\newcommand{\q}[1]{``#1''} % double quotes
\newcommand{\cited}[1]{\q{#1}} % direct citation
\newcommand{\grass}[1]{GRASS module \texttt{#1}} % GRASS modules
\newcommand{\gdal}[1]{GDAL module \texttt{#1}} % GDAL modules
\newcommand{\saga}[1]{SAGA library \texttt{#1}}  % SAGA libraries
\newcommand{\covar}[1]{\texttt{#1}} % covariates
\newcommand{\rr}{\textsuperscript{\tiny\textregistered}}
% \def\citet{\citeonline}
\let\citet\citeonline
{}

% Hyperlinks and URLs
\def\atcorrbug{\href{http://lists.osgeo.org/pipermail/grass-dev/2014-February/067540.html}{bug}} % atcorr bug
\def\baciaparana{\href{http://pt.wikipedia.org/wiki/Bacia_do_Paran\%C3\%A1}{Bacia Sedimentar do Paraná}}
\def\bgs{\href{http://www.bgs.ac.uk/}{BGS}}
\def\cgiar{\href{http://www.cgiar.org/}{CGIAR}} % Consultative Group for International Agricultural Research
\def\cran{\href{http://cran.us.r-project.org/}{CRAN}} % The Comprehensive R Archive Network
\def\git{\href{https://en.wikipedia.org/wiki/Git_\%28software\%29}{Git}}
\def\github{\url{https://github.com/samuel-rosa/dnos-sm-rs-general/}}
\def\fao{\href{http://www.fao.org/index_en.htm}{FAO}}

\def\gates{%
\href{%
http://www.gatesfoundation.org/what-we-do/global-development/agricultural-development%
}{%
Bill \& Melinda Gates%
}%
}%

\def\gsif{\href{http://www.isric.org/projects/global-soil-information-facilities-gsif}{GSIF}}
\def\gsm{\href{http://www.globalsoilmap.net/}{GlobalSoilMap}}
\def\gsp{\href{http://www.fao.org/globalsoilpartnership/}{GSP}}
\def\geoderma{\href{http://www.journals.elsevier.com/geoderma/}{Geoderma}} % Geoderma

% Intergovernmental Technical Panel on Soils
\def\itps{\href{http://www.fao.org/globalsoilpartnership/intergovernmental-technical-panel-on-soils/}{ITPS}}
\def\isric{\href{http://www.isric.org}{ISRIC}} % ISRIC
\def\inpe{\href{http://www.inpe.br/}{INPE}} % INPE
\def\inpedgi{\href{http://www.dgi.inpe.br/siteDgi_EN/index_EN.php}{INPE-DGI}} % INPE-DGI
\def\iso{\href{http://www.iso.org/iso/catalogue_detail.htm?csnumber=13736}{ISO}}
\def\itaara{\href{http://pt.wikipedia.org/wiki/Itaara}{Itaara}}

% Universal Soil Classification System
\def\iussusc{\href{http://www.iuss.org/index.php?article_id=525}{IUSS}}
\def\usdausc{\href{http://www.nrcs.usda.gov/wps/portal/nrcs/main/soils/survey/class/}{USDA}}

\def\lagoadospatos{\href{http://pt.wikipedia.org/wiki/Lagoa_dos_Patos}{Lagoa dos Patos}}
\def\mma{\href{http://geocatalogo.ibama.gov.br/}{MMA}}
\def\pronassolos{\href{https://goo.gl/zbMK24}{PRONASOLOS}}
\def\redemds{%
\href{%
https://www.embrapa.br/busca-de-noticias/-/noticia/2062813/solo-brasileiro-agora-tem-mapeamento-digital%
}{%
RedeMDS%
}%
}%
\def\riovacacaimirim{\href{http://pt.wikipedia.org/wiki/Rio_Vacaca\%C3\%AD-Mirim}{Rio Vacacaí-Mirim}}
\def\riojacui{\href{http://pt.wikipedia.org/wiki/Rio_Jacu\%C3\%AD}{Rio Jacuí}}
\def\rioguaiba{\href{http://pt.wikipedia.org/wiki/Lago_Gua\%C3\%ADba}{Rio Guaíba}}
\def\santamaria{\href{http://pt.wikipedia.org/wiki/Santa_Maria_\%28Rio_Grande_do_Sul\%29}{Santa Maria}}
\def\tcu{\href{https://www.governancadosolo.gov.br/}{TCU}}
\def\topodata{\href{http://www.dsr.inpe.br/topodata/}{TOPODATA}}
\def\ufsm{\href{http://site.ufsm.br/}{UFSM}}
\def\WorldSoilDay{\href{http://www.un.org/apps/news/story.asp?NewsID=49520}{UN}}

% Variables
\def\geoNew{\texttt{GEO\_25}}
\def\geoOld{\texttt{GEO\_50}}
\def\soilNew{\texttt{SOIL\_25}}
\def\soilOld{\texttt{SOIL\_100}}
\def\demNew{\texttt{ELEV\_10}}
\def\demOld{\texttt{ELEV\_90}}
\def\landOld{\texttt{LU1980}}
\def\landNew{\texttt{LU2009}}
\def\googleearth{Google Earth\textregistered{}}

%%==============================================================================
%% Identificação do trabalho
%%==============================================================================
\titulo{Análise de Fontes de Incerteza na Modelagem Espacial do Solo}
\author{Samuel-Rosa}{Alessandro}
\instituto{Instituto de Agronomia}
\curso{Curso de Pós-Graduação em Agronomia--Ciência do Solo}
\area{Ciência do Solo}
\local{Seropédica}{RJ}{Brasil}

%%==============================================================================
%% Identification of supervisors
%%==============================================================================
\advisor[Professora]{Ph.D.}{Anjos}{Lúcia Helena Cunha dos}{UFRRJ}
\coadvisor[Pesquisador]{Ph.D.}{Vasques}{Gustavo de Mattos}
\coadvisor[Professor]{Ph.D.}{Heuvelink}{Gerardus Bernardus Maria}

%%==============================================================================
%% Information about the defence
%%==============================================================================
\committee[Ph.D.]{Vasques}{Gustavo de Mattos}{EMBRAPA} %% President
\committee[Dr.]{Ceddia}{Marcos Bacis}{UFRRJ} %% Examinator
\committee[Ph.D.]{Teixeira}{Wenceslau Geraldes}{EMBRAPA} %% Examinator
\committee[Ph.D.]{Lopes Assad}{Maria Leonor Ribeiro Casimiro}{UFSCar} %% Examinator
\committee[Ph.D.]{Oliveira}{Ronaldo Pereira de}{EMBRAPA} %% Examinator
\date{24}{Fevereiro}{2016} %% Date of defence

%%=============================================================================
%% Início do documento
%%=============================================================================
\begin{document}

%%=============================================================================
%% Capa e folha de rosto
%%=============================================================================
\maketitle

%%=============================================================================
%% Ficha catalográfica
%%=============================================================================
% Como a CIP vai ser impressa atrás da página de rosto, as margens inner e outer	
% devem ser invertidas.
%\newgeometry{inner=20mm,outer=30mm,top=30mm,bottom=20mm}
%\makeCIP{alessandrosamuel@yahoo.com.br}% email do autor
%\restoregeometry

%Se for usar a catalogação gerada pelo gerador do site da biblioteca comentar as linhas
%acima e utilizar o comando abaixo
%\includeCIP{CIP.pdf}

%%=============================================================================
% Folha de aprovação
%%=============================================================================
\makeapprove

%%=============================================================================
%% Agradecimentos ou Prefácio (opcional)
%%=============================================================================
%% Usar versão estrelada do comando 'chapter'.
\chapter*{Preface}

I was never sure about what a thesis should consist of: I worked on so many things during the four years of my 
doctorate that I found myself somewhat lost when I had to decide what to write in the thesis. There are 
official documents suggesting \emph{how} the thesis should be written, but not exactly \emph{what} should be 
written -- I find the definitions somewhat vague. For example, the manual of our university states that a 
\q{thesis consists of the result of a research which is presented as the final requirement for the completion 
of a doctorate}\footnote{\citeonline{UFRRJ2006}}, which is quite the same thing said by the International 
Organization for Standardization (\iso): a \q{document which presents the author's research and findings and 
submitted by him in support of his candidature for a degree or professional 
qualification}\footnote{\citeonline{ISO1986}}. I tried reading other theses to see if I could get an 
inspiration. I also discussed this matter with my patient supervisors Lúcia Anjos (Universidade Federal Rural 
do Rio de Janeiro, Brazil), Gustavo Vasques (Embrapa Solos, Brazil), and Gerard Heuvelink (ISRIC -- World Soil 
Information, the Netherlands). Unfortunately, for one reason or another, I was never satisfied with the 
outcome.

At first, I was a bit desperate. Have I failed? Has everyone failed? I hoped not! Perhaps the lack of an 
objective, ultimate, universal definition of what a thesis should consist of meant that, as a doctorate 
student, it was my responsibility to construct such a definition. This idea gave me back the long-lost 
excitement to write my thesis. I did not want to follow a boring ritual. I wanted to have fun and be completely 
honest with the reader, as Richard Webster\footnote{\citeonline{Webster2003}} had once suggested. 
As such, I started thinking about all steps given since the start of the doctorate, something like \q{Once upon 
a time in Seropédica...}.

As the title says, this thesis is about a research on the factors determining a soil map to be more or less 
accurate, what I call \emph{sources of uncertainty}. Many of these sources are known, others are still unknown, 
and some are disregarded due to our ignorance -- or by convenience. When I wrote my doctorate research project, 
it seemed appropriate to aim at evaluating what I understood as being the main sources of uncertainty in the 
process of building a model to produce soil maps, a process that I call \emph{soil spatial modelling}. The 
reason was simple: soil spatial modelling using modern techniques was a growing activity in Brazilian 
universities and research centres, and I felt that many \emph{soil spatial modellers} were inclined towards 
using the most expensive data sources as the only way of producing higher accuracy soil maps of the Brazilian 
territory. I was preoccupied about these ideas -- which appeared to be sort of an euphoria about new remote 
sensors -- because I believed that high quality soil maps could be produced if we simply started using the data 
at hand.

Defining the main sources of uncertainty in soil spatial modelling required an operational definition, which 
was given based on the observation that, in general, the main decisions made by soil spatial modellers concern 
the a) calibration observations, b) covariates, and c) model structure. The general objective of evaluating 
these three mains sources of uncertainty was then divided into five specific objectives:

\begin{enumerate}[label=(\Roman*)]
\item Identify appropriate calibration sample sizes and designs for soil spatial modelling;

\item Determine the accuracy of freely available covariates and their suitability to calibrate soil spatial 
models;

\item Identify appropriate covariate selection methods to build linear soil spatial models;

\item Assess the effect of multicollinearity among covariates on the performance of linear soil spatial models;

\item Identify database scenarios in which non-linear soil spatial models are more efficient than linear soil 
spatial models.
\end{enumerate}

The idea was to deal with each of the objectives separately and present the results in individual chapters of 
the thesis which would be submitted for publication in peer reviewed journals. The main expected result was the 
definition of a sound \emph{working protocol} that would allow the construction of efficient soil spatial 
models. My goal was to contribute to national (Brazilian Research Network on Digital Soil Mapping -- \redemds) 
and international (\gsm{} and Global Soil Information Facilities -- \gsif) initiatives, while generating a 
significant amount of bibliographic material to support the teaching of modern soil spatial modelling 
techniques in soil classes at Brazilian universities.

\def\footlatex{\footnote{See more about \LaTeX{} in \href{https://en.wikipedia.org/wiki/LaTeX}{Wikipedia}. The 
\LaTeX{} 
class that I have adapted to compile this thesis is available in 
\href{https://github.com/samuel-rosa/UFRuralRJ}{GitHub}.}}

With time it became clear that the five objectives and the expected results were too ambitious. I certainly was 
overwhelmed by the knowledge of the multiple sources of uncertainty, and felt compelled to develop a very 
thorough study. But I forgot that a doctorate includes more activities than those planned in the research 
project: you take classes, prepare grant proposals, write reports, help colleagues, get involved in other 
projects -- such as adapting the \LaTeX{}\footlatex{} class used to compile this thesis --, publish the papers 
of your master thesis, train undergraduate students, create and maintain the newsletter of a scientific group, 
read many articles and books, learn a couple of computer languages, start a relationship, get sick, and so on. 
Then, one day you realize that two years are already gone by and you still are preparing the database with 
which you will develop your case studies.

\def\footr{\footnote{See more about \texttt{R} in 
\href{https://en.wikipedia.org/wiki/R_\%28programming_language\%29}{Wikipedia}.}}

I know that I was particularly lucky for most of the soil and covariate data already being available for my 
use. This is because I have decided very early to continue using the data that I collected during my master so 
that I could go deeper into the details of modern soil spatial modelling techniques. Looking back, I think that 
this was the right decision. However, the resources needed to properly organize the data before I could 
actually use it were considerable. This effort was in line with my original intent of defining a working 
protocol for constructing soil spatial models, which I guess to have achieved, at least partially. Then I 
realized that I also needed to make my research the most reproducible as possible. The way to go was to make a 
thorough description of all data processing steps, including making available all computer scripts so that they 
could be reused by other people. The result was thousands of lines of computer code, mostly on 
\texttt{R}\footr{}, which I used to indirectly access most of other computer programs. These computer scripts 
have shown to be invaluable for my own applications, and I always hoped that other people will find them useful 
as well. But I then learned that many well known methods of data analysis/processing are not used simply 
because they are not implemented in a (single) software package. As such, making only the computer scripts 
available did seem to be a poor solution. Developing and maintaining a software package in the most popular 
environment for data processing and analysis, i.e. \texttt{R}\footr{}, was a natural decision. Although being 
fun, programming took a lot of resources!

A significant amount of resources was also spent preparing a description of the soil-forming factors and 
processes that determine the soil spatio-temporal distribution in the study area where the case studies were to 
be developed. Such a description is what I call \emph{conceptual model of pedogenesis}. This was another effort 
in line with the definition of a working protocol because I believe that soil spatial modelling is not only 
about making maps, but also constructing soil knowledge. Within the scope of the thesis, this knowledge was 
expected to serve the development of an experiment devoted to meeting the third objective of the research 
project. My intent was to compare automated covariate selection methods with the use of expert knowledge. 
Preliminary tests were conducted with a few experts to help planning the experiment, which was believed to be a 
complex one. Preliminary results were encouraging, but since I needed to give more attention to the first and 
second objectives, I had to temporarily stop working on the third objective.

There also was my poor knowledge on some known topics, which sometimes took me to the wrong direction. For 
example, I wanted to evaluate how much more accurate a soil map is when more accurate covariates are used 
(second objective), the reason being that I was concerned with the fact that the covariates too are in error. 
As such, I collected field data to validate the covariates and correct them for any systematic errors. Only 
later, discussing with Gerard, I understood that 1) the validation data was poor, and 2) in soil spatial 
modelling the covariates are generally taken as they are. The latter is like assuming that the covariates were 
measured without error -- otherwise a technique called error propagation analysis (or uncertainty analysis) can 
be employed to take that error into account. As such, the second objective of the research project needed to be 
reformulated in terms of how to define the different covariates that we had at hand. After many discussions we 
still were unable to reach a satisfactory solution, which did not prevent the study from being developed. Quite 
interesting results were produced, but presenting them also was a challenge: many models and covariates had 
been compared, and we wanted to have a summary way of presenting them, preferentially a figure. We came up with 
a figure that we later called a \emph{model series plot}, i.e. a figure that depicts a series of models ordered 
according to some chosen summary performance statistic. For the purpose of our study, that was a useful figure, 
and I hope that the readers have understood how to interpret it. The reviewers of our paper were fundamental 
for improving the description of the model series plot. Fortunately, they were also able to help deciding upon 
a proper definition for the differences observed between the covariates that were being compared. The study was 
not exactly about their accuracy, but about how they were produced and their level of spatial detail.

Devising an experiment to evaluate the influence of sample size and design on the accuracy of soil maps also 
was a challenge. Most soil data used for soil spatial modelling were produced in the past century (legacy data) 
with observation locations purposively selected by soil spatial modellers using tacit rules. As such, I wanted 
to build an algorithm composed of a set of objective decision rules that would produce spatial samples similar 
to those produced by a soil spatial modeller. I would then simulate budget scenarios for sampling and see how 
different spatial samples would perform regarding soil map accuracy. But how to devise such an algorithm? I 
interviewed the soil spatial modellers that produced the soil data to be used to conduct the case studies, 
carried out a point pattern analysis of the resulting spatial sample configuration, and explored psychological 
concepts to understand the whys of the locations of the sampling points. A lengthy study was carried out, which 
provided evidence that many poorly understood factors influence the decision of soil spatial modellers on where 
to make soil observations. From one perspective, this enables one to plan more efficient soil observation 
campaigns. However, it did not help finding a practical solution for the problem that we had at hand. Perhaps 
it was more appropriate to explore the existing, less complex algorithms that produce spatial samples using 
more objective decision rules formulated with basis on conceptual and operational factors.

\def\foottravel{\footnote{See about the \emph{travelling salesman problem} at 
\href{https://en.wikipedia.org/wiki/Travelling_salesman_problem}{Wikipedia}.}}

We then invited Dick Brus (Alterra, the Netherlands) to participate devising the experiment to evaluate the 
influence of sample size and design on the accuracy of soil maps. After some talks and a bibliographic review, 
we decided for using sampling algorithms that are based on the so-called spatial simulated annealing which are 
commonly used to produce spatial samples for soil spatial modelling. The problem was that we did not know of 
spatial simulated annealing being implemented in any free and open source software package in a way that could 
meet the requirements of our study. Again, the solution was to work on our own implementation of spatial 
simulated annealing, which resulted in a second package for R. Having decided the sampling algorithm to 
work with, we needed to choose a sound method to take sampling costs into account. Because the access time to 
sampling points usually is the major cost component in soil sampling, Gerard and I thought of coupling with 
spatial simulated annealing an algorithm to solve the problem of travelling from one sampling point to the next 
with the least cost\foottravel. This would be a good piece of work, but we soon realized that solving the 
travelling problem was impossible given the available resources. The goal of taking sampling costs into account 
ended up being dropped out.

After some time working on the sampling experiment, which at the time seemed simpler than ever, we came to 
learn that the sampling algorithms that we had chosen had weaknesses -- apparently like any algorithm. So, we 
thought that, perhaps, we could improve on those algorithms! A literature review suggested that we were correct 
and there was room for algorithmic improvements. Working on these improvements took a lot of resources, and I 
guess we came up with interesting, sound solutions. We only needed to know if the algorithmic improvements had 
any practical added value before evaluating the influence of sample size and design of prediction accuracy. It 
seemed appropriate to carry out two experiments, the first to evaluate the algorithmic improvements, the second 
to compare the algorithms. Again, the results were promising, specially from the algorithmic point of view. 
With regard to prediction accuracy, we cannot make high claims because the algorithms were tested using a 
single case study. But this gap should be easily filled since we have made our software package freely 
available for anyone to use. The negative side of these important developments is that carrying out the 
experiment to evaluate covariate selection methods became impossible due to the remaining resources available. 
This was a pity because I visited Murray Lark at the the British Geological Survey (\bgs) headquarters in 
Nottingham, UK, to discuss about that experiment.

Like it happened with the third objective, there was not enough time to conduct experiments to meet the 
fourth and fifth objectives. I think the topic of the fourth objective is a very important one, directly 
related to the problem of selecting covariates, which I really wanted to deal with. I would 
have been very happy if I could discuss the topic at least partially, but perhaps partial solutions may not be 
very useful. As for the fifth objective, I believe that developing a sound globally relevant work on the topic 
requires using several datasets, not a single dataset as I explored in my research project.

As a consequence of all these events, at the end of the doctorate, I was able to meet only the first and second 
\emph{scientific} objectives -- \emph{scientific} in the sense of answering research questions -- of my 
doctorate research project. This may seem little but only because the research project was too ambitious. Aside 
from meeting the original scientific objectives, I guess I made other important contributions that one may call 
\emph{technical} contributions -- \emph{technical} in the sense of practical application -- that were part of 
the original goal of defining a working protocol for soil spatial modelling. This includes, for example, 
documenting the soil and covariate data, as well as their processing steps, and describing the soil-forming 
factors that determine the soil spatio-temporal distribution in the study area. I know that these technical 
contributions do not help meeting any of the original scientific objectives. The same applies to the two 
\texttt{R}-packages that I developed and maintained, and the experiment conducted to understand how soil 
spatial modellers decide upon where to observe the soil, which were not originally planned in the research 
project. But I guess this is still valid, perhaps very important, as it seems to be common in any scientific 
research: you end up doing many things that are quite different from those that you were originally planning to 
do.

So... this is, more or less, the \emph{story} of the research that I have carried out in collaboration with my 
supervisors and co-authors during my doctorate. Perhaps one will find this story biased towards the negative 
aspects of my doctorate. This is not entirely false, specially because I think that I generally tend to be 
happier with stories that have more errors than hits -- because I learn more with the former than with the 
latter. As such, I guess there is nothing to write in this thesis other than what I have done during the 
doctorate that is directly and/or indirectly related to the original research project, have it been planned or 
not, be it a technical or scientific contribution, completed or not. This is what I present as the final 
requirement for the completion of the doctorate.

I do hope that my supervisors and co-authors like the work that I have done in the past four years. Lúcia, 
Gustavo, Gerard, and Dick have been so patient, so understanding, so respectful, that I have no words to 
prepare the deserved thanks. Each one of them with a different background, a different life story, a different 
perspective, working in a different part of the world... I learned a lot with them: soil science, mathematics, 
statistics, informatics, English, politics and science, human relations, and much more. What a pleasure 
experience working with the four of you!

I also hope that the outcome of my doctorate is of interest for the supporting institutions, 
because I would never write this thesis without their support. These are:

\begin{itemize}
 \item Universidade Federal Rural do Rio de Janeiro, through the Post-Graduate Course in Agronomy -- Soil 
 Science and Department of Soil Science, for providing a solid soil science education, unconditionally 
 supporting my research, and finding the means to guarantee my participation in several international events;
 
 \item Ministry of Science and Technology of Brazil, through the CNPq Foundation (Process 140720/2012-0), that
 provided a three-year grant without which I would not be able to develop any research at all;
 
 \item Ministry of Education of Brazil, through the CAPES Foundation (Process ID BEX 11677/13-9), that funded
 my one-year stay in the Netherlands, where most of the research was actually developed;
 
 \item ISRIC -- World Soil Information, for unconditionally supporting my research.
 
 \item Embrapa Soils, for supporting my research.
 
 \item Universidade Federal de Santa Maria, through the Department of Soil Science, for supporting my research.

 \item Ministry of the Environment of Brazil, for providing some of the data that we used.
\end{itemize}

I need to note that six individuals gave important contributions during the preparation of the 
data that were used to develop the case studies. Three of them are from the Universidade 
Federal Rural do Rio de Janeiro: Fabio Paes Leme Ferreira, Anastácia Perci Campos de Almeida, and Mauro Antônio 
Homem Antunes. The other three are from the Universidade Federal de Santa Maria: Ricardo Simão Diniz Dalmolin, 
Jean Michel Moura Bueno, and Luis Fernando Chimelo Ruiz. Along with them, I must thank the development teams 
and module/package authors of the many free and open source software and operating system that were used to 
develop the case studies.

Many other individuals have also given some form of scientific and/or technical contribution, be it through 
the exchange of email messages, chatting during a coffee break, a short visit to their home institutions, or 
any other informal occasion. Their contributions were invaluable for usually raising unforeseen questions, 
pointing to extremely helpful references, and helping me seeing the research problems from another perspective.
Special thanks are due to Ad van Oostrum, Andreas Papritz, Bas Kempen, Bradley Miller, Chantal Hendriks, 
Dominique Arrouays, Edgardo Ramos Medeiros, Eloi Carvalho Ribeiro, Jorge Mendes de Jesus, Madlene Nussbaum, 
Marcos Angelini, Murray Lark, Nicolas Saby, Pablo Miguel, Pedro de Souza Calegaro, Richard Webster, Thomas 
Caspari, Tom Hengl, Titia Mulder, and many others that I might have forgotten.

Having the story of a doctorate to tell depends not only on scientific and/or technical contributions,
but also on the unconditional support of family, friends, and colleagues. My family, which gained several new 
members in the last four years, was strong enough to understand my rare and short visits, and the importance 
of a long stay on the other side of the Atlantic. Although my mother (Elaine), father (Adir), and brother 
(Eduardo) have only a vague idea of what I have been working on, their support was incommensurable. 
South-American and European friends were invaluable: Marcos, Indira, Eloi, Thomas, Nina, Jorge, André, Thiago, 
and Andriéli. There was the amazing, coolest of them all, ISRIC family, which I miss very much, lead by 
our dearest Yolanda. There also were my postgraduate colleagues and teachers. But one friend deserves a special 
thanks: Manoeli! Thank you Manoeli -- by destiny or a strange coincidence --, born and raised in the same small 
town, after completing the bachelor and master together in Brazil, we ended up sharing the same house in The 
Netherlands during the doctorate. There, we continued our philosophical chats about all aspects of life, the 
universe, and everything which we started years ago. It was a pleasure to share a full year with such a 
wonderful friend!

Finally, three little Bacchanalian creatures, one having only a vague idea of my work, the other two not 
speaking any human language, deserve my most sincere thanks: Monique, Tupã, and Dindi! Your love was fire 
through my veins!

\begin{flushright}
 Alessandro Samuel Rosa
 
 Seropédica, February 2016.
\end{flushright}


%%=============================================================================
%% Resumo geral em português
%%=============================================================================
% \def\tituloportugues{Análise de fontes de incerteza na modelagem espacial do solo}
% \def\chavesportugues{Pedometria. Mapeamento Digital do Solo. Dados de Solo e Covariáveis}
% \generalabstracttrue
% \begin{generalabstract}{brazilian}{\tituloportugues}{\chavesportugues}
% 
% \end{generalabstract}

%%=============================================================================
%% English General Abstract
%%=============================================================================
\def\titleEn{Analysis of sources of uncertainty in soil spatial modelling}
\def\nivelEn{Doctor of Science in Agronomy, Soil Science}
\def\keyEn{Pedometrics. Digital Soil Mapping. Soil and Covariate Data}

\generalabstracttrue
\begin{generalabstract}{english}{\titleEn}{\keyEn}{\nivelEn}
Modern soil spatial modelling is based on statistical models to explore the empirical relationship among 
environmental conditions and soil properties. These models are a simplification of reality, and their outcome 
(soil map) will always be in \emph{error}. What a soil map conveys is what we expect the soil to be, 
acknowledging that we are \emph{uncertain} about it. The objective of this thesis is to evaluate important 
sources of uncertainty in spatial soil modelling, with emphasis on soil and covariate data. Case studies 
were developed using data from a catchment located in Southern Brazil. The soil spatial distribution in 
the study area is highly variable, being determined by the geology and geomorphology (coarse spatial 
scales), and by agricultural practices (fine spatial scales). Four topsoil properties were explored: clay 
content, organic carbon content, effective cation exchange capacity and bulk density. Five covariates, each 
with two levels of spatial detail, were used: area-class soil maps, digital elevation models, geologic maps, 
land use maps, and satellite images. These soil and covariate data constitute the \emph{Santa Maria dataset}. 
Two packages for \texttt{R} were created in support of the case studies, the first (\texttt{pedometrics}) 
containing various functions for spatial exploratory data analysis and model calibration, the second 
(\texttt{spsann}) designed for the optimization of spatial samples using simulated annealing. The case studies 
illustrated that existing covariates are suitable for calibrating soil spatial models, and using more 
detailed covariates results in only a modest increase in the prediction accuracy that may not outweigh the 
extra costs. More efficient means of increasing prediction accuracy should be explored, such as obtaining more 
soil observations. For this end, one should use objective means for selecting observation locations to 
minimize the effects of psychological responses of soil modellers to conceptual and operational factors on the 
sampling design. This is because conceptual and operational difficulties encountered in the field determine 
how the motivation of soil modellers shifts between learning/verifying soil-landscape relationships and 
maximizing the number of observations and geographic coverage. For the sole purpose of spatial trend 
estimation, it should suffice to optimize spatial samples aiming only at reproducing the marginal distribution 
of the covariates. For the joint purpose of optimizing sample configurations for spatial trend and variogram 
estimation, and spatial interpolation, one can formulate a sound multi-objective optimization problem using 
robust versions of existing sampling algorithms. Overall, we have learned that a single, universal recipe for 
reducing our uncertainty in soil spatial modelling cannot be formulated. Deciding upon efficient ways of 
reducing our uncertainty requires, first, that we explore the full potential of existing soil and covariate 
data using sound spatial modelling techniques.
\end{generalabstract}

%%=============================================================================
%% Lista de figuras
%%=============================================================================
\listoffigures

%%=============================================================================
%% Lista de tabelas (comentar se não houver)
%%=============================================================================
\listoftables

%%=============================================================================
%% Lista de Apêndices (comentar se não houver)
%%=============================================================================
\listofappendix

%%=============================================================================
%% Sumário
%%=============================================================================
\tableofcontents

%%=============================================================================
%% Início da tese
%%=============================================================================

% Adiciona cada capítulo
\selectlanguage{english}
\setcounter{page}{1}
\artigofalse
\chapter{GENERAL INTRODUCTION}
\label{chap:introduction}

Modern soil spatial modelling is based on using statistical models to explore the empirical relationship among 
environmental conditions and soil properties. These soil spatial models, like any other model, are nothing 
more than a gross simplification of reality. Unless we observe the soil everywhere -- which would destroy the 
soil and render the observations useless --, no matter how large the volume of data is, or how comprehensive 
our background knowledge, it will never be possible to construct a model that explains the entire complexity 
of the soil. Thus, the outcome of a soil spatial model, i.e. a soil map, will always deviate from the 
\q{truth} -- this deviation from the \q{truth} is what we call \emph{error}. What a soil map conveys is what 
we expect the soil to be, not our \emph{certainty} about it.

Because soil spatial modellers aim at using the available resources to produce the most accurate 
representation of the soil, a sensible research programme is to investigate the main causes for soil maps 
being more or less \emph{uncertain}. There are many sources of uncertainty in soil spatial modelling, such as 
the errors that result from using a poor statistical model and from making interpolations and extrapolations 
to predict soil properties at unvisited locations. Another important source of uncertainty is the data used to 
assess the empirical relationship among environmental conditions and soil properties: covariate and soil data.

The general objective of this thesis is to evaluate the soil and covariate data as source of uncertainty in 
soil spatial modelling. This general objective can be divided into specific objectives and their respective 
research questions:

\begin{enumerate}
 \item Determine the suitability of freely available covariates to calibrate soil spatial models.
  \begin{enumerate}[label=(\alph*)]
   \item Does the use of more detailed covariates result in considerably more accurate soil maps?
   \item How does incorporation of spatial dependence in a soil spatial model compare to the gain in 
   prediction accuracy obtained with using more detailed covariates?
   \item Are the answers to these research questions consistent across soil properties?
  \end{enumerate}
 
 \item Identify the factors that determine how field soil spatial modellers select soil observation locations.
  \begin{enumerate}[label=(\alph*)]
   \item Do the factors play the same role along the course of the soil observation process?
   \item Do the main criteria employed for deciding upon the location of soil observations have a pedological 
   origin?
   \item Can point pattern analysis help understanding the purposive sampling strategy traditionally employed
   by field soil spatial modellers?
  \end{enumerate}
 
 \item Identify appropriate calibration sample sizes and designs for soil spatial modelling.
  \begin{enumerate}[label=(\alph*)]
   \item Can theoretical and algorithmic improvements on existing spatial sample optimization algorithms 
   improve the performance of soil models?
   \item How suboptimal is it to use a sample configuration that was optimized to a different purpose than it 
   is going to be used for?
   \item Is the predictive performance of a soil-mapping model estimated using a sample configuration 
   optimized using heuristics poorer than that of another soil-mapping model whose parameters were estimated 
   using a sample configuration optimized using an \textit{a priori} knowledge of the model?
   \item Is it possible to obtain a sample configuration that is efficient in identifying and estimating i) 
   the spatial trend and ii) the variogram model, and iii) making spatial predictions?
   \item How does the sample configuration affect the estimated model parameters and thus the conclusions that 
   can be drawn under the light of the existing conceptual model of pedogenesis?
   \item Are the answers to the research questions above consistent across sample sizes and soil properties?
  \end{enumerate}
\end{enumerate}

The thesis is composed of eight chapters where each of the above mentioned objectives are met. Although there 
is a logical sequence in their presentation, all chapters were planned so that they could be read separately. 
This means that there is some overlap between them, i.e. repeated information. References to specific sections 
of other chapters using coloured (blue) hyperlinks are common.

\autoref{chap:chap02} is a commented review of the literature on soil spatial modelling and its main sources 
of uncertainty. The review starts with a discussion about the efforts made by soil spatial modellers to 
raise awareness about the importance of soil spatial information. These efforts seem to have fuelled a global 
scientific demand for up-to-date, high resolution soil spatial information. The chapter continues with a 
description of soil spatial modelling along human history, suggesting that the goal of producing soil maps 
remains more or less the same since the Neolithic Revolution (ca.~\num{10000}~years). The chapter closes with 
the main sources of uncertainty.

\autoref{chap:chap03} presents the conceptual model of pedogenesis (in Portuguese), which consists of a 
description of the study area that includes an explicit description of soil-forming factors (climate, geology, 
geomorphology, hydrology, land use, and vegetation) and processes that determine the soil spatio-temporal 
distribution. \autoref{chap:chap04} describes the soil data included in the \emph{Santa Maria dataset}, which 
was used to develop the case studies presented in this thesis. The Santa Maria dataset is composed of 
$n = 410$ soil observations compiled from studies carried out between \num{2004} and \num{2013}. These studies 
aimed at producing semi-detailed soil and land use maps, and modelling topsoil carbon stock and vulnerability 
to erosion. A comprehensive description of the covariate data included in the Santa Maria dataset, and their 
processing, is given in \autoref{chap:chap05}. The goal of these three chapters is to provide the bases for 
future soil spatial modelling exercises in the study area and to serve as examples for new soil spatial 
modelling studies developed elsewhere.

Based on an article published in the peer reviewed journal \geoderma, \autoref{chap:chap06} serves the purpose 
of meeting the first objective of the thesis and answering its respective research questions. There, the 
prediction performance of linear soil spatial models calibrated using covariates (area-class soil maps, land 
use maps, geological maps, digital elevation models, and orbital images) available in two levels of detail is 
evaluated. The influence of taking the spatial dependence of the residuals into account is assessed as well. 

\autoref{chap:chap07} presents an approach that aims at helping to understand the purposive sampling strategy 
traditionally employed by field soil modellers, i.e. free survey. This is important because most soil 
spatial modelling projects rely on legacy data, i.e. soil data produced many years ago, whose observation 
locations were purposively selected by soil spatial modellers using poorly formalized tacit rules. The 
chapter, designed to answer the research questions of the second objective of the thesis, shows that the 
location of soil observations were strongly determined by subjective elements unrelated to the local 
soil-landscape relationship. Understanding the reasons behind the location of free survey soil observations 
can help soil spatial modellers designing more efficient data-driven sampling strategies.

The results of the experiment devised to evaluate if improving a popular sampling algorithm results in more 
accurate spatial predictions is presented in \autoref{chap:chap08}. The comparison of five sampling algorithms 
on how they affect estimated model parameters and prediction accuracy is presented in \autoref{chap:chap09}. 
The chapter also introduces a new general purpose sampling algorithm. Both chapters contain only partial 
results, and will be the basis of manuscripts to be submitted to peer reviewed journals, both dealing with the 
second objective of this thesis and its respective research questions.

The sequence of eight chapters is closed with a \emph{General Conclusion} where I highlight the main results 
of the research, contributions and merits of the study. Next, there are two appendices, both devoted to the 
description of the two packages for \texttt{R} developed to support the case studies: \texttt{pedometrics} 
(\autoref{apen:pedometrics}) and \texttt{spsann} (\autoref{apen:spsann}). The first includes miscellaneous 
functions that were put together for ease of use. The second was designed for the optimization of sample 
configurations using spatial simulated annealing. All literature references are presented under a unique 
\emph{Bibliographic References} chapter (\autoref{chap:references}) at the end of the thesis.
 % 1. General introduction
\artigotrue
\chapter{MODELO CONCEITUAL DE PEDOGÊNESE}
\label{chap:chap02}
%\SweaveUTF8


\def\ptkeys{Província Geológica do Paraná, Bacia do DNOS, Rebordo do Planalto, Fatores de formação do solo, 
Pedogênese}

\begin{chapterabstract}{brazilian}{\ptkeys}
 Este é o resumo em português.
\end{chapterabstract}

\def\enkeys{Paraná Geological Province, DNOS Catchment, Plateau Border, Soil formation factors, Pedogenesis}
  
\begin{chapterabstract}{english}{\enkeys}
 This is the English abstract.
\end{chapterabstract}

\formatchapter

\section{APRESENTAÇÃO}
\label{sec:chap02-apresentacao}

\titlenote{Colaboraram na preparação deste documento: Pablo Miguel (UFPel), Jean Michel Moura Bueno (UFSM), 
Ricardo Simão Diniz Dalmolin (UFSM), Andrisa Balbinot (UFSM), Lúcia Helena Cunha dos Anjos (UFRRJ), Gustavo de 
Mattos Vasques (Embrapa Soils), e Gerard B. M. Heuvelink (ISRIC -- World Soil Information).}

A modelagem espacial do solo inicia com a definição de um \emph{modelo conceitual de pedogênese}. Um modelo 
conceitual de pedogênese constitui uma representação verbal da realidade sob estudo que inclui a descrição 
explícita dos fatores e processos de formação do solo que determinam as características do solo e o seu padrão 
de distribuição espaço-temporal. Isso requer a reunião de toda a informação ambiental disponível e aplicação 
dos conceitos de relação solo paisagem, desenvolvimento do solo em catenas, ou outro modelo teórico de 
explicação da variação espacial do solo.

O presente documento apresenta o modelo conceitual de pedogênese da bacia de captação do reservatório do 
DNOS/CORSAN (Departamento Nacional de Obras de Saneamento/Companhia Riograndense de Saneamento), localizada na 
divisa entre os municípios de \itaara{} (ao norte) e \santamaria{} (ao sul), na porção sul da \baciaparana{},
estado do Rio Grande do Sul, Brasil (\autoref{fig:chap02-location}). A bacia de captação do reservatório do 
DNOS/CORSAN corresponde à cabeceira da bacia hidrográfica do \riovacacaimirim{}, tributário do \riojacui{} e, 
consequentemente, do \rioguaiba{} e da \lagoadospatos{}. A bacia de captação do reservatório do DNOS/CORSAN
cobre uma área de \SI{\pm29}{\square\kilo\metre} e alimenta um reservatório com volume máximo 
de \SI{\pm3800000}{\cubic\metre} em uma área inundada de \SI{0,74}{\square\kilo\metre}. Este reservatório 
contribui com até \SI{30}{\percent} do abastecimento de água da cidade de Santa Maria \cite{Dias2003, 
DillEtAl2004, Miguel2010}.

\begin{figure}[!ht]
\centering
\begin{minipage}[b]{95mm}
\subcaption{}
 % 2. Uncertainty in Soil Spatial Modelling
\artigotrue
\chapter{THE SANTA MARIA DATASET. PART I -- SOIL DATA}
\chapternote{Collaborated in the preparation of this manuscript: Pablo Miguel (UFPel), Jean Michel Moura Bueno 
(UFSM), Ricardo Simão Diniz Dalmolin (UFSM), Andrisa Balbinot (UFSM), Lúcia Helena Cunha dos Anjos (UFRRJ), 
Gustavo de Mattos Vasques (Embrapa Solos), Gerard B. M. Heuvelink (ISRIC -- World Soil Information), and Ad 
van Oostrum (ISRIC -- World Soil Information).}
\shorttitle{Soil Data}
\label{chap:chap04}
%\SweaveUTF8


%\def\ptkeys{}
%\begin{chapterabstract}{brazilian}{\ptkeys}
% Este é o resumo em português.
%\end{chapterabstract}

\def\enkeys{Spatial soil modelling. Purposive sampling. Legacy data. Soil field description. Soil laboratory 
analysis}
  
\begin{chapterabstract}{english}{\enkeys}
The Santa Maria dataset comprises soil data from $n = 410$ point soil observations made between \num{2008} and 
\num{2013} in the catchment of the reservoir of the \textit{Departamento Nacional de Obras de 
Saneamento}-\textit{Companhia Riograndense de Saneamento} (DNOS-CORSAN), located in the southern Brazilian 
state of Rio Grande do Sul. These soil data were produced during the development of research projects that 
aimed at producing semi-detailed soil and land use maps, and predicting topsoil carbon stock and vulnerability 
to erosion. All observation locations were selected purposively or by convenience. Several environmental 
features were described at the observation locations, such as land use, geology, soil classification, slope, 
drainage condition, presence of coarse fragments and rock outcrops, soil coverage with vegetation, among 
other peculiarities of each observation location that were not recorded in a systematic way. Soil samples were 
submitted to laboratory analysis to determine the soil organic carbon content, particle size distribution, 
bulk density, and the content of exchangeable bases (calcium, magnesium, potassium, and sodium) and acidity. 
The effective cation exchange capacity was calculated as the sum of exchangeable bases and acidity. The soil 
data is freely available in a repository hosted in GitHub. These include the identification of all observation 
locations, their geographic coordinates, and field and laboratory data. The number of laboratory replicates 
and the sample standard deviation is provided as well.
\end{chapterabstract}

\formatchapter

\section{INTRODUCTION}
\label{sec:chap04-introduction}

This manuscript presents a description of the soil data contained in the \emph{Santa Maria dataset}. The Santa 
Maria dataset comprises soil data from $n = 410$ soil observations made between \num{2004} and \num{2013} in 
the catchment of the reservoir of the \textit{Departamento Nacional de Obras de Saneamento}-\textit{Companhia 
Riograndense de Saneamento} (DNOS-CORSAN), henceforth called \emph{DNOS catchment}, located in the southern 
border of the Plateau of the Paraná Sedimentary Basin, in the city of Santa Maria, state of Rio Grande do Sul, 
Brazil.

The soil observations cover the northern sector of the DNOS catchment -- an area of \SI{\pm2000}{\hectare}, 
which corresponds to \SI{\pm60}{\percent} of the entire catchment. These soil data were produced during the 
development of research projects that aimed at producing semi-detailed soil and land use maps (\scale{25000}) 
\cite{Pedron2005, Miguel2010, SamuelRosaEtAl2011a, MiguelEtAl2012}, and predicting topsoil carbon stock and 
vulnerability to erosion \cite{Samuel-Rosa2009, MouraBueno2012, Miguel2013}.

% Footnote %%%%%
\def\foottropics{\footnote{The reader should be aware that soil science evolved in Brazil following a somewhat 
different pathway than in the countries of the northern hemisphere due to the specific soil features of 
tropical and subtropical regions. Methods have been adapted along the years, possibly leading to nomenclature 
mismatches. The reader is invited to contribute to solve any problems in this document.}}

The description presented in this manuscript includes the procedures for soil sampling and description, as 
well as the analytical methods employed\foottropics{}. Soil data is described using summary plots with 
descriptive statistics. Finally, a description of the structure of the database and of how it can be accessed 
and used is presented. 

\section{FIELD SAMPLING}
\label{sec:chap04-sampling}

The Santa Maria dataset is composed of three subsets which are described in the next three sections. Together, 
these subsets yield a sampling density of about \num{\pm0.18}~observations per hectare, with an average 
separation distance between two neighbouring points of \SI{180}{\metre}, minimum and maximum separation 
distances of \num{18} and \SI{328}{\metre}, \SI{95}{\percent} of neighbouring observations being separated by 
more than \SI{49}{\metre}.

\subsection{Subset I}

The first subset ($n = 340$, \autoref{fig:chap04-subsets-I-III}) was produced between 2008 and 20011 as part 
of projects that aimed at producing semi-detailed soil and land use maps, and predicting topsoil carbon stock 
and vulnerability to erosion \cite{Samuel-Rosa2009, SamuelRosaEtAl2011a, MiguelEtAl2012, Moura-BuenoEtAl2012, 
Samuel-RosaEtAl2013}. The researchers faced several difficulties with a budget cut and shortage of workforce. 
They also had restricted access to several areas due to geographic barriers and prohibition of access by some 
landowners. These difficulties forced the researchers to reduce the originally aimed number of observations 
($n = 500$) during the development of the project.

All observation locations were selected purposively or by convenience. Tacit knowledge 
(\autoref{sec:chap02-discrete} and \autoref{sec:chap07-elicited}) was the main tool to choose the observation 
locations, a process that was carried out in the office using \SI{1}{\metre} spatial resolution Google 
Earth\rr{} imagery of the years of \num{2008} and \num{2009}. The main goal of the researchers was to obtain a 
sample that they understood as being representative of the different landforms, land uses, and soil taxa 
present in the DNOS catchment. They also wanted the observations to be spread throughout the entire DNOS 
catchment.

% Footnote %%%%%
\def\footsupport{\footnote{\emph{Sample support} refers to the shape, size and orientation of sampling units, 
the latter being the smallest single entity that we are able or choose to observe in the universe of interest, 
i.e. the sampling region. A discrete universe such as a forest is defined by the collection of these single 
entities. However, by definition, such single entities have no real, physical existence in continuous 
universes such as the soil -- their \q{existence} require our prior, more or less arbitrary, definition of 
their shape, size and orientation. This definition is usually based on theoretical and practical 
considerations. For example, the sampling unit can be defined as a roughly polygonal block that is large 
enough to encompass the pattern of small-scale local vertical (\SI{\leq2}{\m}) and horizontal 
(1--\SI{10}{\square\metre}) variability of soil properties -- a pedon. Depending on the size of the sampling 
unit relative to the universe of interest, the sample support is referred to as \emph{areal} or \emph{point} 
sample support. A pedon of \SI{10}{\square\metre} area observed in an agricultural field of \SI{1}{\hectare} 
corresponds to the \emph{areal sample support}. However, the same pedon observed in a catchment of 
\SI{200}{\hectare} would correspond to the \emph{point sample support}.}}

At the observation locations, the researchers defined an area of \SI{\pm100}{\metre\squared} within which they 
opened three soil pits up to a depth of \SI{20}{\centi\metre}. Soil samples were collected up to a depth of 
\SI{20}{\centi\metre}, the depth being measured with a ruler. The resulting sampling depth of Subset I varies 
from \num{2} to \SI{20}{\centi\metre}, with an average of \SI{17}{\centi\metre}. This variation of the 
vertical sampling support\footsupport{} was not a problem for the researchers because their goal was to sample 
the \emph{topsoil}. The topsoil was defined as the topmost soil layer, with a thickness equal or inferior to 
\SI{20}{\centi\metre}, being the soil layer most susceptible to degradation induced by poor agricultural 
practices and land use changes.

Soil samples from the three pits opened in each sampling area were used to produce a composite sample which 
was used for laboratory analyses. Subsurface soil features were observed with an auger in each pit, and the 
average (continuous variables) or most common (categorical variables) value recorded. Note that soil sampling 
was done using an areal horizontal support -- an area of \SI{\pm100}{\metre\squared}. However, the shape and 
exact area of the sampling units are unknown, and georeferencing took place at point support.

\begin{figure}[!ht]
\centering
\includegraphics[width=0.90\textwidth]{fig/chap04-subsets-I-III}
\caption[Spatial distribution of \emph{Subset I} and \emph{Subset III}.]{Spatial distribution of the soil 
observations contained in \emph{Subset I} ($n = 340$, black solid circles) and \emph{Subset III} ($n = 
10$, red stars) of the Santa Maria dataset. The drainage network is shown in the background (blue 
dashed line) to give an idea of how the locations of soil observations is related to terrain features.}
\label{fig:chap04-subsets-I-III}
\end{figure}

Georeferencing was done in the field using a Global Navigation Satellite System (GNSS) receiver with a 
horizontal positional error of less than \SI{8}{\metre} positioned approximately at the centre of the
sampling area. Sometimes, the horizontal positional error was larger than \SI{8}{\metre} due the effects
of vegetation, terrain, and satellite configuration. In these cases, observation locations were georeferenced 
in the office using \SI{1}{\metre} spatial resolution Google Earth\rr{} imagery with positional horizontal 
error of \SI{6}{\metre} (\autoref{tab:chap05-google-geo-val}).

Every observation was identified with a number in increasing order, following the order in which the 
observations were made (\num{001}--\num{340}). A total of \num{17}~field campaigns were carried out, yielding 
an observation density of about \num{18}~observations per \si{\kilo\metre\squared} (\autoref{chap:chap07}).

\subsection{Subset II}
\label{sec:chap04-subset-ii}

The second subset ($n = 60$, \autoref{fig:chap04-subset-II}) was produced in the years \num{2012} and 
\num{2013}, and was intended to constitute an independent dataset for validation purposes. Because of the many 
access limitations (geographic barriers and prohibition by landowners) and shortage of workforce, budget, 
infrastructure and time faced in previous field campaigns, researchers chose to employ transect (cluster) 
sampling \cite{MiguelEtAl2012, Moura-BuenoEtAl2012, Samuel-RosaEtAl2013}. They started defining the population 
of transects using their knowledge of the study area, taking into account the factors that they thought 
determined the spatial distribution of soil properties. Each researcher (three) delineated $m = 60$ easily 
accessible, straight transects of \SI{400}{\metre} following the spatial gradients of selected environmental 
features (topography, geology, vegetation, land use, and soils), totalling 180 transects. Accordingly, 
knowledge of existing roads, human settlements, water bodies, and other access limitations was used as well. 
The activity was carried out using \SI{1}{\metre} spatial resolution Google Earth\rr{} imagery of the years of 
\num{2008} and \num{2009}.

\begin{figure}[!ht]
\centering
\includegraphics[width=0.90\textwidth]{fig/chap04-subset-II}
\caption[Twelve transects were selected using simple random sampling to yield $n = 60$ validation 
observations]{Three soil spatial modellers manually drew 180 straight transects (black dotted lines) aligned 
in the direction of maximum expected spatial variation of environmental conditions. They avoided locations 
where it was known that geographic barriers or landowners would impede the access to make soil observations. 
Twelve transects were probabilistically selected using simple random sampling to yield $n = 60$ validation 
observations (red solid circles) separated by equidistant intervals of \SI{100}{\metre}. The drainage network 
is shown in the background (blue dashed line) to give an idea of how the location and direction of transects 
is related to terrain features.}
\label{fig:chap04-subset-II}
\end{figure}

Twelve out of the $m = 180$ transects were randomly selected using as many iterations as necessary until 
there were no intersecting transects, and there was at least one transect in each of the three major 
morphostructural units of the DNOS catchment (\textit{Planalto}, \textit{Rebordo do Planalto}, and 
\textit{Depressão Periférica}) (\autoref{chap:chap03}). Finally, $n = 5$ observation locations, separated by 
equidistant intervals of \SI{100}{\metre}, were selected in each transect. Observation locations were named 
with a number in increasing order, following the order in which the observations were made, starting from 
\num{341} (\num{341}--\num{400}).

The location of the observations was identified in the field using a GNSS receiver with a horizontal 
positional error of less than \SI{8}{\metre}. Soil sampling and description was carried out using the same 
procedure used with \emph{Subset I}, except for the fact that a single soil pit was opened within a radius of 
\SI{2}{\m} from the predefined observation location. More accurate geographic coordinates were collected in 
the field using a Differential Global Positioning System (DGPS) with a horizontal positional error of less 
than \SI{1}{\centi\metre}.

\subsection{Subset III}

The third subset ($n = 10$) contains data compiled from the studies of \citet{Pedron2005} and 
\citet{Miguel2010}, specifically from the uppermost A horizon of modal soil profiles (point support) whose 
locations were purposively selected using tacit knowledge after a preliminary area-class soil map had been 
produced and/or the observations included in \emph{Subset I} had been made.

\citet{Pedron2005} and \citet{Miguel2010} aimed at observation locations that they understood as being most 
representative of the soil mapping units depicted in their respective area-class soil maps. A single soil 
sample was taken from each of the described soil horizons and used for laboratory analysis. The resulting 
thickness of the uppermost A horizons varies from \num{12} to \SI{30}{\centi\metre}, with a mean of 
\SI{22.6}{\centi\metre}. Georeferencing was carried using a GNSS receiver with a horizontal positional error 
of less than \SI{8}{\metre} positioned at the observation location. Data are identified in the Santa Maria 
dataset using the same identification that was used in the studies from which they were compiled.

\section{FIELD DESCRIPTION}

Several environmental features were described at the observation locations. This sections present a summary
description of how this description was done, specially for subsets I and II, which have not been documented 
before. For subset III, a thorough explanation of how field description was done is given by 
\citet{Pedron2005} and \citet{Miguel2010}.

Despite the different origins of the datasets, soil sampling and description guidelines are very similar. As 
such, merging field descriptions from subset III with those of subset I and II was easy, rarely requiring 
conceptual translations and adaptations -- this practice is reported when used.

Finally, the code used in the database to identify each of the variables described in the field is presented 
between parenthesis using fixed-width or monospace font.

\subsection{Land Use and Vegetation}

Land use (\texttt{LAND}) was assessed at the time of sampling using data collected in the field. Five land 
uses were identified using nomenclature of \citet{FAO2006} (\autoref{fig:chap04-land}):

\begin{description}
\item[\texttt{animal husbandry}] Native grasslands used for animal husbandry.
\item[\texttt{crop agriculture}] Annual and biannual crop agriculture.
\item[\texttt{forestry}] Plantations of \textit{Eucalyptus spp.} and \textit{Pinus spp.}.
\item[\texttt{native forest}] Primary or secondary native forests.
\item[\texttt{shrubland}] Abandoned areas with predominance of shrub-sized vegetation, known in Brazil as 
\emph{capoeira}.
\end{description}

\begin{figure}[!ht]
\centering
\includegraphics[width=0.60\textwidth]{fig/chap04-land}
\caption[Distribution of land use types in the Santa Maria dataset.]{Distribution of land use types in the 
Santa Maria dataset. Most soil observations were made in areas used for animal husbandry although more that 
half of the area is occupied with native forest (\autoref{sec:chap05-land-use}).}
\label{fig:chap04-land}
\end{figure}

Other land uses are observed in the study area such as human settlements and water bodies 
(\autoref{sec:chap05-land-use}). However, due to access constraints, soil observations were not made in areas 
under these land uses.

\subsection{Geology}

Soil parent material (\texttt{PARENT}) was inferred in the field from direct observation of soil properties 
and local environmental features. Two classes were identified using nomenclature of \cite{FAO2006}:

\begin{description}
\item \texttt{igneous} Soil derived from the \textit{in sutu} weathering of igneous rocks.
\item \texttt{sedimentary} Soil derived from the \textit{in sutu} weathering of sedimentary rocks, or from 
sediments of igneous and/or sedimentary rocks.
\end{description}

Underlying geologic formation (\texttt{GEO}) and lithology (\texttt{LITHO}) were inferred based on soil 
properties and environmental features observed in the field, and on existing area-class soil maps 
(\autoref{sec:chap05-soil-maps}) geologic maps (\autoref{sec:chap05-geo-maps}).

\subsection{Soil Classification}

The most likely taxon (\texttt{TAXON}) of the Brazilian System of Soil Classification (SiBCS) 
\cite{SantosEtAl2013a} was inferred in the field using data obtained from direct observation of soil 
properties (\SI{20}{\centi\metre}-deep soil pits and auger holes down to the diagnostic subsurface horizon or 
bedrock) and local environmental features. These data were then interpreted using the bases and concepts of 
the SiBCS to identify the most likely taxon up to the second taxonomic level of the SiBCS. Further levels of 
the SiBCS were not considered because making any sort of inference would require data that were not observable 
in the field. Eleven taxa were identified (\autoref{fig:chap04-taxon}):

\begin{description}
\item[\texttt{CX}] Cambissolo Háplico. Moderately developed soil.
\item[\texttt{GX}] Gleissolo Háplico. Poorly drained greyish soil with a somewhat constant clay content 
throughout the profile.
\item[\texttt{PA}] Argissolo Amarelo. Soil with significant increase of the clay content with depth, and with 
a yellowish B horizon.
\item[\texttt{PBAC}] Argissolo Bruno-Acinzentado. Soil with significant increase of the clay content with 
depth, and with an upper B horizon slightly darker than the lower B horizons.
\item[\texttt{PV}] Argissolo Vermelho. Soil with significant increase of the clay content with depth, and with 
a reddish B horizon.
\item[\texttt{PVA}] Argissolo Vermelho-Amarelo. Soil with significant increase of the clay content with depth, 
and with a reddish-yellowish B horizon.
\item[\texttt{RL}] Neossolo Litólico. Poorly developed soil.
\item[\texttt{RQ}] Neossolo Quartzarênico. Deep sandy soil derived from sediments.
\item[\texttt{RR}] Neossolo Regolítico. Poorly to moderately developed soil.
\item[\texttt{RY}] Neossolo Flúvico. Poorly developed soil derived from alluvial sediments.
\item[\texttt{SX}] Planossolo Háplico. Poorly drained greyish soil with significant increase of the clay 
content with depth.
\end{description}

\begin{figure}[!ht]
\centering
\includegraphics[width=0.60\textwidth]{fig/chap04-taxon}
\caption[Distribution of soil taxa in the Santa Maria dataset.]{Distribution of soil taxa in the Santa Maria 
dataset. Most soil observations were classified as Neossolo Litólico and Argissolo Bruno-Acinzentado. The 
proportions approximately agree with the information conveyed by existing area-class soil maps 
(\autoref{fig:chap05-soil-maps}).}
\label{fig:chap04-taxon}
\end{figure}

\subsection{Slope}

The slope gradient (\texttt{SLOPE}, \si{\percent}) was measured using a clinometer, the observer and target 
being at a constant height above the ground (\autoref{fig:chap04-slope}). The distance between observer and 
target was between \SI{30}{\metre} (dense forests) and \SI{50}{\metre} (open fields).

\begin{figure}[!ht]
\centering
\includegraphics[width=0.60\textwidth]{fig/chap04-slope}
\caption[Distribution of slope gradient in the Santa Maria dataset.]{Distribution of slope gradient in the 
Santa Maria dataset. Most soil observations were made in areas with a slope gradient $\SI{<25}{\percent}$.}
\label{fig:chap04-slope}
\end{figure}

\subsection{Drainage}

Soil drainage status (\texttt{DRAIN}) was inferred visually from soil features observed with an auger using 
the classification scheme proposed by \citet{SantosEtAl2013}. Four drainage classes were identified 
(\autoref{fig:chap04-drain}):

\begin{description}
\item[\texttt{poorly}] Water is removed from the soil so slowly that the profile remains wet for much of the 
time.

\item[\texttt{somewhat poorly}] Water is removed slowly from the soil, so that it remains wet for a 
significant period, but not during most of the year.

\item[\texttt{moderately well}] Water is removed from the soil somewhat slowly, so that the profile remains 
wet for small but significant period of time.

\item[\texttt{well}] Water is removed from the soil with ease but not rapidly.
\end{description}

\begin{figure}[!ht]
\centering
\includegraphics[width=0.60\textwidth]{fig/chap04-drain}
\caption[Distribution of drainage classes in the Santa Maria dataset.]{Distribution of drainage classes in the 
Santa Maria dataset. Most soil observations were made in areas with well drained soil.}
\label{fig:chap04-drain}
\end{figure}

\subsection{Coarse Fragments and Rock Outcrops}

Presence of coarse fragments (\texttt{FRAG}) -- soil material of diameter \SI{>2}{\milli\metre} -- was 
described as a binary variable, that is, a value of \num{1} (one) was annotated when coarse fragments were 
present, and \num{0} (zero) otherwise. The same approach was adopted to describe the presence of rock outcrops 
(\texttt{ROCK}). The quantity of coarse fragments (\texttt{GRAVEL}, \si{\percent}) was estimated visually in 
some observation points.

It is worth noting that the approach employed to describe the presence of coarse fragments and rock outcrops 
is not in line with the standard soil description guidelines currently used in Brazil. The reason for 
recording only their presence/absence is that the actual content was not of primary interest at the time of 
sampling.

\subsection{Canopy}

% TODO: add three figures as examples of each class.
Soil coverage with vegetation, an idea of the density of stand or plant cover, (\texttt{CANOPY}) was inferred 
visually in the field using three classes:

\begin{description}
\item[low] \SI{<25}{\percent}
\item[medium] 25--\SI{75}{\percent}
\item[high] \SI{>75}{\percent}
\end{description}

\subsection{Additional Information}

Additional information was recorded at each observation location during the field campaigns. They refer to 
peculiarities of each observation location and were not recorded in a systematic way.

\section{LABORATORY ANALYSIS}
\label{sec:chap04-laboratory}

Several laboratory analysis were performed with the soil samples collected in the DNOS catchment. This 
sections present a summary description of how this analyses were done, specially for subsets I and II, which 
have not been documented before. For subset III, a thorough explanation of how laboratory analyses were done 
is given by \citet{Pedron2005} and \citet{Miguel2010}.

Despite the different origins of the datasets, laboratory analyses protocols are very similar. As such, 
merging the results of laboratory analyses from subset III with those of subset I and II was easy, rarely 
requiring the use of conversion factors -- this practice is reported when used. In all three datasets, soil 
samples were air dried, crushed and passed through a \SI{2}{\milli\metre}-sieve prior to laboratory analyses. 
For datasets I and II, one or more laboratory replicates were used to enable calculating analytical errors.

The same coding standard used with field description variables is used here, i.e. the code used in the 
database is presented between parenthesis using fixed-width or monospace font.

\subsection{Soil Organic Fraction}
\label{sec:chap04-organic}

The soil organic carbon content (\texttt{ORCA}, \si{\gram\per\kilo\gram}) was determined using wet combustion 
\cite{YeomansEtAl1988, Mebius1960, TedescoEtAl1995, ClaessenEtAl1997}.

% Footnote %%%%%
\def\footsulfochromic{\footnote{See a detailed description of the sulfochromic solution, or chromic acid, at 
\href{http://en.wikipedia.org/wiki/Chromic_acid}{Wikipedia}.}}

Sample aliquots of \num{0.050} to \SI{0.500}{\g} were placed in glass digestion tubes (\SI{80}{\ml}). The 
amount of sample used varied according to the \texttt{ORCA} estimated by visual interpretation of soil colour. 
Every digestion tube received an aliquot of \SI{10}{\ml} of sulfochromic solution\footsulfochromic{} 
[\SI{0.067}{\mole\per\liter} potassium bichromate solution (\ce{K2Cr2O7}) in the presence of concentrated 
sulphuric acid (\ce{H2SO4})] and a small reflux funnel to avoid loss of reagent during digestion. A digestion 
block with capacity for \num{40}~samples was used: \num{36}~tubes with soil sample plus \num{3}~tubes with 
blank plus \num{1}~tube with \ce{H2SO4} and a thermometer for temperature check. Digestion at 
\SI{150}{\celsius} last \SI{30}{\minute}. Three blanks were prepared and set aside at room temperature to 
estimate the loss of reagent due to heat in the digestion block.

After digestion the tubes were set aside at room temperature to cool down. Next, the solution was transferred 
to Erlenmeyer flasks (\SI{250}{\ml}) with \SI{60}{\ml} of distilled water and \SI{2}{\ml} of concentrated 
orthophosphoric acid [\ce{H3PO4}] and \num{3}~drops of \SI{1}{\percent}~diphenylamine. The solution was 
titrated using \SI{0.1}{\mole\per\liter} ammonium ferrous sulphate (\ce{FeSO4(NH4)2 * 6H2O}) until persistent 
green colour. The results were multiplied by \num{1.11} to correct the estimated soil organic carbon content 
to the standard analytical method (dry combustion).

\begin{figure}[!ht]
\centering
\includegraphics[width=0.60\textwidth]{fig/chap04-orca}
\caption[Distribution of organic carbon content in the Santa Maria dataset.]{Distribution of organic carbon 
content in the Santa Maria dataset. The distribution is severely skewed.}
\label{fig:chap04-orca}
\end{figure}

Observations compiled from \citet{Pedron2005} had their soil organic matter content determined instead of the 
organic carbon content. Sample aliquots of \SI{2.5}{\ml} were placed in Erlenmeyer flasks (\SI{50}{\ml}). 
Every Erlenmeyer flask received an aliquot of \SI{15}{\ml} of \SI{0.067}{\mole\per\liter} sulfochromic 
solution (\ce{Na2Cr2O7 + H2SO4}). The flasks were heated in a water bath at 75--\SI{80}{\celsius} during 
\SI{30}{\minute} and shaken for \SI{5}{\minute}. A water aliquot of \SI{15}{\ml} was added to the flask and 
let overnight (15--\SI{18}{\hour}).

In the next day, an aliquot of \SI{3.0}{\ml} was sampled to a small plastic cup with \SI{3.0}{\ml} of 
distilled water. The absorbency of the supernatant was measured at \SI{645}{\nano\metre}. The results were 
transformed to organic carbon content assuming that \SI{58}{\percent} of the organic matter is composed of 
organic carbon, the result assumed to be equivalent to soil organic carbon content measured using the standard 
analytical method. The results are expressed using a volume-basis and were converted to a mass-basis using a
1:1 relation because the mass of the sample aliquot used in the analyses is unknown.

\subsection{Particle Size Analysis}
\label{chap:chap04-granulometry}

\def\footsuzuki{\footnote{As far as I know, a comprehensive description of this method has not been 
published so far, neither in Portuguese nor in English. You can visit the homepage of the Soil Physics 
Laboratory of the Universidate Federal de Santa Maria at \url{https://coral.ufsm.br/fisicadosolo/} to get more 
information about the method or contact their developers.}}

Particle size analysis was performed using the pipette method, with the sand fraction (\texttt{SAND}, 
\SIrange{0.053}{2}{\milli\metre}, \si{\gram\per\kilo\gram}) determined by wet sieving, and the silt fraction 
(\texttt{SILT}, \SIrange{0.002}{0.053}{\milli\metre}, \si{\gram\per\kilo\gram}) calculated by difference. 
The analytical procedure is an adaptation\footsuzuki{} of the method of the Soil Conservation Service of 
the United States Department of Agriculture \cite{UnitedStates1972} made by the Soil Physics Laboratory of the 
\textit{Universidade Federal de Santa Maria} \cite{SuzukiEtAl2004, SuzukiEtAl2004a}.

First, a sample aliquot of \SI{20}{\gram} was placed in a \SI{100}{\milli\liter} glass container (height: 
\SI{10.5}{\centi\metre}; diameter: \SI{2.75}{\centi\metre}; weight: \SI{85}{\gram}). Two nylon spheres with a 
diameter of \SI{1.71}{\centi\metre} and weighting \SI{3.04}{\gram} (density: \SI{1.11}{\g\per\cm\cubic}) were 
added to act as physical disaggregating agents. Then, an aliquot of \SI{10}{\milli\liter} of 
\SI{1}{\mole\per\liter} sodium hydroxide (\ce{NaOH}) solution was added to act as chemical dispersing agent 
along with \SI{40}{\milli\liter} of distilled water. The glass container was closed with a plastic cap, 
manually shaken for \SI{10}{\second}, and placed in a horizontal mechanical shaker with capacity for 
\num{85}~samples. The suspension was left to stand overnight (\SI{10}{\hour}). In the next day the suspension 
was submitted to horizontal mechanical agitation during \SI{4}{\hour} at \si{120} cycles per minute 
\cite{SuzukiEtAl2004, SuzukiEtAl2004a}.

After horizontal agitation, the suspension was poured in a plastic graduated cylinder with capacity for 
\SI{1000}{\milli\liter} using a glass funnel and a metal sieve to hold the two nylon spheres. The suspension 
in 
the graduated cylinder was completed to \SI{1000}{\milli\liter} and homogenized using a hand stirrer 
(\SI{30}{\second}). The suspension was allowed to stand until sedimentation was complete. The time needed was 
calculated using the Stokes’ law with the temperature measured in a graduated cylinder filled with distilled 
water.

%TODO provide a more detailed description of how CLAY was determined as well as of the oxidative
%treatment with H2O2.

The clay fraction (\texttt{CLAY}, \SI{<0.002}{\milli\metre}, \si{\gram\per\kilo\gram}) was determined by the 
pipette method. Soil samples with organic matter content \SI{>5}{\percent} were submitted to oxidative 
treatment with hydrogen peroxide (\ce{H2O2}) prior to the analysis following the recommendations of
\citeonline{ClaessenEtAl1997}.

%The sand fraction was separated into five size classes:
%
%\begin{itemize}
%\item \SIrange{1.00}{2.00}{\milli\metre}: very coarse sand;
%\item \SIrange{0.50}{1.00}{\milli\metre}: coarse sand;
%\item \SIrange{0.25}{0.50}{\milli\metre}: median sand;
%\item \SIrange{0.106}{0.25}{\milli\metre}:fine sand;
%\item \SIrange{0.053}{0.106}{\milli\metre}: very fine sand.
%\end{itemize}

% The clay fraction (\textless0.002~mm) was initially determined by the pipette method without any 
% pretreatment. A 1~mol~L$^{-1}$ NaOH solution was used as the dispersing agent, with the addition of two 
% nylon spheres as disaggregating agent plus horizontal mechanical agitation during 4~hours 
% \cite{SuzukiEtAl2004}.

% A propor{\c{c}}{\~{a}}o da fra{\c{c}}{\~{a}}o argila dispersa em {\'{a}}gua foi determinada conforme 
% descrito acima para a fra{\c{c}}{\~{a}}o argila total. A diferen{\c{c}}a {\'{e}} que n{\~{a}}o foi usado o 
% agente dispersante (NaOH) e o agente desagregante (esferas de nylon) \cite{ClaessenEtAl1997}.

\subsection{Soil Density}
\label{chap:chap04-bude}

% TODO: Provide a more detailed description of how this is done.
The bulk soil density (\texttt{BUDE}, \si{\mega\gram\per\cubic\metre}) was determined using the core method 
with a metallic ring (height: \SI{3}{\centi\metre}; diameter: \SI{5}{\centi\metre}) as described by 
\citeonline{ClaessenEtAl1997}. The bulk soil density was not determined in the locations where the soil was 
very shallow or stony.

\subsection{Exchangeable Bases and Acidity}
\label{chap:chap04-ecec}

The exchangeable calcium (\texttt{CALC}, \si{\milli\mole\per\kilo\gram}) and magnesium (\texttt{MAGN}, 
\si{\milli\mole\per\kilo\gram}) were determined by atomic absorption spectroscopy after extraction with 
\SI{1.0}{\mole\per\liter} \ce{KCl} solution using the method described by \citeonline{ClaessenEtAl1997}. 
The exchangeable sodium (\texttt{SODI}, \si{\milli\mole\per\kilo\gram}) and potassium (\texttt{POTA}, 
\si{\milli\mole\per\kilo\gram}) were extracted with a \SI{0.05}{\mole\per\liter} \ce{HCl} solution plus 
\SI{0.025}{\mole\per\liter} \ce{H2SO} (Mehlich-\num{1} solution). Both were quantified by means of flame 
atomic emission spectrometry using the method described by \citeonline{TedescoEtAl1995}.

The exchangeable acidity (\texttt{EXAC}, \si{\milli\mole\per\kilo\gram}) was extracted using the same 
\SI{1.0}{\mole\per\liter} \ce{KCl} solution used to extract the exchangeable calcium and magnesium. It was 
determined by titrimetry with \SI{0.025}{\mole\per\liter} \ce{NaOH} solution as described by 
\citeonline[p.~103]{ClaessenEtAl1997}.

% TODO: Include POAC in the database and improve the description of how it was determined.
% The potential acidity (POAC, \si{\milli\mole\per\kilo\gram}) was determined with \SI{1.0}{\mole\per\liter} 
% calcium acetate solution at pH~\num{7.0} and titrated with \SI{0.0606}{\mole\per\liter} \ce{NaOH} solution 
% as described by \citeonline{ClaessenEtAl1997}.

The effective cation exchange capacity (ECEC, \si{\milli\mole\per\kilo\gram}) was defined as the sum of 
exchangeable bases and exchangeable acidity, i.e. 

\begin{equation*}
 \texttt{ECEC} = \texttt{CALC} + \texttt{MAGN} + \texttt{POTA} + \texttt{SODI} + \texttt{EXAC}.
\end{equation*}


% TODO: Provide a more detailed description of how these are calculated and include the data in the database.
% The sum of exchangeable bases (BASES) is given by the sum of the exchangeable calcium, magnesium, sodium and 
% potassium. The effective cation exchange capacity (ECEC) is given by the exchangeable acidity plus the 
% sum of exchangeable bases. The potential cation exchange capacity (CEC) is given by the potential acidity 
% plus the sum of exchangeable bases. Note that the standard method for determining exchangeable bases relies 
% on the use of barium chloride [BaCl$_2$]. The base saturation (BASA) is given by the sum of exchangeable 
% bases divided by the potential cation exchange capacity. The saturation of the ECEC with exchangeable 
% acidity, or the aluminum saturation (ALSA), is given by the sum of exchangeable bases divided by the 
% effective cation exchange capacity. The results are multiplied by 100. 

% \begin{figure}[!ht]
% \centering
% <<echo = FALSE>>=
% options(useFancyQuotes = FALSE)
% tmp <- read.table(
%  '~/projects/dnos-sm-rs/dnos-sm-rs-general/data/labData.csv', sep = ";",
%  header = TRUE, na.strings = 'na')
% lattice::trellis.par.set(
%  fontsize = list(text = 16, points = 15), axis.line = list(lwd = 0.01),
%  layout.widths = list(left.padding = 0, right.padding = 0),
%  layout.heights = list(top.padding = 0, bottom.padding = 0))
% aa <- pedometrics::plotHD(tmp$CLAY, xlab = 'CLAY')
% bb <- pedometrics::plotHD(tmp$ORCA, xlab = 'ORCA')
% cc <- pedometrics::plotHD(tmp$ECEC, xlab = 'ECEC')
% dd <- pedometrics::plotHD(na.exclude(tmp$BUDE), xlab = "BUDE")
% @
% \begin{minipage}[b]{63mm}
% \subcaption{}
% \centering
% <<intro-clay, fig = TRUE, echo = FALSE>>=
% print(aa)
% @
% \end{minipage}
% \begin{minipage}[b]{63mm}
% \subcaption{}
% \centering
% <<intro-orca, fig = TRUE, echo = FALSE>>=
% print(bb)
% @
% \end{minipage}
% \begin{minipage}[b]{63mm}
% \subcaption{}
% \centering
% <<intro-ecec, fig = TRUE, echo = FALSE>>=
% print(cc)
% @
% \end{minipage}
% \begin{minipage}[b]{63mm}
% \subcaption{}
% \centering
% <<intro-bude, fig = TRUE, echo = FALSE>>=
% print(dd)
% @
% \end{minipage}
% \caption{The four soil properties explored in this thesis: (a) clay content, (b) organic carbon
% content, (c) effective cation exchange capacity, and (d) bulk density. Each panel shows the sample
% histogram and summary statistics of the soil properties in their original scale ($\lambda = 1$), as
% well as the theoretical probability density function so that we can assess how good is the fit of
% the normal distribution to the data -- a product of the \Rpackage{pedometrics}.}
% \label{fig:intro-soil-properties}
% \end{figure}

\section{DATASET STRUCTURE}

The soil data is 
freely available as comma-separated values (CSV) files in a repository hosted in \dnosgeneral{}. Files 
\texttt{fieldData.csv} and \texttt{labData.csv} contain the identification of all observation locations, their 
geographic coordinates (latlong, WGS1984), and field and laboratory data, respectively. Files 
\texttt{fieldMetadata.csv} and \texttt{labMetadata.csv} contain the metadata. Every soil property is 
identified with a code composed of three or four capital letters. For example, soil organic carbon is 
identified with \texttt{ORCA}. A column containing the number of laboratory replicates is identified with the 
code of the soil property followed by the letter \q{N}. The column containing the sample standard deviation is 
identified in the same manner, but using \q{SD}. For example, \texttt{ORCA\_N} and \texttt{ORCA\_SD}.


\section{CONCLUSIONS}

The main goal of documenting the soil data contained in the Santa Maria dataset was to provide the reader the 
basis to understand the soil data used in the thesis, and also to support future soil spatial modelling 
exercises in 
the catchment of the DNOS reservoir.

As a result of an ongoing collaborative effort, this documentation will be improved in the near future as new 
studies are developed. We plan to include new figures to exemplify how field soil sampling was carried out. 
Details of non-standard soil description and analysis methods will likely be extended. This includes the 
oxidative 
treatment with \ce{H2O2} to which soil samples were submitted prior to particle size distribution analysis. 
For 
cases such as the ECEC, determined using a non-standard method, we plan to develop a study to calibrate a 
model 
to convert our results to the standard method for determining exchangeable bases, which uses barium chloride 
(\ce{BaCl2}) for saturation.

Other already existing soil data will also be included in the Santa Maria dataset and documented as well. 
These 
data have not been used in any study so far, including the potential acidity, sum of exchangeable bases, 
potential 
cation exchange capacity, base saturation, aluminium saturation, and the five size classes of the sand 
fraction.

Once a comprehensive documentation of the existing soil data has been constructed, we will prepare a basic 
spatial exploratory soil data analysis. We hope that our effort to properly document the soil data that we 
produced,
and make it freely available for use, will serve as an example for future soil spatial modelling studies 
developed 
elsewhere.

 % 4. The Santa Maria dataset -- soil data
\artigotrue
\chapter{On the Uncertainty of Digital Soil Mapping - Model Structure}
\label{chap:chapter05}

\begin{chapterabstractPOR}{Pedometria, Incerteza, Estrutura do modelo}
Este capítulo abordará a identificação de cenários de bancos de dados relativos ao número de observações de calibração disponíveis nos quais modelos não-lineares apresentam desempenho melhor do que modelos lineares.
\end{chapterabstractPOR}

\begin{chapterabstractENG}{Pedometrics, Uncertainty, Model structure}
This chapter will deal with identifying database scenarios regarding the number of calibration observations available in which non-linear models present better performance than linear models.
\end{chapterabstractENG}

\section{INTRODUCTION}

This chapter will deal with identifying database scenarios regarding the number of calibration observations available in which non-linear models present better performance than linear models.

\section{MATERIAL AND METHODS}

\subsection{Database}

Seven subsets of calibration observations will be used to simulate database scenarios regarding the number of calibration observations available to build DSM models. These subsets will contain $n=$50, 100, 150, 200, 250, 300 and 350 calibration observations and will be constructed using the criteria described in item Chapter 2. A suite of $n=64$ environmental co-variates will be derived from seventeen data layers to fit trend models. Orthogonalization of predictor variables will not be considered.

\subsection{Model Structure}

Two types of trend model structures will be evaluated: linear and non-linear. The linear structure will be represented by multivariate linear regression model with estimation of parameters by ordinary least squares (OLS) (package \texttt{stats}), while the non-linear structure will be represented by three models. These are:

\begin{itemize}
\item An artificial neural network with multilayer perceptron (MLP) architecture and training by error backpropagation (package \texttt{RSNNS});
\item A regression tree using a one-step lookahead construction method with splits aiming at the reduction of the residual sum of squares (package \texttt{rpart});
\item A random forest implementing Breiman's algorithm (package \texttt{randomForest}).
\end{itemize}

\subsection{Model Building}

Four trend models will be fitted for each soil properties (particle-size distribution, organic carbon content and cation exchange capacity). Each of these trend models will be fitted using one of the four model structures described above. The residuals will be used to fit a variogram model and interpolated using simple kriging (package \texttt{gstat}). Final prediction map will be obtained by adding kriged predictions to the predictions made by the trend model. Model assessment will involve analyzing summary statistics of each method.

\subsection{Assessment of Competing Models}

Four competing models will be build for every soil property. Their analysis will include evaluating the differences among the sets of environmental co-variates included in the trend model under the light of the conceptual model of pedogenesis. Differences in variogram model parameters will also be searched. Coupled with the analysis of the spatial pattern of predicted values and prediction error variance maps, these analysis will help defining a degree of uncertainty about model specification due to trend model structure. Prediction accuracy  will be evaluated for all models using independent field data obtained through probabilistic sampling ($n=60$). Error statistics (mean error, mean squared error, and mean squared deviation ratio) of pairs of competing models will be compared under the null hypothesis that the expected value of the estimated mean difference is zero. The pedological information content of trend models will be evaluated eliciting the opinion of five experts. % 5. The Santa Maria dataset -- covariate data
\selectlanguage{brazilian}
\artigotrue
\chapter{MODELO CONCEITUAL DE PEDOGÊNESE}
\label{chap:chap03}
%\SweaveUTF8


\def\ptkeys{Província Geológica do Paraná. Bacia do DNOS. Rebordo do Planalto. Fatores de formação do solo. 
Pedogênese}

\begin{chapterabstract}{brazilian}{\ptkeys}
O presente documento apresenta o modelo conceitual de pedogênese -- descrição explícita dos fatores e 
processos de formação do solo que determinam as características do solo e o seu padrão de distribuição 
espaço-temporal -- da bacia de captação do reservatório do DNOS/CORSAN (Departamento Nacional de Obras de 
Saneamento/Companhia Riograndense de Saneamento), localizada no sul do Brasil. O clima é 
subtropical úmido sem estação seca definida. O relevo é plano a montanhoso (variação de altitude entre 139 e 
\SI{475}{\m}), com vales encaixados que influenciam a precipitação e o fluxo radiativo nas diferentes 
superfícies geomórficas. A geologia é constituída pela sequência de três formações geológicas: rochas 
sedimentares 
(arenito fluvial), seguidas de rochas ígneas (basaltos-andesitos toleíticos e vitrófilos, riólitos-riodacitos 
granofíricos) intercaladas por rochas sedimentares (arenito eólico). Depósitos do Quaternário aparecem nas 
partes mais baixas. A geomorfologia atual é resultado dos processos erosivos do Terciário e Quaternário. A 
dissecação atual é fraca devido ao clima que favorece a instalação e permanência de vegetação exuberante. Três 
unidades morfoestruturais são identificadas: no topo, o Planalto, com relevo suave-ondulado a ondulado, 
seguido pelo Rebordo do Planalto, com ampla variação altimétrica, declividade acentuada e escarpas abruptas; na 
base, a Depressão Periférica, com formas agradacionais de planície fluvial. Nas partes altas, a rede de 
drenagem apresenta padrão bem definido, geralmente retangular, determinada pelas falhas e/ou fraturas. Já nas 
áreas mais baixas, devido aos processos de deposição sedimentar e erosão fluvial, sua configuração é sinuosa. 
Ali 
encontram-se um lençol freático próximo da superfície do solo e cursos de água perenes. O uso da terra para 
produção agrossilvopastoril foi intenso em tempos pretéritos e resultou em forte degradação do solo. O 
abandono 
de muitas áreas degradadas permitiu a regeneração da vegetação natural, resultando na atual ocupação com 
florestas e vegetação secundária de \SI{\pm60}{\percent}. Em geral, o solo é pouco profundo devido ao 
predomínio de condições de forte declividade. É comum encontrar solo raso mesmo em áreas de maior estabilidade 
como fruto da degradação pelo uso agrícola. O solo é mais profundo no Planalto, nos terraços do Rebordo, nas 
coxilhas (colinas) de relevo suave-ondulado a ondulado, e nas planícies aluviais. A textura é mais fina e 
homogênea ao 
longo do perfil quando desenvolvido a partir de rochas vulcânicas. As características do solo nas planícies 
aluviais são determinadas pela presença constante de lençol freático próximo da superfície.
\end{chapterabstract}

% \def\enkeys{Paraná Geological Province, DNOS Catchment, Plateau Border, Soil formation factors, Pedogenesis}
%   
% \begin{chapterabstract}{english}{\enkeys}
% This document presents the conceptual model of pedogenesis -- an explicit description of soil-forming 
% factors and processes that determine the spatio-temporal distribution of soil properties  -- of the 
% catchment of the DNOS/CORSAN reservoir, located in southern Brazil. Climate is subtropical humid without a 
% dry season. Relief varies between plain and mountainous, with enclosed valleys (elevation ranging between 
% \num{139} and \SI{475}{\metre} above sea level) that determine rainfall volume and radiative flux on 
% different surfaces. The geology is composed of a sequence of three geological formations: consolidated 
% sedimentary rocks (fluvial sandstone), followed by basic and acid igneous rocks (andesite-basalt and 
% rhyolite-rhyodacite), interlayered with consolidated sedimentary rocks (aeolian sandstone). Unconsolidated 
% colluvial deposits of the Quaternary period occur in the lower portions of the landscape. Current 
% geomorphology is a result of erosive processes of the Tertiary and Quaternary. Landscape dissection is weak 
% due to the current climate that favours the installation and maintenance of an exuberant vegetation. There 
% are three morphostructural units: at the top, the \textit{Planalto} (Plateau), with gently-rolling to 
% sloping relief, followed by the \textit{Rebordo do Planalto} (Plateau Border), with wide altimetric 
% variation, steep slopes and abrupt cliffs; at the bottom, the \textit{Depressão Periférica} (Peripheral 
% Depression), composed of aggradational fluvial plains. In higher altitudes, the drainage network has a well 
% defined pattern, generally rectangular, determined by the faults and/or fractures. In the lower areas, its 
% configuration is sinuous due to sediment deposition and fluvial erosion, with the presence of water table 
% close to the surface and perennial water streams. Land use for agrosilvopastoral production was intense in 
% past times, resulting in severe soil degradation. Recent abandonment of many degraded areas allowed the 
% regeneration of natural vegetation, resulting in \SI{\pm60}{\percent} of the area being now occupied with 
% forest and secondary vegetation. The soil is predominantly shallow due to the dominance of steep slopes. 
% Even in gently-sloping terrain it is common to find shallow soils as a result of soil degradation, Deeper 
% soil can be found in the Planalto, in the terraces of the Rebordo, and in the small hills with 
% gently-rolling slopes and alluvial plains. Soil texture is finer and more homogeneous throughout the soil 
% profile in soil developed from igneous rocks. Soil features in the alluvial plains are determined by the 
% constant presence of the water table close to the surface.
% \end{chapterabstract}

\formatchapter

\section{APRESENTAÇÃO}
\label{sec:chap03-apresentacao}

\titlenote{Colaboraram na preparação deste documento: Pablo Miguel (UFPel), Jean Michel Moura Bueno (UFSM), 
Ricardo Simão Diniz Dalmolin (UFSM), Andrisa Balbinot (UFSM), Lúcia Helena Cunha dos Anjos (UFRRJ), Gustavo de 
Mattos Vasques (Embrapa Soils), e Gerard B. M. Heuvelink (ISRIC -- World Soil Information).}

A modelagem espacial do solo inicia com a definição de um \emph{modelo conceitual de pedogênese}. Um modelo 
conceitual de pedogênese constitui uma representação verbal da realidade sob estudo que inclui a descrição 
explícita dos fatores e processos de formação do solo que determinam as características do solo e o seu padrão 
de distribuição espaço-temporal. Isso requer a reunião de toda a informação ambiental disponível e aplicação 
dos conceitos de relação solo-paisagem, desenvolvimento do solo em catenas, ou outro modelo teórico de 
explicação da variação espacial do solo.

O presente documento apresenta o modelo conceitual de pedogênese da bacia de captação do reservatório do 
DNOS/CORSAN (Departamento Nacional de Obras de Saneamento/Companhia Riograndense de Saneamento), localizada na 
divisa entre os municípios de \itaara{} (ao norte) e \santamaria{} (ao sul), na porção sul da \baciaparana{},
estado do Rio Grande do Sul, Brasil (\autoref{fig:chap03-location}). A bacia de captação do reservatório do 
DNOS/CORSAN corresponde à cabeceira da bacia hidrográfica do \riovacacaimirim{}, tributário do \riojacui{} e, 
consequentemente, do \rioguaiba{} e da \lagoadospatos{}. A bacia de captação do reservatório do DNOS/CORSAN
cobre uma área de \SI{\pm29}{\square\kilo\metre} e alimenta um reservatório com volume máximo 
de \SI{\pm3800000}{\cubic\metre} em uma área inundada de \SI{0,74}{\square\kilo\metre}. Este reservatório 
contribui com até \SI{30}{\percent} do abastecimento de água da cidade de Santa Maria \cite{Dias2003, 
DillEtAl2004, Miguel2010}.

\begin{figure}[!ht]
\centering
\begin{minipage}[b]{95mm}
\subcaption{}
 % 3. Modelo conceitual de pedogênese
\selectlanguage{english}
\artigotrue
\chapter{DO MORE DETAILED COVARIATES DELIVER MORE ACCURATE SOIL MAPS?}
\chapternote{This chapter is based on A.~Samuel-Rosa, G.B.M.~Heuvelink, G.M.~Vasques, L.H.C.~Anjos. Do 
more detailed environmental covariates deliver more accurate soil maps? \emph{Geoderma}, v.243--244, 
p.214--227, 2015. Terms and expressions have been modified to match the standard terminology used throughout 
the thesis without compromising the content of the original text. Footnotes were added where a definition 
required correction or clarification.}
\shorttitle{Using More Detailed Covariates}
\label{chap:chap06}

% User defined commands
\def\elev{\texttt{ELEV}} % elevation
\def\slp{\texttt{SLP}}   % slope
\def\asp{\texttt{ASP}}   % aspect
\def\nor{\texttt{NOR}}   % northernness
\def\acc{\texttt{ACC}}   % flow accumulation
\def\twi{\texttt{TWI}}   % topographic wetness index
\def\spi{\texttt{SPI}}   % stream power index
\def\tpi{\texttt{TPI}}   % topographic position index
\def\ndvi{\texttt{NDVI}} %
\def\savi{\texttt{SAVI}} %
\def\sibcs{Brazilian System of Soil Classification}

% \def\portuguesekeys{Mapeamento Digital do Solo. Modelo Linear Misto. Informação Auxiliar. Seleção de 
% Variáveis. Acurácia do Modelo. Custo do Mapeamento do Solo}

% \begin{chapterabstract}{brazilian}{\portuguesekeys}
% Neste estudo nós avaliamos se investir em covariáveis espacialmente mais detalhadas aumenta a acurácia dos 
% mapas do solo. Nós usamos um estudo de caso no sul do Brasil para mapear o conteúdo de argila (CLAY), o 
% conteúdo de carbono orgânico (SOC), e capacidade de troca de cátions efetiva (ECEC) da camada superficial do 
% solo de uma área de \SI{\sim2000}{\hectare} localizada na borda do planalto da Bacia Sedimentar do Paraná. 
% Cinco covariáveis, cada uma com dois níveis de detalhe espacial, foram usadas: mapa areal-categórico de solo,
% modelos digitais de elevação (DEM), mapas geológicos, mapas de uso da terra, e imagens de satélite. Trinta e 
% dois modelos de regressão linear múltipla foram calibrados para cada propriedade do solo usando todas as 
% combinações de detalhe espacial das covariáveis. Para cada combinação, \textit{stepwise regression} foi 
% usada para selecionar as variáveis preditoras incorporadas no modelo. A avaliação dos modelos foi feita 
% usando o R-quadrado ajustado da regressão. O modelo de referência, calibrado com a versão menos detalhada de 
% cada covariável, e o modelo com o melhor desempenho, foram usados para calibrar dois modelos lineares mistos 
% para cada propriedade do solo. Parâmetros dos modelos foram estimados usando máxima verossimilhança 
% restrita. Predições espaciais foram realizadas usando o melhor preditor linear não-enviesado empírico. 
% Validação-cruzada foi usada para validar os modelos de regressão linear múltipla e dos modelos lineares 
% mistos de referência e com melhor desempenho. Os resultados mostram que para CLAY a acurácia da predição não 
% aumentou consideravelmente por usar covariáveis mais detalhadas. A quantidade de variância explicada 
% aumentou apenas \SI{\sim2}{\pp} (pontos percentuais), menos do que obtido pela inclusão do passo de 
% krigagem, que explicou \SI{4}{\pp}. Por outro lado, a predição de SOC e ECEC aumentou em \SI{\sim13}{\pp} 
% quando o modelo de referência foi substituído pelo modelo com melhor desempenho. Em geral, o aumento no 
% desempenho preditivo foi modesto e pode não sobrepor os custos adicionais do uso de covariáveis mais 
% detalhadas. Pode ser mais eficiente investir recursos adicionais na coleta de mais observações do solo, ou 
% no aumento do detalhe apenas da covariável que tem o efeito de aumento mais forte. Em nosso estudo, a última 
% funcionaria apenas para SOC e ECEC pelo investimento em um mapa de uso da terra mais detalhado e, 
% possivelmente, também em um mapa geológico e DEM mais detalhados.
% \end{chapterabstract}

\def\englishkeys{Digital Soil Mapping. Linear Mixed Model. Auxiliary Information. Variable Selection. Model
Accuracy. Soil Mapping Cost}
  
\begin{chapterabstract}{english}{\englishkeys}
In this study we evaluated whether investing in more spatially detailed covariates improves the accuracy of 
soil maps. We used a case study from Southern Brazil to map clay content (CLAY), organic carbon content 
(SOC), 
and effective cation exchange capacity (ECEC) of the topsoil for a \SI{\approx2000}{\hectare} area located on 
the edge of the plateau of the Paraná Sedimentary Basin. Five covariates, each with two levels of spatial 
detail were used: area-class soil maps, digital elevation models (DEM), geologic maps, land use maps, and 
satellite images. Thirty-two multiple linear regression models were calibrated for each soil property using 
all 
spatial detail combinations of the covariates. For each combination, stepwise regression was used to select 
predictor variables incorporated in the model. Model evaluation was done using the adjusted R-square of the 
regression. The baseline model, calibrated with the less detailed version of each covariate, and the best 
performing model were used to calibrate two linear mixed models for each soil property. Model parameters were 
estimated using restricted maximum likelihood. Spatial prediction was performed using the empirical best 
linear 
unbiased predictor. Validation of baseline and best performing linear multiple regression and linear mixed 
models was done using cross-validation. Results show that for CLAY the prediction accuracy did not 
considerably 
improve by using more detailed covariates. The amount of variance explained increased only \num{\sim2} 
percentage points (\si{\pp}), less than that obtained by including the kriging step, which explained 
\SI{4}{\pp}. On the other hand, prediction of SOC and ECEC improved by \SI{\sim13}{\pp} when the baseline 
model 
was replaced by the best performing model. Overall, the increase in prediction performance was modest and may 
not outweigh the extra costs of using more detailed covariates. It may be more efficient to spend extra 
resources on collecting more soil observations, or increasing the detail of only those covariates that have 
the 
strongest improvement effect. In our  case study, the latter would only work for SOC and ECEC, by investing in 
a more detailed land use map and possibly also a more detailed geologic map and DEM.
\end{chapterabstract}

\formatchapter

\section{INTRODUCTION}
\label{sec:chap06-intro}

Modern soil mapping relies on the use of statistical models to produce digital representations of spatial 
soil distribution using point soil observations and spatially exhaustive covariates \cite{McBratneyEtAl2003, 
ScullEtAl2003, Florinsky2012}. Three important weaknesses in the statistical soil distribution modelling 
approach can be pointed out. First, it requires sufficient and appropriately distributed point soil data 
within 
the area being mapped \cite{CarreEtAl2007a}. Second, the model structure explores only the empirical 
relationship among environmental conditions and soil properties, being less comprehensive than soil-landscape 
process models \cite{Grunwald2009}. Last, the covariates are only approximations of the true environmental 
conditions that helped shape the soil. They serve only as proxies (surrogates) of the current environmental 
conditions, which in many cases are different from the past conditions under which pedogenesis took place 
\cite{HeuvelinkEtAl2001}. In spite of these weaknesses, modern soil mapping techniques have proven very 
successful in the past decades in producing soil property maps that capture the main patterns of soil spatial 
variation \cite{MooreEtAl1993, McBratneyEtAl2000, Grunwald2009}.

More recently, there has been a growing interest in understanding how the characteristics of the covariates 
influence the success of soil mapping -- this study contributes to this effort. It is commonly accepted that 
the more resources are spent on the construction of a covariate and the more spatial information it has, the 
more accurately it describes the environmental conditions \cite{HupyEtAl2004, HenglEtAl2013a}. It is also 
generally believed that such \emph{more detailed} covariates will be more valuable for soil mapping and lead 
to 
more accurate soil property predictions \cite{CavazziEtAl2013, MaynardEtAl2014}. If these more detailed 
covariates convey more information and represent more adequately the environmental conditions -- the drivers 
of 
soil forming processes --, then it is fair to expect that they improve the accuracy of the resulting soil 
maps. 
However, some studies have shown the contrary \cite{ThompsonEtAl2001, EldeiryEtAl2008, KimEtAl2014}. For 
example, the window size at which DEM derivatives are calculated can be more important than the spatial 
resolution of the DEM \cite{Wood1996, ZhuEtAl2008, BehrensEtAl2010a}. The uncertainty about the added value of 
using more detailed covariates is of concern for those seeking to use resources efficiently, because using 
more 
detailed covariates generally increases soil mapping costs \cite{ShiEtAl2012}.

The objective of this study was to evaluate whether investing in more detailed covariates improves the 
accuracy of soil maps. The main difference of our study to previous ones is that we use a rigorous statistical 
approach to assess the added value of using five more detailed covariates simultaneously. We used a  case study 
in Brazil to compare the accuracy of maps of the clay content, organic carbon content and effective cation 
exchange capacity of the topsoil as obtained from regression kriging on the five covariates, whereby each 
covariate was evaluated on two levels of spatial detail.

\section{MATERIAL AND METHODS}
\label{sec:chap06-methods}

\subsection{Study Area and Soil Data}
\label{subsec:chap06-soil-data}

The study area constitutes a small catchment (\SI{\sim2000}{\hectare}) located on the southern edge of the 
plateau of the Paraná Sedimentary Basin, Rio Grande do Sul, Brazil (\autoref{fig:chap06-location}). The 
climate is classified as Cfa (K\"oppen -- subtropical humid without a dry season) with mean annual temperature 
of \SI{19.3}{\celsius}, and mean annual precipitation of \SI{1708}{\mm}, well distributed throughout the year 
\cite{Maluf2000}. Relief varies between plain (slope between \num{0} and \SI{3}{\percent}) and mountainous 
(slope between \num{45} and \SI{100}{\percent}), and elevations range between \num{140} and \SI{475}{\m}. 
Geology consists of basic, intermediate and acid igneous rocks (rhyolite-rhyodacite and andesite-basalt) of 
the Cretaceous period, consolidated sedimentary rocks (aeolian and fluvial sandstones) of the Triassic and 
Jurassic periods, and non-consolidated (fluvial and colluvial deposits) of the Quaternary period 
\cite{GasparettoEtAl1988, MacielFilho1990, Sartori2009}. Native semi-deciduous forests occupy more than 
half of the area, followed by native grassland used for animal husbandry, semi-deciduous shrubland, annual 
crop agriculture, forestry (Eucalyptus), urban areas, and artificial water bodies \cite{SamuelRosaEtAl2011a}.

\begin{figure}[!ht]
 \centering
 \begin{minipage}[b]{95mm}
  \subcaption{}
  \label{fig:chap06-brazil}
  \centering
  \includegraphics[width=90mm]{fig/chap06-FIG1a}
 \end{minipage}
 \begin{minipage}[b]{95mm}
  \subcaption{}
  \label{fig:chap06-points}
  \centering
  \includegraphics[width=90mm]{fig/chap06-FIG1b}
 \end{minipage}
 \caption[Location of the study area.]{Location of the study area in Santa Maria (a) and spatial distribution 
of the point soil observations and drainage network (b).}
 \label{fig:chap06-location}
\end{figure}

A dataset containing $n = 350$ point soil observations collected between \num{2004} and \num{2011} 
\cite{PedronEtAl2006b, SamuelRosaEtAl2011a, MiguelEtAl2012, Samuel-RosaEtAl2013} was used in this study 
(available at \url{https://github.com/samuel-rosa/dnos-sm-rs-general}). Sampling locations were selected 
purposively and by convenience \cite{Samuel-RosaEtAl2014b}. Three soil pits were opened within an area of 
\SI{\pm100}{\m\square} at most sampling locations to obtain composite samples of the topsoil for laboratory 
analysis. Soil was collected to a depth of \SI{20}{\cm} or less when soil depth was smaller than \SI{20}{\cm}. 
A few observations ($n = 10$) correspond to individual samples collected up to \SI{30}{\cm}. Sampling depth 
ranges from \num{2} to \SI{30}{\cm}, with a mean of \SI{17.3}{\cm}. We assumed that the vertical, horizontal 
and temporal support differences between soil samples is negligible for the purpose of this study.

Three soil properties (fine earth fraction, \SI{<2}{\mm}) were explored: clay content (CLAY, 
\si{\gram\per\kilo\gram}), organic carbon content (SOC, \si{\gram\per\kilo\gram}), and effective cation 
exchange capacity (ECEC, \si{\milli\mole\per\kilo\gram}). CLAY was determined by the pipette method. SOC was 
determined using wet digestion. ECEC was calculated as the sum of exchangeable bases plus exchangeable 
acidity. The soil properties selected were expected to present different patterns of spatial variation and 
correlation with the most dominant factors of soil formation \cite{Jenny1941} in the area: organisms 
(\textit{O}), relief (\textit{R}), and parent material (\textit{P}). CLAY was presumed to have a stronger 
relation with \textit{P}, while SOC was expected to be more correlated with \textit{O}. Because the soils of 
the study area were strongly eroded due to intense agriculture in the \num{20}th century, both CLAY and SOC 
were also expected to be closely related with \textit{R}. Finally, ECEC was expected to be strongly correlated 
with \textit{P} and \textit{O}, which is supported by its natural relationship with both CLAY and SOC.

\begin{figure}[!ht]
 \centering
 \begin{minipage}[b]{63mm}
  \subcaption{}
  \centering
  \includegraphics[width=63mm]{fig/chap06-FIG2a}
 \end{minipage}
 \begin{minipage}[b]{63mm}
  \subcaption{}
  \centering
  \includegraphics[width=63mm]{fig/chap06-FIG2d}
 \end{minipage}
 \begin{minipage}[b]{63mm}
  \subcaption{}
  \centering
  \includegraphics[width=63mm]{fig/chap06-FIG2b}
 \end{minipage}
 \begin{minipage}[b]{63mm}
  \subcaption{}
  \centering
  \includegraphics[width=63mm]{fig/chap06-FIG2e}
 \end{minipage}
 \begin{minipage}[b]{63mm}
  \subcaption{}
  \centering
  \includegraphics[width=63mm]{fig/chap06-FIG2c}
 \end{minipage}
 \begin{minipage}[b]{63mm}
  \subcaption{}
  \centering
  \includegraphics[width=63mm]{fig/chap06-FIG2f}
 \end{minipage}
 \caption[Summary statistics of CLAY, SOC, and ECEC.]{Histogram, empirical density function, and summary 
statistics of CLAY (a, b), SOC (c, d), and ECEC (e, f) in the original (left) and Box-Cox feature spaces 
(right).}
 \label{fig:chap06-soil-properties}
\end{figure}

Point soil data, here denoted by $Y(s)$, showed a positive skew (\autoref{fig:chap06-soil-properties}) and was 
normalized, $Y'(s)$, using the Box-Cox family of power transformations, where $Y'(s) = (Y(s)^{\lambda} - 1) / 
\lambda$, if $\lambda > 0$, and $Y'(s) = log(Y(s))$, if $\lambda = 0$ \cite{DiggleEtAl2007}. Lambda 
($\lambda$) values were selected empirically \cite{FoxEtAl2011}. Because the resulting distribution of the 
back-transform (see \autoref{subsec:chap06-validation}) has no expectation when $\lambda < 0$ 
\cite{RibeiroEtAl2001}, a logarithm transformation ($\lambda = 0$) was used when a negative $\lambda$ was 
estimated (SOC and ECEC).

\subsection{Covariates}
\label{subsec:chap06-sources}

Five freely available covariates were evaluated in this study, each with two levels of spatial detail: 
area-class soil maps (\texttt{soil}), geologic maps (\texttt{geo}), land use maps (\texttt{land}), digital 
elevation models (\texttt{dem}), and satellite images (\texttt{sat}). Each pair was composed of covariates 
that were produced separately from scratch using different data sources and/or production methods, thus 
demanding different amounts of resources (time, workforce, budget, technology, etc.). In this study, the level 
of spatial detail of a covariate is a function of the components of its production process such as the 
cartographic ratio (\texttt{soil}, \texttt{geo} and \texttt{land}), spatial sampling support (\texttt{sat}), 
number and diversity of data sources explored (\texttt{dem}), and quantity of spatial data used (all five). 
Thus, the reader should bear in mind that our definition of spatial detail is broader than spatial resolution 
or spatial scale. It should also not be confounded with spatial support \cite{WebsterEtAl2007} or thematic 
detail \cite{Rossiter2000}.

\def\footcovars{\footnote{In statistical terms, the terms \emph{covariate} and \emph{predictor variable} are 
synonymous, and the reason for the use given in this study is purely operational.}}

The covariates were transformed to predictor variables\footcovars{} that were used in the geostatistical 
modelling. Since the transformation is different for categorical and continuous covariates, the procedures are 
explained below for each type separately.

\subsubsection{Categorical predictor variables}
\label{subsubsec:chap06-categorical-covars}

Area-class soil maps, geologic maps and land use maps are categorical covariates (factors). Mapping units are 
the $k$ factor levels that are transformed to as many dummy (indicator, binary) variables as there are factor 
levels, before model calibration. Each dummy variable receives a value equal to one (\num{1}) when a given 
class is present, and zero (\num{0}) otherwise \cite{Everitt2006}. If the number of point soil observations 
falling inside the spatial domain of a mapping unit is too small to accurately estimate a regression 
coefficient (we used a threshold of $n = 15$ observations), the mapping unit is merged with a similar mapping 
unit prior to calculating dummy variables. The resulting generalized categorical covariate maps are shown in 
\autoref{fig:chap06-cat-covars}. The binary maps are the categorical predictor variables.

\noindent\textit{Soil maps}. The less detailed soil map (\soilOld) was published with a \scale{100000} and 
has five mapping units \cite{AzolinEtAl1988} (\autoref{fig:chap06-soil-old}). It was produced using existing 
soil maps and technical reports (\scale{750000}) \cite{Brasil1973}, aerial photographs (\scale{60000}), 
topographic maps (\scale{50000}), and sparse point soil observations along the road network. The more detailed 
soil map (\soilNew) was prepared with a \scale{25000} and has eight mapping units \cite{MiguelEtAl2012} 
(\autoref{fig:chap06-soil-new}). It was produced using high spatial resolution satellite images 
(\SI{65}{\cm}), existing soil maps and technical reports published with a \scale{50000} \cite{Poelking2007} 
and \num{1}:\num{25000} \cite{PedronEtAl2006b}, topographic maps (\scale{25000}), and descriptions from 
\num{\sim350} point soil observations. Five dummy predictor variables were derived from \soilOld{} and seven 
from \soilNew{} (\autoref{tab:chap06-soil-covars}).

\begin{figure}[!ht]
 \centering
 \begin{minipage}[b]{63mm}
  \subcaption{Cartographic scale: \num{1}:\num{100000}}
  \label{fig:chap06-soil-old}
  \centering
  \includegraphics[width=60mm]{fig/chap06-FIG3a}
 \end{minipage}
 \begin{minipage}[b]{63mm}
  \subcaption{Cartographic scale: \num{1}:\num{25000}}
  \label{fig:chap06-soil-new}
  \centering
  \includegraphics[width=60mm]{fig/chap06-FIG3d}
 \end{minipage}    
 \begin{minipage}[b]{63mm}
  \subcaption{Cartographic scale: \num{1}:\num{50000}}
  \label{fig:chap06-geo-old}
  \centering
  \includegraphics[width=60mm]{fig/chap06-FIG3b}
 \end{minipage}
 \begin{minipage}[b]{63mm}
  \subcaption{Cartographic scale: \num{1}:\num{25000}}
  \label{fig:chap06-geo-new}
  \centering
  \includegraphics[width=60mm]{fig/chap06-FIG3e}
 \end{minipage}
 \begin{minipage}[b]{63mm}
  \subcaption{Cartographic scale: \num{1}:\num{500000}}
  \label{fig:chap06-land-old}
  \centering
  \includegraphics[width=60mm]{fig/chap06-FIG3c}
 \end{minipage}
 \begin{minipage}[b]{63mm}
  \subcaption{Cartographic scale: \num{1}:\num{2000}}
  \label{fig:chap06-land-new}
  \centering
  \includegraphics[width=60mm]{fig/chap06-FIG3f}
 \end{minipage}
 \caption[Area-class soil maps, geologic maps, and land use maps compared in the study. ]{Area-class soil maps 
(a, b), geologic maps (c, d), and land use maps (e, f) compared in our study. The less and more detailed 
version are displayed at the left and right, respectively. Legend abbreviations and derived dummy variables are 
described in Tables \ref{tab:chap06-soil-covars}--\ref{tab:chap06-land-covars}.}
 \label{fig:chap06-cat-covars}
\end{figure}

\noindent\textit{Geologic maps}. The less detailed geologic map (\geoOld) was produced using topographic maps 
with \scale{50000} \cite{GasparettoEtAl1988} (\autoref{fig:chap06-geo-old}). The more detailed geologic map 
(\geoNew) was produced using topographic maps with \scale{25000}, and includes the location of overlaying 
Quaternary sedimentary deposits \cite{MacielFilho1990} (\autoref{fig:chap06-geo-new}). \geoNew{} did not cover 
a small part in the North of the study area, where \geoOld{} was used instead (this strategy was approved by 
experts on the local geology). The mapping unit of both geologic maps depicting the Caturrita Formation was 
used indirectly by deriving dummy predictor variables from all other individual mapping units. Three dummy 
predictor variables were derived from \geoOld{} and four from \geoNew{} (\autoref{tab:chap06-geology-covars}).

\noindent\textit{Land use maps}. The less detailed land use map (\landOld) was produced by manually digitizing 
land use data included in topographic maps with a \scale{25000} \cite{DSG1980, DSG1992, DSG1992a} 
(\autoref{fig:chap06-land-old}). The more detailed land use map (\landNew) was prepared (\scale{2000}) by 
manual digitization using \SI{65}{\cm} spatial resolution satellite images covering the years \num{2008} and 
\num{2009} \cite{SamuelRosaEtAl2011a} (\autoref{fig:chap06-land-new}). Mapping units depicting human 
settlements and water bodies ($n = 0$) were not masked out from the prediction grid and were merged with other 
mapping units to derive dummy predictor variables. Five dummy predictor variables were derived from \landNew{} 
and two from \landOld{} (\autoref{tab:chap06-land-covars}).

\subsubsection{Continuous predictor variables}
\label{subsubsec:chap06-continuous-covars}

\def\arcgis{\href{http://resources.arcgis.com/en/help/main/10.1/index.html}{ArcGIS}}

The less detailed DEM (\demOld) is the hole-filled SRTM DEM version~\num{4} \cite{JarvisEtAl2008} 
(\autoref{fig:chap06-dem-old}). The spatial sampling support of the SRTM DEM is \SI{1}{\arcsecond} 
(\SI{\sim30}{\m}), but elevation data were aggregated to \SI{3}{\arcsecond} (\SI{\sim90}{\m}) for public 
release in regions outside the United States \cite{ReuterEtAl2007}. The more detailed DEM (\demNew) was 
produced by interpolating contour lines with vertical spacing of \SI{10}{\m} along with data about the 
drainage network, lakes and peaks digitized from topographic maps with \scale{25000} 
(\autoref{fig:chap06-dem-new}). Interpolation to \SI{5}{\m} pixel size was performed using a hydrologically 
correct algorithm implemented in \arcgis{} software by ESRI \cite{Hutchinson1989}. Contour line artefacts were 
minimized using a seven by seven low-pass filter (\grass{r.neighbors}). The window size was chosen such that 
the smoothed DEM best matched the original contour map while also respecting the original drainage network 
pattern.

\input{chap/tab/chap06-TAB1.tex}

\ctable[
 caption  = {Description of the $p = 7$ dummy predictor variables derived from the two geologic maps.},
 cap      = {Predictor variables derived from geologic maps.},
 label    = tab:chap06-geology-covars,
 notespar,
 pos      = !ht,
 maxwidth = \textwidth,
 % doinside = \scriptsize\setstretch{1.1}
 doinside = \small
 ]{l p{0.85\textwidth} l}{
 \tnote[a]{Minimum Legible Delineation calculated following \citet{Rossiter2000}.}
 }{ \FL
 Code & Mapping unit(s) included and Description\tmark[a] \ML
 
 \multicolumn{2}{p{0.98\linewidth}}{Source: \citet{GasparettoEtAl1988}. Cartographic scale: 
 \num{1}:\num{50000}. Minimum Legible Delineation: \SI{10}{\hectare}.} \NN
 
 \texttt{GEO\_50a} & \textit{SG-I}. Inferior Sequence of the Serra Geral Formation. Composed mainly of basic 
 igneous rocks (tholeiitic basalt and andesite). It is likely to be related with high CLAY and ECEC. \NN
 \texttt{GEO\_50b} & \textit{SG-S}. Superior Sequence of the Serra Geral Formation. Composed mainly of acid 
 igneous rocks (granophyric rhyolite and rhyodacite). It is likely to be related with moderate to high CLAY 
 and ECEC. \NN
 \texttt{GEO\_50c} & \textit{BT}. Botucatu Formation. Composed mainly of aeolian sandstones. It is likely to 
 be related with low CLAY and ECEC. \NN
 Other & \textit{CT} depicts the Caturrita Formation, which is composed mainly of fluvial sandstones. \NN
 & \NN
 
 \multicolumn{2}{p{0.98\linewidth}}{Source: \citet{MacielFilho1990}. Cartographic scale: 
 \num{1}:\num{25000}. Minimum Legible Delineation: \SI{2.5}{\hectare}.} \NN
 
 \texttt{GEO\_25a} & \textit{SG-I}. Inferior Sequence of the Serra Geral Formation. \NN
 \texttt{GEO\_25b} & \textit{SG-S}. Superior Sequence of the Serra Geral Formation. \NN
 \texttt{GEO\_25c} & \textit{BT}. Botucatu Formation. \NN
 \texttt{GEO\_25d} & \textit{QD}. Quaternary deposits of fluvial, alluvial, and colluvial origin. It can help 
 explaining the low CLAY of soils supposedly derived from igneous rocks. \NN
 Other & \textit{CT} depicts the Caturrita Formation. \LL
 }


\input{chap/tab/chap06-TAB3.tex}

\begin{figure}[!ht]
 \centering
 \begin{minipage}[b]{63mm}
  \subcaption{Spatial resolution: \SI{90}{\m}}
  \label{fig:chap06-dem-old}
  \centering
  \includegraphics[width=60mm]{fig/chap06-FIG4a}
 \end{minipage}
 \begin{minipage}[b]{63mm}
  \subcaption{Spatial resolution: \SI{30}{\m}}
  \label{fig:chap06-sat-old}
  \centering
  \includegraphics[width=60mm]{fig/chap06-FIG4b}
 \end{minipage}
 \begin{minipage}[b]{63mm}
  \subcaption{Vertical spacing of contours: \SI{10}{\m}}
  \label{fig:chap06-dem-new}
  \centering
  \includegraphics[width=60mm]{fig/chap06-FIG4c}
 \end{minipage}
 \begin{minipage}[b]{63mm}
  \subcaption{Spatial resolution: \SI{5}{\m}}
  \label{fig:chap06-sat-new}
  \centering
  \includegraphics[width=60mm]{fig/chap06-FIG4d}
 \end{minipage}
 \caption[Digital elevation models and satellite images compared in the study.]{Digital elevation models (a, c) 
and satellite images, depicted using the normalized difference vegetation index (b, d), compared in our study. 
The less detailed version is displayed at the top, while the more detailed version is shown on the bottom.}
 \label{fig:chap06-con-covars}
\end{figure}

Eight DEM derivatives were calculated: elevation (\elev), slope (\slp), aspect (\asp), northernness (\nor), 
flow accumulation (\acc), topographic wetness index (\twi), stream power index (\spi), and topographic 
position index (\tpi). \slp{} and \asp{} were calculated using \grass{r.param.scale} with seven window sizes 
(sampling support, analysis scale): \num{3}, \num{7}, \num{15}, \num{31}, \num{63}, \num{127}, and \num{255}. 
\asp{} was scaled to the standard \num{0}--\ang{360} range and orientation, and was transformed to \nor{} 
using $\texttt{NOR} = abs(\ang{180} - \texttt{ASP})$. \twi{} and \spi{} were calculated using \slp{} 
calculated with different window sizes, and \acc{} calculated using \grass{r.watershed}. \tpi{} was calculated 
using \saga{ta\_morphometry} with the same seven window sizes. The combination of DEM derivatives (\elev, 
\slp, \nor, \twi, \spi, and \tpi) and window sizes yielded $p = 36$ continuous predictor variables from each 
DEM.

The less detailed satellite image was acquired by the Landsat-\num{5} Thematic Mapper on December \num{26}, 
\num{2010} (available at Instituto Nacional de Pesquisas Espaciais - Divisão de Geração de Imagens -- 
\inpedgi) (\autoref{fig:chap06-sat-old}). It has \SI{8}{\bit} radiometric resolution and \SI{\sim30}{\m} 
spatial resolution. Spectral bands were orthorectified (Geomatica OrthoEngine) and radiometrically corrected 
(\grass{i.landsat.toar}). The more detailed satellite image comes from the RapidEye constellation (available 
at Ministério do Meio Ambiente -- \mma) (\autoref{fig:chap06-sat-new}). It was acquired on November \num{16}, 
\num{2012}, has \SI{16}{\bit} radiometric resolution, \SI{6.5}{\m} spatial resolution, and was orthorectified 
to \SI{5}{\m} spatial resolution. Both images were atmospherically (6S atmospheric model 
\cite{VermoteEtAl1997}, \grass{i.atcorr}) and topographically corrected (\grass{i.topo.corr}). Derived 
predictor variables are the spectral bands (except the thermal band) and vegetation indices (normalized 
difference vegetation index - NDVI, and soil-adjusted vegetation index - SAVI). Eight continuous predictor 
variables were derived from the Landsat-5~TM image and nine from the RapidEye image.

\subsubsection{Additional processing}
\label{subsubsec:chap06-sources-processing}

Soil maps, geologic maps, land use maps, and satellite images were registered with the prediction grid 
(\SI{5}{\m} pixel size) using nearest neighbour resampling. \demOld{} was registered using cubic resampling 
\cite{Samuel-RosaEtAl2013c}. Systematic positional errors were corrected using affine transformation 
\cite{Samuel-RosaEtAl2014}.

\subsection{Linear Mixed Model of Spatial Variation}
\label{subsec:chap06-lmm}

We model each of the soil properties of interest as the outcome of a spatial stochastic process. The model is 
composed of fixed and random effects \cite{HeuvelinkEtAl2001, LarkEtAl2006}. We use the point soil 
observations and spatially exhaustive predictor variables to calibrate the model and predict the outcome of 
the spatial stochastic process at unobserved locations. This fixed effect (deterministic trend), 
$m(\textbf{s})$, describes that part of the spatial variation of the soil property that is explained by the 
covariates. We assume here that is a linear function of the predictor variables. The random effect (stochastic 
residuals, latent variables), $e(\textbf{s})$, describes that part of the spatial variation that cannot be 
explained by the covariates \cite{Cressie1993}. It is represented by a spatially correlated, Gaussian 
distributed random variable, that is assumed stationary in the mean and covariance. Thus, the linear mixed 
model of spatial variation that we employed is given by

\begin{equation}\label{eqn:chap06-lmm}
 Y'(\textbf{s}) = m(\textbf{s}) + e(\textbf{s}) 
                = \sum_{j=0}^{p} \beta_{j}\cdot X_{j}(\textbf{s}) + e(\textbf{s}),
\end{equation}

\noindent{where $Y'(\textbf{s})$ is the soil property after Box-Cox transformation, $m(\textbf{s})$ and 
$e(\textbf{s})$ are defined as above, $\beta_{j}$ are the regression model coefficients, and 
$X_{j}(\textbf{s})$ is the regression model matrix, with $j = 0, 1, 2, \ldots, p$, $p$ being the number of 
predictor variables. Variable $X_{0}(\textbf{s})$ is taken as unity so that $\beta_{0}$ is the intercept.}

\subsubsection{Model selection}

We calibrated $k = 2^5 = 32$ multiple linear regression models for each soil property (fitted using ordinary 
least squares, OLS) to model the deterministic trend for each combination of the five covariates (recall from 
\autoref{sec:chap06-intro} that each covariate is available at two levels of spatial detail, hence $2^5$ 
combinations). The number of predictor variables used to calibrate each model varied among combinations 
between $p = 52$ and $p = 62$, because more detailed covariates enabled the derivation of a larger number of 
predictor variables (except the DEM). Backward VIF (variance inflation factor) selection followed by stepwise 
AIC (Akaike's Information Criterion) selection were used to select predictor variables to enter the models 
\cite{Samuel-RosaEtAl2014c, VenablesEtAl2002}.

The $k = 32$ multiple linear regression models calibrated for each soil property were ranked using the ratio 
between the regression sum of squares and the total sum of squares. Because stepwise regression results in 
biased models \cite{Harrell2001a}, the ratio of sum of squares was adjusted (${R}^{2}_{adj}$) using the number 
of predictor variables initially offered to enter the model instead of the reduced number of predictor 
variables that entered the model. Next, the five covariates were ranked based on how their level of spatial 
detail related with the calibration of models with improved predictive performance. The relation between the 
level of spatial detail of the covariates and model performance was evaluated using a graphical output called 
\emph{model series plot} (\Rpackage{pedometrics}, \citet{Samuel-RosaEtAl2014c}). Pedological evaluation 
of predictor variables included in the models was omitted because this was beyond our objectives.

The multiple linear regression model calibrated using only the less detailed covariates, which we call the 
\emph{baseline} model, and the multiple linear regression model with the highest ${R}^{2}_{adj}$, which we 
call the \emph{best performing} model, were extended to linear mixed models of spatial variation 
(\autoref{eqn:chap06-lmm}) for each soil property. Estimation of the parameters of the linear mixed models was 
performed using residual (restricted, marginal) maximum likelihood (REML) \cite{RibeiroEtAl2001, 
LarkEtAl2004}. The spatial correlation function adopted was the exponential function (this is equivalent to 
the Matérn correlation function with smoothness parameter $\nu = 0.5$ \cite{Stein1999}).

\subsubsection{Model validation}
\label{subsec:chap06-validation}

Only the \emph{baseline} and \emph{best performing} multiple linear regression and linear mixed models 
calibrated for each soil property were validated. Model validation was performed using leave-one-out 
cross-validation (LOO-CV) \cite{BrusEtAl2011}. All model parameters were re-estimated at each LOO-CV run 
to reduce bias \cite{LaslettEtAl1987}. LOO-CV predicted values were back-transformed from the Box-Cox space 
to the original space of soil properties using stochastic simulation \cite{ChristensenEtAl2001}:

\begin{enumerate}[label=(\Roman*)]
 \item each predicted value and associated prediction error variance were used to simulate $n = \num{20000}$ 
 values from a Gaussian distribution;
 
 \item simulated values were back-transformed using $Y(s) = (Y'(s) \times \lambda + 1)^{1 / \lambda}$, if 
 $\lambda > 0$, and $Y(s) = exp(Y'(s))$, if $\lambda = 0$;
 
 \item the mean and variance of back-transformed simulated values were used as the predicted value and 
 prediction error variance in the original space of soil properties.
\end{enumerate}

Five error statistics were computed from the leave-one-out cross-validation results \cite{JanssenEtAl1995, 
KempenEtAl2010, BrusEtAl2011}. The mean error (\textit{ME}), which measures the prediction bias, the mean 
absolute error (\textit{MAE}) and the root mean squared error (\textit{RMSE}), which measure the prediction 
accuracy, the scaled root mean squared error (\textit{SRMSE}, also known as mean squared deviation ratio), 
which measures how well the prediction error variance matches the squared differences between predicted and 
observed soil property, where $\textit{SRMSE} > 1$ indicates under-estimation, while $\textit{SRMSE} < 1$ 
indicates over-estimation, and the amount of variance explained (\textit{AVE}, also known as coefficient of 
determination or ratio of scatter), which measures the fraction of the overall spread of observed values that 
is explained by the model. The AVE ranges from \num{0} to \num{100}, where $\textit{AVE} = 100$ is the optimal 
value.

\subsubsection{Spatial prediction}
\label{subsec:chap06-prediction}

Only the \emph{baseline} and \emph{best performing} linear mixed models calibrated for each soil property were 
used for spatial prediction. Spatial predictions at a fine grid of \num{\sim800000} point locations were made 
in the Box-Cox space using the best linear unbiased predictor (BLUP) with the empirical estimates of the 
random effects (EBLUP) \cite{LarkEtAl2006}. EBLUP with a fixed effect model is conceptually equivalent to 
kriging with external drift and regression kriging, and mathematically equivalent to kriging with external 
drift and universal kriging. Point predicted values and prediction error variances were back-transformed to the 
original soil property space using stochastic simulation as described above 
(\autoref{subsec:chap06-validation}).

\section{RESULTS}
\label{sec:chap06-results}

\subsection{Model Series Plots}

The model series plot is a graphical description of the relation between the prediction accuracy of multiple 
linear regression models and the covariates used to calibrate them (\autoref{fig:chap06-model-series}). The 
magnitude of improvement in prediction accuracy is depicted in the bottom panel with the ${R}^{2}_{adj}$. The 
top panel is interpreted both horizontally and vertically. In the vertical direction we identify which version 
of each covariate was used to calibrate a given model. The less and the more detailed versions are identified 
by the yellow (bright) and green (dark) colours, respectively. The \emph{baseline} model is identified by the 
column containing only yellow cells, while the column with only green cells represents the model calibrated 
using only the more detailed version of each covariate, which we call the \emph{most detailed} model. The 
first important results that we obtain from the model series plots is that a) the \emph{baseline} model is not 
the model with the lowest ${R}^{2}_{adj}$, which we call the \emph{poorest performing} model, and b) the 
\emph{most detailed} model is not the \emph{best performing} model.

\begin{figure}[!ht]
 \centering
 \begin{minipage}[b]{\textwidth}
  \subcaption{}
  \includegraphics[width=\textwidth]{fig/chap06-FIG5a}
 \end{minipage}
 \begin{minipage}[b]{\textwidth}
  \subcaption{}
  \includegraphics[width=\textwidth]{fig/chap06-FIG5b}
 \end{minipage}
 \begin{minipage}[b]{\textwidth}
  \subcaption{}
  \includegraphics[width=\textwidth]{fig/chap06-FIG5c}
 \end{minipage}
 \caption[Model series plots for CLAY, SOC, and ECEC.]{Model series plots for CLAY (a), SOC (b), and ECEC (c). 
The less and more detailed version of each covariate are identified with the yellow (bright) and green (dark) 
colours, respectively. Multiple linear regression models were ranked using their ${R}^{2}_{adj}$. Triangles 
show the mean ranking of the more detailed covariates (i.e. centre of green cells).}
 \label{fig:chap06-model-series}
\end{figure}

The row-wise analysis of the model series plots shows if a model calibrated with the more detailed version of 
a given covariate has a higher prediction accuracy. This information is retrieved by looking at the row-wise 
distribution of green cells -- these cells represent the $k = 16$ models calibrated using the more detailed 
version of a given covariate, irrespective of the version of the other covariates. The more concentrated the 
green cells are in the right half of the plot, the larger the relative benefit of using the more detailed 
version of that covariate. For example, the top row of the second model series plot shows the SOC models 
calibrated using the two versions of the land use map (\texttt{land}). All green cells are on the right half 
of the plot between rankings \num{1} and \num{16} (see the x axis). The four lower rows show that the green 
cells of the other four covariates are distributed along the entire ranking range (from \num{1} to \num{32}). 
This means that the relative benefit of calibrating a SOC model with a more detailed land use map is larger 
compared to that of using a more detailed version of the other covariates.

The centre of the row-wise distribution of the green cells for each covariate, calculated as the mean ranking, 
is represented by the triangles. The mean ranking quantifies the relative benefit of using a more detailed 
version of each covariate. For example, the mean ranking of the SOC models calibrated using the more detailed 
land use map is about \num{8} (top row), while the mean ranking of the models calibrated using the more 
detailed satellite image (\texttt{sat}) is close to \num{20} (bottom row). Using the more detailed DEM 
(\texttt{dem}) is almost as beneficial as using the more detailed geologic map (\texttt{geo}) -- the mean 
ranking of the SOC models calibrated using the more detailed version of these two covariates is about 
\num{15}--\num{16} (second and third rows). Using the more detailed version of the soil map (\texttt{soil}, 
fourth row) is not as beneficial as using \texttt{land}, \texttt{geo} or \texttt{dem}, but more beneficial 
than using \texttt{sat}. Because the covariates were ranked based on the mean rankings, the covariate 
displayed in the top row of each model series plot is the one which resulted in the largest improvement of the 
prediction accuracy when the more detailed version was used to calibrate the model -- for SOC this is the land 
use map.

For CLAY, calibrating the models with the more detailed soil map resulted in the largest improvement of the 
prediction accuracy relative to the other covariates. The DEM was the second most beneficial covariate (mean 
ranking of \num{15}), but the benefit of using its more detailed version was similar to that of using the more 
detailed version of any other covariate (mean rankings between \num{17} and \num{18}). Nine models had a 
poorer prediction performance than the baseline model, ranked \num{27}th, the poorest performing model being 
that calibrated with the more detailed land use map and satellite image. Despite these patterns, calibrating 
CLAY models with the more detailed version of any covariate resulted in a small improvement of the prediction 
accuracy, as evidenced by the small increases of the ${R}^{2}_{adj}$. The difference between the poorest and 
best performing models is less than \SI{3}{\pp} (percentage points). In comparison, for SOC, by simply 
using the more detailed land use map we already obtained a model ranked \num{9}th, an increase of \SI{8}{\pp} 
in ${R}^{2}_{adj}$ compared to the baseline model, ranked \num{24}th.

The same general pattern observed for SOC models was observed for ECEC models -- the more detailed land use 
map results in the largest improvement of the prediction accuracy. The main difference is that calibrating the 
models with the more detailed geologic map was slightly more beneficial for ECEC (mean ranking of \num{12}) 
than for SOC (mean ranking of \num{14}). The poorest performing ECEC model was that calibrated with the more 
detailed satellite image. Using only the more detailed land use map resulted in an improvement of \SI{6}{\pp} 
in ${R}^{2}_{adj}$ (model ranked \num{7}th), differing from the best performing model by only \SI{2}{\pp}. 
Using the more detailed version of all covariates except the soil map or satellite image resulted in increases 
of about \num{6} and \SI{7}{\pp} in ${R}^{2}_{adj}$, respectively. The baseline model was ranked as 
\num{28}th, which is a higher ranking than the models calibrated with all possible combinations of the more 
detailed satellite image and the more detailed soil map and/or DEM.

The patterns observed in the model series plots resulted from the change (increase or decrease) of the 
importance of each covariate on explaining the variance when the more detailed version was used 
(\autoref{tab:chap06-drop}). We used the \emph{baseline} and \emph{most detailed} models to quantify this 
change. Each model was refitted dropping one covariate at a time. The difference $\Delta$ between the 
${R}^{2}_{adj}$ of the model calibrated with all five $q$ covariates (${R}^{2}_{adj}{}_{q = 5}$) and the model 
calibrated without the $q$-th covariate ($R^{2}_{adj}{}_{q = 5 - 1}$) was calculated. The more positive 
$\Delta{R}^{2}_{adj}$ becomes, the more beneficial the more detailed version of the $q$-th covariate is for 
improving prediction accuracy. For CLAY, \texttt{dem} and \texttt{land} were the most important covariates in 
the \emph{baseline} model, while \texttt{geo} was the least important. The importance of \texttt{soil} and 
\texttt{geo} increased when their more detailed version was used (change of \SI{+0.013}{\pp} for both), while 
\texttt{sat}, \texttt{land} and \texttt{dem} became less important. For SOC and ECEC, \texttt{land} was not 
the most important covariate in the \emph{baseline} model. But it was the covariate whose importance had the 
largest positive shift when the more detailed version was used (\SI{+0.085}{\pp} for SOC and \SI{+0.045}{\pp}
for ECEC). \texttt{sat} became less important when the more detailed version was used -- see its low ranking 
in all model series plots. The increase of the importance of \texttt{geo} was larger for ECEC 
(\SI{+0.026}{\pp}) than for SOC (\SI{+0.013}{\pp}) -- see the difference in the mean ranking of \texttt{geo} 
in 
the SOC (\num{14}) and ECEC (\num{12}) model series plots.

\input{chap/tab/chap06-TAB4.tex}

\subsection{REML Fit of the Variogram Model}

\def\footnugget{\footnote{To be more precise, the small number of point observations separated by short 
distances reduces the ability of modelling the behaviour of the variogram near the origin as a whole.}}

The small improvement in the prediction accuracy of the CLAY linear mixed model calibrated with the more 
detailed covariates is evidenced by \autoref{fig:chap06-lmm}. The shape of the experimental variogram is very 
similar for both \emph{baseline} and \emph{best performing} linear mixed models, which is also true for SOC 
and ECEC. However, the sill variance had a very small reduction for CLAY compared to SOC and ECEC. The last 
two showed a more considerable improvement in prediction accuracy. It can also be seen that the number of 
point observations separated by short distances is very small, reducing the accuracy of the estimate of the 
nugget variance\footnugget{}. The result is that the estimated nugget variance changes rather erratically 
from the \emph{baseline} to the \emph{best performing} models, decreasing for CLAY and SOC, and increasing 
for ECEC.

\begin{figure}[!ht]
 \centering
 \begin{minipage}[b]{90mm}
  \subcaption{}
  \includegraphics[width=90mm]{fig/chap06-FIG6a} 
 \end{minipage}
 \begin{minipage}[b]{90mm}
  \subcaption{}
  \includegraphics[width=90mm]{fig/chap06-FIG6b}
 \end{minipage}
 \begin{minipage}[b]{90mm}
  \subcaption{}
  \includegraphics[width=90mm]{fig/chap06-FIG6c}
 \end{minipage}
 \caption[Linear mixed models for CLAY, SOC, and ECEC.]{Experimental variogram (dots) and REML fit of the 
linear mixed models (line) for CLAY (a), SOC (b), and ECEC (c). Left -- baseline model. Right -- best 
performing model.}
 \label{fig:chap06-lmm}
\end{figure}

\subsection{Validation}

The LOO-CV results indicate that the linear mixed models for CLAY are slightly positively biased, while 
those for SOC and ECEC are slightly negatively biased (\autoref{tab:chap06-cv-stats}). For both CLAY and ECEC, 
the \textit{MAE} shows that these models are more accurate than the multiple linear regression models, 
suggesting that the kriging step improves the prediction accuracy.

\ctable[
 caption = {Statistics$^a$ of the leave-one-out cross-validation of baseline and best performing multiple
 linear regression models (LM) and linear mixed models (LMM).},
 cap     = {Cross-validation of baseline and best performing models.},
 label   = tab:chap06-cv-stats,
 notespar,
 maxwidth = \textwidth,
 pos     = !th,
 % doinside = \scriptsize\setstretch{1.1}
 doinside = \small
 ]{llrrrrr}{
 \tnote[a]{Statistics: mean error (\textit{ME}), mean absolute error (\textit{MAE}), root mean squared error 
 (\textit{RMSE}), scaled root mean squared error (\textit{SRMSE}, unitless), and amount of variance explained 
 (\textit{AVE}, percent).}
 }{\FL
   \multicolumn{1}{l}{Model}&\multicolumn{1}{c}{Type}&\multicolumn{1}{c}{\textit{ME}}&\multicolumn{1}{c}{\textit{MAE}}&\multicolumn{1}{c}{\textit{RMSE}}&\multicolumn{1}{c}{\textit{SRMSE}}&\multicolumn{1}{c}{\textit{AVE}}\ML
   \multicolumn{7}{l}{CLAY (g kg$^{-1}$)}\NN
   ~~Baseline&LM&$ 1.31$&$52.1$&$ 72.1$&$0.89$&$56.8$\NN
   ~~&LMM&$ 0.94$&$48.5$&$ 68.8$&$1.03$&$60.7$\NN
   ~~Best performing&LM&$ 1.59$&$51.3$&$ 70.7$&$0.91$&$58.4$\NN
   ~~&LMM&$ 1.08$&$47.8$&$ 68.1$&$1.03$&$61.5$\ML
   \multicolumn{7}{l}{SOC (g kg$^{-1}$)}\NN
   ~~Baseline&LM&$-0.30$&$10.9$&$ 18.9$&$1.22$&$35.8$\NN
   ~~&LMM&$-0.39$&$11.0$&$ 19.4$&$1.43$&$32.5$\NN
   ~~Best performing&LM&$-0.20$&$10.1$&$ 16.9$&$0.91$&$49.0$\NN
   ~~&LMM&$-0.25$&$10.4$&$ 17.6$&$1.16$&$44.3$\ML
   \multicolumn{7}{l}{ECEC (mmol kg$^{-1}$)}\NN
   ~~Baseline&LM&$-0.88$&$70.6$&$112.4$&$0.97$&$22.3$\NN
   ~~&LMM&$-0.32$&$63.3$&$101.1$&$1.32$&$37.1$\NN
   ~~Best performing&LM&$-0.76$&$64.9$&$101.7$&$0.86$&$36.3$\NN
   ~~&LMM&$-0.29$&$62.6$&$ 97.9$&$1.09$&$41.1$\LL
}


Overall, all models had a moderate to poor prediction performance. The errors are, in absolute values, 
somewhat large, mainly for ECEC. The best \textit{AVE} are about \SI{60}{\percent} for CLAY, \SI{50}{\percent} 
for SOC, and \SI{40}{\percent} for ECEC. In general, the prediction error variance was under-estimated by the 
linear mixed models and over-estimated by the multiple regression models. The best estimates of the prediction 
error variance were obtained by both CLAY linear mixed models, and the ECEC baseline linear regression model.

For CLAY, the increase in the \textit{AVE} was larger when including a kriging step ($\Delta\textit{AVE} = 
\SI{3.9}{\pp}$) than when using more detailed covariates ($\Delta\textit{AVE} = \SI{1.6}{\pp}$). In the case 
of SOC, including a kriging step reduced the \textit{AVE} by \SI{3.2}{\pp}, and for ECEC, both strategies 
increased the \textit{AVE} (\autoref{tab:chap06-cv-stats}).

\subsection{Spatial Prediction}

Both \emph{baseline} and \emph{best performing} linear mixed models captured the same overall pattern of 
spatial variation of the soil properties (\autoref{fig:chap06-kriging}). The main difference is that the 
spatial patterns of the different covariates used to calibrate each model produced different features in the 
prediction maps. For example, the CLAY map produced by the best performing model 
(\autoref{fig:chap06-clay-best-pred}) displays abrupt changes in the predicted values in the north-north-east 
due to the use of the more detailed soil map. Strongly-marked features following the stream network obtained 
through the use of the more detailed DEM are also observed (\autoref{fig:chap06-clay-best-pred} and 
\autoref{fig:chap06-clay-best-var}).

SOC maps (\autoref{fig:chap06-soc-best-pred} and \autoref{fig:chap06-soc-best-var}) show peculiar features in 
the central part of the study area, where predictions reached values as high as \SI{507}{\gram\per\kilo\gram}, 
while the maximum value in the calibration data is \SI{163}{\gram\per\kilo\gram}. The extremely high predicted 
values resulted from the inclusion of the topographic position index derived from the more detailed DEM, using 
a window size of $15 \times 15$~pixels (\texttt{TPI\_10\_15}) to model the deterministic trend. 
\texttt{TPI\_10\_15} values in the point calibration data range from \num{-7} to \SI{6}{\m}, while in the 
central part of the study area they range from \num{12} to \SI{31}{\m}. Thus, feature-space extrapolation 
explains the extremely high predicted values for SOC. Abrupt changes in predicted SOC are also observed at 
locations with low to moderate SOC (\SIrange{40}{80}{\gram\per\kilo\gram}). This is caused by using the more 
detailed land use map.

Predicted ECEC (\autoref{fig:chap06-ecec-base-pred} and \autoref{fig:chap06-ecec-best-pred}) had a large 
dependency on land use and geologic maps. Several features observed in the prediction maps derive from these 
two covariates. The influence of land use is seen in the northern part, while in the western, central, and 
eastern parts the influence of both covariates create an irregular pattern in the spatial distribution of 
ECEC. It is also in these parts that the largest prediction error standard deviations occur, following the 
spatial pattern of the covariates.

\begin{figure}[!ht]
 \centering
 \begin{minipage}[b]{63mm}
  \subcaption{}
  \label{fig:chap06-clay-base-pred}
  \centering
  \includegraphics[width=63mm]{fig/chap06-FIG7a}
 \end{minipage}
 \begin{minipage}[b]{63mm}
  \subcaption{}
  \label{fig:chap06-clay-best-pred}
  \centering
  \includegraphics[width=63mm]{fig/chap06-FIG7d}
 \end{minipage}
 \begin{minipage}[b]{63mm}
  \subcaption{}
  \label{fig:chap06-soc-base-pred}
  \centering
  \includegraphics[width=63mm]{fig/chap06-FIG7b}
 \end{minipage}
 \begin{minipage}[b]{63mm}
  \subcaption{}
  \label{fig:chap06-soc-best-pred}
  \centering
  \includegraphics[width=63mm]{fig/chap06-FIG7e}
 \end{minipage}
 \begin{minipage}[b]{63mm}
  \subcaption{}
  \label{fig:chap06-ecec-base-pred}
  \centering
  \includegraphics[width=63mm]{fig/chap06-FIG7c}
 \end{minipage}
 \begin{minipage}[b]{63mm}
  \subcaption{}
  \label{fig:chap06-ecec-best-pred}
  \centering
  \includegraphics[width=63mm]{fig/chap06-FIG7f}
 \end{minipage}
 \caption[Predicted values for CLAY, SOC and ECEC.]{Predicted values for CLAY (\si{\gram\per\kilo\gram}) (a, 
b), SOC (\si{\gram\per\kilo\gram}) (c, d), and ECEC (\si{\milli\mole\per\kilo\gram}) (e, f) using the 
\emph{baseline} (left) and \emph{best performing} (right) linear mixed models.}
 \label{fig:chap06-kriging}
\end{figure}

\begin{figure}[!ht]
 \centering
 \begin{minipage}[b]{63mm}
  \subcaption{}
  \label{fig:chap06-clay-base-var}
  \centering
  \includegraphics[width=60mm]{fig/chap06-FIG8a}
 \end{minipage}
 \begin{minipage}[b]{63mm}
  \subcaption{}
  \label{fig:chap06-clay-best-var}
  \centering
  \includegraphics[width=60mm]{fig/chap06-FIG8d}
 \end{minipage}
 \begin{minipage}[b]{63mm}
  \subcaption{}
  \label{fig:chap06-soc-base-var}
  \centering
  \includegraphics[width=60mm]{fig/chap06-FIG8b}
 \end{minipage}
 \begin{minipage}[b]{63mm}
  \subcaption{}
  \label{fig:chap06-soc-best-var}
  \centering
  \includegraphics[width=60mm]{fig/chap06-FIG8e}
 \end{minipage}
 \begin{minipage}[b]{63mm}
  \subcaption{}
  \label{fig:chap06-ecec-base-var}
  \centering
  \includegraphics[width=60mm]{fig/chap06-FIG8c}
 \end{minipage}
 \begin{minipage}[b]{63mm}
  \subcaption{}
  \label{fig:chap06-ecec-best-var}
  \centering
  \includegraphics[width=60mm]{fig/chap06-FIG8f}
 \end{minipage}
 \caption[Prediction error standard deviations for CLAY , SOC and ECEC.]{Prediction error standard deviations 
for CLAY (\si{\gram\per\kilo\gram}) (a, b), SOC (\si{\gram\per\kilo\gram}) (c, d), and ECEC 
(\si{\milli\mole\per\kilo\gram}) (e, f) using the \emph{baseline} (left) and \emph{best performing} (right) 
linear mixed models.}
 \label{fig:chap06-kriging-variance}
\end{figure}

The smallest prediction error standard deviations occur at lower elevations, along the three main streams, and 
close to the water outlet in the southern part of the study area. These areas have the highest density of 
point soil observations used to calibrate the models, and the smallest values for all three soil properties. 
While the first determines the accuracy of the EBLUP, the second influences the final accuracy through the 
back-transformation of predicted values.

\section{DISCUSSION}

Our main goal was to evaluate whether investing in more spatially detailed covariates improves the accuracy of 
soil maps. We saw that calibrating the models with more detailed covariates generally has a small to moderate, 
but positive, impact on the predictions. The magnitude of this benefit depends on the magnitude of the 
increase of the spatial detail of the covariate, on the other covariates included in the model, and on the 
soil 
property. However, there seems to be a limit above which the increase of spatial detail has a negative impact 
on the predictions. In the next two subsections we interpret the results from a pedological perspective and 
assess whether the investment in more detailed covariates is worthwhile or if alternatives to improve 
prediction accuracy should be favoured.

\subsection{Spatio-Temporal Controls of Soil Properties}

CLAY was moderately well predicted using less detailed covariates, with small improvement when using the more 
detailed covariates. CLAY was expected to have a strong correlation with topography and parent material. This 
correlation was already considerable when the less detailed DEM and geologic map were used, and improved only 
marginally with the more detailed version. One sensible explanation is that the effective (actual rather than 
theoretical) spatial detail of the two geologic maps was similar, although they had a four-fold difference in 
the size of the minimum legible delineation (see \citet{HenglEtAl2006a} for a discussion on effective 
scale). For the DEM, many studies have already suggested that its resolution may be of secondary importance 
when calculating DEM derivatives for soil mapping \cite{ZhuEtAl2008, BehrensEtAl2010a, MillerEtAl2015}. The 
influence of land use on CLAY is currently small due to reduction of soil erosion in the first decade of the 
\num{21}st century \cite{MiguelEtAl2012, TenCatenEtAl2012b}. A moderate within-field spatial variation may 
exist due to past erosional processes \cite{MouraBueno2012}, but we lack evidence of how well this source of 
variation was captured in the present-time point soil data.

It is worthwhile to consider the influence of the more detailed soil map on predicting CLAY. Due to its 
production process, the more detailed soil map derives a large amount of spatial detail from the geologic map, 
land use map and DEM -- note that the second-best performing model for CLAY included the more detailed 
geologic map instead of the more detailed soil map (\autoref{fig:chap06-model-series}). However, most of the 
additional spatial detail included in the more detailed soil map was probably based on the spatial variation 
of soil texture, because this is a strongly marked soil feature in the area \cite{MiguelEtAl2012}. Soil 
texture 
is one of the most important soil properties used by soil surveyors in the field to identify mapping units 
\cite{Legros2006}. These findings help explain why in the end the more detailed soil map was the most 
beneficial for CLAY instead of the geologic map.

SOC and ECEC were considerably better predicted when more detailed covariates were used. Our expectation that 
SOC and ECEC would have a strong correlation with land use was confirmed by the fact that this covariate 
explained a large amount of the variance and was highly beneficial for improving the predictions. Although the 
available point soil data are limited to the \num{2004}--\num{2011} period, we believe that land use changes 
in the last \num{30}~years \cite{MiguelEtAl2012, TenCatenEtAl2012b} strongly affected SOC and ECEC. Thus, the 
more detailed land use map is likely to have considerably improved model performance because it is up-to-date 
and, possibly, because it has \num{40}~times more spatial detail than its less detailed version. Despite the 
fact that the two land use maps used in this study were from different time periods, which confounds the 
analysis, the results obtained indicate that a more detailed land use map improves the prediction of SOC. For 
example, the areas used for crop agriculture, which are well known for having lower SOC and ECEC 
\cite{Menezes2008, MouraBueno2012}, are not depicted in the less detailed land use map.

We expected SOC to have a stronger correlation with the DEM than with the geologic map due to its strong 
dependence on erosion, but we observed the contrary. This result may be partially explained by the fact that 
there is a strong relation between geology and topography in the study area \cite{Sartori2009}. Due to its 
production process \cite{MacielFilho1990}, the geologic maps can be interpreted as an aggregated version of a 
DEM. A second sensible explanation is that the effect of erosion on SOC is not that large because erosion was 
considerably reduced in the last decade \cite{MiguelEtAl2012, TenCatenEtAl2012b}. A last possible explanation, 
which integrates the previous two, is the existence of a spatial relation between SOC and CLAY, the last being 
strongly correlated with parent material. These relations help explain why the more detailed DEM was almost as 
beneficial as the more detailed geologic map for SOC predictions. In the case of ECEC, our expectation of a 
strong dependency on a more detailed geologic map for producing more accurate predictions was confirmed.

The observed benefit of the more detailed geologic map and DEM for making more accurate CLAY, SOC, and ECEC 
predictions suggests that these soil properties are spatially related in the study area. We also hypothesize 
that the complexity of current land use makes it difficult to achieve SOC and ECEC models with performances 
comparable to CLAY. One important source of variation in forested areas is its use for animal grazing 
\cite{SamuelRosaEtAl2011a}. This influences nutrient cycling and soil nutrient availability 
\cite{SchramaEtAl2013}. Current remote sensing technology is unable to capture the data needed to proxy the 
environmental conditions created by these processes.

\subsection{Using More Detailed Covariates}

More detailed covariates are usually expected to improve predictions in soil mapping \cite{CavazziEtAl2013, 
MaynardEtAl2014}. However, deciding whether to invest or not in more detailed covariates requires careful 
thinking and depends on case-specific elements. We generally saw improvement in the predictions in our study, 
but the improvement was not large and may not outweigh the costs. Also, the models calibrated with the more 
detailed versions of all covariates were not the best performing models. Using more detailed satellite images 
and land use maps degraded CLAY predictions. Although the more detailed soil map had the largest benefit for 
CLAY, it may be too costly and impractical since its production usually requires having available more 
detailed 
versions of all other covariates. For SOC and ECEC, simply using a more detailed land use map resulted in 
considerably more accurate predictions. However, the superior performance may not outweigh the extra costs 
because producing a more detailed land use map usually requires up-to-date field observations and satellite 
images. Thus, the decision to adopt a more detailed covariate for soil mapping will ultimately depend on a 
trade-off between the increased accuracy and the extra budget required. It may also depend on other potential 
applications of the covariates, but this is not our concern here.

One interesting observation is that if a less detailed covariate yields poor predictions, its more detailed 
version has the potential to produce larger improvement in model performance. However, this is only a 
potential, not a guarantee. For instance, \citet{EldeiryEtAl2008} were not able to increase the $R^2 = 
0.31$ of linear regression models of soil salinity by more than \num{0.07}~points using \num{7.5}~times more 
detailed satellite images. On the other hand, model performance is likely to be hardly improved using more 
detailed covariates if their less detailed version has already produced accurate predictions. This agrees with 
findings by \citet{ThompsonEtAl2001} and \citet{KimEtAl2014}.

We also observed that the predictions can be degraded when using the more detailed version of covariates. In 
our study, this happened with the satellite image (all three soil properties), land use map (CLAY) and soil 
map (SOC and ECEC). A (small) benefit was observed only when these covariates were used along with the more 
detailed version of other covariates. As pointed out above, such a small benefit may not outweigh the increase 
in mapping costs. The trade-off between reducing model performance and being beneficial seems to depend on how 
much more spatial detail a covariate will have and on its correlation with the soil property. For example, the 
land use map was strongly correlated with SOC and ECEC, but not with CLAY, and its more detailed version had 
\num{40}~times more spatial detail. It helped improve SOC and ECEC predictions, but degraded CLAY predictions, 
resulting in only a small improvement when used along with the more detailed satellite image and geologic map.

If the influence of a more detailed covariate depends on the increase of spatial detail, then the priority 
should be to improve the spatial detail of the most beneficial covariate. This requires solid subject area 
knowledge because empirical evidence from the \emph{baseline} model may be insufficient. The most beneficial 
covariate is not necessarily that which explained the largest part of the variance in the \emph{baseline} 
model (see \autoref{tab:chap06-drop}). This occurs because increasing the spatial detail reduces the 
correlation between the covariate and the soil property. And also because there is little room to improve a 
correlation that is already high in the \emph{baseline} model. \citet{CavazziEtAl2013} suggest that the 
more detailed covariate has an excess of detail, a \q{noise} that degrades the predictions. This could explain 
the results for \texttt{sat}: higher resolution images can resolve smaller objects (e.g. individual plants) 
whose spectral behaviours are highly variable, adding noise to the \texttt{sat}-soil property correlation; on 
the other hand, lower resolution images capture collections of objects, and thus their variation is smoothed 
out in the pixel, reducing noise.

According to information theory one should optimize (maximize) the correlation between the point soil data and 
the covariates. This was described elsewhere as matching the \q{phenomenon scale} (the spatial pattern of the 
soil property) with the \q{analysis scale} (the spatial pattern of the covariates) \cite{DunganEtAl2002, 
MillerEtAl2014}. Finding the \q{optimum} requires evaluating the strength of the correlation using covariates 
with different levels of spatial detail \cite{DragutEtAl2009, CavazziEtAl2013, MillerEtAl2015}. Our results 
show that this approach may be too costly and impractical. Since modern soil mapping techniques explore only 
the empirical relation among environmental conditions and soil properties \cite{Grunwald2009}, the \q{optimum}
is a \q{conditional optimum} -- conditional on the point soil data available. It does not necessarily mean 
that the most accurate predictions will be made, but only that there is a level of spatial detail at which the 
correlation between the covariate and the point soil data is at a maximum. We suggest that instead more 
comprehensive approaches should be used to explore the full potential of the available covariates (see 
\citet{BehrensEtAl2010a} and \citet{MillerEtAl2015} for examples).

Finally, one must still judge whether the potential improvement in predictions is sufficient given the extra 
costs involved with using more detailed covariates. If the extra budget is spent on deriving more detailed 
covariates, we suggest that it may be better to substantially improve the detail of a less influential
covariate than marginally increase the detail of the most influential covariate. However, other means to spend 
the extra budget should be considered. For instance, it may be more efficient to concentrate on obtaining more 
soil observations. These may focus on better capturing the short range spatial variation \cite{BrusEtAl2007a} 
or improving the representation of the feature space to avoid undesirable extrapolations 
\cite{MinasnyEtAl2006b}.

\section{CONCLUSIONS}

This study has shown that:

\begin{enumerate}[label = (\Roman*)]
 \item Using more detailed covariates results in only a modest increase in the prediction accuracy of linear 
 prediction models;
 
 \item A more detailed covariate has a greater potential to improve prediction accuracy when the soil property
 is poorly predicted by its less detailed version;
 
 \item The impact on prediction accuracy when using the more detailed version of a less important covariate 
 may depend on which other covariates are included in the model;
 
 \item Choosing whether or not to invest in more detailed covariates depends on the strength of the 
 relationship between the covariates and the soil property being modelled, and on the relative difference 
 between the less detailed and the more detailed versions of the covariates.
\end{enumerate}
 % 6. Do more detailed covariates deliver more accurate soil maps?
\artigotrue
\chapter{OPTIMIZATION OF SAMPLE CONFIGURATIONS FOR SPATIAL TREND ESTIMATION FOR SOIL MAPPING}
\label{chap:chap07}

\def\ptkeys{}

\begin{chapterabstract}{brazilian}{\ptkeys}

\end{chapterabstract}

\def\enkeys{}
  
\begin{chapterabstract}{english}{\enkeys}

\end{chapterabstract}

\formatchapter

\section{INTRODUCTION}
\label{sec:chap07-intro}

\titlenote{This chapter is based on the study \textit{spsann -- optimization of sample patterns using spatial 
simulated annealing}, presented at the EGU General Assembly 2015 \cite{Samuel-RosaEtAl2015a}, and 
\textit{Optimization of sample configurations for spatial trend estimation}, presented at Pedometrics 2015 
\cite{Samuel-RosaEtAl2015d}. Also collaborated in the preparation of this document: Dick J. Brus (Alterra, 
Wageningen University and Research Centre, the Netherlands), Gerard B. M. Heuvelink (ISRIC -- World Soil 
Information), Gustavo M. Vasques (Embrapa Soils, Brazil), and Lúcia Helena Cunha dos Anjos (Universidade 
Federal Rural do Rio de Janeiro, Brazil).}

Modern soil mapping is based on using a model of spatial variation composed of two terms,

\begin{equation}
 Y(\boldsymbol{s}) = m(\boldsymbol{s}) + e(\boldsymbol{s}).
\end{equation}\label{eq:chap07-lmm}

\def\footgerard{\footnote{Gerard Heuvelink shared the same opinion during his Richard Webster Medal speech at 
the conference of the Pedometrics Commission of the IUSS, which took place from 14--18 September 2015, in 
Córdoba, Spain.}}

\nointent The first term in the right-hand size in \autoref{eq:chap07-standard-model} is the spatial trend, 
which corresponds to the spatial variation of the soil property $Y(\boldsymbol{s})$ that is explained 
deterministically using spatially exhaustive covariates; the remaining spatial variation of 
$Y(\boldsymbol{s})$ is explained stochastically with the second term \cite{Cressie1993}. Soil scientists 
devoted all their attention to $m(\boldsymbol{s})$ for more than a century \cite{Jenny1961, Florinsky2012}. 
Post-war technological developments in the fields of mathematics, statistics, and informatics, made many soil 
scientists turn their focus to $e(\boldsymbol{s})$ \cite{WebsterEtAl1990}. Recent developments in remote 
sensing and machine-learning algorithms made those soil scientists shift their attention back to 
$m(\boldsymbol{s})$ \cite{MooreEtAl1993} -- but without forgetting of $e(\boldsymbol{s})$ \cite{OdehEtAl1994} 
--, which now usually explains a considerably large proportion of the variation of $Y(\boldsymbol{s})$ 
compared to $e(\boldsymbol{s})$\footgerard. Besides, it is in $m(\boldsymbol{s})$ where we can incorporate most 
of our pedological knowledge \cite{Lark2012}.

Recent studies have shown that using more detailed covariates or more complex machine-learning algorithms can 
deliver more accurate soil maps, but the increase in prediction performance may be modest 
(\cite{Samuel-RosaEtAl2015}) and largely depends on the calibration data \cite{HeungEtAl2016}. Limited to the 
currently available covariates and machine-learning algorithms, and to the existing pedological knowledge, one 
of the major operational issue that needs to be solved in any soil mapping project is how to design an 
efficient spatial sample to estimate $m(\boldsymbol{s})$. The sampling method most commonly used to solve this 
problem is the \emph{conditioned Latin hypercube sampling} (CLHS). The CLHS was developed by Budiman Minasny 
and Alex McBratney at the University of Sydney in 2005, using an idea borrowed from the Latin hypercube 
sampling \cite{McKayEtAl1979, MinasnyEtAl2006b}. The popularity of the CLHS is due to its non\-/probabilistic 
nature, seen as a link with the sampling strategies used in \q{traditional soil survey}, easiness to 
implement, and the high flexibility which makes the addition of new features simple \cite{MinasnyEtAl2010a, 
RoudierEtAl2012, MulderEtAl2013, CarvalhoJuniorEtAl2014, CliffordEtAl2014}.

The CLHS is a heuristic strategy of creating spatial samples that aim at three objectives: ($\mathcal{O}_1$) 
uniform coverage of the marginal distribution of numeric covariates (continuous and discrete data, e.g. 
elevation, slope, etc.), ($\mathcal{O}_2$) proportional sample sizes for the classes of factor covariates 
(binary, categorical, and ordinal data, e.g. geology, land use, etc.), and ($\mathcal{O}_3$) reproduction of 
the linear correlation of numeric covariates. The main idea was that if a spatial sample reproduces the 
marginal distribution of the numeric and factor covariates, as well as the correlation matrix of the numeric 
covariates, it will approximately cover the multivariate distribution of the covariates -- this should put us 
closer to identifying the \q{true} spatial trend if we are (or assume to be) ignorant about its form.

Some critiques of the CLHS appeared in the literature since it was first published. Most of them focused on 
operational difficulties encountered in the field. For example, \citet{CambuleEtAl2013} argued that the 
CLHS is impractical in poorly-accessible areas, but \citet{RoudierEtAl2012} and 
\citet{MulderEtAl2013} showed that this is just a matter of how the algorithm is implemented. And 
\citet{CliffordEtAl2014} presented an algorithm for selecting an alternative sampling point when a CLHS 
sample point is inaccessible. Only recently soil scientists started paying more attention to the theoretical 
and algorithmic aspects of the CLHS. \citet{MinasnyEtAl2010a} demonstrated that, given an assumed known 
linear spatial trend, the CLHS is suboptimal. \citet{CliffordEtAl2014} questioned the importance of 
meeting the third objective ($\mathcal{O}_3$), as well as the mathematical approach used to find a solution 
for all three objectives jointly (see below). Finally, \citet{Brus2015} proposed an alternative method 
for selecting Latin hypercube samples with known inclusion probabilities so that these samples can also be 
used for design-based inference.

Our objective is to propose conceptual and algorithmic improvements on the CLHS, all of which we describe in 
the next section. We then evaluate if the proposed improvements result in a more accurate representation of 
the feature space and spatial predictions.

\section{PROPOSED IMPROVEMENTS}

\subsection{Defining the Marginal Sampling Strata}

Given a \emph{numeric} covariate, the CLHS uses the sample size $n$ to define the number of marginal 
sampling strata $c$, i.e. $c = n$, and the interpolated sample quantiles to define the breakpoints of the 
$c$ marginal sampling strata. The first objective of the CLHS ($\mathcal{O}_1$) is to have exactly one 
sample point falling in each marginal sampling strata. However, depending on the level of discretization of 
the covariate values, the CLHS may produce replicated breakpoints in the regions with a relatively high 
frequency of covariate values. For example, given a sample size of $n = 5$ and a covariate $\boldsymbol{a}$ 
with (ordered integer) values $\boldsymbol{a} = (1, 1, 1, 1, 2, 2, 3, 3, 4, 5, 8, 9, 9, 9, 9)$, the lower and 
upper boundaries of the marginal sampling strata are $\boldsymbol{a}_{mss} = (1.0, 1.0, 2.6, 4.4, 9.0, 9.0)$. 
Because the marginal sampling strata in which a sample point $b_i$ falls is evaluated using the indicator 
function

\begin{equation*}
 b_{sol_i} = 
 \begin{cases}
  1, & \text{if}\ a_{mss_j} \leq b_i \leq a_{mss_{j + 1}}\ \text{and}\ j = 1 \\ 
  1, & \text{if}\ a_{mss_j} < b_i \leq a_{mss_{j + 1}}\ \text{and}\ j > 1 \\ 
  0, & \text{otherwise}
 \end{cases}
\end{equation*}

\noindent where $i = 1, 2, \ldots, n$, and $j = 1, 2, \ldots, c$, the first and last marginal sampling strata 
of $\boldsymbol{a}$ will be empty, and the respective $n‘ = 2$ sample points will be allocated among the other 
three marginal sampling strata, with the set of allocation solutions $\boldsymbol{b}_{sol} = \{(0, 2, 1, 2, 
0), (0, 1, 2, 2, 0), (0, 2, 2, 1, 0)\}$. Ergo, the CLHS will be unable to find the globally optimum allocation 
solution $\boldsymbol{b}_{sol} = (1, 1, 1, 1, 1)$.

We propose defining the marginal sampling strata using only the unique values of the sample quantiles 
estimated with a discontinuous function \cite{HyndmanEtAl1996}. Accordingly, for our example, 
$\boldsymbol{a}_{mss} = (1, 2, 4, 9)$. The number of sample points that should fall in each marginal sampling 
strata is directly proportional to the number of sampling units (grids cells of a raster image) in that 
stratum of the covariate. For $\boldsymbol{a}$, this is $\boldsymbol{b}_{sol} = (2, 1, 2)$. The direct 
consequence of this modifications is that, given a set of $p$ covariates, each of them will potentially have a 
different number of (quasi-equal-size) marginal sampling strata, i.e. $c_i \leq n$, where $i = 1, 2, \ldots, 
p$. This will ultimately depend on the shape of their empirical frequency distribution, on the level of 
discretization of the covariate values, and on the sample size $n$.

\subsection{Measuring the Association/Correlation Between Covariates}

Two of the objectives of the CLHS ($\mathcal{O}_1$ and $\mathcal{O}_3$) are concerned with \emph{numeric} 
covariates, while only one ($\mathcal{O}_2$) focuses on \emph{factor} covariates. $\mathcal{O}_1$ and 
$\mathcal{O}_2$ are mathematically equivalent -- they aim at the coverage of the marginal distribution of the 
numeric and factor covariates --, and $\mathcal{O}_3$ measures the similarity between the population and 
sample correlation matrices of the numeric covariates as estimated with the Pearson`s $r$. The CLHS ignores 
the association among factor covariates, as well as of those with the numeric covariates. This means that the 
CLHS gives more importance to numeric covariates. Such a bias cannot be corrected by simply attributing 
different \emph{weights} to each objective (see below).

We propose to replace the Pearson`s $r$ with the Cramér`s $v$

\begin{equation}
 v =  \sqrt{\frac{\chi^2 / n}{min(ncol - 1, nrow - 1)}},
\end{equation}\label{eq:chap07-cramer}

\noindent where $nrow$ and $ncol$ are the number of rows and columns of the bivariate contingency table, 
$n$ is the sample size, and $\chi^2$ is the chi-squared statistic

\begin{equation}
 \chi^2 = \sum_{i = 1}^{nrow}\sum_{j=1}^{ncol}\frac{(O_{ij} - E_{ij})^2}{E_{ij}},
\end{equation}\label{eq:chap07-chi-squared}

\noindent where $O_{ij}$ and $E_{ij}$ are the observed and expected frequency, respectively, the marginal 
proportions of $O$ being the maximum likelihood estimates of the marginal proportions of $E$ 
\cite{Cramer1946, Agresti2002}. The Cramér`s $v$ is a measure of association between factor covariates that 
ranges from $0$ to $+1$: the closer to $+1$, the larger the association between two factor covariates. 
Accordingly, the only requirement for using the Cramér`s $v$ -- instead of the Pearson`s $r$ -- is that any 
numeric covariate be transformed into a factor covariate, with the factor levels defined using the marginal 
sampling strata. One could still use the Pearson`s $r$ when all covariates are numeric because computing the 
Cramér`s $v$ is more computationally demanding.

\subsection{Aggregating the Objectives}

Sampling for spatial trend estimation is a \emph{multi-objective combinatorial optimization problem} (MOCOP): 
we have to find a spatial sample that meets a list of objectives among an almost infinite set of possible 
spatial samples. An important step for solving a MOCOP is to define each objective as a function, i.e. an 
\emph{objective function} $f_i$ \cite{Arora2011}. An $f_i$ associates a numerical value with each spatial 
sample as a function only of the values of the $p$ covariates used to describe the spatial domain -- also 
known as \emph{design variables} \cite{Arora2011} -- at the $n$ sample points. The lower the objective 
function value, the closer the spatial sample is to meeting the respective objective. Thus, when solving a 
MOCOP, one aims at minimizing the vector of $k$ objective functions \cite{Arora2011}

\begin{equation}
 \boldsymbol{f}(\boldsymbol{X}) = (f_1(\boldsymbol{X}), f_2(\boldsymbol{X}), \ldots, f_k(\boldsymbol{X})),
\end{equation}

\noindent where $\boldsymbol{X}$ is the design matrix, a $n \times p$ matrix subject to the implicit 
constraints imposed by the finiteness of the spatial domain and discreteness of the $p$ design variables. 
These implicit constraints define the set of values that can be assigned jointly to the design variables, i.e. 
the $p$-dimensional \emph{feasible design space} $\mathcal{S}$, which, in turn, defines the set of numerical 
values that can be returned by the objective functions, i.e. the $k$-dimensional \emph{feasible objective 
space} $\mathcal{Z}$ \cite{MarlerEtAl2004}.

Ideally, there is a traceable unique \emph{point cloud} $\boldsymbol{X}^*$ (i.e. a spatial sample with the 
values of the covariates at its sample points) that minimizes all objective functions simultaneously 
\cite{MarlerEtAl2009}. However, in practice such a unique point cloud seldom exists, and if it exists it is 
hard to find. In most cases there is a large set of optima point clouds that map onto a set of optima points 
on $\mathcal{Z}$ because, for example, multiple point clouds can return the very same objective function value 
\cite{Arora2011}. The set of optima point clouds is commonly defined using the concept of \emph{Pareto 
optimality} \cite{MarlerEtAl2004}: a point cloud $\boldsymbol{X}^*$ in $\mathcal{S}$ is Pareto optimum if and 
only if there is no other point cloud $\boldsymbol{X}$ in $\mathcal{S}$ that decreases the value of at least 
one objective function without increasing the value of another objective function.

A reasonable strategy to find a single optimum solution is to aggregate the objective functions into a single 
\emph{utility function} $U$ \cite{MarlerEtAl2005}. The most common aggregation method is the \emph{weighted 
sum} method, which is used in the CLHS. It employs weights to incorporate the \emph{a priori} preferences of 
the user, their relative values reflecting the importance of each objective function \cite{MarlerEtAl2009}. 
Thus, the MOCOP boils down to minimizing the convex combination of objective functions

\begin{equation}
 U = \sum_{i=1}^{k} w_i f_i(\boldsymbol{X}),
\end{equation}\label{eq:chap07-utility}

\noindent which means that the weights $w_i$ are constrained to $w_i > 0$ and $\sum_{i=1}^{k} w_i = 1$ 
\cite{MarlerEtAl2005, MarlerEtAl2009}. An important requirement of the weighted sum method is that the 
objective functions be scaled to the same approximate range of values so that any potential numerical 
dominance can be eliminated or minimized, and the weights can play the desired role \cite{MarlerEtAl2005, 
MarlerEtAl2009}. 

There are several methods to scale the objective functions \cite{MarlerEtAl2005}. The Fortran source code of 
the CLHS shows that, although not mentioned in the original paper, the CLHS scales $\mathcal{O}_1$ and 
$\mathcal{O}_3$ using the \emph{upper-bound approach}, $f_i'' =f_i(\boldsymbol{X}) / f_i^{max}$, where 
$f^{max}_{\mathcal{O}_1} = n \times p^{num}$ and $f^{max}_{\mathcal{O}_3} = 0.5p^{num^2} + p^{num}$, 
$p^{num}$ being the number of numerical covariates. $\boldsymbol{f}^{max}$ is a rough estimate of the 
single worst solution for $\mathcal{O}_1$ and $\mathcal{O}_3$, called the \emph{nadir point cloud} 
\cite{MarlerEtAl2004}. Thus, this transformation results in a non-dimensional objective function with an upper 
limit around 1, and its use relies on the fact that, by definition, the three objective functions yield 
objective function values of very different orders of magnitude: $\mathcal{O}_1$ > $\mathcal{O}_3$ > 
$\mathcal{O}_2$. This is because $\mathcal{O}_1$ uses the number of sample points per strata (0--n), while 
$\mathcal{O}_3$ uses the linear correlation coefficient (-1--1), and $\mathcal{O}_2$ uses the proportion of 
sample points per strata (0--1).

We believe that the \emph{upper-bound approach} is insufficient for a proper scaling of the objective 
functions because $\boldsymbol{f}^{max}$ usually is unattainable -- i.e. it does not correspond to any point 
cloud in $\mathcal{S}$, and/or is far from the Pareto optimum set \cite{MarlerEtAl2004}. Defining 
$\boldsymbol{f}^{max}$ as the median of the objective functions over multiple spatial samples generated by 
simple random sampling \cite{CliffordEtAl2014} is a suboptimal strategy because it only ensures that the 
objective functions will have similar orders of magnitude at the beginning of the optimization, which might 
have a negligible influence in the definition of $\mathcal{Z}$ \cite{MarlerEtAl2005}. Besides, provided the 
optimization algorithm is well designed, the starting point should not influence the solution of the MOCOP 
(see below).

We propose using a more robust approach, i.e. the \emph{upper-lower bound approach},

\begin{equation}
 f_i'' = \frac{f_i(\boldsymbol{X}) - f_i^{\circ}}{f_i^{max} - f_i^{\circ}}
\end{equation}

\nointent where $f_i''$ is the $i$th non-dimensional, scaled objective function constrained between zero 
and one \cite{MarlerEtAl2005}. Because of the above-mentioned problems regarding the definition of 
$\boldsymbol{f}^{max}$, it is more appropriate to use the \emph{Pareto maximum}, $f_i^{max} = max_{1 \leq j 
\leq k} f_ i(\boldsymbol{X}_j^*)$, where $\boldsymbol{X}_j^*$ is the point cloud that minimizes the $j$th 
objective function \cite{MarlerEtAl2005}. In practice, we find the optimum point cloud for each objective 
function individually, and then calculate the objective function value of every other objective function. The 
Pareto maximum of a given objective function is the largest absolute maximum value obtained for that objective 
function. The same applies for $f_i^{\circ}$, the \emph{utopia point} -- the single best solution for the 
$i$th objective function, which exists in the objective space, but usually is unattainable, i.e. it does not 
correspond to any point cloud in $\mathcal{S}$ \cite{Arora2011} -- which is replaced with the Pareto 
minimum. The drawback of this approach is the extra time needed to optimize the $k$ objective functions 
individually.

\subsection{Resulting Problem Definition}

Given the proposed modifications, the problem of sampling for spatial trend estimation for soil mapping is 
redefined using two objective functions,

\begin{equation}
 \text{CORR} = \sum_{i=1}^{p}\sum_{j=1}^{p}|\varphi_{ij} - v_{ij}|,
\end{equation}\label{eq:chap07-corr}

\noindent where $\varphi_{ij}$ and $v_{ij}$ are the population and sample associations (or correlations in 
case all covariates are numeric) at the $i$th row and $j$th column of the $p$-dimensional population and 
sample association (or correlation) matrices, and

\begin{equation}
 \text{DIST} = \sum_{i=1}^{p}\sum_{j=1}^{c_i} |\pi_{ij} - \gamma_{ij}|,
\end{equation}\label{eq:chap07-dist}

\noindent where $\pi_{ij}$ and $\gamma_{ij}$ are the proportion of sample and population points that fall 
in the $j$th class (or marginal sampling strata) of the $i$th covariate, $c_i$ being the number of classes of 
the $i$th covariate. With these two objective functions, we define an utility function $U$ as in 
\autoref{eq:chap07-utility} aiming at a spatial sample that reproduces an 
\textbf{A}ssociation/\textbf{C}orrelation measure and the marginal \textbf{D}istribution of the 
\textbf{C}ovariates,

\begin{equation}
 \text{ACDC} = w_1\text{CORR} + w_2 \text{DIST},
\end{equation}\label{eq:chap07-acdc}

\noindent with $w_1 = w_2 = 0.5$ in the general setting.

\section{CASE STUDY}

We developed a case study to evaluate the proposed improvements and compare them with the original CLHS. It 
was based on using synthetic data derived from a real-world study case \cite{Samuel-RosaEtAl2015}. The study 
site is a small catchment of about \SI{2000}{\hectare} located on the southern edge of the plateau of the 
Paraná Geologic Province, Rio Grande do Sul, Brazil. The real-world dataset contains $n = 350$ point soil 
observations of the topsoil, and includes several soil properties, but only bulk density data 
(BUDE,~\si{\mega\gram\per\metre\cubic) was used ($n = 282$). The dataset also includes several covariates 
derived from area-class soil maps, digital elevation models, geological maps, land use maps, and satellite 
images. All processing steps used to derive the covariates were described by \citet{Samuel-RosaEtAl2015}.

\subsection{Soil Data Generating Process}

In an ideal world, we would create $\mathcal{R} \geq 100$ spatial samples of $\mathcal{N} \geq 2$ sizes 
with each of the $\mathcal{A} \geq 2$ algorithms that we want to compare. Then we would go to the field, 
sample the soil, and measure a property to construct $\mathcal{D} = \mathcal{R} \times \mathcal{N} \times 
\mathcal{A}$ calibration datasets. The same property would be measured at a fixed set of probabilistically 
selected validation sites. Each calibration dataset would be used to calibrate a model, with which we would 
predict at the validation sites. The $\mathcal{A}$ sampling algorithms would then be compared on how well 
they performed, for each of the $\mathcal{N}$ sizes, using the confidence interval of a prediction error 
statistic over all $\mathcal{R}$ spatial samples. Here, the random selection of  spatial samples would be the 
\emph{source of variation} \cite{deGruijterEtAl1990}.

In the real world\dots Because resources are limited, we decided to create only $\mathcal{R} = 1$ spatial 
sample with $\mathcal{N} = 3$ sizes with each of the $\mathcal{A} = 4$ sampling algorithms that we want to 
compare (CORR, DIST, ACDC, and CLHS). The variation had to come from another source: we chose it to be the 
soil property data. We did so using unconditional sequential Gaussian simulation \cite{Goovaerts2001, 
Pebesma2004}. To start, we defined a theoretical (or super-population) model, our \emph{soil data generating 
process}. To be as close to reality as possible, the soil data generating process was defined empirically 
calibrating a (non)linear mixed model to BUDE. The main calibration steps are as follows \cite{Breiman2001, 
LiawEtAl2002, DiggleEtAl2007, Lark2012}:

\begin{enumerate}
 \item Random regression forest: grow $n_{\text{trees}} = 500$ regression trees with a maximum terminal node 
 size of $n_{\text{node size}} = 5$ points, each tree grown using $n = 282$ calibration points randomly 
 selected with replacement from the set of $n = 282$ point soil observations (about $n_{\text{in-bag}} = 178$ 
 unique point soil observations), and $p_{\text{in-bag}} = 4$ covariates randomly selected at each split out 
 of a set of $p = 12$ covariates selected as in \citet{Samuel-RosaEtAl2015}.
 
 \item Out-of-bag predictions: use each of the $n_{\text{trees}} = 500$ regression trees from step (1) to 
 predict BUDE at the  point soil observations not included (out-of-bag) in the respective calibration dataset 
 (about $n_{\text{out-of-bag}} = 104$ point soil observations), and compute the average of the predicted BUDE 
 at each point soil observation (about 184 predicted values for each out-of-bag point).
 
 \item Linear mixed model: assume that the average of the out-of-bag predictions from step (2) are linearly 
 related to BUDE and present insignificant conditional bias, and use them as a covariate in the fixed effects 
 of a linear mixed model (LMM), the random effects modelled using the Whittle-Matérn model, all parameters 
 being estimated by Gaussian restricted maximum likelihood (REML).
\end{enumerate}

The parameters of the LMM are the coefficients $\beta_0$ and $\beta_1$ of the linear trend, which correct 
any linear bias in the random regression forest out-of-bag predictions \cite{LiawEtAl2002}, and the nugget 
($\tau^2$), sill ($\sigma^2$), and range ($\alpha$) of the Whittle-Matérn model. The shape parameter 
($\nu$) of the Whittle-Matérn model was defined separately, by choosing from a set of discrete values 
$\nu = (0.5, 1.0, 2.0, 4.0, 8.0)$ based on the resulting profile likelihood for $\nu$ and maximized restricted 
log-likelihood, and on the computing time \cite{Stein1999, DiggleEtAl2007}. The fitted LMM 
($\beta_0 = \SI{13.35}{\mega\gram\per\cubic\metre}$, $\beta_1 = 0.91$, 
$\tau^2 = \SI{349.51}{\mega\gram\per\metre\tothe{6}}$, $\sigma^2 = \SI{97.24}{\mega\gram\per\metre\tothe{6}}$, 
$\alpha = \SI{210.99}{\metre}$, $\nu = 2.0$) explained \num{38} and \SI{18}{\percent} of the sample variance of 
BUDE with $m(\boldsymbol{s})$ and $e(\boldsymbol{s})$, respectively (\autoref{fig:chap07-bude-vario}).

\begin{figure}[!ht]
 \centering
 \includegraphics[width=90mm]{fig/chap07-bude-vario}
 \caption[Variogram model representing the stochastic term of the linear mixed model fitted to the soil bulk
 density data.]{Variogram model (red line) representing the stochastic term of the linear mixed model fitted
 to the soil bulk density data. Exponential spacings are used to depict the sample variogram (blue dots) along 
 with  the number of point-pairs in each variogram bin.}
 \label{fig:chap07-bude-vario}
\end{figure}

With the random regression forest and the LMM at hand, we simulated $\mathcal{R} = 1000$ equiprobable 
realizations of BUDE at a fine grid of \num{\sim800000} points covering the entire study area. It is using 
this uncertain reality that we tested our $\mathcal{A} = 4$ sampling algorithms.

\subsection{Sampling and Model Calibration}

We then sampled from each realization using the optimized spatial sample configurations. Because we wanted to 
check the effect of the sample size, each algorithms was run using $$\mathcal{N} = 3$$ sample sizes of $n = 
(100, 200, 400)$, which amounts to $\mathcal{D} = 3000}$ calibration datasets for each algorithm. Each 
calibration dataset was used to calibrate a random regression forest using the same covariates as used in 
simulating the fields.

\section{RESULTS AND DISCUSSION}

\begin{figure}[!ht]
 \centering
 \includegraphics[width=\textwidth]{fig/chap07-energy_corr_dist_acdc_clhs}
 \caption[Objective function values during the optimization of three sample configurations using four sampling 
 algorithms.]{Objective function values during the optimization of sample configurations of size 
 $n = (100, 200, 400$) using sampling algorithms CORR, DIST, ACDC, and CLHS against the number of Markov 
 chains of length $n$.}
 \label{fig:chap07-energy-all}
\end{figure}

\begin{figure}[!ht]
 \centering
 \includegraphics[width=\textwidth]{fig/chap07-energy_acdc_clhs}
 \caption[Region of the feasible objective space explored by pairs of objective functions that compose CLHS 
 and ACDC.]{Region of the feasible objective space $\mathcal{Z}$ explored by pairs of objective functions (x 
 vs  y) that compose CLHS ($\mathcal{O}_1$, $\mathcal{O}_2$, and $\mathcal{O}_3$) and ACDC (CORR and DIST)  
 during the optimization of a sample configuration of size $n = 100$ using $n_{\text{chains}} = 500$ Markov 
 chains of length $n$.}
 \label{fig:chap07-energy-acdc-clhs}
\end{figure}

\begin{figure}[!ht]
 \centering
 \includegraphics[width=\textwidth]{fig/chap07-points_corr_dist_acdc_clhs}
 \caption[Sample configurations optimized using four sampling algorithms.]{Sample configurations of size 
 $n = (100, 200, 400)$ optimized using sampling algorithms CORR, DIST,  ACDC, and CLHS superimposing the
 drainage network.}
 \label{fig:chap07-points}
\end{figure}

\section{FINAL CONSIDERATIONS}


 % 7. Spatial point pattern analysis of soil survey sampling locations
\artigotrue
\chapter{SAMPLING FOR SOIL MAPPING IN \emph{TERRA INCOGNITA}}
\label{chap:chap08}

\def\ptkeys{}

\begin{chapterabstract}{brazilian}{\ptkeys}

\end{chapterabstract}

\def\enkeys{}
  
\begin{chapterabstract}{english}{\enkeys}

\end{chapterabstract}

\formatchapter

\section{INTRODUCTION}

\titlenote{This chapter is based on the study \textit{spsann -- optimization of sample patterns using spatial 
simulated annealing}, presented at the EGU General Assembly 2015 \cite{Samuel-RosaEtAl2015a}, and 
\textit{Optimization of sample configurations for variogram estimation}, presented at Pedometrics 2015 
\cite{Samuel-RosaEtAl2015c}. Also collaborated in the preparation of this document: Gerard B. M. Heuvelink 
(ISRIC -- World Soil Information), Dick J. Brus (Alterra, Wageningen University and Research Centre, the 
Netherlands), Gustavo M. Vasques (Embrapa Soils, Brazil), and Lúcia Helena Cunha dos Anjos (Universidade 
Federal Rural do Rio de Janeiro, Brazil).}

The success of soil mapping largely depends on the sampling data because the last are used to 1) estimate the 
spatial trend, 2) estimate the variogram of the residuals, and 3) make spatial predictions by calculating 
conditional distributions. A poor sampling strategy is likely to deliver a poor model and large prediction 
errors, resulting in a waste of financial resources, staff and time \cite{vanGroenigenEtAl1999,  
deGruijterEtAl2006, LanEtAl2010}. This is undesirable because sampling already is the largest contributor to 
the costs of soil mapping \cite{WebsterEtAl1990, vanGroenigenEtAl1999, KempenEtAl2012}.

The focus of this study is on the optimization of spatial sample configurations for soil mapping. We explore a 
scenario in which a) multiple soil properties have to be mapped, b) we are ignorant about the form of the model 
of spatial variation, and c) the operational constraints limit sampling to a single phase. The objective is to 
evaluate the ability of different sampling configuration types and sample sizes to capture the true form of the 
model of spatial variation and make accurate predictions. We also quantify the gain in prediction accuracy by 
combining popular sampling methods in a multi-objective optimization problem. This study addresses a problem 
that many soil scientists involved in soil mapping projects face: how to come up with a spatial sample 
configuration that is effective and robust in situations where we know very little?

\section{PURPOSIVE SAMPLING}

\emph{Purposive sampling} is the non-probability sampling mode by which the sampling locations are selected 
intentionally as to satisfy an \textit{a priori} criterion. This criterion is commonly defined based on the 
model that will be used to infer the structure of spatial variation of a soil property $Y(\boldsymbol{s})$. 
Compared to probability sampling, purposive sampling generally is more efficient for \emph{model-based 
inference} \cite{deGruijterEtAl2006}.

The criterion used to select the sampling locations can be defined based on the chosen statistical model 
\cite{deGruijterEtAl2006, Mueller2007, WebsterEtAl2013}. A set of mathematical and heuristic rules is then 
formalized in the form of a computer algorithm to find the sampling locations that minimize (or maximize) that 
criterion. The more we know about the structure of spatial variation of $Y(\boldsymbol{s})$, the more likely we 
are to obtain the optimum sample configuration given the chosen statistical model.

However, the statistical model is usually unknown before we sample. This is especially common when multiple 
soil properties have to be mapped and the available information is insufficient to decide on the structure of 
the spatial variation. Because we usually want to make the least possible number of assumptions about the model 
structure, the safest solution is to use a space filling design \cite{HenglEtAl2003a, deGruijterEtAl2006, 
Mueller2007, WalvoortEtAl2010}: the locations are selected as to generate a sample that covers the geographic 
and/or feature space(s) as evenly as possible. In areas with very little information on the spatial variation 
of the soil properties of interest, referred to \emph{terra incognita} by \citet{WebsterEtAl2007}, there 
usually are operational constraints that limit the sampling to a single phase. The spatial sample configuration 
has to be optimized to identify the correct model structure, estimate model parameters, and make spatial 
predictions.

\subsection{Sampling for Spatial Trend Estimation}

The spatial trend corresponds to the spatial variation of $Y(\boldsymbol{s})$ that is explained linearly or 
nonlinearly by the covariates. For a linear spatial trend, the sample should cover the extremes of the 
distribution of the covariates \cite{Mueller2007}. For models with interactions and/or higher order terms there 
are the response surface designs \cite{BoxEtAl1951, LeschEtAl1995}. These approaches produce clusters of points 
and ignore the spatial autocorrelation of the residuals \cite{BrusEtAl2007a, Mueller2007}. Optimal sampling 
designs for neural nets, random forests, etc., are yet unknown.

A common solution for spatial trend estimation in \emph{terra incognita} is to use a feature space filling 
sample. \citet{HenglEtAl2003a} sampled along the marginal distribution of the covariates using equal-range 
strata with weights proportional to the frequency distribution. \citet{MinasnyEtAl2007a} sampled equal-variance 
geographic strata created using the variance of the covariates retained in their first principal component.

A more elaborated method, formulated as a multi-objective optimization problem, was developed by 
\citet{MinasnyEtAl2006b} based on the Latin hypercube sampling \cite{McKayEtAl1979}, known as \emph{conditioned 
Latin hypercube sampling} (CLHS). The CLHS is based on sampling along the marginal distribution of the numeric 
and factor covariates using equal-area strata (quantile sampling) and proportionally to the area occupied by 
each level, respectively, and reproducing the linear correlation of the numeric covariates 
\cite{MinasnyEtAl2006b}. The method is very flexible and can be easily extended \cite{MinasnyEtAl2010a, 
RoudierEtAl2012}. Recently, \citet{Samuel-RosaEtAl} proposed conceptual and algorithmic improvements on the 
CLHS. In short, the proposed improvements concern the definition of the marginal sampling strata, the 
measurement of the correlation between covariates, and the aggregation of the objective functions.

\subsection{Sampling for Variogram Estimation}

A variogram model explains the spatially correlated random part of the spatial variation of 
$Y(\boldsymbol{s})$. Several sampling methods exist to identify and/or estimate the variogram and its 
parameters \cite{BrusEtAl1994, deGruijterEtAl2006, Mueller2007, WebsterEtAl2013}. Modern ones focus on maximum 
likelihood estimators \cite{Lark2002, Zimmerman2006, Mueller2007}. Their limitation is that a minimum knowledge 
about the form of the variogram is required. A Bayesian approach was suggested to account for the uncertainty 
of the estimated variogram \cite{DiggleEtAl2006, MarchantEtAl2006, ZhuEtAl2006}. But it is hard to implement 
for multiple variables simultaneously, and the uncertainty is likely to increase with the number of parameters 
that need to be estimated.

Sampling for variogram estimation should concentrate on relevant pairwise distances \cite{MuellerEtAl1999, 
Lark2002}. But how to do that when we are ignorant about the shape of the variogram? \citet{BreslerEtAl1982, 
Russo1984, WarrickEtAl1987} proposed a conservative solution focusing on the method of moments: the points 
should be located as to match a uniform distribution of pairwise distances. Their claim was that the sample 
would be globally optimal for an infinite set of unknown variograms. This has not been proved mathematically 
nor corroborated by empirical evidence. The resulting sample usually is redundant (poorly informative), 
concentrating most of the points in a single large cluster, with a few scattered points -- many of the 
point-pairs are computed using the same subset of points.

\subsection{Sampling for Spatial Interpolation}

Kriging is the best unbiased linear predictor of soil properties \cite{LarkEtAl2006}. The overall prediction 
accuracy depends on spreading the sample points as uniformly as possible throughout the study area. This is 
because for a stationary isotropic random field the kriging variance is a function only of the distance between 
sample points \cite{Cressie1993}. Regular sampling grids are commonly used to obtain a uniform geographic 
coverage, although triangular equilateral grids are more efficient \cite{WebsterEtAl2007}. Regular grids 
usually are inappropriate for irregularly shaped areas \cite{WalvoortEtAl2010}.

The regression-kriging approach for soil mapping \cite{HenglEtAl2007b} lead to the development of sampling 
methods that account for both feature and geographic spaces. \citet{HenglEtAl2003a} proposed sampling 
iteratively in the feature space and keeping the sample configuration with the best geographic coverage. 
\citet{MinasnyEtAl2006b} developed a sampling strategy for spatial trend estimation and claimed that the 
geographic space could be considered as well. \citet{MinasnyEtAl2007a} suggested that a geographic 
stratification based on the variance of the covariates would take into consideration the geographic coverage. 
These methods are suboptimal for spatial interpolation because they essentially operate in the feature space.

Efficient optimization of sample configurations for spatial interpolation depends upon minimizing a 
distance-based metric \cite{RoyleEtAl1998}. One such metric is the Mean Squared Shortest Distance (MSSD) 
between sample and prediction points \cite{BrusEtAl2006}. It is equivalent to finding, for each prediction 
point, the nearest neighbouring sample point. This metric can be minimized using the \textit{k}-means 
clustering algorithm \cite{WalvoortEtAl2010}, which is computationally fast, but sensitive to local optima 
solutions.

\section{PROPOSED INNOVATIONS AND MODIFICATIONS}

We believe that there is room to improve on the existing methods and propose innovations for sampling for soil 
mapping in \emph{terra incognita}. Our proposed innovations and modifications have been implemented in the 
publicly available \Rpackage{spsann} (\url{https://cran.r-project.org/web/packages/spsann}).

\subsection{Sampling for Spatial Trend Estimation}

We consider the method of \cite{MinasnyEtAl2006b} to be the most suited to sample for spatial trend estimation 
for soil mapping in \emph{terra incognita}.

\subsection{Sampling for Variogram Estimation}

We propose that sampling to estimate the variogram model for soil mapping in \emph{terra incognita} should be 
based on placing several small clusters scattered throughout the spatial domain as to maximize the amount of 
information. The most relevant pairwise distances are those that enable an accurate estimate of the behaviour 
of the variogram near the origin. We use exponentially spaced lags defined up to the circumradius $r$ of the 
bounding box of the area. The exponential spacings are created sequentially from the largest to the smallest 
lag by halving the immediately preceding larger lag, resulting in narrower lags in the left side of the 
variogram. It works as follows:

\begin{enumerate}
 \item Find $r$. Use the result to define the upper bound (UB) of the first rightmost lag.
 \item Halve $r$. Use the result to define the lower bound (LB) of the first rightmost lag.
 \item Go to the next lag.
 \item Set the LB of the last lag as the UB of the current lag.
 \item Halve the UB. Use the result to define the LB of the current lag.
 \item Proceed as in 3--5 till the UB and LB of leftmost lag have been defined.
\end{enumerate}


We define seven exponentially spaced lag-distance classes up to half the 
diagonal of the spatial domain. They are created sequentially by halving the immediately preceding larger lag 
\cite{TruongEtAl2013}. Our objective is to place the points as to have each of them contributing to all lags. 
The criterion to be minimized is the sum of differences between the vectors of the pre-specified 
$\boldsymbol{l}^*$ and observed $\boldsymbol{l}$ distributions of unique \textbf{P}oints \textbf{P}er 
\textbf{L}ag

\begin{equation}
 \text{PPL} = \sum_{i = 1}^{n} w_i (l_i^* - l_i),
\end{equation}\label{eq:chap08-ppl}

\noindent where $\boldsymbol{w}$ is a vector of weights for the $n$ lag-distance classes.

\subsection{Sampling for Spatial Interpolation}

The MSSD seems to be the most suited criterion to optimize sample configurations for spatial interpolation for 
soil mapping in \emph{terra incognita}. However, the available algorithms cannot be used to formulate 
multi-objective optimization problems. We suggest using the spatial simulated annealing algorithm instead, 
eliminating the sensitivity to local optima solutions \cite{KirkpatrickEtAl1983, Groenigen1999a}.

\subsection{Sampling for Soil Mapping in \emph{Terra Incognita}}

We propose a heuristic, general-purpose method to design sample configurations for soil mapping in \emph{terra 
incognita}. Like sampling for spatial trend estimation, it is based on solving a MOOP. An utility function is 
defined aggregating the four objective functions described above so that the sample points SPAN the feature, 
variogram and geographic spaces,

\begin{equation}
\text{SPAN} = w_1 \text{CORR} + w_2 \text{DIST} + w_3 \text{PPL} + w_4 \text{MSSD}, 
\end{equation}\label{eq:chap08-span}

with $w_1 = w_2$ and $w_1 + w_2 = w_3 + w_4$ in the \emph{terra incognita} setting.

\section{CASE STUDY}

The study was developed using synthetic data derived from a real-world study case described by 
\citet{Samuel-RosaEtAl2015}. The study site is a small catchment of about \SI{2000}{\hectare} located on the 
southern edge of the plateau of the Paraná Sedimentary Basin, Rio Grande do Sul, Brazil 
(\autoref{fig:chap08-location}). The real-world dataset contains $n = 350$ point soil observations of the 
topsoil, and includes several soil properties, but only two were explored in this study: clay content (CLAY, 
\si{\gram\pre\kilo\gram}) and bulk density (BUDE, \si{\mega\gram\per\cubic\metre}). The dataset also includes 
several covariates derived from area-class soil maps, digital elevation models, geological maps, land use maps, 
and satellite images. All preprocessing steps and methods used to derive the covariates were described by 
\citet{Samuel-RosaEtAl2015}.

\begin{figure}[!ht]
 \centering
 \includegraphics[width = 90mm]{fig/chap08-location}
 \caption{Location of the real-world study area in Santa Maria, southern Brazil.}
 \label{fig:chap08-location}
\end{figure}

\subsection{Soil Data Generating Process}

We assumed the soil properties ($Y$) to be a function of the interplay of environmental conditions defined by 
the climate, organisms, relief, parent material, time, and other unknown players \cite{Jenny1994, 
McBratneyEtAl2003, Florinsky2012}. Because our pedologial knowledge and data available still are limited to 
build such a complex \emph{mechanistic model}, we assumed the soil properties to be the outcome of a spatial 
stochastic process composed of the additive combination of fixed and random effects, i.e. $Y(\boldsymbol{s}) = 
m(\boldsymbol{s}) + e(\boldsymbol{s})$. Here the soil property is a random variable $Y(\boldsymbol{s})$, 
$m(\boldsymbol{s})$ is a deterministic trend, and $e(\boldsymbol{s})$ is a spatially correlated, Gaussian 
distributed random variable, that is stationary in the mean and covariance \cite{HeuvelinkEtAl2001}.




\section{FINAL CONSIDERATIONS}

The main free and open source implementation of the CLHS is the \Rpackage{clhs} \cite{RoudierEtAl2012}. The 
package extended the CLHS considering the use of a cost surface that penalizes sampling locations that are 
difficult to access. Other researchers have also implemented the CLHS as to consider a cost function 
\cite{MulderEtAl2013, CliffordEtAl2014}, but none is available as a free software package.
 % 8. Optimization of sample configurations for spatial trend estimation for soil mapping
\artigotrue
\chapter{SAMPLING FOR SOIL MAPPING IN \emph{TERRA INCOGNITA}}
\shorttitle{Sampling in \emph{Terra Incognita}}
\label{chap:chap09}

%\def\ptkeys{Recozimento simulado, Otimização multi-objetivo, Estimativa do variogram, Predição espacial}
%\begin{chapterabstract}{brazilian}{\ptkeys}
%Este é o resumo em português.
%\end{chapterabstract}

\def\enkeys{Simulated annealing. Multi-objective optimization. Variogram estimation. Spatial prediction}
  
\begin{chapterabstract}{english}{\enkeys}
This study addresses a problem that many soil spatial modellers face: how to come up with an efficient spatial 
sample configuration to (I) estimate the spatial trend, (II) estimate the variogram of the residuals, and (III) 
make spatial predictions in situations where we know very little. The proposed solution is to formulate a sound 
multi-objective optimization problem using robust versions of existing sampling algorithms. The aimed spatial 
sample should reproduce the marginal distribution of the covariates such that the spatial trend can be 
accurately estimated. It should also contain several small clusters scattered throughout the sampling region to 
enable making an accurate estimate of the behaviour of the variogram, specially near the origin. Finally, it 
should cover the sampling region in the most uniform way such that the average prediction error variance is the 
least possible. This multi-objective optimization problem could be solved using spatial simulated annealing as 
implemented in the \Rpackage{spsann}.
\end{chapterabstract}

\formatchapter

\section{INTRODUCTION}

\titlenote{This chapter is based on the study \textit{spsann -- optimization of sample patterns using spatial 
simulated annealing}, presented at the EGU General Assembly 2015 \cite{Samuel-RosaEtAl2015a}, and 
\textit{Optimization of sample configurations for variogram estimation}, presented at Pedometrics 2015 
\cite{Samuel-RosaEtAl2015c}. Also collaborated in the preparation of this document: Gerard B. M. Heuvelink 
(ISRIC -- World Soil Information), Dick J. Brus (Alterra, Wageningen University and Research Centre, the 
Netherlands), Gustavo M. Vasques (Embrapa Soils, Brazil), and Lúcia Helena Cunha dos Anjos (Universidade 
Federal Rural do Rio de Janeiro, Brazil).}

The success of soil mapping largely depends on the sampling data, which are generally used to 1) estimate the 
spatial trend, 2) estimate the variogram of the residuals, and 3) make spatial predictions by calculating 
conditional distributions. A poor sampling strategy is likely to deliver a poor model and large prediction 
errors, resulting in a waste of financial resources, staff and time \cite{vanGroenigenEtAl1999,  
deGruijterEtAl2006, LanEtAl2010}. This is undesirable because sampling already is the largest contributor to 
the costs of soil mapping \cite{WebsterEtAl1990, vanGroenigenEtAl1999, KempenEtAl2012}.

This study addresses a problem that many soil spatial modellers face: how to come up with a purposive spatial 
sample configuration that is effective and robust in situations where we know very little. We explore a 
scenario in which a) multiple soil properties have to be mapped, b) we are ignorant (or know very little) about 
the form of the model of spatial variation, and c) the operational constraints limit sampling to a single 
phase. The study starts with a review of the purposive sampling strategies employed by soil spatial modellers 
to meet one or more of the three objectives for which sampling data are used under the proposed scenario, i.e. 
to estimate the spatial trend and the variogram, and make spatial predictions. Based on theoretical and 
operational features, we indicate the purposive sampling strategies that we believe to be the most appropriate 
for each purpose and try to formulate a purposive sampling strategy that addresses all three objectives 
jointly. The chapter ends with a suggestion on how to test the performance of the proposed purposive sampling 
algorithms.

% The objective is to evaluate the ability of different sampling configuration types and sample sizes to 
% capture the true form of the model of spatial variation and make accurate predictions. We also quantify the 
% gain in prediction accuracy by combining popular sampling methods in a multi-objective optimization problem. 
% This study addresses a problem that many soil spatial modellers face: how to come up with a spatial sample 
% configuration that is effective and robust in situations where we know very little? To start, we present a 
% short review on purposive sampling methods for spatial sample optimization.

\section{PURPOSIVE SAMPLING}

\emph{Purposive sampling} is the non-probability sampling mode by which the sampling locations are selected 
intentionally as to satisfy an \textit{a priori} criterion. This criterion is commonly defined based on the 
\emph{model} that will be used to infer the structure of spatial variation of a soil property 
$Y(\boldsymbol{s})$. Compared to probability sampling, purposive sampling generally is more efficient for 
\emph{model-based inference} \cite{deGruijterEtAl2006}.

The criterion used to select the sampling locations can be defined based on the chosen statistical model 
\cite{deGruijterEtAl2006, Mueller2007, WebsterEtAl2013}. A set of mathematical and heuristic rules is then 
formalized in the form of a computer algorithm to find the sampling locations that minimize (or maximize) that 
criterion. The more we know about the structure of spatial variation of $Y(\boldsymbol{s})$, the more likely we 
are to obtain the optimum sample configuration given the chosen statistical model.

However, the statistical model is usually unknown before we sample. This is especially common when multiple 
soil properties have to be mapped and the available information is insufficient to decide on the structure of 
the spatial variation. Because we usually want to make the least possible number of assumptions about the model 
structure, the safest solution is to use a space filling design \cite{HenglEtAl2003a, deGruijterEtAl2006, 
Mueller2007, WalvoortEtAl2010}: the locations are selected as to generate a sample that covers the geographic 
and/or feature space(s) as evenly as possible. In areas with very little information on the spatial variation 
of the soil properties of interest, referred to as \emph{terra incognita} by \citet{WebsterEtAl2007}, where 
usually there are operational constraints that limit the sampling to a single phase, an efficient spatial 
sample configuration should be optimized to identify the correct model structure, estimate model parameters, 
and make spatial predictions.

\subsection{Sampling for Spatial Trend Estimation}

The spatial trend corresponds to the spatial variation of $Y(\boldsymbol{s})$ that is explained linearly or 
non-linearly by the covariates. For a linear spatial trend, the sample should cover the extremes of the 
distribution of the covariates \cite{Mueller2007}. For models with interactions and/or higher order terms there 
are the response surface designs \cite{BoxEtAl1951, LeschEtAl1995}. These approaches produce clusters of points 
and ignore the spatial autocorrelation of the residuals \cite{BrusEtAl2007a, Mueller2007}. Optimal sampling 
designs for neural nets, random forests, etc., are yet unknown.

A common solution for spatial trend estimation in \emph{terra incognita} is to use a feature space filling 
sample. \citet{HenglEtAl2003a} sampled along the marginal distribution of the covariates using equal-range 
strata with weights proportional to the frequency distribution. \citet{MinasnyEtAl2007a} sampled equal-variance 
geographic strata created using the variance of the covariates retained in their first principal component.

A more elaborated method, formulated as a multi-objective optimization problem composed of three objective 
functions, was developed by \citet{MinasnyEtAl2006b} based on Latin hypercube sampling (LHS), a non purposive, 
probability sampling method \cite{McKayEtAl1979}. The method, known as conditioned Latin hypercube sampling 
(CLHS), is based on sampling along the marginal distribution of the numeric and factor covariates using 
equal-area strata (quantile sampling) and proportionally to the area occupied by each level, respectively, and 
reproducing the linear correlations among the numeric covariates \cite{MinasnyEtAl2006b}. CLHS is very flexible 
and can be easily extended, two important reasons for its popularity \cite{MinasnyEtAl2010a, RoudierEtAl2012}.

Recently, \citet{Samuel-RosaEtAl2016} proposed conceptual and algorithmic improvements on the CLHS. Algorithmic 
improvements concern the definition of the marginal sampling strata, the measurement of the correlation between 
covariates, and the aggregation of the objective functions. These have resulted in a more numerically stable 
sampling algorithm which not necessarily translates into more accurate spatial predictions. Conceptually, the 
goal of the method was reformulated as aiming at a spatial sample that reproduces an association/correlation 
measure and the marginal distribution of the covariates (ACDC). This was presented as a more appropriate 
definition of the method, while the original denomination given by \citet{MinasnyEtAl2006b} was appointed as 
being misleading because it can lead one to think of CLHS as a (non purposive) probability sampling method 
\citet{Samuel-RosaEtAl2016}.

\subsection{Sampling for Variogram Estimation}

A variogram model explains the spatially correlated random part of the spatial variation of 
$Y(\boldsymbol{s})$. Several sampling methods exist to identify and/or estimate the variogram and its 
parameters \cite{BrusEtAl1994, deGruijterEtAl2006, Mueller2007, WebsterEtAl2013}. Modern ones focus on maximum 
likelihood estimators \cite{Lark2002, Zimmerman2006, Mueller2007}, their limitation being that a minimum 
knowledge about the form of the variogram is required. A Bayesian approach was suggested to account for the 
uncertainty of the estimated variogram \cite{DiggleEtAl2006, MarchantEtAl2006, ZhuEtAl2006}. But it is hard to 
implement for multiple variables simultaneously, and the uncertainty is likely to increase with the number of 
parameters that need to be estimated.

Sampling for variogram estimation should concentrate on relevant pairwise distances \cite{MuellerEtAl1999, 
Lark2002}. But how to do that when we are ignorant about the shape of the variogram? \citet{BreslerEtAl1982, 
Russo1984, WarrickEtAl1987} proposed a conservative solution: the points should be located as to match a 
uniform distribution of pairwise distances. Their claim was that the sample would be globally optimal for an 
infinite set of unknown variograms. This has not been proven mathematically nor corroborated by empirical 
evidence. The resulting sample usually is redundant (poorly informative), concentrating most of the points in a 
single large cluster, with a few scattered points -- many of the point-pairs are computed using the same subset 
of points.

Another critique to the idea of \citet{BreslerEtAl1982, Russo1984, WarrickEtAl1987} is that it was rooted on 
the use of the method-of-moments to fit a continuous function to the binned empirical variogram. Nowadays we 
recognize that the estimates of the method-of-moments are affected by the correlation between the sequence of 
classes of pairwise distances and that more robust methods exist \cite{DiggleEtAl2002}. Maximum likelihood 
methods estimate model parameters using all data points, avoiding the need for an \textit{ad hoc} definition of 
classes of pairwise distances that generally smooth out the structure of the spatial process  
\cite{Lark2000}.

\subsection{Sampling for Spatial Interpolation}

Kriging is the best unbiased linear predictor of soil properties \cite{LarkEtAl2006}. Overall better prediction 
accuracy depends on spreading the sample points as uniformly as possible throughout the study area. This is 
because for a stationary isotropic random field the kriging variance is a function only of the distance between 
sample points \cite{Cressie1993}. Regular sampling grids are commonly used to obtain a uniform geographic 
coverage, although triangular equilateral grids are more efficient \cite{WebsterEtAl2007}. Their main weakness 
is the inefficient coverage of the geographic space when the sampling region is irregularly shaped and/or 
contains irregularly shaped non-sampling areas \cite{WalvoortEtAl2010}.

The regression-kriging approach for soil mapping \cite{HenglEtAl2007b} lead to the development of sampling 
methods that account for both feature and geographic spaces. \citet{HenglEtAl2003a} proposed sampling 
iteratively in the feature space and keeping the sample configuration with the best geographic coverage. 
\citet{MinasnyEtAl2006b} developed a sampling strategy for spatial trend estimation and claimed that the 
geographic space could be considered as well. \citet{MinasnyEtAl2007a} suggested that a geographic 
stratification based on the variance of the covariates would take into consideration the geographic coverage. 
These methods are suboptimal for spatial interpolation because they essentially operate in the feature space.

It has been suggested that efficient optimization of sample configurations for spatial interpolation depends 
upon minimizing a distance-based metric \cite{RoyleEtAl1998}. One such metric is the mean squared shortest 
distance (MSSD) between sample and prediction points, which is equivalent to finding, for each point in the 
prediction grid, the nearest neighbouring sample point  \cite{BrusEtAl2006}. Because the MSSD takes into 
account all points in the prediction grid, its minimization produces a spatial sample that uniformly covers the 
geographic space, irrespective of the sampling region being irregularly shaped and/or containing 
irregularly shaped non-sampling areas \cite{WalvoortEtAl2010}. However, this renders the method computationally 
expensive because a large distance matrix has to be computed every time a candidate spatial sample is 
generated.

The problem of minimizing the MSSD can be speeded up reformulating it in terms of an unsupervised 
classification problem. The objects to be classified are the points of the prediction grid and the 
classification variables are the x- and y-coordinates, the cluster centers defining the sampling locations 
\cite{WalvoortEtAl2010}. This classification problem can be solved using the \textit{k}-means clustering 
algorithm, which is computationally fast, but sensitive to local optima solutions.

\section{WHICH SAMPLING ALGORITHM?}

\subsection{Sampling for Spatial Trend Estimation}

There are multiple methods for designing optimum spatial sample configurations for spatial trend estimation in 
\emph{terra incognita}, each with different complexity levels. The method of \citet{MinasnyEtAl2006b} is very 
well suited, as indicated by its popularity. Because \citet{Samuel-RosaEtAl2016} produced a more numerically 
stable sampling algorithm, we suggest that the improved version of the CLHS called ACDC should be used instead.

Different from CLHS, ACDC is a multi-objective optimization problem composed of only two objective functions,

\begin{equation}\label{eqn:chap09-corr} % SEE IMAGE BELOW
 \text{CORR} = \sum_{i=1}^{p}\sum_{j=1}^{p}|\varphi_{ij} - v_{ij}|,
\end{equation}

\noindent where $\varphi_{ij}$ and $v_{ij}$ are the population and sample associations (or correlations in case 
all covariates are numeric) at the $i$th row and $j$th column of the $p$-dimensional population and sample 
association (or correlation) matrices, and

\begin{equation}\label{eqn:chap09-dist} % SEE IMAGE BELOW
 \text{DIST} = \sum_{i=1}^{p}\sum_{j=1}^{c_i} |\pi_{ij} - \gamma_{ij}|,
\end{equation}

\noindent where $\pi_{ij}$ and $\gamma_{ij}$ are the proportion of sample and population points that fall in 
the $j$th class (or marginal sampling strata) of the $i$th covariate, $c_i$ being the number of classes of the 
$i$th covariate. As such, ACDC is defined as follows:

\begin{equation}\label{eqn:chap08-acdc} % SEE IMAGE BELOW
 \text{ACDC} = w_1\text{CORR} + w_2 \text{DIST},
\end{equation}

\noindent with weights $w_1 = w_2 = 0.5$ when we do not have \emph{a priori} preferences towards the objective 
functions.

\subsection{Sampling for Variogram Estimation}

An efficient and robust method for variogram estimation in \emph{terra incognita} seems to be missing. For that 
end, we propose that sampling should be based on placing several small clusters scattered throughout the 
spatial domain as to maximize the amount of information carried by the sample. The most relevant pairwise 
distances are those that 1) enable an accurate estimate of the 
behaviour of the variogram near the origin and 2) produce a low estimate of the nugget variance. Our main 
concern 
is with the fact that the shape of the variogram and the estimated nugget variance determine the smoothness of 
the spatial predictions \cite{WebsterEtAl2007}.

The design of our sampling algorithm starts with the method proposed by \citet{WarrickEtAl1987}. Instead of 
equidistant, we use exponentially spaced lag-distance classes defined up to the circumradius $r$ of the 
bounding box of the area as proposed by \citet{TruongEtAl2013}. The exponential spacings are created 
sequentially from the largest to the smallest lag by halving the immediately preceding larger lag, resulting in 
narrower lags in the left side of the variogram. It works as follows:

\begin{enumerate}
 \item Find $r$. Use the result to define the upper bound of the first rightmost lag.
 \item Halve $r$. Use the result to define the lower bound of the first rightmost lag.
 \item Go to the next lag.
 \item Set the lower bound of the last lag as the upper bound of the current lag.
 \item Halve the upper bound. Use the result to define the lower bound of the current lag.
 \item Proceed as in 3--5 until the upper and lower bounds of the leftmost lag have been defined.
\end{enumerate}

\noindent The number of lag-distance classes depends on the wanted size of the smallest lag. For general 
applications, it seems appropriate to use a maximum of seven lags, where the size of the smallest lag will be 
approximately \SI{2}{\percent} of the circumradius $r$ of the bounding box of the area. Having defined the 
bounds of the lag-distance classes, our objective is to find a spatial sample configuration such that every 
point forms pairs with points that are separated by distances that fall in each of the (seven) lag-distance 
classes. In other words, we aim at each sample point contributing with at least one point-pair in each 
lag-distance class. For example, with seven lags and $n = 100$ sample points, the aimed solution is 
$\boldsymbol{l}^* = (100, 100, 100, 100, 100, 100, 100)$, i.e. a uniform distribution. When optimizing the 
sample configuration, the criterion to be minimized is the sum of differences between the vectors of the aimed 
$\boldsymbol{l}^*$ and observed $\boldsymbol{l}$ distributions of unique \textbf{P}oints \textbf{P}er 
\textbf{L}ag

\begin{equation}\label{eqn:chap08-ppl} % SEE IMAGE BELOW
 \text{PPL} = \sum_{i = 1}^{q} l_i^* - l_i,
\end{equation}

\noindent where $\boldsymbol{l}^* = n$ for a uniform distribution, $n$ being the number of sample points. This 
proposed modification differs from the original implementation of \citet{WarrickEtAl1987} by the fact that the 
later aims at a uniform distribution of the number of \emph{point-pairs} per lag-distance class.

\subsection{Sampling for Spatial Interpolation}

There are not many strategies for optimizing spatial samples for spatial interpolation in \emph{terra 
incognita} as there are for spatial trend estimation. Among the existing options, the MSSD seems to be the most 
suited criterion. The main reasons for this are 1) its direct link with the need for minimizing the prediction 
error variance when making spatial predictions, and 2) its flexibility to deal with irregularly shaped sampling 
regions containing also irregularly shaped non-sampling areas.

The MSSD is given by

\begin{equation}% SEE IMAGE BELOW
 \text{MSSD} = \frac{1}{N} \sum_{i = 1}^{N} min_j(D_{ij}^2),
\end{equation}

\noindent where $N$ is the number of points in the prediction grid, $D_{ij}^2$ is the squared Euclidean 
distance between the $i$ point in the prediction grid and the $j$ sampling point computed using the x- and 
y-coordinates, and $min_j$ refers to taking the minimum over all $j$’s for each $i$.

\section{SAMPLING IN \emph{TERRA INCOGNITA}}

We propose a heuristic, general-purpose method to design sample configurations for soil mapping in \emph{terra 
incognita}. Like sampling for spatial trend estimation (ACDC), it is based on solving a multi-objective 
combinatorial optimization problem (MOCOP) (\autoref{eqn:chap08-mocop}). A utility function is defined 
aggregating the three criteria described above using the weighted sum method (\autoref{eqn:chap08-utility})
so that the sample points cover, extend over, spread over, SPAN the feature, variogram and geographic spaces,

\begin{equation}\label{eqn:chap08-span} % SEE IMAGE BELOW
\text{SPAN} = w_1 \text{CORR} + w_2 \text{DIST} + w_3 \text{PPL} + w_4 \text{MSSD}, 
\end{equation}

\noindent with $w_1 = w_2$ and $w_1 + w_2 = w_3 + w_4$ in the \emph{terra incognita} setting. Before 
aggregation, each objective function is scaled to the same approximate range of values using the upper-lower 
bound approach with the Pareto minimum and maximum values (\autoref{eqn:chap08-pareto-min-max}) so that any 
potential numerical dominance can be eliminated or minimized, and the weights can play the desired role 
\cite{MarlerEtAl2005, MarlerEtAl2009}. Solving the multi-objective optimization problem of sampling in 
\emph{terra incognita} (SPAN) can be solved using spatial simulated annealing as implemented in the 
\Rpackage{spsann}. Simulated annealing is a popular method with widespread use to solve combinatorial 
optimization problems due to its robustness against local optima and easiness of implementation 
\cite{MetropolisEtAl1953, KirkpatrickEtAl1983, Cerny1985, AartsEtAl1989, Groenigen1999a}.

\section{CONCLUSIONS}

This chapter presented sound strategies for optimizing spatial samples to estimate the spatial trend and the 
variogram, and make spatial predictions when we know very little about the soil spatial distribution. Overall, 
the main requirement is the formulation of a sound multi-objective combinatorial optimization problem (MOCOP) 
using robust versions of existing sampling algorithms. The aimed spatial sample should reproduce the marginal 
distribution of the covariates such that the spatial trend can be accurately estimated. This can be achieved 
using the recently improved version of the conditioned Latin hypercube sampling algorithm (ACDC).

The spatial sample should also contain several small clusters scattered throughout the sampling region, the 
reason being the need for an accurate estimate of the behaviour of the variogram near the origin. An efficient 
metric for achieving such objective is the number of unique points that form point-pairs in each of the 
exponentially spaced lag-distance classes of the sample variogram (PPL). Optimally, every point would form 
point-pairs in each lag class. Finally, the sampling region should be covered the most uniformly possible such 
that the average prediction error variance is the least possible. For that end, one can minimize the distance 
between every prediction point and its nearest neighbouring sampling point (MSSD).

We believe that the proposed general purpose sampling strategy need to be evaluated compared to (i) the single 
version of each objective function, (ii) popular sampling designs such as regular grids, and (iii) sampling 
algorithms that assume the model of soil spatial variation as known. The effect of sample size need to be 
addressed as well. Unfortunately, resource restrictions hinder the execution of such an evaluation using field 
data because sampling costs are high. A reasonable solution is to use synthetic data derived from a real-world 
case study. The first step would consist of defining soil data generating processes using existing point soil 
observations and spatially exhaustive covariates. At least two generating processes should be defined, 
possibly 
using different soil properties, such that the deterministic and random components of soil spatial variation 
have different forms. For example, linear and non-linear trends coupled with exponential and Gaussian 
variograms. Next, the generating processes would be used to produce multiple realizations of the soil data, 
from which we would sample using the spatial samples under comparison. Evaluation of sampling strategies would 
consist of measuring how well the spatial samples (i) capture the true form of the soil data generating 
process and (ii) make spatial predictions. The outcome of such an exercise should help us understanding on how 
to decide upon sampling strategies when we want to learn about the soil-landscape relationships and make 
accurate predictions.

% \section{CASE STUDY}
% 
% The study was developed using synthetic data derived from a real-world study case described by 
% \citet{Samuel-RosaEtAl2015}. The study site is a small catchment of about \SI{2000}{\hectare} located on the 
% southern edge of the Plateau of the Paraná Sedimentary Basin, Rio Grande do Sul, Brazil 
% (\autoref{fig:chap08-location}). The real-world dataset contains $n = 350$ point soil observations of the 
% topsoil, and includes several soil properties, but only two were explored in this study: clay content (CLAY, 
% \si{\gram\per\kilo\gram}) and bulk density (BUDE, \si{\mega\gram\per\cubic\metre}). The dataset also 
% includes several covariates derived from area-class soil maps, digital elevation models, geological maps, 
% land use maps, and satellite images. All preprocessing steps and methods used to derive the covariates were 
% described by \citet{Samuel-RosaEtAl2015}.

% \begin{figure}[!ht]
%  \centering
%  \includegraphics[width = 0.6\textwidth]{fig/chap02-location}
%  \caption{Location of the real-world study area in Santa Maria, southern Brazil.}
%  \label{fig:chap08-location}
% \end{figure}
% 
% \subsection{Soil Data Generating Process}
% 
% We assumed the soil properties ($Y$) to be a function of the interplay of environmental conditions defined 
% by climate, organisms, relief and parent material through time \cite{Jenny1941, McBratneyEtAl2003, 
% Florinsky2012}. Because our pedologial knowledge and data available still are too limited to build such a 
% complex \emph{mechanistic model}, we assumed the soil properties to be the outcome of a spatial stochastic 
% process composed of the additive combination of fixed and random effects, i.e. $Y(\boldsymbol{s}) = 
% m(\boldsymbol{s}) + e(\boldsymbol{s})$. Here the soil property is a random variable $Y(\boldsymbol{s})$, 
% $m(\boldsymbol{s})$ is a deterministic trend, and $e(\boldsymbol{s})$ is a spatially correlated, Gaussian 
% distributed random variable, that is stationary in the mean and covariance \cite{HeuvelinkEtAl2001}.

% Using the above formulation of the \emph{mixed model of spatial variation}, we defined two \emph{soil 
% data generating processes} assuming a different form in $m(\boldsymbol{s})$ and $e(\boldsymbol{s})$. In 
% practice, we used the real-world soil data to calibrate linear and nonlinear mixed models, which are then 
% taken as the exact mathematical representations of the true (stochastic) processes giving rise to the soil 
% and its properties. The linear mixed model is that constructed by \citet{Samuel-RosaEtAl2015} using CLAY ($n 
% = 350$) in the Box-Cox space due to its strong skewness, and corresponds to what the authors called their 
% \emph{best performing} linear mixed model. The non-linear mixed model was constructed by 
% \citet{Samuel-RosaEtAl2016} using BUDE ($n = 282$) in its original untransformed scale, the non-linear trend 
% being defined using the out-of-bag predictions of a random regression forest model. The spatially dependent 
% stochastic residuals of CLAY and BUDE were modelled using the Whittle-Matérn model, the difference being 
% that the shape parameter was set to $\nu = (0.5, 2)$, respectively. All model parameters were estimated by 
% Gaussian restricted maximum likelihood (REML) \cite{LarkEtAl2004, DiggleEtAl2007}. The two models also 
% differ by the spatially correlated variance (SCV) of the stochastic residuals, which is small for BUDE 
% ($\text{SCV} \cong 0.20$) and large for CLAY ($\text{SCV} \cong 0.85$).

% Sequential unconditional Gaussian simulation was used to create $\mathcal{R} = 1000$ equiprobable 
% realizations 
% of an isotropic Gaussian random field of CLAY and BUDE 
% % (\autoref{fig:chap08-realizations}) 
% using the same settings described in \autoref{subsec:simulation} \cite{Samuel-RosaEtAl2016}.

% \begin{figure}[!ht]
%  \centering
%  \includegraphics[width = \textwidth]{fig/chap02-realizations}
%  \caption{Example realizations of the soil data generating processes with a) linear and b) non-linear 
%  trends.}
%  \label{fig:chap08-realizations}
% \end{figure}

% \subsection{Sampling Scenarios}
% 
% We defined $\mathcal{A} = 6$ sampling designs with $\mathcal{N} = 3$ sample sizes $\boldsymbol{n} = (100, 
% 200, 
% 400)$. These sample sizes correspond to the moderately high inspection density (1 sample point per 20, 10, 
% and 
% \SI{5}{\hectare}, respectively) recommended for the production of soil maps published at a \scale{25000} 
% \cite{Rossiter2000}. The baseline design was obtained by simple random sampling (SRS), which we understand 
% as the poorest sampling design. The second sampling design was obtained by systematic grid sampling (SGS), 
% one 
% of the most commonly used sampling design for soil mapping.

% Another three sampling designs were defined minimizing each of the three criteria pointed above: ACDC, PPL, 
% and MSSD. Our multi-objective sampling design was obtained minimizing the three criteria simultaneously 
% (SPAN). Since we assumed to be ignorant about the structure of the soil data generating processes, all 
% criteria 
% were considered equally important and received the same weights. Note that for ACDC and SPAN, different 
% sample 
% configurations were optimized for CLAY and BUDE because the algorithms use the same set of covariates used 
% to 
% calibrate the soil data generating processes.
% 
% The combination of sampling designs and sample sizes resulted in 18~point sample sets. Each of these point 
% sample sets were used to sample from the $\mathcal{R} = 1000$ simulated realities of CLAY and BUDE. This 
% yielded \num{36000}~calibration datasets. The baseline designs were used to quantify the gain in prediction 
% accuracy with the use of our multi-objective sampling design.

% We also defined the true optimum sampling design for CLAY. In this case we assumed the structure of the 
% CLAY data generating process to be known. The criterion minimized was the mean universal kriging variance 
% \cite{BrusEtAl2007a}. The optimum design was also explored at three sample sizes and used to sample from 
% the $\mathcal{R} = 1000$ simulated realities of CLAY with linear trend. This yielded another 
% 3000~calibration datasets. The true optimum sampling designs was used to quantify how suboptimal our 
% multi-objective sampling design is.

% \subsection{Spatial Trend and Variogram Analysis}
% 
% \subsection{Evaluation of Sampling Algorithms}
% 
% We will evaluate how sampling design and sample size influence spatial prediction accuracy. The goal will be 
% to evaluate how our multi-objective sampling design compares with the baseline designs. For the spatial 
% stochastic process with linear trend, we will also compare our multi-objective sampling design with the true 
% optimum design. For these purposes, the same evaluation strategy used by \cite{Samuel-RosaEtAl2016} will be 
% employed.

% The variation in model parameter estimation will also be evaluated for both linear mixed model and 
% regression-kriging model plotting together the 1000 variogram models calibrated with each sampling design 
% and sample size superimposed by the true variogram model. The variation of each variogram model parameter 
% will 
% be 
% summarized using box-and-whisker plots. For the regression-kriging model, box-and-whisker plots will be used 
% to summarize the variation of the importance ranking of each predictor variable. For the linear mixed model, 
% we will summarize the coefficients of the linear trend. These statistics will be compared with the true 
% model 
% parameters.
% 
% \section{FINAL CONSIDERATIONS}
% 
% Experimental results are not available yet. So far, the main outcome of the study is the creation and 
% maintenance of the \Rpackage{spsann}, devoted to the optimization of sample patterns using spatial simulated 
% annealing. \texttt{spsann} offers many optimizing criteria for sampling for variogram estimation (PPL), 
% spatial trend estimation (CORR, DIST, ACDC, and CLHS), and spatial interpolation (MSSD) in \emph{terra 
% incognita}. It also includes the mean or maximum universal kriging variance (MUKV) as an optimizing 
% criterion 
% for spatial interpolation when the model of spatial variation is known. ACDC, PPL, and MSSD were combined 
% into 
% the 
% SPAN algorithm to jointly optimize sampling for variogram and spatial trend estimation, and spatial 
% interpolation when we are ignorant about the model of spatial variation.
% 


 % 9. Sampling for soil mapping in terra incognita
\artigofalse
\chapter{GENERAL CONCLUSIONS}
\shorttitle{General Conclusions}
\label{chap:chap10}

This thesis has made a pedological contribution with the development of a comprehensive description of the 
soil-forming factors and processes that determine the spatio-temporal distribution of soil properties in the 
Santa Maria case study area. The conceptual model of pedogenesis, presented in \autoref{chap:chap03}, showed 
that the spatial distribution of soil properties is highly variable, even when under the same land use. At 
coarse spatial scales, this spatial variation is determined by the geological and geomorphological diversity of 
the area, while at fine spatial scales, past and current (poor) agricultural practices seem to play a major 
role. Along with the conceptual model of pedogenesis, \autoref{chap:chap04} and \autoref{chap:chap05} 
constitute a technical contribution of this thesis. These chapters provide the basis for soil spatial modelling 
exercises in the study area.

\autoref{chap:chap06} demonstrated that existing, freely available covariates are suitable for calibrating soil 
spatial models. It was shown that using more detailed covariates results in only a modest increase in the 
prediction accuracy of linear soil spatial models. The observed increase is comparable to the effect of 
incorporating spatial dependence in the soil spatial model, and may not outweigh the extra costs of using more 
detailed covariates. In general, a more detailed covariate has a greater potential to improve 
prediction accuracy when a soil property is poorly predicted by its less detailed version. However, the 
magnitude of the improvement may depend on which other covariates are included in the model. Choosing whether 
or not to invest in more detailed covariates depends on the strength of the relationship between the covariates 
and the soil property being modelled, and on the relative difference between the less, and more detailed 
versions of the covariates. It is likely better to substantially improve the detail of a less influential 
covariate than marginally increase the detail of the most influential covariate. However, one should always 
consider if more efficient means of increasing prediction accuracy exist (e.g. obtaining more soil 
observations).

\autoref{chap:chap07} showed that several factors influence how field soil spatial modellers decide upon where 
to place soil observation locations. These are of three types: conceptual, operational, and psychological. The 
first concerns the knowledge of the soil spatial modellers about soil-landscape relationships, and seems to be 
connected with the years of field experience. The second relates to the available resources (infrastructure, 
workforce, and budget) to make soil observations, as well as to access restrictions imposed by landowners and 
geographic barriers, for example. The third relates to how the soil modellers perceive their surrounding 
physical environment and how the course of their motivation shifts during the soil observation process. Point 
pattern analysis helped understanding that there is a trade-off between conceptual and operational factors, 
which determines how the motivation of field soil modellers shifts focus towards one or another immediate goal. 
Depending on the focal goal, the resulting sample configuration resembles a random (learning/verifying 
soil-landscape relationships -- means-focused motivation) or a regular (maximizing the number of observations 
and geographic coverage -- outcome-focused motivation) point pattern.

\autoref{chap:chap08} showed that the conditioned Latin hypercube sampling algorithm, a popular algorithm used 
to optimize spatial sample configurations for spatial trend estimation, can be considerably improved. Compared 
to the original CLHS, our proposed modifications resulted in a sampling algorithm with an improved numerical 
behaviour, but this does not necessarily translates into improved prediction accuracy. For instance, sample 
size has a larger influence on prediction accuracy than the sampling algorithm. However, aiming only at the 
association/correlation between covariates degrades prediction accuracy possibly because the coverage of the 
geographic space is poorer. As such, when optimizing a sample configuration for spatial trend estimation, it 
should suffice to aim only at reproducing the marginal distribution of the covariates. This should be done 
using only the non-empty marginal sampling strata.

\autoref{chap:chap09} showed how to optimize sample configurations for spatial trend and variogram estimation, 
and spatial interpolation in situations where we know very little about the soil spatial distribution. The only 
requirement is that one formulates a sound multi-objective optimization problem using robust versions of 
existing sampling algorithms. The resulting spatial sample should reproduce the marginal distribution of the 
covariates such that the spatial trend can be accurately estimated. It should also contain several small 
clusters scattered throughout the spatial domain to enable making an accurate estimate of the behaviour of the 
variogram, specially near the origin. Finally, it should cover the sampling region in the most uniform way such 
that the average prediction error variance is the least possible.

This thesis has also contributed with two packages for the software environment for statistical computing and 
graphics \texttt{R}. The first package, called \texttt{pedometrics} (\autoref{appen:pedometrics}), contains 
various functions for spatial exploratory data analysis and model calibration designed for the development of 
this thesis. The second package, called \texttt{spsann} (\autoref{appen:spsann}), contains functions to 
optimize sample configurations to identify and estimate the variogram and spatial trend, and make spatial 
predictions. The latter was developed as part of \autoref{chap:chap08} and \autoref{chap:chap09}. Both are 
freely available and can be obtained from The Comprehensive R Archive Network (\cran).

Overall, this thesis showed that the complex interplay between soil and covariate data can have a large 
influence on the accuracy of soil maps. A single, universal, cost-effective recipe for reducing uncertainty 
in soil spatial modelling seems out of range. The case studies suggested that solutions are case specific and 
primarily depend on the existing soil and covariate data. Obtaining more soil samples showed to be an efficient 
strategy provided the available resources allow extra sampling. Otherwise, deciding upon cost-effective ways of 
reducing uncertainty requires, first, that we explore the full potential of existing soil and covariate 
data using robust spatial modelling techniques. Such an exercise requires a comprehensive knowledge of the 
soil-landscape relationships, as well as a thorough documentation of the soil and covariate data so that their 
weaknesses and strengths can be easily identified. Then, the decision of whether to invest on improving the 
quality of soil or covariate or both data sources will depend upon the trade-off between the increased 
data/prediction quality and the amount of resources required.
 % 10. General conclusion

%%=============================================================================
%% Referências
%%=============================================================================
\selectlanguage{brazilian}
\shorttitle{Referências Bibliográficas}
\bibliography{ref/biblio}\label{chap:references}

%%=============================================================================
%% Apêndices
%%=============================================================================
\appendix
\artigofalse
\chapter{R-PACKAGE SPSANN: OPTIMIZATION OF SAMPLE CONFIGURATIONS USING SPATIAL SIMULATED ANNEALING}
\label{apen:spsann}

\includepdf[pages=-,pagecommand={}]{chap/spsann.pdf}

% \section{Objective functions}
% 
% \subsection{Spatial trend identification and estimation}
% 
% \subsubsection{DIST}
% 
% Reproducing the marginal distribution of the numeric covariates depends upon
% the definition of marginal sampling strata. These marginal sampling strata 
% are also used to define the factor levels of all numeric covariates that  
% are passed together with factor covariates. Two types of marginal sampling 
% strata can be used: \textit{equal-area} and \textit{equal-range}.
% 
% \textit{Equal-area} marginal sampling strata are defined using the sample 
% quantiles estimated with the \texttt{quantile}-function of the 
% \textbf{stats}-package using a discontinuous function (\texttt{type = 3}). This 
% is to avoid creating breakpoints that do not occur in the population of 
% existing covariate values.
% 
% Depending on the level of discretization of the covariate values, 
% the \texttt{quantile}-function produces repeated breakpoints. A breakpoint 
% will be repeated if that value has a relatively high frequency in the 
% population of covariate values. The number of repeated breakpoints increases 
% with the number of marginal sampling strata. Repeated breakpoints result in
% empty marginal sampling strata. To avoid this, only the unique breakpoints 
% are used.
% 
% \textit{Equal-range} marginal sampling strata are defined by breaking the range
% of covariate values into pieces of equal size. Depending on the level of 
% discretization of the covariate values, this method creates breakpoints that
% do not occur in the population of existing covariate values. Such breakpoints
% are replaced by the nearest existing covariate value identified using 
% Euclidean distances.
% 
% Like the equal-area method, the equal-range method can produce empty marginal
% sampling strata. The solution used here is to merge any empty marginal 
% sampling strata with the closest non-empty marginal sampling strata. This is
% identified using Euclidean distances as well.
% 
% The approaches used to define the marginal sampling strata result in each 
% numeric covariate having a different number of marginal sampling strata, 
% some of them with different area/size. Because the goal is to have a sample 
% that reproduces the marginal distribution of the covariate, each marginal 
% sampling strata will have a different number of sample points. The wanted 
% distribution of the number of sample points per marginal strata is estimated 
% empirically as the proportion of points in the population of existing 
% covariate values that fall in each marginal sampling strata.
% 
% \subsubsection{CORR}
% 
% The \textit{correlation} between two numeric covariates is measured using the 
% Pearson's \textit{r}, a descriptive statistic that ranges from $-1$ to $+1$. 
% This statistic is also known as the linear correlation coefficient.
% 
% When the set of covariates includes factor covariates, all numeric covariates 
% are transformed into factor covariates. The factor levels are defined 
% using the marginal sampling strata created from one of the two methods 
% available (equal-area or equal-range strata).
% 
% The \textit{association} between two factor covariates is measured using the 
% Cramér's \textit{v}, a descriptive statistic that ranges from $0$ to $+1$. The 
% closer to $+1$ the Cramér's \textit{v} is, the stronger the association between 
% two factor covariates. The main weakness of using the Cramér's \textit{v} is 
% that, while the Pearson's \textit{r} shows the degree and direction of the 
% association between two covariates (negative or positive), the Cramér's 
% \textit{v} only measures the degree of association (weak or strong).
% 
% \subsection{Variogram identification and estimation}
% 
% PPL: points and pairs; minimum and distribution
% 
% \subsection{Spatial interpolation}
% 
% MKV and MSSD
% 
% \subsection{Multi-objective optimization}
% 
% ACDC: CORR and DIST;
% CLHS;
% SPAN: CORR, DIST, PPL, and MSSD;
% 
% A method of solving a multi-objective optimization problem is to aggregate 
% the objective functions into a single \textit{utility function}. In the
% \textbf{spsann}-package, the aggregation is performed using the \textit{weighted 
% sum method}, which incorporates in the weights the preferences of the user 
% regarding the relative importance of each objective function.
% 
% The weighted sum method is affected by the relative magnitude of the 
% different function values. The objective functions implemented in the
% \textbf{spsann}-package have different units and orders of magnitude. The 
% consequence is that the objective function with the largest values will have 
% a numerical dominance in the optimization. In other words, the weights will 
% not express the true preferences of the user, and the meaning of the utility 
% function becomes unclear.
% 
% A solution to avoid the numerical dominance is to transform the objective
% functions so that they are constrained to the same approximate range of 
% values. Several function-transformation methods can be used and the 
% \textbf{spsann}-package offers a few of them. The \textit{upper-lower-bound 
% approach} requires the user to inform the maximum (nadir point) and minimum 
% (utopia point) absolute function values. The resulting function values will 
% always range between 0 and 1.
% 
% Using the \textit{upper-bound approach} requires the user to inform only the
% nadir point, while the utopia point is set to zero. The upper-bound approach
% for transformation aims at equalizing only the upper bounds of the objective 
% functions. The resulting function values will always be smaller than or equal
% to 1.
% 
% Sometimes, the absolute maximum and minimum values of an objective function 
% can be calculated exactly. This seems not to be the case of the objective 
% functions implemented in the \textbf{spsann}-package. If the user is 
% uncomfortable with informing the nadir and utopia points, there is the option
% for using \textit{numerical simulations}. It consists of computing the 
% function value for many random sample configurations. The mean function 
% value is used to set the nadir point, while the the utopia point is set to
% zero. This approach is similar to the upper-bound approach, but the function
% values will have the same orders of magnitude only at the starting point of 
% the optimization. Function values larger than one are likely to occur during 
% the optimization. We recommend the user to avoid this approach whenever 
% possible because the effect of the starting point on the optimization as a 
% whole usually is insignificant or arbitrary.
% 
% The \textit{upper-lower-bound approach} with the \textit{Pareto maximum and 
% minimum values} is the most elegant solution to transform the objective 
% functions. However, it is the most time consuming. It works as follows:
% 
% \enumerate{
%   \item Optimize a sample configuration with respect to each objective
%   function that composes the MOOP;
%   \item Compute the function value of every objective function that composes 
%   the MOOP for every optimized sample configuration;
%   \item Record the maximum and minimum absolute function values computed for 
%   each objective function that composes the MOOP -- these are the Pareto
%   maximum and minimum.
% }
% 
% For example, consider that a MOOP is composed of two objective functions: A 
% and B. The minimum absolute value for function A is obtained when the sample
% configuration is optimized with respect to function A. This is the Pareto
% minimum of function A. Consequently, the maximum absolute value for function
% A is obtained when the sample configuration is optimized regarding function
% B. This is the Pareto maximum of function A. The same logic applies for 
% function B.
% 
% \section{Generation mechanism}
% 
% The \textit{generation mechanism} corresponds to the set of formal rules used 
% to randomly perturb the sample configuration to create a new solution out of the
% current one. This is done by adding random noise to the coordinates of one of 
% the sample points, a process known as \textit{jittering}.
% 
% Before we jitter a given sample point, we have to define the maximum quantity 
% of random noise that can be added to its coordinates, i.e. the area within 
% which it can be moved around. In principle, this area corresponds to a rectangle
% centred at the sample point that ignores the presence of non-sampling areas 
% (e.g. buildings and water bodies) and the finiteness of the sampling region. We 
% call this the \textit{neighbourhood}.
% 
% Once we know the size of the neighbourhood, we have to decide upon how much 
% noise will be added to the coordinates of our sampling point, i.e. to choose a
% candidate location in the neighbourhood. This can be done in two different ways.
% We can use an \textit{infinite} set of candidate locations, that is, any 
% location in the neighbourhood can be selected as the candidate location for our 
% sample point. After a candidate location is selected, we check if it falls 
% within the sampling region but does not fall within a non-sampling area. These 
% checks usually are computationally demanding, the reason why this method is not 
% implemented in the \textbf{spsann}-package.
% 
% A more efficient way of selecting a candidate location is to first identify the
% set of \textit{effective} candidate locations for our sample point in the 
% neighbourhood. This can be done using a \textit{finite} set of candidate 
% locations. A finite set of candidate locations is created by discretizing the 
% sampling region beforehand, that is, creating a fine grid of points that serve 
% as candidate locations during the entire search for the optimum sample 
% configuration. This is the least computationally demanding jittering method 
% because, by definition, the candidate location will always fall within the 
% sampling region and out of non-sampling areas.
% 
% Using a finite set of candidate locations has two main disadvantages. First, not
% all locations in the sampling region can enter the sample. The sample points are
% limited to a finite set of regularly spaced candidate locations which is not 
% guaranteed to include the \textit{true} global optimum sample configuration. 
% Second, when a sample point is jittered, it may be that the selected candidate 
% location already is occupied by a sample point. When this happens, another 
% candidate location has to be sought in the neighbourhood because we cannot have 
% more than one sample point in the same location. In the worst case, most (or 
% all) candidate locations in the neighbourhood are already occupied by a sample 
% point -- in general, the more points there are in the sample (or the smaller 
% the size of the neighbourhood (see bellow), the more likely it is that the 
% selected candidate location already is occupied by a sample point. If a 
% candidate location is not found, our sample point is kept in its original 
% location.
% 
% The \textbf{spsann}-package uses a more elegant method based on using a finite
% set of candidate locations coupled with a form of \textit{two-stage random 
% sampling} as implemented in the \texttt{spsample}-function of the 
% \textbf{spcosa}-package \citep{WalvoortEtAl2010}. The fine grid of points that 
% cover the sampling region can be understood as being the centre nodes of a 
% finite set of grid cells (or pixels of a raster image). In the first stage, one 
% of the candidate 'grid cells' is selected with replacement in the neighbourhood,
% i.e. independently of already being occupied by another sample point. The 
% candidate location for our sample point is selected in the second stage within 
% that 'grid cell' by simple random sampling. This method guarantees that a sample
%  point can be placed at \textit{almost} any location within the sampling region.
% It also discards the need to worry if the candidate location already is occupied
% by a sample point, possibly speeding up the computations.
% 
% In order to increase its computational efficiency, the \textbf{spsann}-package 
% uses a decrement function to reduce the size of the neighbourhood as the search
% for the optimum sample configuration evolves. The reason for this is that, as 
% the search evolves and approaches its end, it is likely that moving a sample 
% point over a short distance contributes more to finding the global optimum than 
% moving it over larger distances \citep{GroenigenEtAl1998}. The decrement 
% function determines that the size of the neighbourhood is reduced linearly at 
% the end of each chain $k_i$,
% 
% \begin{equation}
%   x_{max} = x_{max\,0} - k_i / k * x_{max\,0} - x_{min} + x_{dim}
% 
%   y_{max} = y_{max\,0} - k_i / k * y_{max\,0} - y_{min} + y_{dim}
% \end{equation}
% 
% where $x_{max}$ and $y_{max}$ are the dimensions of the neighbourhood in the 
% next chain, i.e. the maximum allowed shifts in the x- and y-coordinates, 
% $x_{max\,0}$ and $y_{max\,0}$ are the dimensions of the neighbourhood in the 
% first chain, $x_{min}$ and $y_{min}$ are the minimum required shifts in the x- 
% and y-coordinates, $x_{dim}$ and $y_{dim}$ are the grid spacings in the x- and 
% y-coordinates, i.e. the grid cell size, and $k$ is the total number of chains.
% The default settings stablish that the size of the neighbourhood in the first
% chain is equal to half the maximum distance in the x- and y-coordinates of the
% entire sampling region, and that the minimum jitter is equal to zero, i.e. that
% the grid cell were the sample point is located can be selected as well. With 
% these settings, at the end of the search, the neighbourhood will be constrained 
% to the set of nine grid cells composed of that in which the sample point falls
% and its eight surrounding grid cells.
% 
% \section{Annealing schedule}
% 
% The \textit{annealing schedule} corresponds to a set of formal rules that 
% determine how the probability of accepting inferior sample configurations is 
% decreased as the search for the globally optimum sample configuration evolves.
% 
 % R-package spsann
\artigofalse
\chapter{R-PACKAGE PEDOMETRICS: PEDOMETRIC TOOLS AND TECHNIQUES}
\label{apen:pedometrics}

\includepdf[pages=-,pagecommand={}]{chap/pedometrics.pdf}
 % R-package pedometrics
\artigofalse
\chapter{INTRODUÇÃO GERAL}
\shorttitle{Introdução Geral}
\label{appen:introduction-pt}

Modern soil spatial modelling is based on using statistical models to explore the empirical relationship among 
environmental conditions and soil properties. These soil spatial models, like any other model, are nothing 
more than a simplification of reality. Unless we observe the soil everywhere -- which would destroy the 
soil and render the observations useless --, no matter how large the volume of data is, or how comprehensive 
our background knowledge, it will \emph{never} be possible to construct a model that explains the entire 
complexity of the soil. Thus, the outcome of a soil spatial model, i.e. a soil map, will \emph{always} deviate 
from the \q{truth} -- this deviation from the \q{truth} is what we call \emph{error}. What a soil map conveys 
is what we expect the soil to be, acknowledging that there is \emph{uncertainty} about it.

Because soil spatial modellers aim at using the available resources to produce the most accurate 
representation of the soil, a sensible research programme is to investigate the main causes for soil maps 
being more or less \emph{uncertain}. There are many sources of uncertainty in soil spatial modelling, such as 
the errors that result from using a poor statistical model or from making interpolations and extrapolations 
to predict soil properties at unvisited locations. Another important source of uncertainty is the data used to 
assess the empirical relationship among environmental conditions and soil properties: covariate and soil data.

The general objective of this thesis is to evaluate important sources of uncertainty in soil spatial modelling 
with emphasis on soil and covariate data. This general objective can be divided into specific objectives and 
their respective research questions:

\begin{enumerate}[label=(\Roman*)]
 \item Determine the suitability of freely available covariates to calibrate soil spatial models.
  \begin{enumerate}[label=(\alph*)]
   \item Does the use of more detailed covariates result in considerably more accurate soil maps?
   \item How does incorporation of spatial dependence in a soil spatial model compares to the gain in 
   prediction accuracy obtained from using more detailed covariates?
   \item Are the answers to these research questions consistent across soil properties?
  \end{enumerate}
 
 \item Identify the factors that determine how field soil spatial modellers select soil observation locations.
  \begin{enumerate}[label=(\alph*)]
   \item Which factors are considered for deciding upon the location of soil observations, and do they have a 
   pedological origin?
   \item Do the factors play the same role along the course of the soil observation process?
   \item Can point pattern analysis help understanding the purposive sampling strategy traditionally employed
   by field soil spatial modellers?
  \end{enumerate}

\item Identify appropriate calibration sample sizes and designs for soil spatial modelling.
\begin{enumerate}[label=(\alph*)]
\item Can the conditioned Latin hypercube sampling algorithm be improved, and does this improvement deliver 
more accurate soil spatial predictions?
\item Which are the most theoretically sound sampling algorithms for spatial trend estimation, variogram 
estimation, and spatial prediction when we know very little about the soil spatial variation?
\item Can these sampling algorithms be used to construct a general purpose sampling algorithm?

%   \item How suboptimal is it to use a sample configuration that was optimized for a different purpose?
%   \item Is the predictive performance of a soil-mapping model estimated using a sample configuration 
%   optimized using heuristics poorer than that of another soil-mapping model whose parameters were estimated 
%   using a sample configuration optimized using an \textit{a priori} knowledge of the model?
%   \item Is it possible to obtain a sample configuration that is efficient in identifying and estimating i) 
%   the spatial trend and ii) the variogram model, and iii) making spatial predictions?
%   \item How does the sample configuration affect the estimated model parameters and thus the conclusions 
% that 
%   can be drawn under the light of the existing conceptual model of pedogenesis?
%   \item Are the answers to the research questions above consistent across sample sizes and soil properties?
  \end{enumerate}
\end{enumerate}

The thesis is composed of eight chapters where each of the above mentioned objectives are met. Although there 
is a logical sequence in their presentation, all chapters were planned so that they could be read separately. 
This means that there is some overlap between them, i.e. repeated information. References to specific sections 
of other chapters using coloured (blue) hyperlinks are common.

\autoref{chap:chap02} is a commented review of the literature on soil spatial modelling and its main sources 
of 
uncertainty. The review starts with a discussion about the efforts made by soil spatial modellers to raise 
awareness about the importance of soil spatial information. These efforts seem to have fuelled a global 
scientific demand for up-to-date, high resolution soil spatial information. The chapter continues with a 
description of soil spatial modelling along human history, suggesting that the goal of producing soil maps 
remains more or less the same since the Neolithic Revolution (ca.~\num{10000}~years). The chapter closes with 
the main sources of uncertainty.

\autoref{chap:chap03} presents the conceptual model of pedogenesis (in Portuguese), which consists 
of a description of the study area that includes an explicit description of soil-forming factors (climate, 
geology, geomorphology, hydrology, land use, and vegetation) and processes that determine the soil 
spatio-temporal distribution. \autoref{chap:chap04} describes the soil data included in the 
\emph{Santa Maria dataset}, which was used to develop the case studies presented in this thesis. The Santa
Maria dataset is composed of $n = 410$ soil observations compiled from studies carried out between \num{2004} 
and \num{2013}. These studies aimed at producing semi-detailed soil and land use maps, and modelling topsoil 
carbon stock and vulnerability to erosion. A comprehensive description of the covariate data included in the 
Santa Maria dataset, and their processing, is given in \autoref{chap:chap05}. Beyond describing 
the data used in the thesis, the goal of these two chapters, along with the conceptual model of pedogenesis, 
is 
to provide the basis for future soil spatial modelling exercises in the study area, and to serve as examples 
for new soil spatial modelling studies developed elsewhere.

Based on an article published in the peer reviewed journal \geoderma, \autoref{chap:chap06} serves 
the purpose of meeting the first objective of the thesis and answering its respective research questions. The 
prediction performance of linear soil spatial models calibrated using covariates (area-class soil maps, land 
use maps, geological maps, digital elevation models, and satellite images) available in two levels of detail 
is evaluated. The influence of taking the spatial dependence of the residuals into account is assessed as 
well. 

\autoref{chap:chap07} presents an approach that aims at helping to understand the purposive 
sampling strategy traditionally employed by field soil modellers, i.e. free survey. This is important because 
many soil spatial modelling projects rely on legacy data, i.e. soil data produced previously and made 
available 
(publicly or not), whose observation locations were purposively selected by soil spatial modellers using 
poorly documented tacit rules. The chapter is designed to answer the research questions of the second 
objective 
of the thesis. Point pattern analysis is used to characterize the spatial sample configuration, 
whereas theories borrowed from Psychology are used to elaborate on the subjective factors involved in 
selecting 
soil observation locations.

Objective 3 and its research questions are addressed in \autoref{chap:chap08} and \autoref{chap:chap09}. In 
\autoref{chap:chap08}, three improved sampling algorithms are compared to the original conditioned Latin 
hypercube sampling algorithm on how they affect geographic coverage, estimated model parameters and prediction 
accuracy. The influence of sample size is also discussed. \autoref{chap:chap09} presents the most efficient 
sampling strategies for spatial trend estimation, variogram estimation, and spatial prediction when we know 
very little about the soil spatial variation. The chapter closes with a new general purpose sampling algorithm 
that aims at the three objectives jointly.

The sequence of eight chapters is closed with \hyperref[chap:chap10]{General Conclusions} where I highlight 
the main research results and contributions of the study. Next, there are two appendices, both devoted to the 
description of the two packages for \texttt{R} developed to support the thesis: \texttt{spsann} 
(\autoref{appen:spsann}) and \texttt{pedometrics} (\autoref{appen:pedometrics}). The first was designed for 
the optimization of sample configurations using spatial simulated annealing. The second includes miscellaneous 
functions that were put together for ease of use. All literature references are presented under a unique list 
of \hyperref[chap:references]{Bibliographic References} at the end of the thesis.

 % Introdução Geral
\artigofalse
\chapter{CONCLUSÃO GERAL}
\shorttitle{Conclusão Geral}
\label{appen:conclusion-pt}
 % Conclusão Geral

\end{document}
