%\listfiles

%% Tipo de documento e a classe a ser usada para sua formatação.
\documentclass[tese,english]{UFRuralRJ}

%% Um tipo específico de monografia pode ser informado como parâmetro opcional:
%\documentclass[tese]{UFRuralRJ}

%% A opção 'openright' força inícios de capítulos em páginas ímpares
%\documentclass[openright]{UFRuralRJ}

%% Use a opção 'twoside' para gerar uma versão frente-e-verso
%\documentclass[twoside]{UFRuralRJ}

%%==============================================================================
%% Pacotes - língua, codificação e fonte
%%==============================================================================

\usepackage[english]{babel}
\usepackage[T1]{fontenc} %% Conjunto de caracteres correto
%\usepackage{times} %% Usar fonte Adobe Times Roman, equivalente à Times New Roman
\usepackage[utf8]{inputenc} %% Para acentuação correta

%%==============================================================================
%% Pacotes - formatação de equações, números, elementos químicos
%%==============================================================================

\usepackage{amsmath,latexsym,amssymb}
\usepackage[range-phrase = --, binary-units = true]{siunitx} %% Sistema Internacional de Unidades
\DeclareSIUnit\pp{pp} % pencentual point

% Access bold symbols in maths mode
\usepackage{bm}

% Elementos químicos
\usepackage[version=4]{mhchem}

%%==============================================================================
%% Pacotes - formatação de figuras
%%==============================================================================

%% Importar figuras corretamente
\usepackage{graphicx}

%% Diretório onde estão as figuras dos capítulos
\graphicspath{{chap/}}

% Im­proved in­ter­face for float­ing ob­jects
\usepackage{float}
\usepackage{wrapfig}

\usepackage{Sweave}

%\bibliographystyle{abbrvnat}
%\setcitestyle{authoryear,round}


%%==============================================================================
%% Pacotes - formatação de hyperlinks
%%==============================================================================
%% Opção 'hidelinks' disponível no pacote 'hyperref' a partir da versão 
%% 2011-02-05  6.82a. 'hidelinks' retira os retângulos do entorno das palavras
%% com links.

\usepackage[%hidelinks%, 
            bookmarksopen=true,linktoc=none,colorlinks=true,
            linkcolor=blue,citecolor=blue,filecolor=magenta,urlcolor=blue,
            pdftitle={Analysis of Sources of Uncertainty in Soil Mapping},
            pdfauthor={Alessandro Samuel-Rosa},
            pdfsubject={Tese de Doutorado},
            pdfkeywords={Pedometria, Modelos, Incerteza}
            ]{hyperref}
\usepackage[hyphenbreaks]{breakurl} % lidar com url longa

%%TODO: Margens conforme MDT UFSM 7ª edição. Corrigir no arquivo UFRuralRJ.cls 
%%      para funcionar a opção twoside *PENDENTE*
%\usepackage[inner = 30mm, outer = 20mm, top = 30mm, bottom = 20mm]{geometry}

%% Se o pacote 'hyperref' acima foi carregado, a linha abaixo corrige um bug na 
%% hora de montar o sumário da lista de figuras e tabelas. Comente a linha se o
%% pacote 'hyperref' não foi carregado.
\input{macros/bugcaption}

%%==============================================================================
%% Pacotes - formatação da bibliografia de acordo com as normas da ABNT
%%==============================================================================

% IMPORTANTE: O pacote 'abntex2cite' precisa, obrigatoriamente, ser carregado
% depois do pacote 'hyperref'
\usepackage[alf]{abntex2cite}
%\renewcommand{\authorcapstyle}{\small}

%%==============================================================================
%% Pacotes - formatação de verbatim
%%==============================================================================
%% O ambiente verbatim é o ambiente onde são inseridos exemplos de código fonte.
%% Está opção adiciona cor de fundo ao ambiente verbatim.

\let\oldv\verbatim
\let\oldendv\endverbatim
\def\verbatim{\par\setbox0\vbox\bgroup\oldv}
\def\endverbatim{\oldendv\egroup\fboxsep0pt 
                 \noindent\colorbox[gray]{0.95}{\usebox0}\par}

%%==============================================================================
%% Packages - other
%%==============================================================================

% Include PDF documents in LaTeX
\usepackage{pdfpages}

% Place selected parts of a document in landscape
\usepackage{lscape}

% Publication quality tables in LaTeX
\usepackage{booktabs}

% Flexible typesetting of table and figure floats using key/value directives
\usepackage{ctable}

% Customising captions in floating environments
\usepackage[font=footnotesize,labelfont=bf,compatibility=false]{caption}

% Support for sub-captions
\usepackage[skip=0pt,position=top,singlelinecheck=off,justification=raggedright, 
            font+=footnotesize]{subcaption}

% A range of dash commands for compound words
\usepackage[shortcuts]{extdash}

% Con­trol lay­out of item­ize, enu­mer­ate, de­scrip­tion
\usepackage{enumitem}

%%==============================================================================
%% User-defined macros
%%==============================================================================

\newcommand{\Rpackage}[1]{\texttt{R}-package \texttt{#1}} % reference to R-packages
\newcommand{\refsec}[1]{\hyperref[sec:#1]{Section \ref{sec:#1}}} % link to a section in the document
\newcommand{\reffig}[1]{\hyperref[fig:#1]{Figure \ref{sec:#1}}} % link to a figure in the document
\newcommand{\scale}[1]{cartographic scale of 1:\num{#1}} % scale
\newcommand{\scales}[2]{cartographic scales of 1:\num{#1} and 1:\num{#2}} % scales
\newcommand{\q}[1]{``#1''} % double quotes
\newcommand{\cited}[1]{\q{#1}} % direct citation
\newcommand{\grass}[1]{GRASS module \texttt{#1}} % GRASS modules
\newcommand{\gdal}[1]{GDAL module \texttt{#1}} % GDAL modules
\newcommand{\saga}[1]{SAGA library \texttt{#1}}  % SAGA libraries
\newcommand{\covar}[1]{\texttt{#1}} % covariates
\newcommand\titlenote[1]{%
 \begingroup
 \renewcommand\thefootnote{}\footnote{#1}%
 \addtocounter{footnote}{-1}%
 \endgroup
}
% \def\citet{\citeonline}
\let\citet\citeonline
{}

% Hyperlinks and URLs
\def\atcorrbug{\href{http://lists.osgeo.org/pipermail/grass-dev/2014-February/067540.html}{bug}} % atcorr bug
\def\baciaparana{\href{http://pt.wikipedia.org/wiki/Bacia_do_Paran\%C3\%A1}{Bacia Sedimentar do Paraná}}
\def\bgs{\href{http://www.bgs.ac.uk/}{BGS}}
\def\cgiar{\href{http://www.cgiar.org/}{CGIAR}} % Consultative Group for International Agricultural Research
\def\dnosgeneral{\href{https://github.com/samuel-rosa/dnos-sm-rs-general/tree/master/data}{GitHub}}
\def\gsif{\href{http://www.isric.org/projects/global-soil-information-facilities-gsif}{GSIF}}
\def\globalsoilmap{\href{http://www.globalsoilmap.net/}{GlobalSoilMap}}
\def\geoderma{\href{http://www.journals.elsevier.com/geoderma/}{Geoderma}} % Geoderma
\def\inpe{\href{http://www.inpe.br/}{INPE}} % INPE
\def\inpedgi{\href{http://www.dgi.inpe.br/siteDgi_EN/index_EN.php}{INPE-DGI}} % INPE-DGI
\def\iso{\href{http://www.iso.org/iso/catalogue_detail.htm?csnumber=13736}{ISO}}
\def\itaara{\href{http://pt.wikipedia.org/wiki/Itaara}{Itaara}}
\def\lagoadospatos{\href{http://pt.wikipedia.org/wiki/Lagoa_dos_Patos}{Lagoa dos Patos}}
\def\mma{\href{http://geocatalogo.ibama.gov.br/}{MMA}}
\def\redemds{\href{https://goo.gl/m8QWUm}{RedeMDS}}
\def\riovacacaimirim{\href{http://pt.wikipedia.org/wiki/Rio_Vacaca\%C3\%AD-Mirim}{Rio Vacacaí-Mirim}}
\def\riojacui{\href{http://pt.wikipedia.org/wiki/Rio_Jacu\%C3\%AD}{Rio Jacuí}}
\def\rioguaiba{\href{http://pt.wikipedia.org/wiki/Lago_Gua\%C3\%ADba}{Rio Guaíba}}
\def\santamaria{\href{http://pt.wikipedia.org/wiki/Santa_Maria_\%28Rio_Grande_do_Sul\%29}{Santa Maria}}
\def\topodata{\href{http://www.dsr.inpe.br/topodata/}{TOPODATA}}
\def\ufsm{\href{http://site.ufsm.br/}{UFSM}}

% Variables
\def\geoNew{\texttt{GEO\_25}}
\def\geoOld{\texttt{GEO\_50}}
\def\soilNew{\texttt{SOIL\_25}}
\def\soilOld{\texttt{SOIL\_100}}
\def\demNew{\texttt{ELEV\_10}}
\def\demOld{\texttt{ELEV\_90}}
\def\landOld{\texttt{LU1980}}
\def\landNew{\texttt{LU2009}}
\def\googleearth{Google Earth\textregistered{}}

%%==============================================================================
%% Identificação do trabalho
%%==============================================================================
\titulo{Analysis of Sources of Uncertainty in Soil Mapping}
\author{Samuel-Rosa}{Alessandro}
\instituto{Instituto de Agronomia}
\curso{Curso de Pós-Graduação em Agronomia -- Ciência do Solo}
\area{Ciência do Solo}
\grau{Doutor em Agronomia -- Ciência do Solo}
\nivel{Doutorado em Agronomia -- Ciência do Solo}
\local{Serop\'edica}{RJ}{Brasil}

%%==============================================================================
%% Identificação dos orientadores
%%==============================================================================
\advisor[Professora]{Drª.}{Anjos}{Lúcia Helena Cunha dos}{UFRRJ}
\orientadoratrue
\coadvisor[Professor]{Dr.}{Vasques}{Gustavo de Mattos}
\coadvisor[Professor]{Dr.}{Heuvelink}{Gerard B M}
\coorientadorestrue

%%==============================================================================
%% Informações sobre a defesa
%%==============================================================================
\committee[Dr.]{Banca}{Melhor Presidente da}{UFRRJ} %% Presidente
\committee[Dra.]{Banca}{Melhor Integrante da}{UFRRJ} %% Examinador
\committee[Dr.]{Banca}{Melhor Integrante da}{UFRRJ} %% Examinador
\committee[Dra.]{Banca}{Outra Melhor Integrante da}{MEFR} %% Examinador
\committee[Dr.]{Banca}{Outro Melhor Integrante da}{MEFR} %% Examinador
\date{14}{September}{2015} %% Data da defesa

%%==============================================================================
%% Outros itens
%%==============================================================================

% \keyword{Pedometria}
% \keyword{Modelos preditivos}
% \keyword{Incerteza}

%%=============================================================================
%% Início do documento
%%=============================================================================
\begin{document}

%%=============================================================================
%% Capa e folha de rosto
%%=============================================================================
\maketitle

%%=============================================================================
%% Ficha catalográfica
%%=============================================================================
% Como a CIP vai ser impressa atrás da página de rosto, as margens inner e outer	
% devem ser invertidas.
%\newgeometry{inner=20mm,outer=30mm,top=30mm,bottom=20mm}
%\makeCIP{alessandrosamuel@yahoo.com.br}% email do autor		
%\restoregeometry

%Se for usar a catalogação gerada pelo gerador do site da biblioteca comentar as linhas
%acima e utilizar o comando abaixo
%\includeCIP{CIP.pdf}

%%=============================================================================
% Folha de aprovação
%%=============================================================================
% \makeapprove

%%=============================================================================
%% Dedicatória (opcional)
%%=============================================================================
% \clearpage\mbox{}\vfill\hspace{80mm}\begin{minipage}{76mm}\begin{flushright}{\em
% Àqueles que financiaram meus estudos...
% \par
% ...DEDICO!
% }\end{flushright}\end{minipage}

%%=============================================================================
%% Agradecimentos ou Prefácio (opcional)
%%=============================================================================
%% Usar versão estrelada do comando 'chapter'.
\chapter*{Preface}

I was never sure about what a thesis should consist of: I worked on so many things during the four years of my 
doctorate that I found myself somewhat lost when I had to decide what to write in the thesis. There are 
official documents suggesting \emph{how} the thesis should be written, but not exactly \emph{what} should be 
written -- I find the definitions somewhat vague. For example, the manual of our university states that a 
\q{thesis consists of the result of a research which is presented as the final requirement for the completion 
of a doctorate}\footnote{\citeonline{UFRRJ2006}}, which is quite the same thing said by the International 
Organization for Standardization (\iso): a \q{document which presents the author's research and findings and 
submitted by him in support of his candidature for a degree or professional 
qualification}\footnote{\citeonline{ISO1986}}. I tried reading other theses to see if I could get an 
inspiration. I also discussed this matter with my patient supervisors Lúcia Anjos (Universidade Federal Rural 
do Rio de Janeiro, Brazil), Gustavo Vasques (Embrapa Solos, Brazil), and Gerard Heuvelink (ISRIC -- World Soil 
Information, the Netherlands). Unfortunately, for one reason or another, I was never satisfied with the 
outcome.

At first, I was a bit desperate. Have I failed? Has everyone failed? I hoped not! Perhaps the lack of an 
objective, ultimate, universal definition of what a thesis should consist of meant that, as a doctorate 
student, it was my responsibility to construct such a definition. This idea gave me back the long-lost 
excitement to write my thesis. I did not want to follow a boring ritual. I wanted to have fun and be completely 
honest with the reader, as Richard Webster\footnote{\citeonline{Webster2003}} had once suggested. 
As such, I started thinking about all steps given since the start of the doctorate, something like \q{Once upon 
a time in Seropédica...}.

As the title says, this thesis is about a research on the factors determining a soil map to be more or less 
accurate, what I call \emph{sources of uncertainty}. Many of these sources are known, others are still unknown, 
and some are disregarded due to our ignorance -- or by convenience. When I wrote my doctorate research project, 
it seemed appropriate to aim at evaluating what I understood as being the main sources of uncertainty in the 
process of building a model to produce soil maps, a process that I call \emph{soil spatial modelling}. The 
reason was simple: soil spatial modelling using modern techniques was a growing activity in Brazilian 
universities and research centres, and I felt that many \emph{soil spatial modellers} were inclined towards 
using the most expensive data sources as the only way of producing higher accuracy soil maps of the Brazilian 
territory. I was preoccupied about these ideas -- which appeared to be sort of an euphoria about new remote 
sensors -- because I believed that high quality soil maps could be produced if we simply started using the data 
at hand.

Defining the main sources of uncertainty in soil spatial modelling required an operational definition, which 
was given based on the observation that, in general, the main decisions made by soil spatial modellers concern 
the a) calibration observations, b) covariates, and c) model structure. The general objective of evaluating 
these three mains sources of uncertainty was then divided into five specific objectives:

\begin{enumerate}[label=(\Roman*)]
\item Identify appropriate calibration sample sizes and designs for soil spatial modelling;

\item Determine the accuracy of freely available covariates and their suitability to calibrate soil spatial 
models;

\item Identify appropriate covariate selection methods to build linear soil spatial models;

\item Assess the effect of multicollinearity among covariates on the performance of linear soil spatial models;

\item Identify database scenarios in which non-linear soil spatial models are more efficient than linear soil 
spatial models.
\end{enumerate}

The idea was to deal with each of the objectives separately and present the results in individual chapters of 
the thesis which would be submitted for publication in peer reviewed journals. The main expected result was the 
definition of a sound \emph{working protocol} that would allow the construction of efficient soil spatial 
models. My goal was to contribute to national (Brazilian Research Network on Digital Soil Mapping -- \redemds) 
and international (\gsm{} and Global Soil Information Facilities -- \gsif) initiatives, while generating a 
significant amount of bibliographic material to support the teaching of modern soil spatial modelling 
techniques in soil classes at Brazilian universities.

\def\footlatex{\footnote{See more about \LaTeX{} in \href{https://en.wikipedia.org/wiki/LaTeX}{Wikipedia}. The 
\LaTeX{} 
class that I have adapted to compile this thesis is available in 
\href{https://github.com/samuel-rosa/UFRuralRJ}{GitHub}.}}

With time it became clear that the five objectives and the expected results were too ambitious. I certainly was 
overwhelmed by the knowledge of the multiple sources of uncertainty, and felt compelled to develop a very 
thorough study. But I forgot that a doctorate includes more activities than those planned in the research 
project: you take classes, prepare grant proposals, write reports, help colleagues, get involved in other 
projects -- such as adapting the \LaTeX{}\footlatex{} class used to compile this thesis --, publish the papers 
of your master thesis, train undergraduate students, create and maintain the newsletter of a scientific group, 
read many articles and books, learn a couple of computer languages, start a relationship, get sick, and so on. 
Then, one day you realize that two years are already gone by and you still are preparing the database with 
which you will develop your case studies.

\def\footr{\footnote{See more about \texttt{R} in 
\href{https://en.wikipedia.org/wiki/R_\%28programming_language\%29}{Wikipedia}.}}

I know that I was particularly lucky for most of the soil and covariate data already being available for my 
use. This is because I have decided very early to continue using the data that I collected during my master so 
that I could go deeper into the details of modern soil spatial modelling techniques. Looking back, I think that 
this was the right decision. However, the resources needed to properly organize the data before I could 
actually use it were considerable. This effort was in line with my original intent of defining a working 
protocol for constructing soil spatial models, which I guess to have achieved, at least partially. Then I 
realized that I also needed to make my research the most reproducible as possible. The way to go was to make a 
thorough description of all data processing steps, including making available all computer scripts so that they 
could be reused by other people. The result was thousands of lines of computer code, mostly on 
\texttt{R}\footr{}, which I used to indirectly access most of other computer programs. These computer scripts 
have shown to be invaluable for my own applications, and I always hoped that other people will find them useful 
as well. But I then learned that many well known methods of data analysis/processing are not used simply 
because they are not implemented in a (single) software package. As such, making only the computer scripts 
available did seem to be a poor solution. Developing and maintaining a software package in the most popular 
environment for data processing and analysis, i.e. \texttt{R}\footr{}, was a natural decision. Although being 
fun, programming took a lot of resources!

A significant amount of resources was also spent preparing a description of the soil-forming factors and 
processes that determine the soil spatio-temporal distribution in the study area where the case studies were to 
be developed. Such a description is what I call \emph{conceptual model of pedogenesis}. This was another effort 
in line with the definition of a working protocol because I believe that soil spatial modelling is not only 
about making maps, but also constructing soil knowledge. Within the scope of the thesis, this knowledge was 
expected to serve the development of an experiment devoted to meeting the third objective of the research 
project. My intent was to compare automated covariate selection methods with the use of expert knowledge. 
Preliminary tests were conducted with a few experts to help planning the experiment, which was believed to be a 
complex one. Preliminary results were encouraging, but since I needed to give more attention to the first and 
second objectives, I had to temporarily stop working on the third objective.

There also was my poor knowledge on some known topics, which sometimes took me to the wrong direction. For 
example, I wanted to evaluate how much more accurate a soil map is when more accurate covariates are used 
(second objective), the reason being that I was concerned with the fact that the covariates too are in error. 
As such, I collected field data to validate the covariates and correct them for any systematic errors. Only 
later, discussing with Gerard, I understood that 1) the validation data was poor, and 2) in soil spatial 
modelling the covariates are generally taken as they are. The latter is like assuming that the covariates were 
measured without error -- otherwise a technique called error propagation analysis (or uncertainty analysis) can 
be employed to take that error into account. As such, the second objective of the research project needed to be 
reformulated in terms of how to define the different covariates that we had at hand. After many discussions we 
still were unable to reach a satisfactory solution, which did not prevent the study from being developed. Quite 
interesting results were produced, but presenting them also was a challenge: many models and covariates had 
been compared, and we wanted to have a summary way of presenting them, preferentially a figure. We came up with 
a figure that we later called a \emph{model series plot}, i.e. a figure that depicts a series of models ordered 
according to some chosen summary performance statistic. For the purpose of our study, that was a useful figure, 
and I hope that the readers have understood how to interpret it. The reviewers of our paper were fundamental 
for improving the description of the model series plot. Fortunately, they were also able to help deciding upon 
a proper definition for the differences observed between the covariates that were being compared. The study was 
not exactly about their accuracy, but about how they were produced and their level of spatial detail.

Devising an experiment to evaluate the influence of sample size and design on the accuracy of soil maps also 
was a challenge. Most soil data used for soil spatial modelling were produced in the past century (legacy data) 
with observation locations purposively selected by soil spatial modellers using tacit rules. As such, I wanted 
to build an algorithm composed of a set of objective decision rules that would produce spatial samples similar 
to those produced by a soil spatial modeller. I would then simulate budget scenarios for sampling and see how 
different spatial samples would perform regarding soil map accuracy. But how to devise such an algorithm? I 
interviewed the soil spatial modellers that produced the soil data to be used to conduct the case studies, 
carried out a point pattern analysis of the resulting spatial sample configuration, and explored psychological 
concepts to understand the whys of the locations of the sampling points. A lengthy study was carried out, which 
provided evidence that many poorly understood factors influence the decision of soil spatial modellers on where 
to make soil observations. From one perspective, this enables one to plan more efficient soil observation 
campaigns. However, it did not help finding a practical solution for the problem that we had at hand. Perhaps 
it was more appropriate to explore the existing, less complex algorithms that produce spatial samples using 
more objective decision rules formulated with basis on conceptual and operational factors.

\def\foottravel{\footnote{See about the \emph{travelling salesman problem} at 
\href{https://en.wikipedia.org/wiki/Travelling_salesman_problem}{Wikipedia}.}}

We then invited Dick Brus (Alterra, the Netherlands) to participate devising the experiment to evaluate the 
influence of sample size and design on the accuracy of soil maps. After some talks and a bibliographic review, 
we decided for using sampling algorithms that are based on the so-called spatial simulated annealing which are 
commonly used to produce spatial samples for soil spatial modelling. The problem was that we did not know of 
spatial simulated annealing being implemented in any free and open source software package in a way that could 
meet the requirements of our study. Again, the solution was to work on our own implementation of spatial 
simulated annealing, which resulted in a second package for R. Having decided the sampling algorithm to 
work with, we needed to choose a sound method to take sampling costs into account. Because the access time to 
sampling points usually is the major cost component in soil sampling, Gerard and I thought of coupling with 
spatial simulated annealing an algorithm to solve the problem of travelling from one sampling point to the next 
with the least cost\foottravel. This would be a good piece of work, but we soon realized that solving the 
travelling problem was impossible given the available resources. The goal of taking sampling costs into account 
ended up being dropped out.

After some time working on the sampling experiment, which at the time seemed simpler than ever, we came to 
learn that the sampling algorithms that we had chosen had weaknesses -- apparently like any algorithm. So, we 
thought that, perhaps, we could improve on those algorithms! A literature review suggested that we were correct 
and there was room for algorithmic improvements. Working on these improvements took a lot of resources, and I 
guess we came up with interesting, sound solutions. We only needed to know if the algorithmic improvements had 
any practical added value before evaluating the influence of sample size and design of prediction accuracy. It 
seemed appropriate to carry out two experiments, the first to evaluate the algorithmic improvements, the second 
to compare the algorithms. Again, the results were promising, specially from the algorithmic point of view. 
With regard to prediction accuracy, we cannot make high claims because the algorithms were tested using a 
single case study. But this gap should be easily filled since we have made our software package freely 
available for anyone to use. The negative side of these important developments is that carrying out the 
experiment to evaluate covariate selection methods became impossible due to the remaining resources available. 
This was a pity because I visited Murray Lark at the the British Geological Survey (\bgs) headquarters in 
Nottingham, UK, to discuss about that experiment.

Like it happened with the third objective, there was not enough time to conduct experiments to meet the 
fourth and fifth objectives. I think the topic of the fourth objective is a very important one, directly 
related to the problem of selecting covariates, which I really wanted to deal with. I would 
have been very happy if I could discuss the topic at least partially, but perhaps partial solutions may not be 
very useful. As for the fifth objective, I believe that developing a sound globally relevant work on the topic 
requires using several datasets, not a single dataset as I explored in my research project.

As a consequence of all these events, at the end of the doctorate, I was able to meet only the first and second 
\emph{scientific} objectives -- \emph{scientific} in the sense of answering research questions -- of my 
doctorate research project. This may seem little but only because the research project was too ambitious. Aside 
from meeting the original scientific objectives, I guess I made other important contributions that one may call 
\emph{technical} contributions -- \emph{technical} in the sense of practical application -- that were part of 
the original goal of defining a working protocol for soil spatial modelling. This includes, for example, 
documenting the soil and covariate data, as well as their processing steps, and describing the soil-forming 
factors that determine the soil spatio-temporal distribution in the study area. I know that these technical 
contributions do not help meeting any of the original scientific objectives. The same applies to the two 
\texttt{R}-packages that I developed and maintained, and the experiment conducted to understand how soil 
spatial modellers decide upon where to observe the soil, which were not originally planned in the research 
project. But I guess this is still valid, perhaps very important, as it seems to be common in any scientific 
research: you end up doing many things that are quite different from those that you were originally planning to 
do.

So... this is, more or less, the \emph{story} of the research that I have carried out in collaboration with my 
supervisors and co-authors during my doctorate. Perhaps one will find this story biased towards the negative 
aspects of my doctorate. This is not entirely false, specially because I think that I generally tend to be 
happier with stories that have more errors than hits -- because I learn more with the former than with the 
latter. As such, I guess there is nothing to write in this thesis other than what I have done during the 
doctorate that is directly and/or indirectly related to the original research project, have it been planned or 
not, be it a technical or scientific contribution, completed or not. This is what I present as the final 
requirement for the completion of the doctorate.

I do hope that my supervisors and co-authors like the work that I have done in the past four years. Lúcia, 
Gustavo, Gerard, and Dick have been so patient, so understanding, so respectful, that I have no words to 
prepare the deserved thanks. Each one of them with a different background, a different life story, a different 
perspective, working in a different part of the world... I learned a lot with them: soil science, mathematics, 
statistics, informatics, English, politics and science, human relations, and much more. What a pleasure 
experience working with the four of you!

I also hope that the outcome of my doctorate is of interest for the supporting institutions, 
because I would never write this thesis without their support. These are:

\begin{itemize}
 \item Universidade Federal Rural do Rio de Janeiro, through the Post-Graduate Course in Agronomy -- Soil 
 Science and Department of Soil Science, for providing a solid soil science education, unconditionally 
 supporting my research, and finding the means to guarantee my participation in several international events;
 
 \item Ministry of Science and Technology of Brazil, through the CNPq Foundation (Process 140720/2012-0), that
 provided a three-year grant without which I would not be able to develop any research at all;
 
 \item Ministry of Education of Brazil, through the CAPES Foundation (Process ID BEX 11677/13-9), that funded
 my one-year stay in the Netherlands, where most of the research was actually developed;
 
 \item ISRIC -- World Soil Information, for unconditionally supporting my research.
 
 \item Embrapa Soils, for supporting my research.
 
 \item Universidade Federal de Santa Maria, through the Department of Soil Science, for supporting my research.

 \item Ministry of the Environment of Brazil, for providing some of the data that we used.
\end{itemize}

I need to note that six individuals gave important contributions during the preparation of the 
data that were used to develop the case studies. Three of them are from the Universidade 
Federal Rural do Rio de Janeiro: Fabio Paes Leme Ferreira, Anastácia Perci Campos de Almeida, and Mauro Antônio 
Homem Antunes. The other three are from the Universidade Federal de Santa Maria: Ricardo Simão Diniz Dalmolin, 
Jean Michel Moura Bueno, and Luis Fernando Chimelo Ruiz. Along with them, I must thank the development teams 
and module/package authors of the many free and open source software and operating system that were used to 
develop the case studies.

Many other individuals have also given some form of scientific and/or technical contribution, be it through 
the exchange of email messages, chatting during a coffee break, a short visit to their home institutions, or 
any other informal occasion. Their contributions were invaluable for usually raising unforeseen questions, 
pointing to extremely helpful references, and helping me seeing the research problems from another perspective.
Special thanks are due to Ad van Oostrum, Andreas Papritz, Bas Kempen, Bradley Miller, Chantal Hendriks, 
Dominique Arrouays, Edgardo Ramos Medeiros, Eloi Carvalho Ribeiro, Jorge Mendes de Jesus, Madlene Nussbaum, 
Marcos Angelini, Murray Lark, Nicolas Saby, Pablo Miguel, Pedro de Souza Calegaro, Richard Webster, Thomas 
Caspari, Tom Hengl, Titia Mulder, and many others that I might have forgotten.

Having the story of a doctorate to tell depends not only on scientific and/or technical contributions,
but also on the unconditional support of family, friends, and colleagues. My family, which gained several new 
members in the last four years, was strong enough to understand my rare and short visits, and the importance 
of a long stay on the other side of the Atlantic. Although my mother (Elaine), father (Adir), and brother 
(Eduardo) have only a vague idea of what I have been working on, their support was incommensurable. 
South-American and European friends were invaluable: Marcos, Indira, Eloi, Thomas, Nina, Jorge, André, Thiago, 
and Andriéli. There was the amazing, coolest of them all, ISRIC family, which I miss very much, lead by 
our dearest Yolanda. There also were my postgraduate colleagues and teachers. But one friend deserves a special 
thanks: Manoeli! Thank you Manoeli -- by destiny or a strange coincidence --, born and raised in the same small 
town, after completing the bachelor and master together in Brazil, we ended up sharing the same house in The 
Netherlands during the doctorate. There, we continued our philosophical chats about all aspects of life, the 
universe, and everything which we started years ago. It was a pleasure to share a full year with such a 
wonderful friend!

Finally, three little Bacchanalian creatures, one having only a vague idea of my work, the other two not 
speaking any human language, deserve my most sincere thanks: Monique, Tupã, and Dindi! Your love was fire 
through my veins!

\begin{flushright}
 Alessandro Samuel Rosa
 
 Seropédica, February 2016.
\end{flushright}


%%=============================================================================
%% Biografia (opcional)
%%=============================================================================
% \chapter*{Biography}
% O autor nasceu, cresceu e escreveu uma tese.

%%=============================================================================
%% Epígrafe (opcional)
%%=============================================================================
% \clearpage\mbox{}\vfill\hspace{80mm}\begin{minipage}{76mm}\begin{flushright}{\em
% ``Fazer é a melhor forma de dizer.''
% \par
% Autor desconhecido
% }\end{flushright}\end{minipage}

%%=============================================================================
%% Resumo geral em português
%%=============================================================================
\def\tituloportugues{Análise das Fontes de Incerteza no Mapeamento do Solo}
\def\chavesportugues{Pedometria, Modelos Preditivos, Incerteza}
\generalabstracttrue
\begin{generalabstract}{brazilian}{\tituloportugues}{\chavesportugues}
A tradução da representação verbal de modelos conceituais que descrevem a 
realidade para a forma de representação matemática destes modelos é um passo 
chave no mapeamento digital do solo (MDS). O desenvolvimento de tais 
representações matemáticas é afetado por diversas fontes de incerteza. Estas 
fontes podem ser classificadas, didaticamente, em três categoriais: a) 
observações de calibração, b) co-variáveis ambientais, e c) estrutura do 
modelo. O objetivo geral desta proposta de pesquisa de doutorado é avaliar 
estas três principais fontes de incerteza no MDS sob diferentes cenários de 
banco de dados relativos ao número de observações de calibração e à qualidade 
das co-variáveis ambientais usadas para ajustas os modelos de MDS. Um banco de
dados de uma área piloto em estudos de MDS no sul do Brasil será usada para 
desenvolver uma série de testes empíricos para alcançar estes objetivos. 
Primeiro será avaliada a qualidade das co-variáveis ambientais disponíveis 
gratuitamente e a sua adequabilidade para o MDS. Em seguida, sete conjuntos 
de observações de calibração com tamanhos variados serão avaliados em relação 
ao seu efeito sobre o desempenho dos modelos de MDS. Métodos de seleção de 
co-variáveis também serão avaliados quanto à sua aderência ao modelo conceitual
de pedogênese. Os efeitos da multicolinearidade sobre os modelos lineares de 
tendência e a solução de ortogonalização serão profundamente avaliados. Por 
fim, mas não menos importante, modelos não-lineares serão avaliados em relação 
a sua capacidade de apresentar melhor desempenho do que modelo lineares em 
cenários de banco de dados pobres. Os resultados permitirão entender como os 
fatores incertos influenciam os modelos de MDS e a definição de um protocolo 
de trabalho apropriado. Isso será uma contribuição direta à Rede Brasileira de 
Pesquisa em Mapeamento Digital do Solo, ao consórcio GlobalSoilMap.net, ao 
Global Soil Information Facilities, e ao projeto da FAO Aliança Global pelo 
Solo. Além disso, permitirá o uso da informação do solo com maior segurança 
sobre a sua confiabilidade, o que é crítico para tomadores de decisão.
\end{generalabstract}

%%=============================================================================
%% Resumo geral em inglês
%%=============================================================================
\def\tituloingles{Analysis of Sources of Uncertainty in Soil Mapping}
\def\chavesingles{Pedometrics, Predictive Models, Uncertainty}
\generalabstracttrue
\begin{generalabstract}{english}{\tituloingles}{\chavesingles}
Translating verbal representations of conceptual models that describe reality 
into mathematical representations of these models is a key step in digital soil 
mapping (DSM). The development of such mathematical representations is affected 
by several sources of uncertainty. These sources can be classified into three 
categories for didactic purposes: a) calibration observations, b) environmental
co-variates and c) model structure. The general objective of this PhD research 
proposal is to evaluate these three main sources of uncertainty in DSM under 
different database scenarios regarding the number of calibration observations 
and the quality of environmental co-variates used to fit DSM models. A database 
from a pilot area in DSM studies in southern Brazil will be used to develop a 
series of empirical tests to meet these objective. First, the quality of freely 
available environmental co-variates and their suitability for DSM will be 
assessed. Next, seven sets of calibration observations of varying sizes will be 
evaluated as to how they affect the performance of DSM models. Co-variate 
selection methods will also be evaluated on how they comply with the conceptual 
model of pedogenesis. Multicollinearity effects on linear trend models and the 
orthogonalization solution will be deeply evaluated. Last, but not least, 
non-linear models will be evaluated if they present better performance than 
linear models in poor database scenarios. The results will enable to understand 
how uncertain factors influence DSM models and the definition of a proper 
working protocol. This will be a direct contribution to the Brazilian Research 
Network on Digital Soil Mapping, the GlobalSoilMap.net consortium, the Global 
Soil Information Facilities, and of FAO's project Global Soil Partnership. 
Besides, it will ensure the use of the soil information with greater confidence 
about its reliability, which is critical for decision makers.
\end{generalabstract}

%% Lista de Ilustrações (opc)
%% Lista de Símbolos (opc)
%% Lista de Anexos e Apêndices (opc)

%%=============================================================================
%% Lista de figuras
%%=============================================================================
\listoffigures

%%=============================================================================
%% Lista de tabelas (comentar se não houver)
%%=============================================================================
\listoftables

%%=============================================================================
%% Lista de Apêndices (comentar se não houver)
%%=============================================================================
\listofappendix

%%=============================================================================
%% Lista de Anexos (comentar se não houver)
%%=============================================================================
%\listofannex

%%=============================================================================
%% Lista de abreviaturas e siglas
%%=============================================================================
%\begin{listofabbrv}{UbiComp}
%   \item [DSM] Digital Soil Mapping
%   \item [UbiComp] Computação Ubíqua
%\end{listofabbrv}

%%=============================================================================
%% Lista de simbolos (opcional)
%%=============================================================================
%Simbolos devem aparecer conforme a ordem em que aparecem no texto
% o parametro deve ser o símbolo mais longo
%\begin{listofsymbols}{teste}
%  \item [$\varnothing$] vazio
%  \item [$\Gamma$]  Gama
%  \item [$\forall$] Para todo
%\end{listofsymbols}

%%=============================================================================
%% Sumário
%%=============================================================================
\tableofcontents

%%=============================================================================
%% Início da tese
%%=============================================================================
\setlength{\baselineskip}{1.5\baselineskip}

% Adiciona cada capitulo
\setcounter{page}{1}
\artigofalse
\chapter{General introduction}
\label{chap:introduction}
%\usepackage[utf8]{inputenc}


\section{Background and motivation}

\subsection{Demand for soil information}
\label{sec:intro-demand}

Soil modellers/mappers have complained for many years about the lack of government funding, not only
in Brazil \cite{Dalmolin1999,Ker1999,KerEtAl2003,Mendonca-SantosEtAl2003,Ramos2003,Espindola2008}, but in
many countries around the world \cite{Basher1997,HarteminkEtAl2008,Grunwald2009,SanchezEtAl2009,Finke2012}.
They pointed out several reasons to explain the general lack of interest in soil information by governments
after the 1980's. But all of them seem to agree on one point: cutting down the budget for soil
modelling/mapping fundamentally was an economic decision in which the general public was not given the
chance to participate \cite{SamuelRosa2012}.

The soil science landscape has changed dramatically since the last decade. Soil science is now 
experiencing a period of renaissance \cite{HarteminkEtAl2008}. Soils are back on the agenda 
\cite{Kempen2011}. Soil is now a hot topic. For example, the United Nations 
(\href{http://www.un.org/apps/news/story.asp?NewsID=49520#.VnremV6alz0}{UN}) declared 5 December 
the World Soil Day and 2015 the International Year of Soils \cited{in an effort to raise awareness 
and promote more sustainable use of this critical resource}. A global consortium, the 
\href{http://www.globalsoilmap.net/}{GlobalSoilMap}, was created with the goal of producing 
\cited{a new digital soil map of the world using state-of-the-art and emerging technologies}. 
The Food and Agriculture Organization (\href{http://www.fao.org/index_en.htm}{FAO}) launched a 
Global Soil Partnership (\href{http://www.fao.org/globalsoilpartnership/}{GSP}) for 
\cited{leading to the adoption of sustainable development goals for soils}. The International 
Union of Soil Sciences (\href{http://www.iuss.org/index.php?article_id=525}{IUSS}) created a working group,
funded by the United States Department of Agriculture 
(\href{http://www.nrcs.usda.gov/wps/portal/nrcs/main/soils/survey/class/}{USDA}) to develop a 
Universal Soil Classification System, \cited{a common language to describe soils that can be used 
internationally}. An Intergovernmental Technical Panel on Soils 
(\href{http://www.fao.org/globalsoilpartnership/intergovernmental-technical-panel-on-soils/}{ITPS})
was composed, counting with soil experts from all regions of the world \cited{to provide 
scientific and technical advice and guidance on global soil issues to the Global Soil Partnership}.
The \href{http://www.gatesfoundation.org/what-we-do/global-development/agricultural-development}{Bill 
\& Melinda Gates} foundation handed out an \SI{18}[\$]~million grant \cited{to map most parts in 
Sub-Sahara Africa, and make all Sub-Saharan Africa data available}. The International Soil 
Reference and Information Centre (ISRIC) launched its Global Soil Information Facilities 
(\href{http://www.isric.org/projects/global-soil-information-facilities-gsif}{GSIF}), a 
\cited{framework for production of open soil data}. In Brazil, soil scientists created the 
Brazilian Network for Research in Digital Soil Mapping (\href{https://goo.gl/m8QWUm}{RedeMDS}) with 
the objective of \cited{generating synergy among Brazilian soil scientists to advance the 
research in digital soil mapping}. And the Federal Court of Accounts held a 
\href{https://www.governancadosolo.gov.br/}{Soil Governance Conference}, where a new National 
Program for Soil Survey and Interpretation of Brazil (\href{https://goo.gl/zbMK24}{PRONASOLOS}) was 
announced.

Most soil modellers/mappers seem to agree that there is a \q{renewed recognition} of the importance 
of soils for humanity and the environment, a global understanding that soil information is needed to solve 
those that have been named as the five major problems of our time: food insecurity, climate change, 
environmental degradation, water scarcity, and threats to the biodiversity \cite{SanchezEtAl2009}. 
This \q{renewed recognition} would be the reason for an increasing worldwide societal demand for 
up-to-date, high resolution soil information \cite{OmutoEtAl2013}. Although it is out of the scope 
of this thesis to analyse whether this worldwide demand exists or not, it is worth pointing that the above 
mentioned initiatives are coordinated by soil scientists themselves, where, again, the general 
public seem not to be involved. Thus, perhaps the current increased funding for soil modelling/mapping and 
soil-related research is more a result of a growing presence of (soil) scientists in all levels of 
public administration (e.g. scientific and technical advisory boards) than of a global understanding
of the importance of soils for humanity and the environment. Otherwise we could expect soil 
degradation to have been diminished, at least marginally, in the last years considering the already 
existing body of knowledge on soil management and conservation \cite{Blanco-CanquiEtAl2010}. A good
example of this \emph{science-driven} demand for soil information is given by \citeonline{Kempen2011} in the 
third paragraph of the introduction of his thesis.

In Brazil, I guess the main reason for restarting the national soil survey program, today called 
PRONASOLOS, which is expected to have a budget of \SI{8}[R\$]~billions, is the same old reason: the
economic pressure of large multinational corporations to occupy the Cerrado and Amazon biomes, 
\q{the last agriculture frontier} \cite{Macarini2005,Silva2005}. One could argue that transforming part of
these biomes into agricultural land is needed to feed a growing world population which is expected 
to reach about nine billion people by \num{2050} \cite{SanchezEtAl2009}. But FAO has already recognized 
that the problem of food insecurity is more due to the lack of political will than to the
lack of food \cite{FAO2005,FAO2009,FAO2015} -- it is well known that about \SI{50}{\percent} of the
produced food is lost, even in rich countries such as Switzerland \cite{BerettaEtAl2013}. The 
Brazilian states that compose \q{the last agriculture frontier}, called MATOPIBA (States of 
Marranhão, Tocantins, Piauí, and Bahia), are among the poorest Brazilian states and have a long 
history of land conflicts due to the conservative development model adopted in Brazil 
\cite{ComissaoPastoraldaTerra2015}. Many Brazilian politicians are in favour of changing the 
Brazilian legislation to easy the acquisition of agricultural land (up to \SI{100000}{\hectare}) by 
multinational corporations in this region, using the discourse that foreign investments are needed 
to solve the major issues in the region \cite{SECOM2015}. Simply put, I believe that the growing 
demand for up-to-date, high resolution soil information in Brazil has an economic motivation which 
is not necessarily beneficial for the general Brazilian population
\cite{ComissaoPastoraldaTerra2015,SECOM2015}. It is in this contradictory scenario that I present this
thesis!

In the next two sections I give an overview of the models of spatial variation used for soil 
modelling/mapping (\refsec{intro-soil-mapping}) and discuss about some of the sources of uncertainty in
soil-mapping projects (\refsec{uncertainty}). Then I present the content of the thesis 
(\refsec{thesis-content}), including the objectives and research questions that guided my work 
during most of the past four years (\refsec{thesis-objectives}), and the study area 
(\refsec{intro-study-area}) and database (\refsec{intro-database}) that I used to develop the case 
studies. I end this \textit{General Introduction} with the outline of the thesis (\refsec{thesis-outline}).

\subsection{Mapping the soil}
\label{sec:intro-soil-mapping}

Technology plays a determinant role on how we perceive the world around us -- see, for example, 
\citeonline{Hartemink2009}. When early farmers, during the Neolithic Revolution, ca.~\num{10000}~years 
ago, first observed that soil properties varied in space, I guess they soon figured out 
that such variation was related to other environmental features and affected crop yields. This 
early, rough, approximate understanding -- a \emph{model} -- of the spatial soil variation was 
fundamental for choosing -- \emph{predicting} -- the most appropriate locations to start and maintain 
human settlements, some of which became great, long-standing empires 
\cite{MazoyerEtAl2008,BrevikEtAl2010,Churchman2010}. Archaeological research provides evidence that 
several of these empires had more formal \emph{soil classification systems} and \emph{soil survey programs}, 
in most cases for taxation purposes \cite{Barrera-BassolsEtAl2003} -- a practice that lasts till today.

A lot happened since the Neolithic Revolution \cite{BrevikEtAl2010} -- from bone to spacecraft, as 
in Kubrick's \href{https://www.youtube.com/watch?v=qtbOmpTnyOc}{\textit{2001: A Space Odyssey}} --, 
and the knowledge constructed with the multiple soil studies was fundamental for the development 
of agriculture and increase of food production -- although many farmers still live in Neolithic 
conditions \cite{MazoyerEtAl2008}. If we adopt an integrative view, soil maps produced during this 
long period of human history seem to fit into what we call today the \emph{discrete model of 
spatial variation}. The discrete model of spatial variation explains the variation of soil 
properties in space using mutually exclusive mapping units that are separated by sharply defined, 
crisp boundaries (i.e. polygons) \cite{Heuvelink1996,Legros2006}. The soil in each mapping unit 
is more or less homogeneous with regard to its properties at the time of mapping. These properties, 
which are generally used to name the mapping unit along with other environmental features, can be 
characterized using one or more direct observations made within the domain of the mapping unit \cite{WebsterEtAl1990,Legros2006}.

A key step was given about \num{130}~years ago with the formalization of the approximate 
understanding of the spatial soil variation using scientific parlance (i.e. mathematization), which 
was continuously enhanced till about \num{75}~year ago \cite{Jenny1941,Florinsky2012}. 
Accordingly, a soil property $s$ at any point in space is a function of the so-called 
\emph{soil-forming factors},

\begin{equation}\label{eqn:intro-clorpt}
s = f(cl, o, r, p, t, \ldots),
\end{equation}

\noindent
i.e. a soil property $s$ is \emph{determined} by the interplay of environmental conditions defined
by climate ($cl$), organisms ($o$), relief ($r$), parent material ($p$), time ($t$), and other unknown 
players ($\ldots$) \cite{Jenny1941}. When \citeonline{Jenny1941} formulated \autoref{eqn:intro-clorpt},
his goal was to organize the large volume of already existing soil data/knowledge by means of 
numerical laws and quantitative theories -- instead of soil maps, taxonomic classifications, and 
soil-forming processes -- to enable treating it mathematically (i.e. using empirical correlation). 
\citeonline{Jenny1941} proposed that solving \autoref{eqn:intro-clorpt} depended on the soil scientist' 
skills to select suitable study areas and locations for making observation. He also knew that direct 
application of \autoref{eqn:intro-clorpt} for soil mapping was impossible at his time because it 
required soil-forming factors to be exhaustively known everywhere. \autoref{eqn:intro-clorpt} 
remains unsolved, but the concept of soil-forming factors was employed in most of the subsequent soil 
studies around the world, resulting in the enhancement of taxonomic classifications, theories about 
soil-forming processes, and production of soil maps using the discrete model of spatial variation 
\cite{Schelling1970,Hudson1992,BockheimEtAl2000,Legros2006,KrasilnikovEtAl2009b,HarteminkEtAl2013}.

Soil mapping using the discrete model of spatial variation and the idea that soil properties were 
determined by soil-forming factors had its weaknesses. Today we point three main weaknesses, which 
were recognized or understood only using postwar scientific/technological developments 
\cite{HeuvelinkEtAl2001,McBratneyEtAl2003,ScullEtAl2003}:

\begin{enumerate}
\item \q{soil bodies} were described as discrete, homogeneous entities,
\item uncertainty (errors) about mapped soil properties was disregarded, and
\item soil-mapping rules could not be shared with others.
\end{enumerate}

\noindent
Such weaknesses are evidenced, for example, by the fact that different soil scientists would produce 
considerably different soil maps without being able to explain why \cite{Legros2006,BazagliaFilhoEtAl2013}.
As such, many soil scientists decided to explore the new developments in the
fields of mathematics, statistics, and informatics, already successfully employed by the mining 
industry \cite{Matheron1969}, to explain the spatial variation of soil properties \cite{WebsterEtAl1990}.

Soil maps would now be produced using a computer, requiring soil-mapping rules to be formalized in the 
form of a computer script, which is the mean used to establish the communication between the soil 
modeller/mapper (a human being) and the data processing environment (a computer). This meant that 
soil-mapping rules could be effectively shared with others -- provided they had a computer. Taking the 
uncertainty about the mapped soil property into account was also possible. The only requirement was that a 
soil property be treated as a stochastic variable \cite{Cressie1993}, i.e. as if it was impossible to be 
completely sure about its absolute value but only to have an idea of its most likely value. As such, the 
definition of mapping units, which was based on using the knowledge of the soil-forming factors, could be 
viewed as a process that aims at minimizing the within-unit variance (and maximizing the between-unit
variance) of that stochastic variable \cite{VoltzEtAl1990}. Accordingly, the most likely value of a
soil property in a given mapping unit is the mean over the multiple observations made in that 
mapping unit \cite{VoltzEtAl1990,Cressie1993}. It follows that the uncertainty about the value of
the mapped soil property at a given location is a measure of how much in error we would be if the true
value is replaced with the mean of all values observed in the mapping unit, i.e. is the same 
everywhere.

The only solution for avoiding the description of \q{soil bodies} as discrete, homogeneous entities 
was to ignore the soil-forming factors and use the \emph{continuous model of spatial variation}.
The continuous model of spatial variation gives a large importance to the existing observation 
locations, stating that the spatial variation of soil properties is gradual and depends only on the 
separation distance between locations, and possibly on a spatial trend defined by the geographic 
coordinates \cite{WebsterEtAl1990,Cressie1993}. In the simplest case, the most likely value of a 
soil property in a given location is defined by a constant mean computed over all observations made 
in the mapping region, plus a random variable with mean zero and (spatial) variance that depends on 
the separation distance between locations \cite{Cressie1993}. The main idea underlying the 
continuous model of spatial variation is that soil properties at nearby locations are more similar 
than at locations that are far apart \cite{WebsterEtAl1990}. Thus, we err less if we replace the
true value of the mapped soil property at a given location with a value observed at a nearby 
location than with a value observed at a distant location. It follows that the uncertainty about 
the mapped soil property is larger the farther from existing soil observations, i.e. it is 
spatially varying \cite{Cressie1993}.

Availability of general-purpose computers fuelled the use and development of the continuous model of
spatial variation, specially in rich European, North American, and Oceanian countries 
\cite{HeuvelinkEtAl2001,McBratneyEtAl2003,ScullEtAl2003}. But limitations in its prediction 
performance and developments in remote sensing and machine-learning algorithms helped many soil 
scientists to understand that the continuous model of spatial variation had limitations too. For 
instance, it is unable to capture abrupt changes in the values of soil properties that occur, for 
example, between agricultural fields, parent materials, land uses, and so on 
\cite{SteinEtAl1988,VoltzEtAl1990}. Because, contrary to the discrete model of spatial variation, the
continuous model of spatial variation largely ignores the existing pedological knowledge 
\cite{Grunwald2009,Lark2012}. Soil scientists also understood that the discrete model of spatial 
variation was more efficient than previously thought \cite{BregtEtAl1987}. First, because it was now
possible to employ Jenny's equation of soil-forming factors for soil mapping using remote sensing 
products as surrogates of the factors of soil formation \cite{MooreEtAl1993}. Second, machine-learning 
algorithms enabled identifying complex spatial patterns that before could only be identified by an 
experienced soil scientist \cite{McKenzieEtAl1999}. The most logical step was to combine the 
strengths of both discrete and continuous models of spatial variation into a single model -- the 
\emph{mixed model of spatial variation} --, that is, inclusion of the existing pedological 
knowledge and consideration of the spatial continuity of soil property values.

The mixed model of spatial variation -- also called regression-kriging, kriging with external drift,
universal kriging, hybrid approach for soil mapping, pedometric mapping, digital soil mapping, 
predictive soil mapping, and so on \cite{Hengl2003,McBratneyEtAl2003,ScullEtAl2003} -- can be 
viewed as a generalization of previously existing models of spatial variation, by which a soil 
property $Y(\boldsymbol{s})$ at a given location $\boldsymbol{s}$ is modelled as the outcome of a 
spatial stochastic process \cite{Cressie1993,HeuvelinkEtAl2001,LarkEtAl2006}. Accordingly, the 
model is composed of \emph{fixed} and \emph{random effects}. The fixed effects, a deterministic 
large-scale spatial trend, $m(\boldsymbol{s})$, describes the portion of the spatial
variation of the soil property that is explained with the factors of soil formation as suggested by 
the empirical correlation calculated using point soil observations and spatially exhaustive 
covariates. The random effects, also known as stochastic residuals or latent variables, 
$e(\boldsymbol{s})$, describe the portion of the spatial variation of the soil property 
that cannot be explained with the covariates but is spatially correlated, the form and degree of 
this spatial correlation possibly being interpreted pedologically \cite{Lark2012}. Thus

\begin{equation}\label{eq:intro-mixed-model}
 Y(\boldsymbol{s}) = m(\boldsymbol{s}) + e(\boldsymbol{s}).
\end{equation}

\autoref{eq:intro-mixed-model} possesses a great flexibility that makes easy to explore newly developed 
statistical and data-mining methods, generally resulting in better performance than the constituent 
models alone, as well as integrating the existing pedological knowledge provided it is translated into a
mathematical form \cite{OdehEtAl1994,OdehEtAl1995,Heuvelink1996,McBratneyEtAl2000,HenglEtAl2004,Lopez-GranadosEtAl2005,WebsterEtAl2007,Grunwald2009,Lark2012}. These features, and the development 
of the Internet, promoted the rapid popularization of the mixed model of spatial variation, and many 
recent large scale soil-mapping projects already successfully employed the mixed model of spatial 
variation \cite{PoggioEtAl2014,NussbaumEtAl2014,HenglEtAl2015}.

Unfortunately, along with the success of the mixed model of spatial variation, came a rupture 
between soil modellers/mappers pertaining to different \q{schools}, building a negative atmosphere in many 
countries (see examples from Brazil [\url{https://groups.google.com/forum/#!forum/soil-mapping}] and
Spain [\url{http://www.madrimasd.org/blogs/universo/2009/10/07/126094}]). On one side, this was 
caused by the (presumptuous) assumption that the mixed model of spatial variation possibly 
represented a \q{paradigm shift} in soil science \cite{McBratneyEtAl2003} and that it is the 
ultimate soil mapping method, superior to all others \cite{MinasnyEtAl2016}. This assumption is 
certainly untrue, specially in many poorer regions where the mixed model of spatial variation seems 
useless because farmers already obtain satisfactory yields and properly manage their soils using 
indigenous/local knowledge 
\cite{Barrera-BassolsEtAl2003,Barrera-BassolsEtAl2006,HillyerEtAl2006,CorreiaEtAl2007,ValeJuniorEtAl2007}.
 Perhaps the strong criticism made against soil scientist that 
refused to adopt the mixed model of spatial variation, using somewhat pejorative arguments 
\cite{HeuvelinkEtAl2001,Mendonca-SantosEtAl2003}, helped building this unfruitful scenario. On 
the other side, soil scientists disliked loosing importance in the research field in which they worked for 
decades, perhaps very afraid of the new technological developments due to their poor knowledge of 
mathematics, statistics, and informatics \cite{Webster2001,SamuelRosa2012}.

Although all that is very common when a new theory or method appears 
\cite{Russell1932,Feyerabend1977,Kuhn2011}, I think we have reached a point in which nothing else 
can be gained by 
playing one soil scientist against the other. Such a (noble) understanding was already shared by 
\citeonline{Jenny1941} when comparing \cited{soil geographers} and \cited{soil functionalists}.
The view that I
try to maintain from beginning to end of this thesis is that, despite the technological developments,
the activity of modelling/mapping the soil remained essentially the same throughout human history. Soil 
maps still serve the same old purpose of representing our limited understanding about the spatial 
organization of the soil in the natural environment in a simplified manner, as well as giving 
insights about how the soil came to be and how they should be managed 
\cite{Jenny1941,Hudson1992,Legros2006,Blanco-CanquiEtAl2010,Grunwald2010}. This is the reason why,
irrespective of the method/model
used to produce soil maps, I use the (integrative) expression \emph{soil mapping}. In support to 
this view, based on the existing body of knowledge on soil mapping and the currently available 
technologies, and believing in the importance of aiming at a more reproducible research, I have 
tentatively defined a general (didactic) sequence of seven steps that I believe should be
followed in a soil-mapping project.

\noindent\textit{Step 0}. Identify a reality or problem entity, the geographic region for which 
there is a demand of spatial soil information. Target soil properties are appointed as well as the 
required accompanying output information (e.g. metadata). Key modelling decisions are taken in this step
such as the support (punctual or areal), spatial resolution (and possibly the cartographic scale),
coordinate reference system, etc. Depending on how well defined the demand is, the model of spatial 
variation can also be specified, i.e. discrete, continuous, or mixed. Data policy is discussed (What data
should the public? How to make data public? How to implement the data policy?) and agreed upon. Finally,
the available infrastructure, budget, time, and workforce are specified so that next steps can be 
appropriately planned as to fulfil the demand.

\noindent\textit{Step 1}. Develop a conceptual model of pedogenesis, a verbal representation of the 
reality or problem entity including the explicit description of soil-forming factors and processes 
that determine the spatio-temporal distribution of soil properties. This requires gathering the most
of the existing environmental information contained in scientific articles, technical reports, 
books, websites, local knowledge, as well as existing maps of the soil, land use, geology, digital 
elevation models, satellite images, aerial photographs, among others. Environmental information is used
to articulate pedogenetic concepts. Provided that any of the existing soil data is available, an 
exploratory data analysis can help unravelling soil-landscape relationships. The poorer the volume of 
existing environmental information, and the less experienced the soil modeller/mapper is, the more 
import an exploratory field campaign is to help understanding the existing soil-landscape relationships.

\noindent\textit{Step 2}. Define the model of spatial variation, a translation of the conceptual 
model of pedogenesis into a set of possible mathematical representations. Depending on 
how well defined the demand was, the model of spatial variation was already specified in 
\textit{Step 0}. Provided the volume of existing environmental information and legacy soil data is 
moderate to large and/or the soil modeller/mapper is very experienced and/or the available budget 
allows carrying out exploratory field campaigns, a single model of spatial variation is 
defined, i.e. discrete, continuous, or mixed. Assuming that the mixed model of spatial variation is 
chosen, the statistical and/or data-mining models that will be used to represent the discrete 
and continuous components are specified, taking into account the feasibility of meeting their 
requirements given the available soil data, infrastructure, budget, time, and workforce. If multiple
models or statistical and/or data-mining models are chosen, the pedological and 
statistical criteria for identifying the best performing model are defined.

\noindent\textit{Step 3}. Prepare the modelling database, a collection of soil and covariate data 
needed to estimate the parameters and validate the chosen statistical and/or data-mining
models. If required, this includes preparing a sampling plan with formally defined selection rules, 
making properly documented field soil observations, and running replicated laboratory analyses. 
Soil data from different sources are harmonized. Covariates are selected using the conceptual model 
of pedogenesis and empirical evidence. Both soil and covariate data are assessed regarding the need 
for non-linear transformations to meet the requirements of the chosen statistical and/or 
data-mining models, and to improve their empirical correlation. Several of this tasks can be (and 
usually are) carried out with the aid of a data processing environment (a computer).

\noindent\textit{Step 4}. Estimate the parameters of the statistical and/or data-mining 
models, a task that essentially depends on translating the set of possible mathematical 
representations of the conceptual model of pedogenesis into a computer representation, that is, a 
computer code or script. Developing a well documented computer code that describes all processing 
steps ease re-design, future consistency checks, correction of mistakes, and 
dissemination/reproducibility. Calibrated models are evaluated using statistical criteria defined in
\textit{Step 2} such as goodness-of-fit measures. Best performing models are evaluated regarding their
tenability (pedological evaluation), which includes visually assessing draft soil maps, and how well
they represent the range of possible mathematical models. Failure in this last assessment suggests 
that the model requires adjustments, possibly more calibration data, or that it can be discarded.

\noindent\textit{Step 5}. Validate the statistical and/or data-mining models, preferentially
predicting the values of the modelled soil properties at a set of independent, probabilistically 
selected observation locations for which the true values are known. If an independent set of 
observation locations is unavailable, validation is performed using leave-one-out cross-validation. 
The best performing model, selected using the statistical criteria defined in \textit{Step 2}, is 
assumed to be the best mathematical representation of the reality under study. If two or more 
models present similar prediction performance and have a considerably different structure, then it 
is assumed that the best mathematical representation of the reality under study is given by the 
aggregated version of these models. If previous steps have already allowed selecting a single best 
performing model, statistical validation is used only to assess model accuracy.

\noindent\textit{Step 6}. Make spatial predictions, the application of the best performing 
model(s) to predict soil properties values at unvisited locations. If demanded, the uncertainty 
about the predicted soil properties values (i.e. the prediction interval) is estimated too. 
Maps of the target soil properties as well as the required accompanying output information 
(e.g. point soil observations, covariate maps, uncertainty maps, metadata, computer scripts) 
are delivered to the users of the soil information, and possibly used to populate a spatial soil 
information system, where they are made available for inspection using different visualization 
techniques. Provided there is infrastructure, budget, time, and workforce available, modelling
steps can be re-designed and the outputs updated at the user request.

\noindent\textit{Step 7}. Reformulate the conceptual model of pedogenesis, the use
of the knowledge gained during the previous steps till the production of the soil property 
maps, which give insights about the reality or problem entity under study, to correct 
and/or improve the description of soil-forming factors and processes that determine the 
spatio-temporal distribution of soil properties. If demanded, the reformulated conceptual model 
of pedogenesis is delivered to the users of the soil information as well to help in scenario analysis
and decision making.

\subsection{Sources of uncertainty}
\label{sec:intro-uncertainty}

Soil-mapping models, like any other model, are nothing more than a gross simplification of reality.
This means that soil-mapping models are unable to explain the spatio-temporal soil variation in 
its entirety, but only a small part of it \cite{Heuvelink1998a,Legros2006}. When we use a 
soil-mapping model to produce continuous representations of soil properties across space and/or time,
i.e. soil maps, these continuous representations will inexorably deviate from the \q{truth}. What 
the soil map presents is our most likely expectation about the soil properties -- not our 
\emph{certainty} about them. The deviation from the \q{truth} is what we call \emph{error}. Many
examples from the soil-mapping literature show that, irrespective of the soil property, soil-mapping
models have a quite variable predictive performance, usually explaining between \SI{15}{\percent} 
(or less) and (rarely more than) \SI{75}{\percent} of the spatio-temporal soil variation
\cite{MooreEtAl1993,OdehEtAl1994,GesslerEtAl1995,McKenzieEtAl1999,GobinEtAl2001,SumflethEtAl2008,SunEtAl2012,ViscarraRosselEtAl2013,NussbaumEtAl2014,HenglEtAl2015,GaschEtAl2015,HeungEtAl2016}.
The main reason for a soil map to be in error is that the background knowledge and data used to 
construct the soil-mapping model is very limited -- we have to try our best with the available 
resources. Unless we observe the soil everywhere, which would destroy the soil and render the 
observations useless, no matter how large the volume of data is, or how comprehensive our background
knowledge is, it will never be possible to construct a model that explains the entire complexity of 
the soil \cite{Tukey1997}. This means that our knowledge about the soil, and the world as a whole, 
will always be only partial \cite{Box1993}. Because we cannot eliminate the uncertainty of a soil 
map, they can always be considered as wrong, the difference being that some might be useful 
\cite{Box1976}.

Soil modellers/mappers aim at producing the most accurate representations of the soil given the 
available resources. Thus, a reasonable research program is the identification of the causes for 
soil maps being more or less uncertain. For instance, the error that results from making 
extrapolations and interpolations to predict the soil properties at unvisited locations is an 
important source of uncertainty \cite{HeuvelinkEtAl1999,RefsgaardEtAl2006}. In the case of the 
data-centred soil-mapping models, such as the mixed model of spatial variation, these errors are 
larger the farther we are from the existing observations. Thus, the most efficient way of reducing 
these errors is to increase the number of observations and improve the spatial coverage of the mapping
region \cite{BrusEtAl2007a}. However, most soil-mapping projects must rely on using only soil 
data produced many years ago 
\cite{KempenEtAl2009,HenglEtAl2014,PoggioEtAl2014,NussbaumEtAl2014,MulderEtAl2016}, when most sampling
locations were chosen by soil modellers/mappers using non-explicit
location rules, usually placing more samples in complex and less known areas \cite{Rossiter2000}.

On the contrary, if the budget of the soil-mapping project includes (additional) sampling, soil 
modellers/mappers have to decide upon the number and spatial configuration of the sample 
\cite{deGruijterEtAl2006,WebsterEtAl2013}. Unfortunately this is not an easy task, except if the 
goal is on modelling/mapping very few soil properties that can be rapidly measured using field 
sensors, and for which a model of spatial (co)variation can be assumed known \cite{MarchantEtAl2006}.
In most cases, several difficult-to-measure soil properties have to be modelled/mapped, many of 
which have a poorly known structure of spatial (co)variation. If using the mixed model of spatial 
variation, the chosen spatial sample configuration has to be appropriate for estimating the spatial 
trend \cite{HenglEtAl2003a,MinasnyEtAl2006b} and the variogram model 
\cite{WarrickEtAl1987,WebsterEtAl1992,Lark2002}, making spatial predictions
\cite{YfantisEtAl1987,WalvoortEtAl2010} and, (cross)validating the model/map \cite{BrusEtAl2011}, four 
often conflicting objectives. Besides, one
must be careful not to decide upon collecting an insufficient or an exaggerated number of samples. 
Sub-sampling can result in soil-mapping models with little utility, while over-sampling can produce
modelling benefits that do not outweigh the sampling costs \cite{vanGroenigenEtAl1999}.

Soil data may not only poorly cover the geographic and/or feature spaces \cite{HenglEtAl2003a}, but
also have significant laboratory and positional errors \cite{NelsonEtAl2011}. Besides, some soil 
properties may naturally contain more errors due to their conceptual definition and analytical 
procedures employed. Take particle-size distribution as an example. First, the errors are propagated
to the fraction obtained by difference (silt). Second, pre-treatments, such as organic matter 
oxidation, can change mineral structure \cite{MikuttaEtAl2005a}. And third, ignoring that the 
particle-size distribution is a compositional datum can introduce bias in the predictions 
\cite{LarkEtAl2007}.

Another important source of uncertainty is the covariates used to calibrate the soil-mapping models.
First, because they contain varying levels of error \cite{HeuvelinkEtAl1989}. These errors derive 
from the various methods of data generation, analytical procedures and, inherent characteristics of 
each site. For example, digital elevation models usually present larger errors in areas with steep 
slopes, rough terrains, and great elevations, and with dense forest cover or urbanized 
\cite{Florinsky1998,Toutin2000,FisherEtAl2006}. Interpolation of elevation data using kriging 
can produce spurious artefacts \cite{HenglEtAl2009} compared to hydrologically consistent 
algorithms \cite{Hutchinson1989}. And stereoscopic correlation techniques produce digital elevation
models with poorer quality than interferometric synthetic aperture radar \cite{HirtEtAl2010}.

Deciding upon which covariates to include in the soil-mapping model is another important source of 
uncertainty. This is specially important now days, when the number of available (uncertain) 
covariates increases every week, many of which possibly being statistically redundant. A more
pedologically sound approach for selecting the covariates consists in eliciting the knowledge of a 
few experts \cite{LarkEtAl2007a}, optimally more than five \cite{MeyerEtAl2001}. An expert is 
every soil modeller/mapper with long practical experience in soil mapping \cite{MeyerEtAl2001}. 
If the mapping region is unknown to the experts, then a sound conceptual model of pedogenesis must be
provided. However, the understanding about the mapping region may be limited to a point that enables
building only a very uncertain conceptual model of pedogenesis. For example, a unstable landscape that
consistently suffers natural and/or anthropogenic alterations is geomorphologically more complex than 
a stable one \cite{Schumm1979}. As the landscape is rejuvenated, thus increasing complexity, establishing 
soil-landscape relationships becomes very difficult \cite{StreckEtAl2008}. Similar uncertainty 
in the conceptual model of pedogenesis can arise in very old, stable surfaces that have gone 
through many environmental modifications, but were not affected by Pleistocene glaciations 
\cite{MckenzieEtAl2006}. These landscapes commonly have polygenetic soils that present properties
reflecting today but ancient vegetation and climate \cite{PainEtAl1995,Ker1998}.

A commonly used approach to select the covariates to enter a soil-mapping model is automated selection.
Because automated selection algorithms are available in most software packages \cite{Harrell2001a}, 
are relatively easy to use \cite{DraperEtAl1971}, deliver satisfactory results, 
\cite{HenglEtAl2004}, and are needed to automate soil-mapping routines \cite{HenglEtAl2014}. The 
main uncertainty here is on which method to use. Some methods analyse all possible combinations of 
covariates. Others simulate the process of natural selection \cite{AndersenEtAl2010}. 
Cross-validation selects covariates that produce the best predictions on test sets 
\cite{GuyonEtAl2003}. Other methods take into account the order in which the covariates are added 
to (forward selection) or removed from (backward elimination) the model \cite{LarkEtAl2007a}. And 
the stepwise method adds and removes covariates until no further addition or removal results in 
significant changes in the model \cite{DraperEtAl1998}. Many other methods exist and have been 
already used in soil-mapping projects \cite{PoggioEtAl2013,NussbaumEtAl2014}. Some prefer to use 
dimensionality reduction techniques to reduce the number of covariates \cite{Massy1965} before 
running the covariate selection algorithm \cite{tenCatenEtAl2011a,HenglEtAl2014}. However, there 
are evidences that the number of problems associated with using automated covariate selection 
methods, and dimensionality reduction techniques, can be greater than the number of advantages 
\cite{FarrarEtAl1967,Jackson1993,Chatfield1995,Edirisooriya1995,Harrell2001a,Jolliffe2002,Peres-NetoEtAl2005,LarkEtAl2007a}. 
The most evident being the fact that each method produces a 
different set of covariates which unnecessarily have any physical or biological relation with the 
soil property being modelled/mapped.
 
Soil modellers/mappers also have to chose the form of the model that will be used to model the 
empirical relation between covariates and soil properties. Early soil-mapping projects were based 
solely on mental models \cite{Hudson1992}. Later, many soil-mapping projects used statistical 
models that assume a linear relation \cite{MooreEtAl1993,OdehEtAl1994}. Developments in 
informatics introduced new forms of modelling this relation. Now days many soil-mapping projects 
employ machine-learning methods such as regression trees, artificial neural networks, random forests, 
support vector machines, among many others \cite{HeungEtAl2016}. The main reason being that these
non-linear models have a greater ability to capture more complex site-specific soil-landscape relations
\cite{Grunwald2009}. In fact, the relation between soil properties and covariates seldom is 
completely linear \cite{McKenzieEtAl1999}. Once again, the problem is that each model produces a 
different soil map, although the validation statistics may suggest that the differences in their 
performance are insignificant \cite{HeungEtAl2016}. A solution is to aggregate the predictions of 
the many models, but this will not necessarily reduce our uncertainty. Besides, machine-learning 
methods are more difficult to implement and interpret, possibly being more prone to error, and can 
be devoid of any pedological information \cite{Grunwald2009}.

As we have seen, the sources of uncertainty in soil-mapping projects are multiple and the list that I
present is far from being comprehensive -- others will do a better job. The main message is that we 
cannot eliminate the uncertainty of a soil map and, as such, our knowledge about the soil will always 
be limited. Despite of this, soil models/maps -- and the entire body of human knowledge -- are still 
needed to guide our every-day actions. So, instead of eliminating uncertainty, the real quest is for 
understanding it and its sources. Because while studying the multiple sources of uncertainty -- the 
\emph{known unknowns} --, instead of bringing them all into light, turning them into sources of error 
which we have a good understanding of -- the \emph{known knowns} --, we end up uncovering other sources
of error that we were not aware of -- the \emph{unknown unknowns} \cite{Wikipedia2015}. If a known 
(or unknown) unknown happens to become a known known, then action can be taken to reduce our 
uncertainty. For example, if we gain knowledge that a source of error has a systematic nature, then a
corrective measure can be taken right away.

A most comprehensive way of dealing with the known unknowns is \emph{error propagation analysis}, 
also called \emph{uncertainty analysis} \cite{HeuvelinkEtAl1989,Taylor1997}. This is done taking
our uncertainty into account through the soil-modelling steps, seeing how it propagates, and 
evaluating its impact on the uncertainty of the output soil map. This would allow us identifying the
main source of uncertainty, so that we could try to take corrective measures to improve its quality.
But such an exercise is cumbersome \cite{NelsonEtAl2011}, and the efforts required may not
outweigh the benefits, this being one of the reasons why it is rarely carried out. Another important 
reason for error propagation analysis being unpopular is the common ignorance and lack of 
understanding -- perhaps prejudice -- about error and uncertainty \cite{Wechsler2003,Heuvelink2005}.
But the most important reason seem to be that statistical packages and data analysis environments do 
not include -- if they ever will or should include -- a simple routine, a button, to run a complete 
uncertainty analysis \cite{HeuvelinkEtAl2006b}.

Since soil maps still are useful, despite being in error, most soil-mapping projects adopt a very
pragmatic approach, and the uncertainty is almost completely ignored \cite{McBratneyEtAl2003,ScullEtAl2003}.
In other words, we assume that all sources of uncertainty are insignificant. 
Positional and analytical errors are disregarded, covariates are taken as certain, and so on. These 
data are used to calibrate a few models, whose parameters are assumed to be estimated without error 
\cite{DiggleEtAl1998}. The soil-mapping model with the best validation statistics, generally 
chosen using (the optimistic) cross-validation, is selected to make spatial predictions at unvisited
locations \cite{BrusEtAl2011}. Errors in the resulting map are regarded as being due to interpolation
error. Most soil-mapping models will output an estimate of this uncertainty, i.e. a measure of what 
we do not know about the modelled/mapped soil property such as the kriging prediction error variance
\cite{HeuvelinkEtAl1989}. But even such a measure is nothing more than a model of our uncertainty 
whose quality needs proper assessment \cite{Goovaerts2001}. Because soil-mapping models that ignore
the spatial autocorrelation of the prediction errors generally are optimistic about our uncertainty,
i.e. they estimate that we known more than we actually do. And geostatistical models can either 
under- or overestimate the uncertainty depending on the available data and modelling decisions 
\cite{Lark2000a}.

Continually ignoring the uncertainty in soil-mapping projects is quite a dangerous choice 
\cite{HeuvelinkEtAl1999}. It can induce the layman -- and even other scientists -- to think that 
soil modellers/mappers produce perfect, complete answers to soil-related issues. The most likely
consequence is what already happened in the end of the 1980's in most countries when funds for
soil-modelling/mapping research were almost completely extinguished 
\cite{Basher1997,Dalmolin1999,Ker1999,Ramos2003,HarteminkEtAl2008,Finke2012}: if the soil map is a perfect
representation of reality, then once it is done, soil modellers/mappers become useless. Nowadays, many 
software packages and data analysis environments include a basic routine to take into account, at 
least, one source of uncertainty \cite{ChristensenEtAl2002,Papritz2015,RibeiroJrEtAl2015}. If a 
more elaborated, problem-specific software is required, then there is the free and open source 
software community. Free access to the scientific literature is guaranteed by many governments, 
universities, and libraries that spend a significant amount of resources every year on subscriptions
and maintenance of digital repositories. Anyone with a true interest in contributing to the body of 
human knowledge must recognize the urgent need to stop neglecting, perhaps consciously denying, what
we already known -- the \emph{unknown knowns} \cite{Zizek2006}.

\section{Content of the thesis}
\label{sec:thesis-content}

\subsection{Objectives and research questions}
\label{sec:thesis-objectives}

There are several sources of uncertainty that affect the development of soil-mapping models. Many 
of this sources are known, others are still unknown, and some are disregarded due to our ignorance
(or by convenience). The list that I gave in the previous section is far from comprehensive, but it 
suggests that we face a complex scenario. Accordingly, when I wrote the research project that
guided the development of the present thesis (2012-2013), it seemed appropriate to aim at evaluating 
the main sources of uncertainty in soil-mapping projects. This required an operational definition, 
which was given based on the observation that, in general, the main decisions made by soil 
modellers/mappers are with regard to the

\begin{enumerate}[label=(\alph*)]
\item calibration observations,
\item covariates, and
\item model structure.
\end{enumerate}

I then divided the general objective of evaluating these three mains sources of uncertainty into five 
specific objectives with their respective research questions:

\begin{enumerate}
\item Identify appropriate calibration sample sizes and designs for soil-mapping projects.\newline
Research question: How calibration sample size and design affect which covariates enter the 
soil-mapping model, its prediction accuracy, and the monetary cost of the soil-mapping project?

\item Determine the accuracy of freely available covariates and their suitability to calibrate 
soil-mapping models.\newline
Research question: How accurate are the freely available covariates and how much uncertainty 
reduction in the soil-mapping models is achieved when more detailed covariates are used?

\item Identify appropriate covariate selection methods to build linear soil-mapping models.\newline
Research question: How covariate selection methods affect which covariates enter the soil-mapping 
model and its prediction accuracy?

\item Assess the effect of multicollinearity among covariates on the performance of linear 
soil-mapping models.\newline
Research question: How strongly correlated are the freely available covariates and is prediction 
accuracy improved when they are transformed to their principal components?

\item Identify database scenarios in which non-linear soil-mapping models are more efficient than 
linear soil-mapping models.\newline
Research question: In which database scenarios do non-linear soil-mapping models have better 
performance than linear soil-mapping models?
\end{enumerate}

The main expected result was the definition of a working protocol that would allows the construction
of soil-mapping models while taking into account the main factors that influence their performance. 
My goal was to contribute to national (Brazilian Research Network on Digital Soil Mapping) and 
international (GlobalSoilMap and Global Soil Information Facilities) initiatives, while generating a
significant amount of bibliographic material support the teaching of modern soil-mapping techniques 
in Brazilian universities, which is still incipient these days.

With time it became clear that the five objectives and the expected results were quite ambitious. 
Perhaps, I was overwhelmed by the knowledge of the multiple sources of uncertainty at the same time 
that the demand for soil information was increasing, and felt compelled to develop a very through 
study of the sources of uncertainty. The reason for my mistake is simple: the unknown unknowns and
the unknown knowns. First, a few unknown sources of error were uncovered during the development of 
this study, and I decided to confront them because they were closely linked to the sources that were
originally planned to be studied. There also was my poor knowledge on some known topics, which 
sometimes took me to the wrong direction, and my blind optimism regarding the available (personal, 
material, financial, and psychological) resources. Finally, I came to learn only late that many 
well known methods of data analysis/processing are not used simply because they are not implemented
in a software package -- programming took a lot of my time. As a consequence of these events, the 
resulting thesis deals only with a modified version of the first and second, and only scratches 
the surface of the third and fifth, above-mentioned specific objectives and their respective 
research questions, i.e.

\begin{enumerate}
\item Determine the suitability of freely available covariates to calibrate soil-mapping models.

	\begin{enumerate}[label=(\alph*)]
	\item Does the use of more detailed covariates result in considerably more accurate soil 
	maps?

	\item How does incorporation of spatial dependence in a soil-mapping model compare to the 
	gain in prediction accuracy obtained with using more detailed covariates?

	\item Are the answers to these research questions consistent across soil properties?
	\end{enumerate}

\item Identify appropriate calibration sample sizes and designs for soil-mapping projects.

	\begin{enumerate}[label=(\alph*)]
	\item Can theoretical and algorithmic improvements on existing spatial sample optimization 
	algorithms improved the performance of soil-mapping models?
	
	\item How suboptimal is it to use a sample configuration that was optimized to a different 
	purpose than that that it is going to be use with?

	\item Is it the predictive performance of a soil-mapping model estimated using a sample 
	configuration optimized using heuristics poorer than that of another soil-mapping model 
	whose parameters were estimated using a sample configuration optimized using an 
	\textit{a priori} knowledge of the model?
	
	\item Is it possible to obtain a sample configuration that is efficient in identifying and
	estimating i) the spatial trend and ii) the variogram model, and iii) making spatial 
	predictions?
	
	\item How does the sample configuration affect the estimated model parameters and thus the
	conclusions that can be drawn under the light of the existing conceptual model of 
	pedogenesis?
	
	\item Are the answers to these research questions consistent across sample sizes and soil 
	properties?
	\end{enumerate}
\end{enumerate}

\subsection{Study area}
\label{sec:intro-study-area}

The study area used to develop the case studies is located in the southern edge of the plateau of the 
Paraná Sedimentary Basin, in the state of Rio Grande do Sul, Brazil. It consists of a small catchment
(about \SI{2000}{\hectare}) that represents \SI{60}{\percent} of the entire catchment of one of the 
water reservoirs of the city of Santa Maria (\reffig{intro-location}). Climate is subtropical humid 
without a dry season (Cfa, Köppen climate classification), with mean annual temperature of 
\SI{19.3}{\celsius} -- Summer temperatures can reach \SI{>40}{\celsius}, while Winter can have negative
temperatures \cite{HeldweinEtAl2009} --, and mean annual precipitation of \SI{1708}{\milli\metre} well
distributed throughout the year \cite{Maluf2000}. The rainfall pattern is advanced, which is 
characterized by high intensity at the beginning of the precipitation event \cite{MehlEtAl2001}. 
High intensity rainfalls occur from the end of Spring till the beginning of Autumn 
\cite{MouraBueno2012}. Relief varies between plain (slope between \num{0} and \SI{3}{\percent}) and
mountainous (slope between \num{45} and \SI{100}{\percent}), with enclosed valleys (elevation 
ranging between \num{139} and \SI{475}{\metre} above sea level) that determine rainfall volume and 
radiative flux on different surfaces.

\begin{figure}[!ht]
    \centering
    \begin{minipage}[b]{95mm}
      \subcaption{}
      \label{fig:brazil}
      \centering
      \includegraphics[width=90mm]{chap01FIG1a}
    \end{minipage}
    \begin{minipage}[b]{95mm}
      \subcaption{}
      \label{fig:points}
      \centering
      \includegraphics[width=90mm]{chap01FIG1b}
    \end{minipage}
  \caption{The study area. The top panel shows the location of the study area in the central region 
  of the southernmost Brazilian state, Rio Grande do Sul. The bottom panel uses a RapidEye image
  of the year of 2012 in perspective to give an overview of the land use and terrain features.
  A large collection of images of the study area are available at the web-page of the group
  \href{http://www.panoramio.com/group/130903}{Águas do Perau} on Panoramio.}
  \label{fig:intro-location}
\end{figure}

The geology of the study area is composed of three geological formations plus Quaternary deposits
\cite{GasparettoEtAl1988,MacielFilho1990,Sartori2009}. Consolidated sedimentary rocks (fluvial 
sandstone -- Caturrita Formation) of the Triassic period appear at elevations bellow \SI{\pm200}{\metre}.
Basic and acid igneous rocks (andesite-basalt and rhyolite-rhyodacite -- Serra Geral Formation) of the
Cretaceous period appear at elevations between \num{\pm200} and \SI{\pm350}{\metre}, and above 
\SI{\pm350}{\metre}, respectively. Between these rocks there are layers of consolidated sedimentary 
rocks (aeolian sandstone -- Botucatu Formation) of the Jurassic period. Unconsolidated colluvial 
deposits of igneous and sedimentary material eroded further upslope during the Quaternary period 
occur at elevations below \SI{\pm300}{\metre}, while recent fluvial deposits can be found close to 
drainages.

Current geomorphology is a result of erosive processes of the Tertiary and Quaternary, wherein the 
landscape sculpting was determined by alternations between humid, semi-arid and arid climates 
\cite{Sartori2009}. Landscape dissection is weak due to the current climate that favours the 
installation and maintenance of an exuberant vegetation \cite{Sartori2009,NascimentoEtAl2010}.
There are three large morphostructural units \cite{GasparettoEtAl1988,NascimentoEtAl2010}:

\begin{enumerate}[label=(\alph*)]
	\item the Planalto, with gently-rolling to slopping relief, composed predominately of 
	denudational forms with flat surfaces, convex tops, and gently rolling convex slopes, often 
	embedded in faults and/or fractures,

	\item the Rebordo do Planalto, with wide altimetric variation, steep slopes and abrupt 
	cliffs, composed predominately of denudational forms with convex, acute tops, presenting 
	straight slopes with large vertical drops often interrupted by terraces, generally embedded
	in faults and/or fractures, and

	\item the Depressão Periférica, composed of aggradational fluvial plains and denudational 
	forms, the last with convex tops and flat surfaces, presenting concave elongated slopes.
\end{enumerate}

The drainage network has a well defined pattern, generally regular, determined by the faults and/or
fractures \cite{Bortoluzzi1974,GasparettoEtAl1988,NascimentoEtAl2010}. In the lower areas, its
configuration is snake-like due to sediment deposition and fluvial erosion 
\cite{PaivaEtAl2001,SutiliEtAl2009}. Here the plain to gently-sloping, long, concave slopes favour the
occurrence of a 
water table close to the surface and the maintenance of perennial water streams. In the upper, 
undulating areas the rectilinear drainage network has little activity in periods of drought. Floods 
are characterized by high flow velocity and volume due to the steep slopes and \SI{\pm120}{\metre} 
vertical drop from the upper to the lower part of the catchment \cite{PaivaEtAl2001,SutiliEtAl2009}.

Land use for agrosylvopastoral production was intense in past times, resulting in severe soil 
degradation. Recent abandonment of many degraded areas allowed the regeneration of natural 
vegetation. The area occupied with forest and secondary vegetation nowadays is \SI{\pm60}{\percent},
specially in difficult to access areas, with steep slopes, shallow and stony soil 
\cite{SamuelRosaEtAl2011a}. Many of these areas are still used for livestock during some periods of
the year, which is negative for natural regeneration \cite{ScheneiderEtAl1978,HackEtAl2005}, and 
are traversed by old and degraded trails that drain rainwater. About \SI{\pm30}{\percent} of the 
area is used for agrosylvopastoral activities year-round, mainly extensive livestock, which relies 
on natural pastures in areas with deeper, less stony soil, and less sloping relief. Only a few 
agrosylvopastoral activities are carried out in areas with steep slops and shallow soil. Urban
settlements occupy only a small portion of the catchment, mainly close to drainages, the largest
settlements being in the southern region.

The soil in the study area started forming during the Triassic and have a strong dependency on the 
parent material \cite{NascimentoEtAl2010}. It is predominantly shallow (\SI{0.5}{\metre}, Neossolo 
and Cambissolo) due to the dominance of steep slopes, which result in soil formation rates being 
close to or lower than soil erosion rates \cite{DalmolinEtAl2006a}. Even in gently-slopping terrain
it is common to find shallow soils as a result of soil degradation \cite{Moser1990, MouraBueno2012}.
Deeper soil (\SI{>1}{\metre}, Argissolo and Planossolo) can be found in the Planalto, in the 
terraces of the Rebordo do Planalto, and in the small hills with gently-rolling slopes and alluvial 
plains of the Depressão Central \cite{Moser1990,MiguelEtAl2012}. Soil texture is finer and more 
homogeneous throughout the soil profile when the soil has developed from igneous rocks, where the 
presence of iron oxides gives a reddish colour to the soil \cite{MiguelEtAl2012}. Soil features in 
the alluvial plains are determined by seasonal or permanent water-logging, sedimentation, or both 
(Planossolo and Neossolo Flúvico). One such feature is the common presence of clay accumulation in
subsurface horizons due to argiluviation, preferential lateral erosion and, possibly, geological
discontinuities \cite{PieriniEtAl2002,MiguelEtAl2012}.

\subsection{Database}
\label{sec:intro-database}

The soil database used to develop the case studies is part of the \emph{Santa Maria dataset} 
(\autoref{apen:database}), a dataset composed of $n = 410$ soil observations made between \num{2004} 
and \num{2013} as part of different research projects. These projects aimed at producing semi-detailed
soil and land use maps (\scale{25000})
\cite{Pedron2005,Miguel2010,SamuelRosaEtAl2011a,MiguelEtAl2012,Samuel-RosaEtAl2013}, and predicting the
vulnerability of the topsoil to erosion \cite{MouraBueno2012,Miguel2013}. The Santa Maria dataset, which is
freely available at ISRIC World Soil Information Service
(\href{http://www.isric.org/data/wosis}{WoSIS}), and is fully documented at the end of this thesis 
(\autoref{apen:database}), is composed of three subsets, but only two of them are used here 
(\reffig{intro-database}).

\begin{figure}[!ht]
\centering
\includegraphics{fig/chap00-intro-database}
\caption{Spatial distribution of the $n = 340$ (black) and $n = 10$ (red) soil observations from 
the Santa Maria dataset used in this thesis. The drainage network is shown in the background to 
give an idea of how sampling locations are related to terrain features.}
\label{fig:intro-database}
\end{figure}

The first dataset ($n = 340$) was produced using purposively selected observation locations (free 
survey). The main goal of the researchers was to obtain a sample that they understood as being 
representative of the different landforms, land uses, and soil taxa present in the study area. They 
also wanted the sample to cover the entire study area. At the observation locations, they defined 
an area of \SI{\approx100}{\metre\squared} within which they opened three soil pits. Soil samples 
were collected up to a depth of \SI{20}{\centi\metre}. The resulting sampling depth varies from 
\num{2} to \SI{20}{\centi\metre}, with a mean of \SI{17}{\centi\metre}. Soil samples from the three 
pits were used to produce a composite sample which was used for laboratory analysis. Georeferencing 
took place in the field using a GNSS signal receiver with a horizontal precision of more than 
\SI{\pm8}{\metre} positioned approximately at the centre of the sampling area. When the GNSS signal 
was affected by vegetation biomass, terrain features and satellite configuration, resulting in a 
horizontal positional error larger than \SI{\pm8}{\metre}, georeferencing was carried out in the 
office using \SI{1}{\metre} spatial resolution Google Earth\textregistered{} satellite image with 
a positional horizontal error of less than \SI{\pm6}{\metre}.

The second dataset ($n = 10$) contains data from the uppermost A horizon of modal soil profiles 
whose locations were purposively selected after the observations of the first dataset had been made, 
and a preliminary area-class soil map had been produced. The researchers aimed at locations that 
they understood as being most representative of the soil mapping units depicted in the area-class 
soil map. A single soil sample was taken from the each described soil horizon and used for 
laboratory analysis. The resulting depth of the uppermost A horizons varies from \num{12} to 
\SI{30}{\centi\metre}, with a mean of \SI{22.6}{\centi\metre}. Georeferencing was carried out as 
described above, the difference being that GNSS signal receiver was positioned at the observation 
location.

Four soil properties contained in the Santa Maria dataset are explored in this thesis: clay content
(CLAY, \si{\gram\per\kilo\gram}), organic carbon content (ORCA, \si{\gram\per\kilo\gram}), effective
cation exchange capacity (ECEC, \si{\milli\mole\per\kilo\gram}), and bulk density (BUDE, 
\si{\mega\gram\per\cubic\metre}). CLAY was determined by the pipette method. ORCA was determined 
using wet digestion. ECEC was calculated as the sum of exchangeable bases plus exchangeable acidity.
BUDE was determined using the core method. The first three soil properties were selected because 
they were expected to present different patterns of spatial variation and correlation with the 
dominant factors of soil formation in the study area: organisms (\textit{O}), relief (\textit{R}), 
and parent material (\textit{P}). CLAY was presumed to have a stronger relation with \textit{P}, 
while ORCA was expected to be more correlated with \textit{O}. Because the soils of the study area 
were strongly eroded due to intense agriculture in the 20th century, both CLAY and ORCA were also 
expected to be closely related with \textit{R}. Finally, ECEC was expected to be strongly correlated
with \textit{P} and \textit{O}, which is supported by its natural relationship with both CLAY and 
ORCA. BUDE was selected because it was the only soil property that approximately follows a normal 
frequency distribution, while the other three have a right-tailed frequency distribution 
(\autoref{intro-soil-properties}). The main weakness of the BUDE data is that it is available at 
only $n = 282$ observations due to soil stoniness.

\begin{figure}[!ht]
\centering
\begin{minipage}[b]{63mm}
\subcaption{}
\centering
\includegraphics{fig/chap00-intro-clay}
\end{minipage}
\begin{minipage}[b]{63mm}
\subcaption{}
\centering
\includegraphics{fig/chap00-intro-orca}
\end{minipage}
\begin{minipage}[b]{63mm}
\subcaption{}
\centering
\includegraphics{fig/chap00-intro-ecec}
\end{minipage}
\begin{minipage}[b]{63mm}
\subcaption{}
\centering
\includegraphics{fig/chap00-intro-bude}
\end{minipage}
\caption{The four soil properties explored in this thesis: (a) clay content, (b) organic carbon 
content, (c) effective cation exchange capacity, and (d) bulk density. Each panel shows the sample 
histogram and summary statistics of the soil properties in their original scale ($\lambda = 1$), as 
well as the theoretical probability density function so that we can assess how good is the fit of 
the normal distribution to the data -- a product of the \Rpackage{pedometrics}.}
\label{fig:intro-soil-properties}
\end{figure}

A list of more than \num{100} covariates were used in this thesis. All of them were derived from 
five freely available sources: area-class soil maps, digital elevation models, geological maps,
land use maps, and orbital images. Each of these sources was explored in two levels of spatial 
detail, here defined by the data sources and/or production methods, which demand different amounts 
of resources (time, workforce, budget, technology, etc.). Specifically, the level of spatial detail 
of a covariate is a function of the components of its production process such as the cartographic 
ratio, spatial sampling support, number and diversity of data sources explored, and quantity of 
spatial data used. This definition of spatial detail is broader than that of spatial resolution
or spatial scale, and should not be confounded with spatial support \cite{WebsterEtAl2007} or 
thematic detail \cite{Rossiter2000}.

The first area-class soil map was published at a \scale{100000} \cite{AzolinEtAl1988}, while the 
second was prepared at the more detailed \scale{25000} \cite{MiguelEtAl2012}. The first digital 
elevation model is the result of interpolating the contour lines of the most recent topographic maps
produced by the Brazilian Army (\scale{25000}) \cite{DSG1980,DSG1992,DSG1992a}, while second 
digital elevation model is the well known SRTM DEM (3 arc-seconds \SI{\approx90}{\metre} spatial 
resolution) produced by NASA’s Jet Propulsion Laboratory in collaboration with the National 
Geospatial-Intelligence Agency \cite{RodriguezEtAl2006}. Data on geology and soil parent material 
comes from most recent geological maps published in the \scales{25000}{50000} 
\cite{MacielFilho1990,GasparettoEtAl1988}. The land use map of the year of 1980 comes from the most 
recent topographic 
map produced by the Brazilian Army already mentioned above. The second land use map refers to the 
years of 2008 and 2009, and was prepared at a \scale{2000} \cite{SamuelRosaEtAl2011a}. The less 
detailed satellite image was acquired by the Landsat-5 Thematic Mapper on December 26, 2010, while
the more detailed satellite image comes from the RapidEye constellation, acquired on November 16, 
2012. A comprehensive description of these covariates and their processing is presented in the 
end of this thesis (\autoref{apen:database}).

\subsection{Outline}
\label{sec:thesis-outline}

This thesis is composed of five chapters: this \emph{General Introduction}, where I present the scope 
of the thesis, three core chapters where I address the objectives and research questions presented
in \refsec{thesis-objectives}, and the last, a \emph{General Conclusion} (\autoref{chap:conclusion})
of the work that I developed during the last four years. The second chapter (\autoref{chap:chapter01}) 
is based on an article published in the peer reviewed journal \geoderma. There I compare linear 
soil-mapping models calibrated using covariates available in two levels of detail, with and without 
taking the spatial dependence of the residuals into account (Questions 1a-c).

The third (\autoref{chap:chapter02}) and fourth chapters (\autoref{chap:chapter03}) are based on 
manuscripts to be submitted to peer reviewed journals, both dealing with the second objective of this
thesis. In the former I evaluate if improving a popular algorithm used to optimize sample configurations
for spatial trend estimation results in more accurate spatial predictions (Questions 2a and 2f). In the
last I compare five methods for designing sample configurations on how they affect estimated model 
parameters and prediction accuracy (Questions 2b and 2e). I also present an algorithm for optimizing 
sample configurations for identifying and estimating the spatial trend and variogram model, and making
spatial predictions (Question 2d). Finally, I evaluate the gain in prediction accuracy when a sample 
configuration is optimized assuming the form of the model to be known (Question 2c).

This thesis also includes a through description of the soil and covariate databases 
(\autoref{apen:database}), and a verbal representation of the conceptual model of pedogenesis 
(\autoref{apen:pedogenesis}), both summarized above in \refsec{database}. \autoref{apen:spsann} contains
a description of the \Rpackage{spsann}, which I designed for the optimization of sample configurations
using spatial simulated annealing. Finally, \autoref{apen:pedometrics} presents a description of the 
\Rpackage{pedometrics}, which includes miscellaneous functions.

All chapters can be read separately, which means that there might be some overlap between them, 
i.e. repeated information. However, there are clear links between this introductory chapter 
and the concluding chapter. \autoref{chap:chapter02}, \autoref{chap:chapter03}, and \autoref{apen:spsann} 
also are closely connected. References to specific sections of other chapters are common. All literature
references are presented under a unique \emph{Bibliographic References} chapter (\autoref{chap:references})
at the end of this thesis.
 % General introduction
% \setcounter{page}{1}
\artigofalse
\chapter{GENERAL INTRODUCTION}
\label{chap:introduction}

Modern soil spatial modelling is based on using statistical models to explore the empirical relationship among 
environmental conditions and soil properties. These soil spatial models, like any other model, are nothing 
more than a gross simplification of reality. Unless we observe the soil everywhere -- which would destroy the 
soil and render the observations useless --, no matter how large the volume of data is, or how comprehensive 
our background knowledge, it will never be possible to construct a model that explains the entire complexity 
of the soil. Thus, the outcome of a soil spatial model, i.e. a soil map, will always deviate from the 
\q{truth} -- this deviation from the \q{truth} is what we call \emph{error}. What a soil map conveys is what 
we expect the soil to be, not our \emph{certainty} about it.

Because soil spatial modellers aim at using the available resources to produce the most accurate 
representation of the soil, a sensible research programme is to investigate the main causes for soil maps 
being more or less \emph{uncertain}. There are many sources of uncertainty in soil spatial modelling, such as 
the errors that result from using a poor statistical model and from making interpolations and extrapolations 
to predict soil properties at unvisited locations. Another important source of uncertainty is the data used to 
assess the empirical relationship among environmental conditions and soil properties: covariate and soil data.

The general objective of this thesis is to evaluate the soil and covariate data as source of uncertainty in 
soil spatial modelling. This general objective can be divided into specific objectives and their respective 
research questions:

\begin{enumerate}
 \item Determine the suitability of freely available covariates to calibrate soil spatial models.
  \begin{enumerate}[label=(\alph*)]
   \item Does the use of more detailed covariates result in considerably more accurate soil maps?
   \item How does incorporation of spatial dependence in a soil spatial model compare to the gain in 
   prediction accuracy obtained with using more detailed covariates?
   \item Are the answers to these research questions consistent across soil properties?
  \end{enumerate}
 
 \item Identify the factors that determine how field soil spatial modellers select soil observation locations.
  \begin{enumerate}[label=(\alph*)]
   \item Do the factors play the same role along the course of the soil observation process?
   \item Do the main criteria employed for deciding upon the location of soil observations have a pedological 
   origin?
   \item Can point pattern analysis help understanding the purposive sampling strategy traditionally employed
   by field soil spatial modellers?
  \end{enumerate}
 
 \item Identify appropriate calibration sample sizes and designs for soil spatial modelling.
  \begin{enumerate}[label=(\alph*)]
   \item Can theoretical and algorithmic improvements on existing spatial sample optimization algorithms 
   improve the performance of soil models?
   \item How suboptimal is it to use a sample configuration that was optimized to a different purpose than it 
   is going to be used for?
   \item Is the predictive performance of a soil-mapping model estimated using a sample configuration 
   optimized using heuristics poorer than that of another soil-mapping model whose parameters were estimated 
   using a sample configuration optimized using an \textit{a priori} knowledge of the model?
   \item Is it possible to obtain a sample configuration that is efficient in identifying and estimating i) 
   the spatial trend and ii) the variogram model, and iii) making spatial predictions?
   \item How does the sample configuration affect the estimated model parameters and thus the conclusions that 
   can be drawn under the light of the existing conceptual model of pedogenesis?
   \item Are the answers to the research questions above consistent across sample sizes and soil properties?
  \end{enumerate}
\end{enumerate}

The thesis is composed of eight chapters where each of the above mentioned objectives are met. Although there 
is a logical sequence in their presentation, all chapters were planned so that they could be read separately. 
This means that there is some overlap between them, i.e. repeated information. References to specific sections 
of other chapters using coloured (blue) hyperlinks are common.

\autoref{chap:chap02} is a commented review of the literature on soil spatial modelling and its main sources 
of uncertainty. The review starts with a discussion about the efforts made by soil spatial modellers to 
raise awareness about the importance of soil spatial information. These efforts seem to have fuelled a global 
scientific demand for up-to-date, high resolution soil spatial information. The chapter continues with a 
description of soil spatial modelling along human history, suggesting that the goal of producing soil maps 
remains more or less the same since the Neolithic Revolution (ca.~\num{10000}~years). The chapter closes with 
the main sources of uncertainty.

\autoref{chap:chap03} presents the conceptual model of pedogenesis (in Portuguese), which consists of a 
description of the study area that includes an explicit description of soil-forming factors (climate, geology, 
geomorphology, hydrology, land use, and vegetation) and processes that determine the soil spatio-temporal 
distribution. \autoref{chap:chap04} describes the soil data included in the \emph{Santa Maria dataset}, which 
was used to develop the case studies presented in this thesis. The Santa Maria dataset is composed of 
$n = 410$ soil observations compiled from studies carried out between \num{2004} and \num{2013}. These studies 
aimed at producing semi-detailed soil and land use maps, and modelling topsoil carbon stock and vulnerability 
to erosion. A comprehensive description of the covariate data included in the Santa Maria dataset, and their 
processing, is given in \autoref{chap:chap05}. The goal of these three chapters is to provide the bases for 
future soil spatial modelling exercises in the study area and to serve as examples for new soil spatial 
modelling studies developed elsewhere.

Based on an article published in the peer reviewed journal \geoderma, \autoref{chap:chap06} serves the purpose 
of meeting the first objective of the thesis and answering its respective research questions. There, the 
prediction performance of linear soil spatial models calibrated using covariates (area-class soil maps, land 
use maps, geological maps, digital elevation models, and orbital images) available in two levels of detail is 
evaluated. The influence of taking the spatial dependence of the residuals into account is assessed as well. 

\autoref{chap:chap07} presents an approach that aims at helping to understand the purposive sampling strategy 
traditionally employed by field soil modellers, i.e. free survey. This is important because most soil 
spatial modelling projects rely on legacy data, i.e. soil data produced many years ago, whose observation 
locations were purposively selected by soil spatial modellers using poorly formalized tacit rules. The 
chapter, designed to answer the research questions of the second objective of the thesis, shows that the 
location of soil observations were strongly determined by subjective elements unrelated to the local 
soil-landscape relationship. Understanding the reasons behind the location of free survey soil observations 
can help soil spatial modellers designing more efficient data-driven sampling strategies.

The results of the experiment devised to evaluate if improving a popular sampling algorithm results in more 
accurate spatial predictions is presented in \autoref{chap:chap08}. The comparison of five sampling algorithms 
on how they affect estimated model parameters and prediction accuracy is presented in \autoref{chap:chap09}. 
The chapter also introduces a new general purpose sampling algorithm. Both chapters contain only partial 
results, and will be the basis of manuscripts to be submitted to peer reviewed journals, both dealing with the 
second objective of this thesis and its respective research questions.

The sequence of eight chapters is closed with a \emph{General Conclusion} where I highlight the main results 
of the research, contributions and merits of the study. Next, there are two appendices, both devoted to the 
description of the two packages for \texttt{R} developed to support the case studies: \texttt{pedometrics} 
(\autoref{apen:pedometrics}) and \texttt{spsann} (\autoref{apen:spsann}). The first includes miscellaneous 
functions that were put together for ease of use. The second was designed for the optimization of sample 
configurations using spatial simulated annealing. All literature references are presented under a unique 
\emph{Bibliographic References} chapter (\autoref{chap:references}) at the end of the thesis.

\selectlanguage{brazilian}
% \artigotrue
\chapter{MODELO CONCEITUAL DE PEDOGÊNESE}
\label{chap:chap02}
%\SweaveUTF8


\def\ptkeys{Província Geológica do Paraná, Bacia do DNOS, Rebordo do Planalto, Fatores de formação do solo, 
Pedogênese}

\begin{chapterabstract}{brazilian}{\ptkeys}
 Este é o resumo em português.
\end{chapterabstract}

\def\enkeys{Paraná Geological Province, DNOS Catchment, Plateau Border, Soil formation factors, Pedogenesis}
  
\begin{chapterabstract}{english}{\enkeys}
 This is the English abstract.
\end{chapterabstract}

\formatchapter

\section{APRESENTAÇÃO}
\label{sec:chap02-apresentacao}

\titlenote{Colaboraram na preparação deste documento: Pablo Miguel (UFPel), Jean Michel Moura Bueno (UFSM), 
Ricardo Simão Diniz Dalmolin (UFSM), Andrisa Balbinot (UFSM), Lúcia Helena Cunha dos Anjos (UFRRJ), Gustavo de 
Mattos Vasques (Embrapa Soils), e Gerard B. M. Heuvelink (ISRIC -- World Soil Information).}

A modelagem espacial do solo inicia com a definição de um \emph{modelo conceitual de pedogênese}. Um modelo 
conceitual de pedogênese constitui uma representação verbal da realidade sob estudo que inclui a descrição 
explícita dos fatores e processos de formação do solo que determinam as características do solo e o seu padrão 
de distribuição espaço-temporal. Isso requer a reunião de toda a informação ambiental disponível e aplicação 
dos conceitos de relação solo paisagem, desenvolvimento do solo em catenas, ou outro modelo teórico de 
explicação da variação espacial do solo.

O presente documento apresenta o modelo conceitual de pedogênese da bacia de captação do reservatório do 
DNOS/CORSAN (Departamento Nacional de Obras de Saneamento/Companhia Riograndense de Saneamento), localizada na 
divisa entre os municípios de \itaara{} (ao norte) e \santamaria{} (ao sul), na porção sul da \baciaparana{},
estado do Rio Grande do Sul, Brasil (\autoref{fig:chap02-location}). A bacia de captação do reservatório do 
DNOS/CORSAN corresponde à cabeceira da bacia hidrográfica do \riovacacaimirim{}, tributário do \riojacui{} e, 
consequentemente, do \rioguaiba{} e da \lagoadospatos{}. A bacia de captação do reservatório do DNOS/CORSAN
cobre uma área de \SI{\pm29}{\square\kilo\metre} e alimenta um reservatório com volume máximo 
de \SI{\pm3800000}{\cubic\metre} em uma área inundada de \SI{0,74}{\square\kilo\metre}. Este reservatório 
contribui com até \SI{30}{\percent} do abastecimento de água da cidade de Santa Maria \cite{Dias2003, 
DillEtAl2004, Miguel2010}.

\begin{figure}[!ht]
\centering
\begin{minipage}[b]{95mm}
\subcaption{}
 % Modelo conceitual de pedogênese
\selectlanguage{english}
% \artigotrue
\chapter{MODELO CONCEITUAL DE PEDOGÊNESE}
\label{chap:chap03}
%\SweaveUTF8


\def\ptkeys{Província Geológica do Paraná. Bacia do DNOS. Rebordo do Planalto. Fatores de formação do solo. 
Pedogênese}

\begin{chapterabstract}{brazilian}{\ptkeys}
O presente documento apresenta o modelo conceitual de pedogênese -- descrição explícita dos fatores e 
processos de formação do solo que determinam as características do solo e o seu padrão de distribuição 
espaço-temporal -- da bacia de captação do reservatório do DNOS/CORSAN (Departamento Nacional de Obras de 
Saneamento/Companhia Riograndense de Saneamento), localizada no sul do Brasil. O clima é 
subtropical úmido sem estação seca definida. O relevo é plano a montanhoso (variação de altitude entre 139 e 
\SI{475}{\m}), com vales encaixados que influenciam a precipitação e o fluxo radiativo nas diferentes 
superfícies geomórficas. A geologia é constituída pela sequência de três formações geológicas: rochas 
sedimentares 
(arenito fluvial), seguidas de rochas ígneas (basaltos-andesitos toleíticos e vitrófilos, riólitos-riodacitos 
granofíricos) intercaladas por rochas sedimentares (arenito eólico). Depósitos do Quaternário aparecem nas 
partes mais baixas. A geomorfologia atual é resultado dos processos erosivos do Terciário e Quaternário. A 
dissecação atual é fraca devido ao clima que favorece a instalação e permanência de vegetação exuberante. Três 
unidades morfoestruturais são identificadas: no topo, o Planalto, com relevo suave-ondulado a ondulado, 
seguido pelo Rebordo do Planalto, com ampla variação altimétrica, declividade acentuada e escarpas abruptas; na 
base, a Depressão Periférica, com formas agradacionais de planície fluvial. Nas partes altas, a rede de 
drenagem apresenta padrão bem definido, geralmente retangular, determinada pelas falhas e/ou fraturas. Já nas 
áreas mais baixas, devido aos processos de deposição sedimentar e erosão fluvial, sua configuração é sinuosa. 
Ali 
encontram-se um lençol freático próximo da superfície do solo e cursos de água perenes. O uso da terra para 
produção agrossilvopastoril foi intenso em tempos pretéritos e resultou em forte degradação do solo. O 
abandono 
de muitas áreas degradadas permitiu a regeneração da vegetação natural, resultando na atual ocupação com 
florestas e vegetação secundária de \SI{\pm60}{\percent}. Em geral, o solo é pouco profundo devido ao 
predomínio de condições de forte declividade. É comum encontrar solo raso mesmo em áreas de maior estabilidade 
como fruto da degradação pelo uso agrícola. O solo é mais profundo no Planalto, nos terraços do Rebordo, nas 
coxilhas (colinas) de relevo suave-ondulado a ondulado, e nas planícies aluviais. A textura é mais fina e 
homogênea ao 
longo do perfil quando desenvolvido a partir de rochas vulcânicas. As características do solo nas planícies 
aluviais são determinadas pela presença constante de lençol freático próximo da superfície.
\end{chapterabstract}

% \def\enkeys{Paraná Geological Province, DNOS Catchment, Plateau Border, Soil formation factors, Pedogenesis}
%   
% \begin{chapterabstract}{english}{\enkeys}
% This document presents the conceptual model of pedogenesis -- an explicit description of soil-forming 
% factors and processes that determine the spatio-temporal distribution of soil properties  -- of the 
% catchment of the DNOS/CORSAN reservoir, located in southern Brazil. Climate is subtropical humid without a 
% dry season. Relief varies between plain and mountainous, with enclosed valleys (elevation ranging between 
% \num{139} and \SI{475}{\metre} above sea level) that determine rainfall volume and radiative flux on 
% different surfaces. The geology is composed of a sequence of three geological formations: consolidated 
% sedimentary rocks (fluvial sandstone), followed by basic and acid igneous rocks (andesite-basalt and 
% rhyolite-rhyodacite), interlayered with consolidated sedimentary rocks (aeolian sandstone). Unconsolidated 
% colluvial deposits of the Quaternary period occur in the lower portions of the landscape. Current 
% geomorphology is a result of erosive processes of the Tertiary and Quaternary. Landscape dissection is weak 
% due to the current climate that favours the installation and maintenance of an exuberant vegetation. There 
% are three morphostructural units: at the top, the \textit{Planalto} (Plateau), with gently-rolling to 
% sloping relief, followed by the \textit{Rebordo do Planalto} (Plateau Border), with wide altimetric 
% variation, steep slopes and abrupt cliffs; at the bottom, the \textit{Depressão Periférica} (Peripheral 
% Depression), composed of aggradational fluvial plains. In higher altitudes, the drainage network has a well 
% defined pattern, generally rectangular, determined by the faults and/or fractures. In the lower areas, its 
% configuration is sinuous due to sediment deposition and fluvial erosion, with the presence of water table 
% close to the surface and perennial water streams. Land use for agrosilvopastoral production was intense in 
% past times, resulting in severe soil degradation. Recent abandonment of many degraded areas allowed the 
% regeneration of natural vegetation, resulting in \SI{\pm60}{\percent} of the area being now occupied with 
% forest and secondary vegetation. The soil is predominantly shallow due to the dominance of steep slopes. 
% Even in gently-sloping terrain it is common to find shallow soils as a result of soil degradation, Deeper 
% soil can be found in the Planalto, in the terraces of the Rebordo, and in the small hills with 
% gently-rolling slopes and alluvial plains. Soil texture is finer and more homogeneous throughout the soil 
% profile in soil developed from igneous rocks. Soil features in the alluvial plains are determined by the 
% constant presence of the water table close to the surface.
% \end{chapterabstract}

\formatchapter

\section{APRESENTAÇÃO}
\label{sec:chap03-apresentacao}

\titlenote{Colaboraram na preparação deste documento: Pablo Miguel (UFPel), Jean Michel Moura Bueno (UFSM), 
Ricardo Simão Diniz Dalmolin (UFSM), Andrisa Balbinot (UFSM), Lúcia Helena Cunha dos Anjos (UFRRJ), Gustavo de 
Mattos Vasques (Embrapa Soils), e Gerard B. M. Heuvelink (ISRIC -- World Soil Information).}

A modelagem espacial do solo inicia com a definição de um \emph{modelo conceitual de pedogênese}. Um modelo 
conceitual de pedogênese constitui uma representação verbal da realidade sob estudo que inclui a descrição 
explícita dos fatores e processos de formação do solo que determinam as características do solo e o seu padrão 
de distribuição espaço-temporal. Isso requer a reunião de toda a informação ambiental disponível e aplicação 
dos conceitos de relação solo-paisagem, desenvolvimento do solo em catenas, ou outro modelo teórico de 
explicação da variação espacial do solo.

O presente documento apresenta o modelo conceitual de pedogênese da bacia de captação do reservatório do 
DNOS/CORSAN (Departamento Nacional de Obras de Saneamento/Companhia Riograndense de Saneamento), localizada na 
divisa entre os municípios de \itaara{} (ao norte) e \santamaria{} (ao sul), na porção sul da \baciaparana{},
estado do Rio Grande do Sul, Brasil (\autoref{fig:chap03-location}). A bacia de captação do reservatório do 
DNOS/CORSAN corresponde à cabeceira da bacia hidrográfica do \riovacacaimirim{}, tributário do \riojacui{} e, 
consequentemente, do \rioguaiba{} e da \lagoadospatos{}. A bacia de captação do reservatório do DNOS/CORSAN
cobre uma área de \SI{\pm29}{\square\kilo\metre} e alimenta um reservatório com volume máximo 
de \SI{\pm3800000}{\cubic\metre} em uma área inundada de \SI{0,74}{\square\kilo\metre}. Este reservatório 
contribui com até \SI{30}{\percent} do abastecimento de água da cidade de Santa Maria \cite{Dias2003, 
DillEtAl2004, Miguel2010}.

\begin{figure}[!ht]
\centering
\begin{minipage}[b]{95mm}
\subcaption{}
 % The Santa Maria dataset -- soil data
% \artigotrue
\chapter{THE SANTA MARIA DATASET. PART I -- SOIL DATA}
\chapternote{Collaborated in the preparation of this manuscript: Pablo Miguel (UFPel), Jean Michel Moura Bueno 
(UFSM), Ricardo Simão Diniz Dalmolin (UFSM), Andrisa Balbinot (UFSM), Lúcia Helena Cunha dos Anjos (UFRRJ), 
Gustavo de Mattos Vasques (Embrapa Solos), Gerard B. M. Heuvelink (ISRIC -- World Soil Information), and Ad 
van Oostrum (ISRIC -- World Soil Information).}
\shorttitle{Soil Data}
\label{chap:chap04}
%\SweaveUTF8


%\def\ptkeys{}
%\begin{chapterabstract}{brazilian}{\ptkeys}
% Este é o resumo em português.
%\end{chapterabstract}

\def\enkeys{Spatial soil modelling. Purposive sampling. Legacy data. Soil field description. Soil laboratory 
analysis}
  
\begin{chapterabstract}{english}{\enkeys}
The Santa Maria dataset comprises soil data from $n = 410$ point soil observations made between \num{2008} and 
\num{2013} in the catchment of the reservoir of the \textit{Departamento Nacional de Obras de 
Saneamento}-\textit{Companhia Riograndense de Saneamento} (DNOS-CORSAN), located in the southern Brazilian 
state of Rio Grande do Sul. These soil data were produced during the development of research projects that 
aimed at producing semi-detailed soil and land use maps, and predicting topsoil carbon stock and vulnerability 
to erosion. All observation locations were selected purposively or by convenience. Several environmental 
features were described at the observation locations, such as land use, geology, soil classification, slope, 
drainage condition, presence of coarse fragments and rock outcrops, soil coverage with vegetation, among 
other peculiarities of each observation location that were not recorded in a systematic way. Soil samples were 
submitted to laboratory analysis to determine the soil organic carbon content, particle size distribution, 
bulk density, and the content of exchangeable bases (calcium, magnesium, potassium, and sodium) and acidity. 
The effective cation exchange capacity was calculated as the sum of exchangeable bases and acidity. The soil 
data is freely available in a repository hosted in GitHub. These include the identification of all observation 
locations, their geographic coordinates, and field and laboratory data. The number of laboratory replicates 
and the sample standard deviation is provided as well.
\end{chapterabstract}

\formatchapter

\section{INTRODUCTION}
\label{sec:chap04-introduction}

This manuscript presents a description of the soil data contained in the \emph{Santa Maria dataset}. The Santa 
Maria dataset comprises soil data from $n = 410$ soil observations made between \num{2004} and \num{2013} in 
the catchment of the reservoir of the \textit{Departamento Nacional de Obras de Saneamento}-\textit{Companhia 
Riograndense de Saneamento} (DNOS-CORSAN), henceforth called \emph{DNOS catchment}, located in the southern 
border of the Plateau of the Paraná Sedimentary Basin, in the city of Santa Maria, state of Rio Grande do Sul, 
Brazil.

The soil observations cover the northern sector of the DNOS catchment -- an area of \SI{\pm2000}{\hectare}, 
which corresponds to \SI{\pm60}{\percent} of the entire catchment. These soil data were produced during the 
development of research projects that aimed at producing semi-detailed soil and land use maps (\scale{25000}) 
\cite{Pedron2005, Miguel2010, SamuelRosaEtAl2011a, MiguelEtAl2012}, and predicting topsoil carbon stock and 
vulnerability to erosion \cite{Samuel-Rosa2009, MouraBueno2012, Miguel2013}.

% Footnote %%%%%
\def\foottropics{\footnote{The reader should be aware that soil science evolved in Brazil following a somewhat 
different pathway than in the countries of the northern hemisphere due to the specific soil features of 
tropical and subtropical regions. Methods have been adapted along the years, possibly leading to nomenclature 
mismatches. The reader is invited to contribute to solve any problems in this document.}}

The description presented in this manuscript includes the procedures for soil sampling and description, as 
well as the analytical methods employed\foottropics{}. Soil data is described using summary plots with 
descriptive statistics. Finally, a description of the structure of the database and of how it can be accessed 
and used is presented. 

\section{FIELD SAMPLING}
\label{sec:chap04-sampling}

The Santa Maria dataset is composed of three subsets which are described in the next three sections. Together, 
these subsets yield a sampling density of about \num{\pm0.18}~observations per hectare, with an average 
separation distance between two neighbouring points of \SI{180}{\metre}, minimum and maximum separation 
distances of \num{18} and \SI{328}{\metre}, \SI{95}{\percent} of neighbouring observations being separated by 
more than \SI{49}{\metre}.

\subsection{Subset I}

The first subset ($n = 340$, \autoref{fig:chap04-subsets-I-III}) was produced between 2008 and 20011 as part 
of projects that aimed at producing semi-detailed soil and land use maps, and predicting topsoil carbon stock 
and vulnerability to erosion \cite{Samuel-Rosa2009, SamuelRosaEtAl2011a, MiguelEtAl2012, Moura-BuenoEtAl2012, 
Samuel-RosaEtAl2013}. The researchers faced several difficulties with a budget cut and shortage of workforce. 
They also had restricted access to several areas due to geographic barriers and prohibition of access by some 
landowners. These difficulties forced the researchers to reduce the originally aimed number of observations 
($n = 500$) during the development of the project.

All observation locations were selected purposively or by convenience. Tacit knowledge 
(\autoref{sec:chap02-discrete} and \autoref{sec:chap07-elicited}) was the main tool to choose the observation 
locations, a process that was carried out in the office using \SI{1}{\metre} spatial resolution Google 
Earth\rr{} imagery of the years of \num{2008} and \num{2009}. The main goal of the researchers was to obtain a 
sample that they understood as being representative of the different landforms, land uses, and soil taxa 
present in the DNOS catchment. They also wanted the observations to be spread throughout the entire DNOS 
catchment.

% Footnote %%%%%
\def\footsupport{\footnote{\emph{Sample support} refers to the shape, size and orientation of sampling units, 
the latter being the smallest single entity that we are able or choose to observe in the universe of interest, 
i.e. the sampling region. A discrete universe such as a forest is defined by the collection of these single 
entities. However, by definition, such single entities have no real, physical existence in continuous 
universes such as the soil -- their \q{existence} require our prior, more or less arbitrary, definition of 
their shape, size and orientation. This definition is usually based on theoretical and practical 
considerations. For example, the sampling unit can be defined as a roughly polygonal block that is large 
enough to encompass the pattern of small-scale local vertical (\SI{\leq2}{\m}) and horizontal 
(1--\SI{10}{\square\metre}) variability of soil properties -- a pedon. Depending on the size of the sampling 
unit relative to the universe of interest, the sample support is referred to as \emph{areal} or \emph{point} 
sample support. A pedon of \SI{10}{\square\metre} area observed in an agricultural field of \SI{1}{\hectare} 
corresponds to the \emph{areal sample support}. However, the same pedon observed in a catchment of 
\SI{200}{\hectare} would correspond to the \emph{point sample support}.}}

At the observation locations, the researchers defined an area of \SI{\pm100}{\metre\squared} within which they 
opened three soil pits up to a depth of \SI{20}{\centi\metre}. Soil samples were collected up to a depth of 
\SI{20}{\centi\metre}, the depth being measured with a ruler. The resulting sampling depth of Subset I varies 
from \num{2} to \SI{20}{\centi\metre}, with an average of \SI{17}{\centi\metre}. This variation of the 
vertical sampling support\footsupport{} was not a problem for the researchers because their goal was to sample 
the \emph{topsoil}. The topsoil was defined as the topmost soil layer, with a thickness equal or inferior to 
\SI{20}{\centi\metre}, being the soil layer most susceptible to degradation induced by poor agricultural 
practices and land use changes.

Soil samples from the three pits opened in each sampling area were used to produce a composite sample which 
was used for laboratory analyses. Subsurface soil features were observed with an auger in each pit, and the 
average (continuous variables) or most common (categorical variables) value recorded. Note that soil sampling 
was done using an areal horizontal support -- an area of \SI{\pm100}{\metre\squared}. However, the shape and 
exact area of the sampling units are unknown, and georeferencing took place at point support.

\begin{figure}[!ht]
\centering
\includegraphics[width=0.90\textwidth]{fig/chap04-subsets-I-III}
\caption[Spatial distribution of \emph{Subset I} and \emph{Subset III}.]{Spatial distribution of the soil 
observations contained in \emph{Subset I} ($n = 340$, black solid circles) and \emph{Subset III} ($n = 
10$, red stars) of the Santa Maria dataset. The drainage network is shown in the background (blue 
dashed line) to give an idea of how the locations of soil observations is related to terrain features.}
\label{fig:chap04-subsets-I-III}
\end{figure}

Georeferencing was done in the field using a Global Navigation Satellite System (GNSS) receiver with a 
horizontal positional error of less than \SI{8}{\metre} positioned approximately at the centre of the
sampling area. Sometimes, the horizontal positional error was larger than \SI{8}{\metre} due the effects
of vegetation, terrain, and satellite configuration. In these cases, observation locations were georeferenced 
in the office using \SI{1}{\metre} spatial resolution Google Earth\rr{} imagery with positional horizontal 
error of \SI{6}{\metre} (\autoref{tab:chap05-google-geo-val}).

Every observation was identified with a number in increasing order, following the order in which the 
observations were made (\num{001}--\num{340}). A total of \num{17}~field campaigns were carried out, yielding 
an observation density of about \num{18}~observations per \si{\kilo\metre\squared} (\autoref{chap:chap07}).

\subsection{Subset II}
\label{sec:chap04-subset-ii}

The second subset ($n = 60$, \autoref{fig:chap04-subset-II}) was produced in the years \num{2012} and 
\num{2013}, and was intended to constitute an independent dataset for validation purposes. Because of the many 
access limitations (geographic barriers and prohibition by landowners) and shortage of workforce, budget, 
infrastructure and time faced in previous field campaigns, researchers chose to employ transect (cluster) 
sampling \cite{MiguelEtAl2012, Moura-BuenoEtAl2012, Samuel-RosaEtAl2013}. They started defining the population 
of transects using their knowledge of the study area, taking into account the factors that they thought 
determined the spatial distribution of soil properties. Each researcher (three) delineated $m = 60$ easily 
accessible, straight transects of \SI{400}{\metre} following the spatial gradients of selected environmental 
features (topography, geology, vegetation, land use, and soils), totalling 180 transects. Accordingly, 
knowledge of existing roads, human settlements, water bodies, and other access limitations was used as well. 
The activity was carried out using \SI{1}{\metre} spatial resolution Google Earth\rr{} imagery of the years of 
\num{2008} and \num{2009}.

\begin{figure}[!ht]
\centering
\includegraphics[width=0.90\textwidth]{fig/chap04-subset-II}
\caption[Twelve transects were selected using simple random sampling to yield $n = 60$ validation 
observations]{Three soil spatial modellers manually drew 180 straight transects (black dotted lines) aligned 
in the direction of maximum expected spatial variation of environmental conditions. They avoided locations 
where it was known that geographic barriers or landowners would impede the access to make soil observations. 
Twelve transects were probabilistically selected using simple random sampling to yield $n = 60$ validation 
observations (red solid circles) separated by equidistant intervals of \SI{100}{\metre}. The drainage network 
is shown in the background (blue dashed line) to give an idea of how the location and direction of transects 
is related to terrain features.}
\label{fig:chap04-subset-II}
\end{figure}

Twelve out of the $m = 180$ transects were randomly selected using as many iterations as necessary until 
there were no intersecting transects, and there was at least one transect in each of the three major 
morphostructural units of the DNOS catchment (\textit{Planalto}, \textit{Rebordo do Planalto}, and 
\textit{Depressão Periférica}) (\autoref{chap:chap03}). Finally, $n = 5$ observation locations, separated by 
equidistant intervals of \SI{100}{\metre}, were selected in each transect. Observation locations were named 
with a number in increasing order, following the order in which the observations were made, starting from 
\num{341} (\num{341}--\num{400}).

The location of the observations was identified in the field using a GNSS receiver with a horizontal 
positional error of less than \SI{8}{\metre}. Soil sampling and description was carried out using the same 
procedure used with \emph{Subset I}, except for the fact that a single soil pit was opened within a radius of 
\SI{2}{\m} from the predefined observation location. More accurate geographic coordinates were collected in 
the field using a Differential Global Positioning System (DGPS) with a horizontal positional error of less 
than \SI{1}{\centi\metre}.

\subsection{Subset III}

The third subset ($n = 10$) contains data compiled from the studies of \citet{Pedron2005} and 
\citet{Miguel2010}, specifically from the uppermost A horizon of modal soil profiles (point support) whose 
locations were purposively selected using tacit knowledge after a preliminary area-class soil map had been 
produced and/or the observations included in \emph{Subset I} had been made.

\citet{Pedron2005} and \citet{Miguel2010} aimed at observation locations that they understood as being most 
representative of the soil mapping units depicted in their respective area-class soil maps. A single soil 
sample was taken from each of the described soil horizons and used for laboratory analysis. The resulting 
thickness of the uppermost A horizons varies from \num{12} to \SI{30}{\centi\metre}, with a mean of 
\SI{22.6}{\centi\metre}. Georeferencing was carried using a GNSS receiver with a horizontal positional error 
of less than \SI{8}{\metre} positioned at the observation location. Data are identified in the Santa Maria 
dataset using the same identification that was used in the studies from which they were compiled.

\section{FIELD DESCRIPTION}

Several environmental features were described at the observation locations. This sections present a summary
description of how this description was done, specially for subsets I and II, which have not been documented 
before. For subset III, a thorough explanation of how field description was done is given by 
\citet{Pedron2005} and \citet{Miguel2010}.

Despite the different origins of the datasets, soil sampling and description guidelines are very similar. As 
such, merging field descriptions from subset III with those of subset I and II was easy, rarely requiring 
conceptual translations and adaptations -- this practice is reported when used.

Finally, the code used in the database to identify each of the variables described in the field is presented 
between parenthesis using fixed-width or monospace font.

\subsection{Land Use and Vegetation}

Land use (\texttt{LAND}) was assessed at the time of sampling using data collected in the field. Five land 
uses were identified using nomenclature of \citet{FAO2006} (\autoref{fig:chap04-land}):

\begin{description}
\item[\texttt{animal husbandry}] Native grasslands used for animal husbandry.
\item[\texttt{crop agriculture}] Annual and biannual crop agriculture.
\item[\texttt{forestry}] Plantations of \textit{Eucalyptus spp.} and \textit{Pinus spp.}.
\item[\texttt{native forest}] Primary or secondary native forests.
\item[\texttt{shrubland}] Abandoned areas with predominance of shrub-sized vegetation, known in Brazil as 
\emph{capoeira}.
\end{description}

\begin{figure}[!ht]
\centering
\includegraphics[width=0.60\textwidth]{fig/chap04-land}
\caption[Distribution of land use types in the Santa Maria dataset.]{Distribution of land use types in the 
Santa Maria dataset. Most soil observations were made in areas used for animal husbandry although more that 
half of the area is occupied with native forest (\autoref{sec:chap05-land-use}).}
\label{fig:chap04-land}
\end{figure}

Other land uses are observed in the study area such as human settlements and water bodies 
(\autoref{sec:chap05-land-use}). However, due to access constraints, soil observations were not made in areas 
under these land uses.

\subsection{Geology}

Soil parent material (\texttt{PARENT}) was inferred in the field from direct observation of soil properties 
and local environmental features. Two classes were identified using nomenclature of \cite{FAO2006}:

\begin{description}
\item \texttt{igneous} Soil derived from the \textit{in sutu} weathering of igneous rocks.
\item \texttt{sedimentary} Soil derived from the \textit{in sutu} weathering of sedimentary rocks, or from 
sediments of igneous and/or sedimentary rocks.
\end{description}

Underlying geologic formation (\texttt{GEO}) and lithology (\texttt{LITHO}) were inferred based on soil 
properties and environmental features observed in the field, and on existing area-class soil maps 
(\autoref{sec:chap05-soil-maps}) geologic maps (\autoref{sec:chap05-geo-maps}).

\subsection{Soil Classification}

The most likely taxon (\texttt{TAXON}) of the Brazilian System of Soil Classification (SiBCS) 
\cite{SantosEtAl2013a} was inferred in the field using data obtained from direct observation of soil 
properties (\SI{20}{\centi\metre}-deep soil pits and auger holes down to the diagnostic subsurface horizon or 
bedrock) and local environmental features. These data were then interpreted using the bases and concepts of 
the SiBCS to identify the most likely taxon up to the second taxonomic level of the SiBCS. Further levels of 
the SiBCS were not considered because making any sort of inference would require data that were not observable 
in the field. Eleven taxa were identified (\autoref{fig:chap04-taxon}):

\begin{description}
\item[\texttt{CX}] Cambissolo Háplico. Moderately developed soil.
\item[\texttt{GX}] Gleissolo Háplico. Poorly drained greyish soil with a somewhat constant clay content 
throughout the profile.
\item[\texttt{PA}] Argissolo Amarelo. Soil with significant increase of the clay content with depth, and with 
a yellowish B horizon.
\item[\texttt{PBAC}] Argissolo Bruno-Acinzentado. Soil with significant increase of the clay content with 
depth, and with an upper B horizon slightly darker than the lower B horizons.
\item[\texttt{PV}] Argissolo Vermelho. Soil with significant increase of the clay content with depth, and with 
a reddish B horizon.
\item[\texttt{PVA}] Argissolo Vermelho-Amarelo. Soil with significant increase of the clay content with depth, 
and with a reddish-yellowish B horizon.
\item[\texttt{RL}] Neossolo Litólico. Poorly developed soil.
\item[\texttt{RQ}] Neossolo Quartzarênico. Deep sandy soil derived from sediments.
\item[\texttt{RR}] Neossolo Regolítico. Poorly to moderately developed soil.
\item[\texttt{RY}] Neossolo Flúvico. Poorly developed soil derived from alluvial sediments.
\item[\texttt{SX}] Planossolo Háplico. Poorly drained greyish soil with significant increase of the clay 
content with depth.
\end{description}

\begin{figure}[!ht]
\centering
\includegraphics[width=0.60\textwidth]{fig/chap04-taxon}
\caption[Distribution of soil taxa in the Santa Maria dataset.]{Distribution of soil taxa in the Santa Maria 
dataset. Most soil observations were classified as Neossolo Litólico and Argissolo Bruno-Acinzentado. The 
proportions approximately agree with the information conveyed by existing area-class soil maps 
(\autoref{fig:chap05-soil-maps}).}
\label{fig:chap04-taxon}
\end{figure}

\subsection{Slope}

The slope gradient (\texttt{SLOPE}, \si{\percent}) was measured using a clinometer, the observer and target 
being at a constant height above the ground (\autoref{fig:chap04-slope}). The distance between observer and 
target was between \SI{30}{\metre} (dense forests) and \SI{50}{\metre} (open fields).

\begin{figure}[!ht]
\centering
\includegraphics[width=0.60\textwidth]{fig/chap04-slope}
\caption[Distribution of slope gradient in the Santa Maria dataset.]{Distribution of slope gradient in the 
Santa Maria dataset. Most soil observations were made in areas with a slope gradient $\SI{<25}{\percent}$.}
\label{fig:chap04-slope}
\end{figure}

\subsection{Drainage}

Soil drainage status (\texttt{DRAIN}) was inferred visually from soil features observed with an auger using 
the classification scheme proposed by \citet{SantosEtAl2013}. Four drainage classes were identified 
(\autoref{fig:chap04-drain}):

\begin{description}
\item[\texttt{poorly}] Water is removed from the soil so slowly that the profile remains wet for much of the 
time.

\item[\texttt{somewhat poorly}] Water is removed slowly from the soil, so that it remains wet for a 
significant period, but not during most of the year.

\item[\texttt{moderately well}] Water is removed from the soil somewhat slowly, so that the profile remains 
wet for small but significant period of time.

\item[\texttt{well}] Water is removed from the soil with ease but not rapidly.
\end{description}

\begin{figure}[!ht]
\centering
\includegraphics[width=0.60\textwidth]{fig/chap04-drain}
\caption[Distribution of drainage classes in the Santa Maria dataset.]{Distribution of drainage classes in the 
Santa Maria dataset. Most soil observations were made in areas with well drained soil.}
\label{fig:chap04-drain}
\end{figure}

\subsection{Coarse Fragments and Rock Outcrops}

Presence of coarse fragments (\texttt{FRAG}) -- soil material of diameter \SI{>2}{\milli\metre} -- was 
described as a binary variable, that is, a value of \num{1} (one) was annotated when coarse fragments were 
present, and \num{0} (zero) otherwise. The same approach was adopted to describe the presence of rock outcrops 
(\texttt{ROCK}). The quantity of coarse fragments (\texttt{GRAVEL}, \si{\percent}) was estimated visually in 
some observation points.

It is worth noting that the approach employed to describe the presence of coarse fragments and rock outcrops 
is not in line with the standard soil description guidelines currently used in Brazil. The reason for 
recording only their presence/absence is that the actual content was not of primary interest at the time of 
sampling.

\subsection{Canopy}

% TODO: add three figures as examples of each class.
Soil coverage with vegetation, an idea of the density of stand or plant cover, (\texttt{CANOPY}) was inferred 
visually in the field using three classes:

\begin{description}
\item[low] \SI{<25}{\percent}
\item[medium] 25--\SI{75}{\percent}
\item[high] \SI{>75}{\percent}
\end{description}

\subsection{Additional Information}

Additional information was recorded at each observation location during the field campaigns. They refer to 
peculiarities of each observation location and were not recorded in a systematic way.

\section{LABORATORY ANALYSIS}
\label{sec:chap04-laboratory}

Several laboratory analysis were performed with the soil samples collected in the DNOS catchment. This 
sections present a summary description of how this analyses were done, specially for subsets I and II, which 
have not been documented before. For subset III, a thorough explanation of how laboratory analyses were done 
is given by \citet{Pedron2005} and \citet{Miguel2010}.

Despite the different origins of the datasets, laboratory analyses protocols are very similar. As such, 
merging the results of laboratory analyses from subset III with those of subset I and II was easy, rarely 
requiring the use of conversion factors -- this practice is reported when used. In all three datasets, soil 
samples were air dried, crushed and passed through a \SI{2}{\milli\metre}-sieve prior to laboratory analyses. 
For datasets I and II, one or more laboratory replicates were used to enable calculating analytical errors.

The same coding standard used with field description variables is used here, i.e. the code used in the 
database is presented between parenthesis using fixed-width or monospace font.

\subsection{Soil Organic Fraction}
\label{sec:chap04-organic}

The soil organic carbon content (\texttt{ORCA}, \si{\gram\per\kilo\gram}) was determined using wet combustion 
\cite{YeomansEtAl1988, Mebius1960, TedescoEtAl1995, ClaessenEtAl1997}.

% Footnote %%%%%
\def\footsulfochromic{\footnote{See a detailed description of the sulfochromic solution, or chromic acid, at 
\href{http://en.wikipedia.org/wiki/Chromic_acid}{Wikipedia}.}}

Sample aliquots of \num{0.050} to \SI{0.500}{\g} were placed in glass digestion tubes (\SI{80}{\ml}). The 
amount of sample used varied according to the \texttt{ORCA} estimated by visual interpretation of soil colour. 
Every digestion tube received an aliquot of \SI{10}{\ml} of sulfochromic solution\footsulfochromic{} 
[\SI{0.067}{\mole\per\liter} potassium bichromate solution (\ce{K2Cr2O7}) in the presence of concentrated 
sulphuric acid (\ce{H2SO4})] and a small reflux funnel to avoid loss of reagent during digestion. A digestion 
block with capacity for \num{40}~samples was used: \num{36}~tubes with soil sample plus \num{3}~tubes with 
blank plus \num{1}~tube with \ce{H2SO4} and a thermometer for temperature check. Digestion at 
\SI{150}{\celsius} last \SI{30}{\minute}. Three blanks were prepared and set aside at room temperature to 
estimate the loss of reagent due to heat in the digestion block.

After digestion the tubes were set aside at room temperature to cool down. Next, the solution was transferred 
to Erlenmeyer flasks (\SI{250}{\ml}) with \SI{60}{\ml} of distilled water and \SI{2}{\ml} of concentrated 
orthophosphoric acid [\ce{H3PO4}] and \num{3}~drops of \SI{1}{\percent}~diphenylamine. The solution was 
titrated using \SI{0.1}{\mole\per\liter} ammonium ferrous sulphate (\ce{FeSO4(NH4)2 * 6H2O}) until persistent 
green colour. The results were multiplied by \num{1.11} to correct the estimated soil organic carbon content 
to the standard analytical method (dry combustion).

\begin{figure}[!ht]
\centering
\includegraphics[width=0.60\textwidth]{fig/chap04-orca}
\caption[Distribution of organic carbon content in the Santa Maria dataset.]{Distribution of organic carbon 
content in the Santa Maria dataset. The distribution is severely skewed.}
\label{fig:chap04-orca}
\end{figure}

Observations compiled from \citet{Pedron2005} had their soil organic matter content determined instead of the 
organic carbon content. Sample aliquots of \SI{2.5}{\ml} were placed in Erlenmeyer flasks (\SI{50}{\ml}). 
Every Erlenmeyer flask received an aliquot of \SI{15}{\ml} of \SI{0.067}{\mole\per\liter} sulfochromic 
solution (\ce{Na2Cr2O7 + H2SO4}). The flasks were heated in a water bath at 75--\SI{80}{\celsius} during 
\SI{30}{\minute} and shaken for \SI{5}{\minute}. A water aliquot of \SI{15}{\ml} was added to the flask and 
let overnight (15--\SI{18}{\hour}).

In the next day, an aliquot of \SI{3.0}{\ml} was sampled to a small plastic cup with \SI{3.0}{\ml} of 
distilled water. The absorbency of the supernatant was measured at \SI{645}{\nano\metre}. The results were 
transformed to organic carbon content assuming that \SI{58}{\percent} of the organic matter is composed of 
organic carbon, the result assumed to be equivalent to soil organic carbon content measured using the standard 
analytical method. The results are expressed using a volume-basis and were converted to a mass-basis using a
1:1 relation because the mass of the sample aliquot used in the analyses is unknown.

\subsection{Particle Size Analysis}
\label{chap:chap04-granulometry}

\def\footsuzuki{\footnote{As far as I know, a comprehensive description of this method has not been 
published so far, neither in Portuguese nor in English. You can visit the homepage of the Soil Physics 
Laboratory of the Universidate Federal de Santa Maria at \url{https://coral.ufsm.br/fisicadosolo/} to get more 
information about the method or contact their developers.}}

Particle size analysis was performed using the pipette method, with the sand fraction (\texttt{SAND}, 
\SIrange{0.053}{2}{\milli\metre}, \si{\gram\per\kilo\gram}) determined by wet sieving, and the silt fraction 
(\texttt{SILT}, \SIrange{0.002}{0.053}{\milli\metre}, \si{\gram\per\kilo\gram}) calculated by difference. 
The analytical procedure is an adaptation\footsuzuki{} of the method of the Soil Conservation Service of 
the United States Department of Agriculture \cite{UnitedStates1972} made by the Soil Physics Laboratory of the 
\textit{Universidade Federal de Santa Maria} \cite{SuzukiEtAl2004, SuzukiEtAl2004a}.

First, a sample aliquot of \SI{20}{\gram} was placed in a \SI{100}{\milli\liter} glass container (height: 
\SI{10.5}{\centi\metre}; diameter: \SI{2.75}{\centi\metre}; weight: \SI{85}{\gram}). Two nylon spheres with a 
diameter of \SI{1.71}{\centi\metre} and weighting \SI{3.04}{\gram} (density: \SI{1.11}{\g\per\cm\cubic}) were 
added to act as physical disaggregating agents. Then, an aliquot of \SI{10}{\milli\liter} of 
\SI{1}{\mole\per\liter} sodium hydroxide (\ce{NaOH}) solution was added to act as chemical dispersing agent 
along with \SI{40}{\milli\liter} of distilled water. The glass container was closed with a plastic cap, 
manually shaken for \SI{10}{\second}, and placed in a horizontal mechanical shaker with capacity for 
\num{85}~samples. The suspension was left to stand overnight (\SI{10}{\hour}). In the next day the suspension 
was submitted to horizontal mechanical agitation during \SI{4}{\hour} at \si{120} cycles per minute 
\cite{SuzukiEtAl2004, SuzukiEtAl2004a}.

After horizontal agitation, the suspension was poured in a plastic graduated cylinder with capacity for 
\SI{1000}{\milli\liter} using a glass funnel and a metal sieve to hold the two nylon spheres. The suspension 
in 
the graduated cylinder was completed to \SI{1000}{\milli\liter} and homogenized using a hand stirrer 
(\SI{30}{\second}). The suspension was allowed to stand until sedimentation was complete. The time needed was 
calculated using the Stokes’ law with the temperature measured in a graduated cylinder filled with distilled 
water.

%TODO provide a more detailed description of how CLAY was determined as well as of the oxidative
%treatment with H2O2.

The clay fraction (\texttt{CLAY}, \SI{<0.002}{\milli\metre}, \si{\gram\per\kilo\gram}) was determined by the 
pipette method. Soil samples with organic matter content \SI{>5}{\percent} were submitted to oxidative 
treatment with hydrogen peroxide (\ce{H2O2}) prior to the analysis following the recommendations of
\citeonline{ClaessenEtAl1997}.

%The sand fraction was separated into five size classes:
%
%\begin{itemize}
%\item \SIrange{1.00}{2.00}{\milli\metre}: very coarse sand;
%\item \SIrange{0.50}{1.00}{\milli\metre}: coarse sand;
%\item \SIrange{0.25}{0.50}{\milli\metre}: median sand;
%\item \SIrange{0.106}{0.25}{\milli\metre}:fine sand;
%\item \SIrange{0.053}{0.106}{\milli\metre}: very fine sand.
%\end{itemize}

% The clay fraction (\textless0.002~mm) was initially determined by the pipette method without any 
% pretreatment. A 1~mol~L$^{-1}$ NaOH solution was used as the dispersing agent, with the addition of two 
% nylon spheres as disaggregating agent plus horizontal mechanical agitation during 4~hours 
% \cite{SuzukiEtAl2004}.

% A propor{\c{c}}{\~{a}}o da fra{\c{c}}{\~{a}}o argila dispersa em {\'{a}}gua foi determinada conforme 
% descrito acima para a fra{\c{c}}{\~{a}}o argila total. A diferen{\c{c}}a {\'{e}} que n{\~{a}}o foi usado o 
% agente dispersante (NaOH) e o agente desagregante (esferas de nylon) \cite{ClaessenEtAl1997}.

\subsection{Soil Density}
\label{chap:chap04-bude}

% TODO: Provide a more detailed description of how this is done.
The bulk soil density (\texttt{BUDE}, \si{\mega\gram\per\cubic\metre}) was determined using the core method 
with a metallic ring (height: \SI{3}{\centi\metre}; diameter: \SI{5}{\centi\metre}) as described by 
\citeonline{ClaessenEtAl1997}. The bulk soil density was not determined in the locations where the soil was 
very shallow or stony.

\subsection{Exchangeable Bases and Acidity}
\label{chap:chap04-ecec}

The exchangeable calcium (\texttt{CALC}, \si{\milli\mole\per\kilo\gram}) and magnesium (\texttt{MAGN}, 
\si{\milli\mole\per\kilo\gram}) were determined by atomic absorption spectroscopy after extraction with 
\SI{1.0}{\mole\per\liter} \ce{KCl} solution using the method described by \citeonline{ClaessenEtAl1997}. 
The exchangeable sodium (\texttt{SODI}, \si{\milli\mole\per\kilo\gram}) and potassium (\texttt{POTA}, 
\si{\milli\mole\per\kilo\gram}) were extracted with a \SI{0.05}{\mole\per\liter} \ce{HCl} solution plus 
\SI{0.025}{\mole\per\liter} \ce{H2SO} (Mehlich-\num{1} solution). Both were quantified by means of flame 
atomic emission spectrometry using the method described by \citeonline{TedescoEtAl1995}.

The exchangeable acidity (\texttt{EXAC}, \si{\milli\mole\per\kilo\gram}) was extracted using the same 
\SI{1.0}{\mole\per\liter} \ce{KCl} solution used to extract the exchangeable calcium and magnesium. It was 
determined by titrimetry with \SI{0.025}{\mole\per\liter} \ce{NaOH} solution as described by 
\citeonline[p.~103]{ClaessenEtAl1997}.

% TODO: Include POAC in the database and improve the description of how it was determined.
% The potential acidity (POAC, \si{\milli\mole\per\kilo\gram}) was determined with \SI{1.0}{\mole\per\liter} 
% calcium acetate solution at pH~\num{7.0} and titrated with \SI{0.0606}{\mole\per\liter} \ce{NaOH} solution 
% as described by \citeonline{ClaessenEtAl1997}.

The effective cation exchange capacity (ECEC, \si{\milli\mole\per\kilo\gram}) was defined as the sum of 
exchangeable bases and exchangeable acidity, i.e. 

\begin{equation*}
 \texttt{ECEC} = \texttt{CALC} + \texttt{MAGN} + \texttt{POTA} + \texttt{SODI} + \texttt{EXAC}.
\end{equation*}


% TODO: Provide a more detailed description of how these are calculated and include the data in the database.
% The sum of exchangeable bases (BASES) is given by the sum of the exchangeable calcium, magnesium, sodium and 
% potassium. The effective cation exchange capacity (ECEC) is given by the exchangeable acidity plus the 
% sum of exchangeable bases. The potential cation exchange capacity (CEC) is given by the potential acidity 
% plus the sum of exchangeable bases. Note that the standard method for determining exchangeable bases relies 
% on the use of barium chloride [BaCl$_2$]. The base saturation (BASA) is given by the sum of exchangeable 
% bases divided by the potential cation exchange capacity. The saturation of the ECEC with exchangeable 
% acidity, or the aluminum saturation (ALSA), is given by the sum of exchangeable bases divided by the 
% effective cation exchange capacity. The results are multiplied by 100. 

% \begin{figure}[!ht]
% \centering
% <<echo = FALSE>>=
% options(useFancyQuotes = FALSE)
% tmp <- read.table(
%  '~/projects/dnos-sm-rs/dnos-sm-rs-general/data/labData.csv', sep = ";",
%  header = TRUE, na.strings = 'na')
% lattice::trellis.par.set(
%  fontsize = list(text = 16, points = 15), axis.line = list(lwd = 0.01),
%  layout.widths = list(left.padding = 0, right.padding = 0),
%  layout.heights = list(top.padding = 0, bottom.padding = 0))
% aa <- pedometrics::plotHD(tmp$CLAY, xlab = 'CLAY')
% bb <- pedometrics::plotHD(tmp$ORCA, xlab = 'ORCA')
% cc <- pedometrics::plotHD(tmp$ECEC, xlab = 'ECEC')
% dd <- pedometrics::plotHD(na.exclude(tmp$BUDE), xlab = "BUDE")
% @
% \begin{minipage}[b]{63mm}
% \subcaption{}
% \centering
% <<intro-clay, fig = TRUE, echo = FALSE>>=
% print(aa)
% @
% \end{minipage}
% \begin{minipage}[b]{63mm}
% \subcaption{}
% \centering
% <<intro-orca, fig = TRUE, echo = FALSE>>=
% print(bb)
% @
% \end{minipage}
% \begin{minipage}[b]{63mm}
% \subcaption{}
% \centering
% <<intro-ecec, fig = TRUE, echo = FALSE>>=
% print(cc)
% @
% \end{minipage}
% \begin{minipage}[b]{63mm}
% \subcaption{}
% \centering
% <<intro-bude, fig = TRUE, echo = FALSE>>=
% print(dd)
% @
% \end{minipage}
% \caption{The four soil properties explored in this thesis: (a) clay content, (b) organic carbon
% content, (c) effective cation exchange capacity, and (d) bulk density. Each panel shows the sample
% histogram and summary statistics of the soil properties in their original scale ($\lambda = 1$), as
% well as the theoretical probability density function so that we can assess how good is the fit of
% the normal distribution to the data -- a product of the \Rpackage{pedometrics}.}
% \label{fig:intro-soil-properties}
% \end{figure}

\section{DATASET STRUCTURE}

The soil data is 
freely available as comma-separated values (CSV) files in a repository hosted in \dnosgeneral{}. Files 
\texttt{fieldData.csv} and \texttt{labData.csv} contain the identification of all observation locations, their 
geographic coordinates (latlong, WGS1984), and field and laboratory data, respectively. Files 
\texttt{fieldMetadata.csv} and \texttt{labMetadata.csv} contain the metadata. Every soil property is 
identified with a code composed of three or four capital letters. For example, soil organic carbon is 
identified with \texttt{ORCA}. A column containing the number of laboratory replicates is identified with the 
code of the soil property followed by the letter \q{N}. The column containing the sample standard deviation is 
identified in the same manner, but using \q{SD}. For example, \texttt{ORCA\_N} and \texttt{ORCA\_SD}.


\section{CONCLUSIONS}

The main goal of documenting the soil data contained in the Santa Maria dataset was to provide the reader the 
basis to understand the soil data used in the thesis, and also to support future soil spatial modelling 
exercises in 
the catchment of the DNOS reservoir.

As a result of an ongoing collaborative effort, this documentation will be improved in the near future as new 
studies are developed. We plan to include new figures to exemplify how field soil sampling was carried out. 
Details of non-standard soil description and analysis methods will likely be extended. This includes the 
oxidative 
treatment with \ce{H2O2} to which soil samples were submitted prior to particle size distribution analysis. 
For 
cases such as the ECEC, determined using a non-standard method, we plan to develop a study to calibrate a 
model 
to convert our results to the standard method for determining exchangeable bases, which uses barium chloride 
(\ce{BaCl2}) for saturation.

Other already existing soil data will also be included in the Santa Maria dataset and documented as well. 
These 
data have not been used in any study so far, including the potential acidity, sum of exchangeable bases, 
potential 
cation exchange capacity, base saturation, aluminium saturation, and the five size classes of the sand 
fraction.

Once a comprehensive documentation of the existing soil data has been constructed, we will prepare a basic 
spatial exploratory soil data analysis. We hope that our effort to properly document the soil data that we 
produced,
and make it freely available for use, will serve as an example for future soil spatial modelling studies 
developed 
elsewhere.

 % The Santa Maria dataset -- covariate data
\artigotrue
\chapter{On the Uncertainty of Digital Soil Mapping - Model Structure}
\label{chap:chapter05}

\begin{chapterabstractPOR}{Pedometria, Incerteza, Estrutura do modelo}
Este capítulo abordará a identificação de cenários de bancos de dados relativos ao número de observações de calibração disponíveis nos quais modelos não-lineares apresentam desempenho melhor do que modelos lineares.
\end{chapterabstractPOR}

\begin{chapterabstractENG}{Pedometrics, Uncertainty, Model structure}
This chapter will deal with identifying database scenarios regarding the number of calibration observations available in which non-linear models present better performance than linear models.
\end{chapterabstractENG}

\section{INTRODUCTION}

This chapter will deal with identifying database scenarios regarding the number of calibration observations available in which non-linear models present better performance than linear models.

\section{MATERIAL AND METHODS}

\subsection{Database}

Seven subsets of calibration observations will be used to simulate database scenarios regarding the number of calibration observations available to build DSM models. These subsets will contain $n=$50, 100, 150, 200, 250, 300 and 350 calibration observations and will be constructed using the criteria described in item Chapter 2. A suite of $n=64$ environmental co-variates will be derived from seventeen data layers to fit trend models. Orthogonalization of predictor variables will not be considered.

\subsection{Model Structure}

Two types of trend model structures will be evaluated: linear and non-linear. The linear structure will be represented by multivariate linear regression model with estimation of parameters by ordinary least squares (OLS) (package \texttt{stats}), while the non-linear structure will be represented by three models. These are:

\begin{itemize}
\item An artificial neural network with multilayer perceptron (MLP) architecture and training by error backpropagation (package \texttt{RSNNS});
\item A regression tree using a one-step lookahead construction method with splits aiming at the reduction of the residual sum of squares (package \texttt{rpart});
\item A random forest implementing Breiman's algorithm (package \texttt{randomForest}).
\end{itemize}

\subsection{Model Building}

Four trend models will be fitted for each soil properties (particle-size distribution, organic carbon content and cation exchange capacity). Each of these trend models will be fitted using one of the four model structures described above. The residuals will be used to fit a variogram model and interpolated using simple kriging (package \texttt{gstat}). Final prediction map will be obtained by adding kriged predictions to the predictions made by the trend model. Model assessment will involve analyzing summary statistics of each method.

\subsection{Assessment of Competing Models}

Four competing models will be build for every soil property. Their analysis will include evaluating the differences among the sets of environmental co-variates included in the trend model under the light of the conceptual model of pedogenesis. Differences in variogram model parameters will also be searched. Coupled with the analysis of the spatial pattern of predicted values and prediction error variance maps, these analysis will help defining a degree of uncertainty about model specification due to trend model structure. Prediction accuracy  will be evaluated for all models using independent field data obtained through probabilistic sampling ($n=60$). Error statistics (mean error, mean squared error, and mean squared deviation ratio) of pairs of competing models will be compared under the null hypothesis that the expected value of the estimated mean difference is zero. The pedological information content of trend models will be evaluated eliciting the opinion of five experts. % Do more detailed covariates deliver more accurate soil maps?
% \artigotrue
\chapter{DO MORE DETAILED COVARIATES DELIVER MORE ACCURATE SOIL MAPS?}
\chapternote{This chapter is based on A.~Samuel-Rosa, G.B.M.~Heuvelink, G.M.~Vasques, L.H.C.~Anjos. Do 
more detailed environmental covariates deliver more accurate soil maps? \emph{Geoderma}, v.243--244, 
p.214--227, 2015. Terms and expressions have been modified to match the standard terminology used throughout 
the thesis without compromising the content of the original text. Footnotes were added where a definition 
required correction or clarification.}
\shorttitle{Using More Detailed Covariates}
\label{chap:chap06}

% User defined commands
\def\elev{\texttt{ELEV}} % elevation
\def\slp{\texttt{SLP}}   % slope
\def\asp{\texttt{ASP}}   % aspect
\def\nor{\texttt{NOR}}   % northernness
\def\acc{\texttt{ACC}}   % flow accumulation
\def\twi{\texttt{TWI}}   % topographic wetness index
\def\spi{\texttt{SPI}}   % stream power index
\def\tpi{\texttt{TPI}}   % topographic position index
\def\ndvi{\texttt{NDVI}} %
\def\savi{\texttt{SAVI}} %
\def\sibcs{Brazilian System of Soil Classification}

% \def\portuguesekeys{Mapeamento Digital do Solo. Modelo Linear Misto. Informação Auxiliar. Seleção de 
% Variáveis. Acurácia do Modelo. Custo do Mapeamento do Solo}

% \begin{chapterabstract}{brazilian}{\portuguesekeys}
% Neste estudo nós avaliamos se investir em covariáveis espacialmente mais detalhadas aumenta a acurácia dos 
% mapas do solo. Nós usamos um estudo de caso no sul do Brasil para mapear o conteúdo de argila (CLAY), o 
% conteúdo de carbono orgânico (SOC), e capacidade de troca de cátions efetiva (ECEC) da camada superficial do 
% solo de uma área de \SI{\sim2000}{\hectare} localizada na borda do planalto da Bacia Sedimentar do Paraná. 
% Cinco covariáveis, cada uma com dois níveis de detalhe espacial, foram usadas: mapa areal-categórico de solo,
% modelos digitais de elevação (DEM), mapas geológicos, mapas de uso da terra, e imagens de satélite. Trinta e 
% dois modelos de regressão linear múltipla foram calibrados para cada propriedade do solo usando todas as 
% combinações de detalhe espacial das covariáveis. Para cada combinação, \textit{stepwise regression} foi 
% usada para selecionar as variáveis preditoras incorporadas no modelo. A avaliação dos modelos foi feita 
% usando o R-quadrado ajustado da regressão. O modelo de referência, calibrado com a versão menos detalhada de 
% cada covariável, e o modelo com o melhor desempenho, foram usados para calibrar dois modelos lineares mistos 
% para cada propriedade do solo. Parâmetros dos modelos foram estimados usando máxima verossimilhança 
% restrita. Predições espaciais foram realizadas usando o melhor preditor linear não-enviesado empírico. 
% Validação-cruzada foi usada para validar os modelos de regressão linear múltipla e dos modelos lineares 
% mistos de referência e com melhor desempenho. Os resultados mostram que para CLAY a acurácia da predição não 
% aumentou consideravelmente por usar covariáveis mais detalhadas. A quantidade de variância explicada 
% aumentou apenas \SI{\sim2}{\pp} (pontos percentuais), menos do que obtido pela inclusão do passo de 
% krigagem, que explicou \SI{4}{\pp}. Por outro lado, a predição de SOC e ECEC aumentou em \SI{\sim13}{\pp} 
% quando o modelo de referência foi substituído pelo modelo com melhor desempenho. Em geral, o aumento no 
% desempenho preditivo foi modesto e pode não sobrepor os custos adicionais do uso de covariáveis mais 
% detalhadas. Pode ser mais eficiente investir recursos adicionais na coleta de mais observações do solo, ou 
% no aumento do detalhe apenas da covariável que tem o efeito de aumento mais forte. Em nosso estudo, a última 
% funcionaria apenas para SOC e ECEC pelo investimento em um mapa de uso da terra mais detalhado e, 
% possivelmente, também em um mapa geológico e DEM mais detalhados.
% \end{chapterabstract}

\def\englishkeys{Digital Soil Mapping. Linear Mixed Model. Auxiliary Information. Variable Selection. Model
Accuracy. Soil Mapping Cost}
  
\begin{chapterabstract}{english}{\englishkeys}
In this study we evaluated whether investing in more spatially detailed covariates improves the accuracy of 
soil maps. We used a case study from Southern Brazil to map clay content (CLAY), organic carbon content 
(SOC), 
and effective cation exchange capacity (ECEC) of the topsoil for a \SI{\approx2000}{\hectare} area located on 
the edge of the plateau of the Paraná Sedimentary Basin. Five covariates, each with two levels of spatial 
detail were used: area-class soil maps, digital elevation models (DEM), geologic maps, land use maps, and 
satellite images. Thirty-two multiple linear regression models were calibrated for each soil property using 
all 
spatial detail combinations of the covariates. For each combination, stepwise regression was used to select 
predictor variables incorporated in the model. Model evaluation was done using the adjusted R-square of the 
regression. The baseline model, calibrated with the less detailed version of each covariate, and the best 
performing model were used to calibrate two linear mixed models for each soil property. Model parameters were 
estimated using restricted maximum likelihood. Spatial prediction was performed using the empirical best 
linear 
unbiased predictor. Validation of baseline and best performing linear multiple regression and linear mixed 
models was done using cross-validation. Results show that for CLAY the prediction accuracy did not 
considerably 
improve by using more detailed covariates. The amount of variance explained increased only \num{\sim2} 
percentage points (\si{\pp}), less than that obtained by including the kriging step, which explained 
\SI{4}{\pp}. On the other hand, prediction of SOC and ECEC improved by \SI{\sim13}{\pp} when the baseline 
model 
was replaced by the best performing model. Overall, the increase in prediction performance was modest and may 
not outweigh the extra costs of using more detailed covariates. It may be more efficient to spend extra 
resources on collecting more soil observations, or increasing the detail of only those covariates that have 
the 
strongest improvement effect. In our  case study, the latter would only work for SOC and ECEC, by investing in 
a more detailed land use map and possibly also a more detailed geologic map and DEM.
\end{chapterabstract}

\formatchapter

\section{INTRODUCTION}
\label{sec:chap06-intro}

Modern soil mapping relies on the use of statistical models to produce digital representations of spatial 
soil distribution using point soil observations and spatially exhaustive covariates \cite{McBratneyEtAl2003, 
ScullEtAl2003, Florinsky2012}. Three important weaknesses in the statistical soil distribution modelling 
approach can be pointed out. First, it requires sufficient and appropriately distributed point soil data 
within 
the area being mapped \cite{CarreEtAl2007a}. Second, the model structure explores only the empirical 
relationship among environmental conditions and soil properties, being less comprehensive than soil-landscape 
process models \cite{Grunwald2009}. Last, the covariates are only approximations of the true environmental 
conditions that helped shape the soil. They serve only as proxies (surrogates) of the current environmental 
conditions, which in many cases are different from the past conditions under which pedogenesis took place 
\cite{HeuvelinkEtAl2001}. In spite of these weaknesses, modern soil mapping techniques have proven very 
successful in the past decades in producing soil property maps that capture the main patterns of soil spatial 
variation \cite{MooreEtAl1993, McBratneyEtAl2000, Grunwald2009}.

More recently, there has been a growing interest in understanding how the characteristics of the covariates 
influence the success of soil mapping -- this study contributes to this effort. It is commonly accepted that 
the more resources are spent on the construction of a covariate and the more spatial information it has, the 
more accurately it describes the environmental conditions \cite{HupyEtAl2004, HenglEtAl2013a}. It is also 
generally believed that such \emph{more detailed} covariates will be more valuable for soil mapping and lead 
to 
more accurate soil property predictions \cite{CavazziEtAl2013, MaynardEtAl2014}. If these more detailed 
covariates convey more information and represent more adequately the environmental conditions -- the drivers 
of 
soil forming processes --, then it is fair to expect that they improve the accuracy of the resulting soil 
maps. 
However, some studies have shown the contrary \cite{ThompsonEtAl2001, EldeiryEtAl2008, KimEtAl2014}. For 
example, the window size at which DEM derivatives are calculated can be more important than the spatial 
resolution of the DEM \cite{Wood1996, ZhuEtAl2008, BehrensEtAl2010a}. The uncertainty about the added value of 
using more detailed covariates is of concern for those seeking to use resources efficiently, because using 
more 
detailed covariates generally increases soil mapping costs \cite{ShiEtAl2012}.

The objective of this study was to evaluate whether investing in more detailed covariates improves the 
accuracy of soil maps. The main difference of our study to previous ones is that we use a rigorous statistical 
approach to assess the added value of using five more detailed covariates simultaneously. We used a  case study 
in Brazil to compare the accuracy of maps of the clay content, organic carbon content and effective cation 
exchange capacity of the topsoil as obtained from regression kriging on the five covariates, whereby each 
covariate was evaluated on two levels of spatial detail.

\section{MATERIAL AND METHODS}
\label{sec:chap06-methods}

\subsection{Study Area and Soil Data}
\label{subsec:chap06-soil-data}

The study area constitutes a small catchment (\SI{\sim2000}{\hectare}) located on the southern edge of the 
plateau of the Paraná Sedimentary Basin, Rio Grande do Sul, Brazil (\autoref{fig:chap06-location}). The 
climate is classified as Cfa (K\"oppen -- subtropical humid without a dry season) with mean annual temperature 
of \SI{19.3}{\celsius}, and mean annual precipitation of \SI{1708}{\mm}, well distributed throughout the year 
\cite{Maluf2000}. Relief varies between plain (slope between \num{0} and \SI{3}{\percent}) and mountainous 
(slope between \num{45} and \SI{100}{\percent}), and elevations range between \num{140} and \SI{475}{\m}. 
Geology consists of basic, intermediate and acid igneous rocks (rhyolite-rhyodacite and andesite-basalt) of 
the Cretaceous period, consolidated sedimentary rocks (aeolian and fluvial sandstones) of the Triassic and 
Jurassic periods, and non-consolidated (fluvial and colluvial deposits) of the Quaternary period 
\cite{GasparettoEtAl1988, MacielFilho1990, Sartori2009}. Native semi-deciduous forests occupy more than 
half of the area, followed by native grassland used for animal husbandry, semi-deciduous shrubland, annual 
crop agriculture, forestry (Eucalyptus), urban areas, and artificial water bodies \cite{SamuelRosaEtAl2011a}.

\begin{figure}[!ht]
 \centering
 \begin{minipage}[b]{95mm}
  \subcaption{}
  \label{fig:chap06-brazil}
  \centering
  \includegraphics[width=90mm]{fig/chap06-FIG1a}
 \end{minipage}
 \begin{minipage}[b]{95mm}
  \subcaption{}
  \label{fig:chap06-points}
  \centering
  \includegraphics[width=90mm]{fig/chap06-FIG1b}
 \end{minipage}
 \caption[Location of the study area.]{Location of the study area in Santa Maria (a) and spatial distribution 
of the point soil observations and drainage network (b).}
 \label{fig:chap06-location}
\end{figure}

A dataset containing $n = 350$ point soil observations collected between \num{2004} and \num{2011} 
\cite{PedronEtAl2006b, SamuelRosaEtAl2011a, MiguelEtAl2012, Samuel-RosaEtAl2013} was used in this study 
(available at \url{https://github.com/samuel-rosa/dnos-sm-rs-general}). Sampling locations were selected 
purposively and by convenience \cite{Samuel-RosaEtAl2014b}. Three soil pits were opened within an area of 
\SI{\pm100}{\m\square} at most sampling locations to obtain composite samples of the topsoil for laboratory 
analysis. Soil was collected to a depth of \SI{20}{\cm} or less when soil depth was smaller than \SI{20}{\cm}. 
A few observations ($n = 10$) correspond to individual samples collected up to \SI{30}{\cm}. Sampling depth 
ranges from \num{2} to \SI{30}{\cm}, with a mean of \SI{17.3}{\cm}. We assumed that the vertical, horizontal 
and temporal support differences between soil samples is negligible for the purpose of this study.

Three soil properties (fine earth fraction, \SI{<2}{\mm}) were explored: clay content (CLAY, 
\si{\gram\per\kilo\gram}), organic carbon content (SOC, \si{\gram\per\kilo\gram}), and effective cation 
exchange capacity (ECEC, \si{\milli\mole\per\kilo\gram}). CLAY was determined by the pipette method. SOC was 
determined using wet digestion. ECEC was calculated as the sum of exchangeable bases plus exchangeable 
acidity. The soil properties selected were expected to present different patterns of spatial variation and 
correlation with the most dominant factors of soil formation \cite{Jenny1941} in the area: organisms 
(\textit{O}), relief (\textit{R}), and parent material (\textit{P}). CLAY was presumed to have a stronger 
relation with \textit{P}, while SOC was expected to be more correlated with \textit{O}. Because the soils of 
the study area were strongly eroded due to intense agriculture in the \num{20}th century, both CLAY and SOC 
were also expected to be closely related with \textit{R}. Finally, ECEC was expected to be strongly correlated 
with \textit{P} and \textit{O}, which is supported by its natural relationship with both CLAY and SOC.

\begin{figure}[!ht]
 \centering
 \begin{minipage}[b]{63mm}
  \subcaption{}
  \centering
  \includegraphics[width=63mm]{fig/chap06-FIG2a}
 \end{minipage}
 \begin{minipage}[b]{63mm}
  \subcaption{}
  \centering
  \includegraphics[width=63mm]{fig/chap06-FIG2d}
 \end{minipage}
 \begin{minipage}[b]{63mm}
  \subcaption{}
  \centering
  \includegraphics[width=63mm]{fig/chap06-FIG2b}
 \end{minipage}
 \begin{minipage}[b]{63mm}
  \subcaption{}
  \centering
  \includegraphics[width=63mm]{fig/chap06-FIG2e}
 \end{minipage}
 \begin{minipage}[b]{63mm}
  \subcaption{}
  \centering
  \includegraphics[width=63mm]{fig/chap06-FIG2c}
 \end{minipage}
 \begin{minipage}[b]{63mm}
  \subcaption{}
  \centering
  \includegraphics[width=63mm]{fig/chap06-FIG2f}
 \end{minipage}
 \caption[Summary statistics of CLAY, SOC, and ECEC.]{Histogram, empirical density function, and summary 
statistics of CLAY (a, b), SOC (c, d), and ECEC (e, f) in the original (left) and Box-Cox feature spaces 
(right).}
 \label{fig:chap06-soil-properties}
\end{figure}

Point soil data, here denoted by $Y(s)$, showed a positive skew (\autoref{fig:chap06-soil-properties}) and was 
normalized, $Y'(s)$, using the Box-Cox family of power transformations, where $Y'(s) = (Y(s)^{\lambda} - 1) / 
\lambda$, if $\lambda > 0$, and $Y'(s) = log(Y(s))$, if $\lambda = 0$ \cite{DiggleEtAl2007}. Lambda 
($\lambda$) values were selected empirically \cite{FoxEtAl2011}. Because the resulting distribution of the 
back-transform (see \autoref{subsec:chap06-validation}) has no expectation when $\lambda < 0$ 
\cite{RibeiroEtAl2001}, a logarithm transformation ($\lambda = 0$) was used when a negative $\lambda$ was 
estimated (SOC and ECEC).

\subsection{Covariates}
\label{subsec:chap06-sources}

Five freely available covariates were evaluated in this study, each with two levels of spatial detail: 
area-class soil maps (\texttt{soil}), geologic maps (\texttt{geo}), land use maps (\texttt{land}), digital 
elevation models (\texttt{dem}), and satellite images (\texttt{sat}). Each pair was composed of covariates 
that were produced separately from scratch using different data sources and/or production methods, thus 
demanding different amounts of resources (time, workforce, budget, technology, etc.). In this study, the level 
of spatial detail of a covariate is a function of the components of its production process such as the 
cartographic ratio (\texttt{soil}, \texttt{geo} and \texttt{land}), spatial sampling support (\texttt{sat}), 
number and diversity of data sources explored (\texttt{dem}), and quantity of spatial data used (all five). 
Thus, the reader should bear in mind that our definition of spatial detail is broader than spatial resolution 
or spatial scale. It should also not be confounded with spatial support \cite{WebsterEtAl2007} or thematic 
detail \cite{Rossiter2000}.

\def\footcovars{\footnote{In statistical terms, the terms \emph{covariate} and \emph{predictor variable} are 
synonymous, and the reason for the use given in this study is purely operational.}}

The covariates were transformed to predictor variables\footcovars{} that were used in the geostatistical 
modelling. Since the transformation is different for categorical and continuous covariates, the procedures are 
explained below for each type separately.

\subsubsection{Categorical predictor variables}
\label{subsubsec:chap06-categorical-covars}

Area-class soil maps, geologic maps and land use maps are categorical covariates (factors). Mapping units are 
the $k$ factor levels that are transformed to as many dummy (indicator, binary) variables as there are factor 
levels, before model calibration. Each dummy variable receives a value equal to one (\num{1}) when a given 
class is present, and zero (\num{0}) otherwise \cite{Everitt2006}. If the number of point soil observations 
falling inside the spatial domain of a mapping unit is too small to accurately estimate a regression 
coefficient (we used a threshold of $n = 15$ observations), the mapping unit is merged with a similar mapping 
unit prior to calculating dummy variables. The resulting generalized categorical covariate maps are shown in 
\autoref{fig:chap06-cat-covars}. The binary maps are the categorical predictor variables.

\noindent\textit{Soil maps}. The less detailed soil map (\soilOld) was published with a \scale{100000} and 
has five mapping units \cite{AzolinEtAl1988} (\autoref{fig:chap06-soil-old}). It was produced using existing 
soil maps and technical reports (\scale{750000}) \cite{Brasil1973}, aerial photographs (\scale{60000}), 
topographic maps (\scale{50000}), and sparse point soil observations along the road network. The more detailed 
soil map (\soilNew) was prepared with a \scale{25000} and has eight mapping units \cite{MiguelEtAl2012} 
(\autoref{fig:chap06-soil-new}). It was produced using high spatial resolution satellite images 
(\SI{65}{\cm}), existing soil maps and technical reports published with a \scale{50000} \cite{Poelking2007} 
and \num{1}:\num{25000} \cite{PedronEtAl2006b}, topographic maps (\scale{25000}), and descriptions from 
\num{\sim350} point soil observations. Five dummy predictor variables were derived from \soilOld{} and seven 
from \soilNew{} (\autoref{tab:chap06-soil-covars}).

\begin{figure}[!ht]
 \centering
 \begin{minipage}[b]{63mm}
  \subcaption{Cartographic scale: \num{1}:\num{100000}}
  \label{fig:chap06-soil-old}
  \centering
  \includegraphics[width=60mm]{fig/chap06-FIG3a}
 \end{minipage}
 \begin{minipage}[b]{63mm}
  \subcaption{Cartographic scale: \num{1}:\num{25000}}
  \label{fig:chap06-soil-new}
  \centering
  \includegraphics[width=60mm]{fig/chap06-FIG3d}
 \end{minipage}    
 \begin{minipage}[b]{63mm}
  \subcaption{Cartographic scale: \num{1}:\num{50000}}
  \label{fig:chap06-geo-old}
  \centering
  \includegraphics[width=60mm]{fig/chap06-FIG3b}
 \end{minipage}
 \begin{minipage}[b]{63mm}
  \subcaption{Cartographic scale: \num{1}:\num{25000}}
  \label{fig:chap06-geo-new}
  \centering
  \includegraphics[width=60mm]{fig/chap06-FIG3e}
 \end{minipage}
 \begin{minipage}[b]{63mm}
  \subcaption{Cartographic scale: \num{1}:\num{500000}}
  \label{fig:chap06-land-old}
  \centering
  \includegraphics[width=60mm]{fig/chap06-FIG3c}
 \end{minipage}
 \begin{minipage}[b]{63mm}
  \subcaption{Cartographic scale: \num{1}:\num{2000}}
  \label{fig:chap06-land-new}
  \centering
  \includegraphics[width=60mm]{fig/chap06-FIG3f}
 \end{minipage}
 \caption[Area-class soil maps, geologic maps, and land use maps compared in the study. ]{Area-class soil maps 
(a, b), geologic maps (c, d), and land use maps (e, f) compared in our study. The less and more detailed 
version are displayed at the left and right, respectively. Legend abbreviations and derived dummy variables are 
described in Tables \ref{tab:chap06-soil-covars}--\ref{tab:chap06-land-covars}.}
 \label{fig:chap06-cat-covars}
\end{figure}

\noindent\textit{Geologic maps}. The less detailed geologic map (\geoOld) was produced using topographic maps 
with \scale{50000} \cite{GasparettoEtAl1988} (\autoref{fig:chap06-geo-old}). The more detailed geologic map 
(\geoNew) was produced using topographic maps with \scale{25000}, and includes the location of overlaying 
Quaternary sedimentary deposits \cite{MacielFilho1990} (\autoref{fig:chap06-geo-new}). \geoNew{} did not cover 
a small part in the North of the study area, where \geoOld{} was used instead (this strategy was approved by 
experts on the local geology). The mapping unit of both geologic maps depicting the Caturrita Formation was 
used indirectly by deriving dummy predictor variables from all other individual mapping units. Three dummy 
predictor variables were derived from \geoOld{} and four from \geoNew{} (\autoref{tab:chap06-geology-covars}).

\noindent\textit{Land use maps}. The less detailed land use map (\landOld) was produced by manually digitizing 
land use data included in topographic maps with a \scale{25000} \cite{DSG1980, DSG1992, DSG1992a} 
(\autoref{fig:chap06-land-old}). The more detailed land use map (\landNew) was prepared (\scale{2000}) by 
manual digitization using \SI{65}{\cm} spatial resolution satellite images covering the years \num{2008} and 
\num{2009} \cite{SamuelRosaEtAl2011a} (\autoref{fig:chap06-land-new}). Mapping units depicting human 
settlements and water bodies ($n = 0$) were not masked out from the prediction grid and were merged with other 
mapping units to derive dummy predictor variables. Five dummy predictor variables were derived from \landNew{} 
and two from \landOld{} (\autoref{tab:chap06-land-covars}).

\subsubsection{Continuous predictor variables}
\label{subsubsec:chap06-continuous-covars}

\def\arcgis{\href{http://resources.arcgis.com/en/help/main/10.1/index.html}{ArcGIS}}

The less detailed DEM (\demOld) is the hole-filled SRTM DEM version~\num{4} \cite{JarvisEtAl2008} 
(\autoref{fig:chap06-dem-old}). The spatial sampling support of the SRTM DEM is \SI{1}{\arcsecond} 
(\SI{\sim30}{\m}), but elevation data were aggregated to \SI{3}{\arcsecond} (\SI{\sim90}{\m}) for public 
release in regions outside the United States \cite{ReuterEtAl2007}. The more detailed DEM (\demNew) was 
produced by interpolating contour lines with vertical spacing of \SI{10}{\m} along with data about the 
drainage network, lakes and peaks digitized from topographic maps with \scale{25000} 
(\autoref{fig:chap06-dem-new}). Interpolation to \SI{5}{\m} pixel size was performed using a hydrologically 
correct algorithm implemented in \arcgis{} software by ESRI \cite{Hutchinson1989}. Contour line artefacts were 
minimized using a seven by seven low-pass filter (\grass{r.neighbors}). The window size was chosen such that 
the smoothed DEM best matched the original contour map while also respecting the original drainage network 
pattern.

\input{chap/tab/chap06-TAB1.tex}

\ctable[
 caption  = {Description of the $p = 7$ dummy predictor variables derived from the two geologic maps.},
 cap      = {Predictor variables derived from geologic maps.},
 label    = tab:chap06-geology-covars,
 notespar,
 pos      = !ht,
 maxwidth = \textwidth,
 % doinside = \scriptsize\setstretch{1.1}
 doinside = \small
 ]{l p{0.85\textwidth} l}{
 \tnote[a]{Minimum Legible Delineation calculated following \citet{Rossiter2000}.}
 }{ \FL
 Code & Mapping unit(s) included and Description\tmark[a] \ML
 
 \multicolumn{2}{p{0.98\linewidth}}{Source: \citet{GasparettoEtAl1988}. Cartographic scale: 
 \num{1}:\num{50000}. Minimum Legible Delineation: \SI{10}{\hectare}.} \NN
 
 \texttt{GEO\_50a} & \textit{SG-I}. Inferior Sequence of the Serra Geral Formation. Composed mainly of basic 
 igneous rocks (tholeiitic basalt and andesite). It is likely to be related with high CLAY and ECEC. \NN
 \texttt{GEO\_50b} & \textit{SG-S}. Superior Sequence of the Serra Geral Formation. Composed mainly of acid 
 igneous rocks (granophyric rhyolite and rhyodacite). It is likely to be related with moderate to high CLAY 
 and ECEC. \NN
 \texttt{GEO\_50c} & \textit{BT}. Botucatu Formation. Composed mainly of aeolian sandstones. It is likely to 
 be related with low CLAY and ECEC. \NN
 Other & \textit{CT} depicts the Caturrita Formation, which is composed mainly of fluvial sandstones. \NN
 & \NN
 
 \multicolumn{2}{p{0.98\linewidth}}{Source: \citet{MacielFilho1990}. Cartographic scale: 
 \num{1}:\num{25000}. Minimum Legible Delineation: \SI{2.5}{\hectare}.} \NN
 
 \texttt{GEO\_25a} & \textit{SG-I}. Inferior Sequence of the Serra Geral Formation. \NN
 \texttt{GEO\_25b} & \textit{SG-S}. Superior Sequence of the Serra Geral Formation. \NN
 \texttt{GEO\_25c} & \textit{BT}. Botucatu Formation. \NN
 \texttt{GEO\_25d} & \textit{QD}. Quaternary deposits of fluvial, alluvial, and colluvial origin. It can help 
 explaining the low CLAY of soils supposedly derived from igneous rocks. \NN
 Other & \textit{CT} depicts the Caturrita Formation. \LL
 }


\input{chap/tab/chap06-TAB3.tex}

\begin{figure}[!ht]
 \centering
 \begin{minipage}[b]{63mm}
  \subcaption{Spatial resolution: \SI{90}{\m}}
  \label{fig:chap06-dem-old}
  \centering
  \includegraphics[width=60mm]{fig/chap06-FIG4a}
 \end{minipage}
 \begin{minipage}[b]{63mm}
  \subcaption{Spatial resolution: \SI{30}{\m}}
  \label{fig:chap06-sat-old}
  \centering
  \includegraphics[width=60mm]{fig/chap06-FIG4b}
 \end{minipage}
 \begin{minipage}[b]{63mm}
  \subcaption{Vertical spacing of contours: \SI{10}{\m}}
  \label{fig:chap06-dem-new}
  \centering
  \includegraphics[width=60mm]{fig/chap06-FIG4c}
 \end{minipage}
 \begin{minipage}[b]{63mm}
  \subcaption{Spatial resolution: \SI{5}{\m}}
  \label{fig:chap06-sat-new}
  \centering
  \includegraphics[width=60mm]{fig/chap06-FIG4d}
 \end{minipage}
 \caption[Digital elevation models and satellite images compared in the study.]{Digital elevation models (a, c) 
and satellite images, depicted using the normalized difference vegetation index (b, d), compared in our study. 
The less detailed version is displayed at the top, while the more detailed version is shown on the bottom.}
 \label{fig:chap06-con-covars}
\end{figure}

Eight DEM derivatives were calculated: elevation (\elev), slope (\slp), aspect (\asp), northernness (\nor), 
flow accumulation (\acc), topographic wetness index (\twi), stream power index (\spi), and topographic 
position index (\tpi). \slp{} and \asp{} were calculated using \grass{r.param.scale} with seven window sizes 
(sampling support, analysis scale): \num{3}, \num{7}, \num{15}, \num{31}, \num{63}, \num{127}, and \num{255}. 
\asp{} was scaled to the standard \num{0}--\ang{360} range and orientation, and was transformed to \nor{} 
using $\texttt{NOR} = abs(\ang{180} - \texttt{ASP})$. \twi{} and \spi{} were calculated using \slp{} 
calculated with different window sizes, and \acc{} calculated using \grass{r.watershed}. \tpi{} was calculated 
using \saga{ta\_morphometry} with the same seven window sizes. The combination of DEM derivatives (\elev, 
\slp, \nor, \twi, \spi, and \tpi) and window sizes yielded $p = 36$ continuous predictor variables from each 
DEM.

The less detailed satellite image was acquired by the Landsat-\num{5} Thematic Mapper on December \num{26}, 
\num{2010} (available at Instituto Nacional de Pesquisas Espaciais - Divisão de Geração de Imagens -- 
\inpedgi) (\autoref{fig:chap06-sat-old}). It has \SI{8}{\bit} radiometric resolution and \SI{\sim30}{\m} 
spatial resolution. Spectral bands were orthorectified (Geomatica OrthoEngine) and radiometrically corrected 
(\grass{i.landsat.toar}). The more detailed satellite image comes from the RapidEye constellation (available 
at Ministério do Meio Ambiente -- \mma) (\autoref{fig:chap06-sat-new}). It was acquired on November \num{16}, 
\num{2012}, has \SI{16}{\bit} radiometric resolution, \SI{6.5}{\m} spatial resolution, and was orthorectified 
to \SI{5}{\m} spatial resolution. Both images were atmospherically (6S atmospheric model 
\cite{VermoteEtAl1997}, \grass{i.atcorr}) and topographically corrected (\grass{i.topo.corr}). Derived 
predictor variables are the spectral bands (except the thermal band) and vegetation indices (normalized 
difference vegetation index - NDVI, and soil-adjusted vegetation index - SAVI). Eight continuous predictor 
variables were derived from the Landsat-5~TM image and nine from the RapidEye image.

\subsubsection{Additional processing}
\label{subsubsec:chap06-sources-processing}

Soil maps, geologic maps, land use maps, and satellite images were registered with the prediction grid 
(\SI{5}{\m} pixel size) using nearest neighbour resampling. \demOld{} was registered using cubic resampling 
\cite{Samuel-RosaEtAl2013c}. Systematic positional errors were corrected using affine transformation 
\cite{Samuel-RosaEtAl2014}.

\subsection{Linear Mixed Model of Spatial Variation}
\label{subsec:chap06-lmm}

We model each of the soil properties of interest as the outcome of a spatial stochastic process. The model is 
composed of fixed and random effects \cite{HeuvelinkEtAl2001, LarkEtAl2006}. We use the point soil 
observations and spatially exhaustive predictor variables to calibrate the model and predict the outcome of 
the spatial stochastic process at unobserved locations. This fixed effect (deterministic trend), 
$m(\textbf{s})$, describes that part of the spatial variation of the soil property that is explained by the 
covariates. We assume here that is a linear function of the predictor variables. The random effect (stochastic 
residuals, latent variables), $e(\textbf{s})$, describes that part of the spatial variation that cannot be 
explained by the covariates \cite{Cressie1993}. It is represented by a spatially correlated, Gaussian 
distributed random variable, that is assumed stationary in the mean and covariance. Thus, the linear mixed 
model of spatial variation that we employed is given by

\begin{equation}\label{eqn:chap06-lmm}
 Y'(\textbf{s}) = m(\textbf{s}) + e(\textbf{s}) 
                = \sum_{j=0}^{p} \beta_{j}\cdot X_{j}(\textbf{s}) + e(\textbf{s}),
\end{equation}

\noindent{where $Y'(\textbf{s})$ is the soil property after Box-Cox transformation, $m(\textbf{s})$ and 
$e(\textbf{s})$ are defined as above, $\beta_{j}$ are the regression model coefficients, and 
$X_{j}(\textbf{s})$ is the regression model matrix, with $j = 0, 1, 2, \ldots, p$, $p$ being the number of 
predictor variables. Variable $X_{0}(\textbf{s})$ is taken as unity so that $\beta_{0}$ is the intercept.}

\subsubsection{Model selection}

We calibrated $k = 2^5 = 32$ multiple linear regression models for each soil property (fitted using ordinary 
least squares, OLS) to model the deterministic trend for each combination of the five covariates (recall from 
\autoref{sec:chap06-intro} that each covariate is available at two levels of spatial detail, hence $2^5$ 
combinations). The number of predictor variables used to calibrate each model varied among combinations 
between $p = 52$ and $p = 62$, because more detailed covariates enabled the derivation of a larger number of 
predictor variables (except the DEM). Backward VIF (variance inflation factor) selection followed by stepwise 
AIC (Akaike's Information Criterion) selection were used to select predictor variables to enter the models 
\cite{Samuel-RosaEtAl2014c, VenablesEtAl2002}.

The $k = 32$ multiple linear regression models calibrated for each soil property were ranked using the ratio 
between the regression sum of squares and the total sum of squares. Because stepwise regression results in 
biased models \cite{Harrell2001a}, the ratio of sum of squares was adjusted (${R}^{2}_{adj}$) using the number 
of predictor variables initially offered to enter the model instead of the reduced number of predictor 
variables that entered the model. Next, the five covariates were ranked based on how their level of spatial 
detail related with the calibration of models with improved predictive performance. The relation between the 
level of spatial detail of the covariates and model performance was evaluated using a graphical output called 
\emph{model series plot} (\Rpackage{pedometrics}, \citet{Samuel-RosaEtAl2014c}). Pedological evaluation 
of predictor variables included in the models was omitted because this was beyond our objectives.

The multiple linear regression model calibrated using only the less detailed covariates, which we call the 
\emph{baseline} model, and the multiple linear regression model with the highest ${R}^{2}_{adj}$, which we 
call the \emph{best performing} model, were extended to linear mixed models of spatial variation 
(\autoref{eqn:chap06-lmm}) for each soil property. Estimation of the parameters of the linear mixed models was 
performed using residual (restricted, marginal) maximum likelihood (REML) \cite{RibeiroEtAl2001, 
LarkEtAl2004}. The spatial correlation function adopted was the exponential function (this is equivalent to 
the Matérn correlation function with smoothness parameter $\nu = 0.5$ \cite{Stein1999}).

\subsubsection{Model validation}
\label{subsec:chap06-validation}

Only the \emph{baseline} and \emph{best performing} multiple linear regression and linear mixed models 
calibrated for each soil property were validated. Model validation was performed using leave-one-out 
cross-validation (LOO-CV) \cite{BrusEtAl2011}. All model parameters were re-estimated at each LOO-CV run 
to reduce bias \cite{LaslettEtAl1987}. LOO-CV predicted values were back-transformed from the Box-Cox space 
to the original space of soil properties using stochastic simulation \cite{ChristensenEtAl2001}:

\begin{enumerate}[label=(\Roman*)]
 \item each predicted value and associated prediction error variance were used to simulate $n = \num{20000}$ 
 values from a Gaussian distribution;
 
 \item simulated values were back-transformed using $Y(s) = (Y'(s) \times \lambda + 1)^{1 / \lambda}$, if 
 $\lambda > 0$, and $Y(s) = exp(Y'(s))$, if $\lambda = 0$;
 
 \item the mean and variance of back-transformed simulated values were used as the predicted value and 
 prediction error variance in the original space of soil properties.
\end{enumerate}

Five error statistics were computed from the leave-one-out cross-validation results \cite{JanssenEtAl1995, 
KempenEtAl2010, BrusEtAl2011}. The mean error (\textit{ME}), which measures the prediction bias, the mean 
absolute error (\textit{MAE}) and the root mean squared error (\textit{RMSE}), which measure the prediction 
accuracy, the scaled root mean squared error (\textit{SRMSE}, also known as mean squared deviation ratio), 
which measures how well the prediction error variance matches the squared differences between predicted and 
observed soil property, where $\textit{SRMSE} > 1$ indicates under-estimation, while $\textit{SRMSE} < 1$ 
indicates over-estimation, and the amount of variance explained (\textit{AVE}, also known as coefficient of 
determination or ratio of scatter), which measures the fraction of the overall spread of observed values that 
is explained by the model. The AVE ranges from \num{0} to \num{100}, where $\textit{AVE} = 100$ is the optimal 
value.

\subsubsection{Spatial prediction}
\label{subsec:chap06-prediction}

Only the \emph{baseline} and \emph{best performing} linear mixed models calibrated for each soil property were 
used for spatial prediction. Spatial predictions at a fine grid of \num{\sim800000} point locations were made 
in the Box-Cox space using the best linear unbiased predictor (BLUP) with the empirical estimates of the 
random effects (EBLUP) \cite{LarkEtAl2006}. EBLUP with a fixed effect model is conceptually equivalent to 
kriging with external drift and regression kriging, and mathematically equivalent to kriging with external 
drift and universal kriging. Point predicted values and prediction error variances were back-transformed to the 
original soil property space using stochastic simulation as described above 
(\autoref{subsec:chap06-validation}).

\section{RESULTS}
\label{sec:chap06-results}

\subsection{Model Series Plots}

The model series plot is a graphical description of the relation between the prediction accuracy of multiple 
linear regression models and the covariates used to calibrate them (\autoref{fig:chap06-model-series}). The 
magnitude of improvement in prediction accuracy is depicted in the bottom panel with the ${R}^{2}_{adj}$. The 
top panel is interpreted both horizontally and vertically. In the vertical direction we identify which version 
of each covariate was used to calibrate a given model. The less and the more detailed versions are identified 
by the yellow (bright) and green (dark) colours, respectively. The \emph{baseline} model is identified by the 
column containing only yellow cells, while the column with only green cells represents the model calibrated 
using only the more detailed version of each covariate, which we call the \emph{most detailed} model. The 
first important results that we obtain from the model series plots is that a) the \emph{baseline} model is not 
the model with the lowest ${R}^{2}_{adj}$, which we call the \emph{poorest performing} model, and b) the 
\emph{most detailed} model is not the \emph{best performing} model.

\begin{figure}[!ht]
 \centering
 \begin{minipage}[b]{\textwidth}
  \subcaption{}
  \includegraphics[width=\textwidth]{fig/chap06-FIG5a}
 \end{minipage}
 \begin{minipage}[b]{\textwidth}
  \subcaption{}
  \includegraphics[width=\textwidth]{fig/chap06-FIG5b}
 \end{minipage}
 \begin{minipage}[b]{\textwidth}
  \subcaption{}
  \includegraphics[width=\textwidth]{fig/chap06-FIG5c}
 \end{minipage}
 \caption[Model series plots for CLAY, SOC, and ECEC.]{Model series plots for CLAY (a), SOC (b), and ECEC (c). 
The less and more detailed version of each covariate are identified with the yellow (bright) and green (dark) 
colours, respectively. Multiple linear regression models were ranked using their ${R}^{2}_{adj}$. Triangles 
show the mean ranking of the more detailed covariates (i.e. centre of green cells).}
 \label{fig:chap06-model-series}
\end{figure}

The row-wise analysis of the model series plots shows if a model calibrated with the more detailed version of 
a given covariate has a higher prediction accuracy. This information is retrieved by looking at the row-wise 
distribution of green cells -- these cells represent the $k = 16$ models calibrated using the more detailed 
version of a given covariate, irrespective of the version of the other covariates. The more concentrated the 
green cells are in the right half of the plot, the larger the relative benefit of using the more detailed 
version of that covariate. For example, the top row of the second model series plot shows the SOC models 
calibrated using the two versions of the land use map (\texttt{land}). All green cells are on the right half 
of the plot between rankings \num{1} and \num{16} (see the x axis). The four lower rows show that the green 
cells of the other four covariates are distributed along the entire ranking range (from \num{1} to \num{32}). 
This means that the relative benefit of calibrating a SOC model with a more detailed land use map is larger 
compared to that of using a more detailed version of the other covariates.

The centre of the row-wise distribution of the green cells for each covariate, calculated as the mean ranking, 
is represented by the triangles. The mean ranking quantifies the relative benefit of using a more detailed 
version of each covariate. For example, the mean ranking of the SOC models calibrated using the more detailed 
land use map is about \num{8} (top row), while the mean ranking of the models calibrated using the more 
detailed satellite image (\texttt{sat}) is close to \num{20} (bottom row). Using the more detailed DEM 
(\texttt{dem}) is almost as beneficial as using the more detailed geologic map (\texttt{geo}) -- the mean 
ranking of the SOC models calibrated using the more detailed version of these two covariates is about 
\num{15}--\num{16} (second and third rows). Using the more detailed version of the soil map (\texttt{soil}, 
fourth row) is not as beneficial as using \texttt{land}, \texttt{geo} or \texttt{dem}, but more beneficial 
than using \texttt{sat}. Because the covariates were ranked based on the mean rankings, the covariate 
displayed in the top row of each model series plot is the one which resulted in the largest improvement of the 
prediction accuracy when the more detailed version was used to calibrate the model -- for SOC this is the land 
use map.

For CLAY, calibrating the models with the more detailed soil map resulted in the largest improvement of the 
prediction accuracy relative to the other covariates. The DEM was the second most beneficial covariate (mean 
ranking of \num{15}), but the benefit of using its more detailed version was similar to that of using the more 
detailed version of any other covariate (mean rankings between \num{17} and \num{18}). Nine models had a 
poorer prediction performance than the baseline model, ranked \num{27}th, the poorest performing model being 
that calibrated with the more detailed land use map and satellite image. Despite these patterns, calibrating 
CLAY models with the more detailed version of any covariate resulted in a small improvement of the prediction 
accuracy, as evidenced by the small increases of the ${R}^{2}_{adj}$. The difference between the poorest and 
best performing models is less than \SI{3}{\pp} (percentage points). In comparison, for SOC, by simply 
using the more detailed land use map we already obtained a model ranked \num{9}th, an increase of \SI{8}{\pp} 
in ${R}^{2}_{adj}$ compared to the baseline model, ranked \num{24}th.

The same general pattern observed for SOC models was observed for ECEC models -- the more detailed land use 
map results in the largest improvement of the prediction accuracy. The main difference is that calibrating the 
models with the more detailed geologic map was slightly more beneficial for ECEC (mean ranking of \num{12}) 
than for SOC (mean ranking of \num{14}). The poorest performing ECEC model was that calibrated with the more 
detailed satellite image. Using only the more detailed land use map resulted in an improvement of \SI{6}{\pp} 
in ${R}^{2}_{adj}$ (model ranked \num{7}th), differing from the best performing model by only \SI{2}{\pp}. 
Using the more detailed version of all covariates except the soil map or satellite image resulted in increases 
of about \num{6} and \SI{7}{\pp} in ${R}^{2}_{adj}$, respectively. The baseline model was ranked as 
\num{28}th, which is a higher ranking than the models calibrated with all possible combinations of the more 
detailed satellite image and the more detailed soil map and/or DEM.

The patterns observed in the model series plots resulted from the change (increase or decrease) of the 
importance of each covariate on explaining the variance when the more detailed version was used 
(\autoref{tab:chap06-drop}). We used the \emph{baseline} and \emph{most detailed} models to quantify this 
change. Each model was refitted dropping one covariate at a time. The difference $\Delta$ between the 
${R}^{2}_{adj}$ of the model calibrated with all five $q$ covariates (${R}^{2}_{adj}{}_{q = 5}$) and the model 
calibrated without the $q$-th covariate ($R^{2}_{adj}{}_{q = 5 - 1}$) was calculated. The more positive 
$\Delta{R}^{2}_{adj}$ becomes, the more beneficial the more detailed version of the $q$-th covariate is for 
improving prediction accuracy. For CLAY, \texttt{dem} and \texttt{land} were the most important covariates in 
the \emph{baseline} model, while \texttt{geo} was the least important. The importance of \texttt{soil} and 
\texttt{geo} increased when their more detailed version was used (change of \SI{+0.013}{\pp} for both), while 
\texttt{sat}, \texttt{land} and \texttt{dem} became less important. For SOC and ECEC, \texttt{land} was not 
the most important covariate in the \emph{baseline} model. But it was the covariate whose importance had the 
largest positive shift when the more detailed version was used (\SI{+0.085}{\pp} for SOC and \SI{+0.045}{\pp}
for ECEC). \texttt{sat} became less important when the more detailed version was used -- see its low ranking 
in all model series plots. The increase of the importance of \texttt{geo} was larger for ECEC 
(\SI{+0.026}{\pp}) than for SOC (\SI{+0.013}{\pp}) -- see the difference in the mean ranking of \texttt{geo} 
in 
the SOC (\num{14}) and ECEC (\num{12}) model series plots.

\input{chap/tab/chap06-TAB4.tex}

\subsection{REML Fit of the Variogram Model}

\def\footnugget{\footnote{To be more precise, the small number of point observations separated by short 
distances reduces the ability of modelling the behaviour of the variogram near the origin as a whole.}}

The small improvement in the prediction accuracy of the CLAY linear mixed model calibrated with the more 
detailed covariates is evidenced by \autoref{fig:chap06-lmm}. The shape of the experimental variogram is very 
similar for both \emph{baseline} and \emph{best performing} linear mixed models, which is also true for SOC 
and ECEC. However, the sill variance had a very small reduction for CLAY compared to SOC and ECEC. The last 
two showed a more considerable improvement in prediction accuracy. It can also be seen that the number of 
point observations separated by short distances is very small, reducing the accuracy of the estimate of the 
nugget variance\footnugget{}. The result is that the estimated nugget variance changes rather erratically 
from the \emph{baseline} to the \emph{best performing} models, decreasing for CLAY and SOC, and increasing 
for ECEC.

\begin{figure}[!ht]
 \centering
 \begin{minipage}[b]{90mm}
  \subcaption{}
  \includegraphics[width=90mm]{fig/chap06-FIG6a} 
 \end{minipage}
 \begin{minipage}[b]{90mm}
  \subcaption{}
  \includegraphics[width=90mm]{fig/chap06-FIG6b}
 \end{minipage}
 \begin{minipage}[b]{90mm}
  \subcaption{}
  \includegraphics[width=90mm]{fig/chap06-FIG6c}
 \end{minipage}
 \caption[Linear mixed models for CLAY, SOC, and ECEC.]{Experimental variogram (dots) and REML fit of the 
linear mixed models (line) for CLAY (a), SOC (b), and ECEC (c). Left -- baseline model. Right -- best 
performing model.}
 \label{fig:chap06-lmm}
\end{figure}

\subsection{Validation}

The LOO-CV results indicate that the linear mixed models for CLAY are slightly positively biased, while 
those for SOC and ECEC are slightly negatively biased (\autoref{tab:chap06-cv-stats}). For both CLAY and ECEC, 
the \textit{MAE} shows that these models are more accurate than the multiple linear regression models, 
suggesting that the kriging step improves the prediction accuracy.

\ctable[
 caption = {Statistics$^a$ of the leave-one-out cross-validation of baseline and best performing multiple
 linear regression models (LM) and linear mixed models (LMM).},
 cap     = {Cross-validation of baseline and best performing models.},
 label   = tab:chap06-cv-stats,
 notespar,
 maxwidth = \textwidth,
 pos     = !th,
 % doinside = \scriptsize\setstretch{1.1}
 doinside = \small
 ]{llrrrrr}{
 \tnote[a]{Statistics: mean error (\textit{ME}), mean absolute error (\textit{MAE}), root mean squared error 
 (\textit{RMSE}), scaled root mean squared error (\textit{SRMSE}, unitless), and amount of variance explained 
 (\textit{AVE}, percent).}
 }{\FL
   \multicolumn{1}{l}{Model}&\multicolumn{1}{c}{Type}&\multicolumn{1}{c}{\textit{ME}}&\multicolumn{1}{c}{\textit{MAE}}&\multicolumn{1}{c}{\textit{RMSE}}&\multicolumn{1}{c}{\textit{SRMSE}}&\multicolumn{1}{c}{\textit{AVE}}\ML
   \multicolumn{7}{l}{CLAY (g kg$^{-1}$)}\NN
   ~~Baseline&LM&$ 1.31$&$52.1$&$ 72.1$&$0.89$&$56.8$\NN
   ~~&LMM&$ 0.94$&$48.5$&$ 68.8$&$1.03$&$60.7$\NN
   ~~Best performing&LM&$ 1.59$&$51.3$&$ 70.7$&$0.91$&$58.4$\NN
   ~~&LMM&$ 1.08$&$47.8$&$ 68.1$&$1.03$&$61.5$\ML
   \multicolumn{7}{l}{SOC (g kg$^{-1}$)}\NN
   ~~Baseline&LM&$-0.30$&$10.9$&$ 18.9$&$1.22$&$35.8$\NN
   ~~&LMM&$-0.39$&$11.0$&$ 19.4$&$1.43$&$32.5$\NN
   ~~Best performing&LM&$-0.20$&$10.1$&$ 16.9$&$0.91$&$49.0$\NN
   ~~&LMM&$-0.25$&$10.4$&$ 17.6$&$1.16$&$44.3$\ML
   \multicolumn{7}{l}{ECEC (mmol kg$^{-1}$)}\NN
   ~~Baseline&LM&$-0.88$&$70.6$&$112.4$&$0.97$&$22.3$\NN
   ~~&LMM&$-0.32$&$63.3$&$101.1$&$1.32$&$37.1$\NN
   ~~Best performing&LM&$-0.76$&$64.9$&$101.7$&$0.86$&$36.3$\NN
   ~~&LMM&$-0.29$&$62.6$&$ 97.9$&$1.09$&$41.1$\LL
}


Overall, all models had a moderate to poor prediction performance. The errors are, in absolute values, 
somewhat large, mainly for ECEC. The best \textit{AVE} are about \SI{60}{\percent} for CLAY, \SI{50}{\percent} 
for SOC, and \SI{40}{\percent} for ECEC. In general, the prediction error variance was under-estimated by the 
linear mixed models and over-estimated by the multiple regression models. The best estimates of the prediction 
error variance were obtained by both CLAY linear mixed models, and the ECEC baseline linear regression model.

For CLAY, the increase in the \textit{AVE} was larger when including a kriging step ($\Delta\textit{AVE} = 
\SI{3.9}{\pp}$) than when using more detailed covariates ($\Delta\textit{AVE} = \SI{1.6}{\pp}$). In the case 
of SOC, including a kriging step reduced the \textit{AVE} by \SI{3.2}{\pp}, and for ECEC, both strategies 
increased the \textit{AVE} (\autoref{tab:chap06-cv-stats}).

\subsection{Spatial Prediction}

Both \emph{baseline} and \emph{best performing} linear mixed models captured the same overall pattern of 
spatial variation of the soil properties (\autoref{fig:chap06-kriging}). The main difference is that the 
spatial patterns of the different covariates used to calibrate each model produced different features in the 
prediction maps. For example, the CLAY map produced by the best performing model 
(\autoref{fig:chap06-clay-best-pred}) displays abrupt changes in the predicted values in the north-north-east 
due to the use of the more detailed soil map. Strongly-marked features following the stream network obtained 
through the use of the more detailed DEM are also observed (\autoref{fig:chap06-clay-best-pred} and 
\autoref{fig:chap06-clay-best-var}).

SOC maps (\autoref{fig:chap06-soc-best-pred} and \autoref{fig:chap06-soc-best-var}) show peculiar features in 
the central part of the study area, where predictions reached values as high as \SI{507}{\gram\per\kilo\gram}, 
while the maximum value in the calibration data is \SI{163}{\gram\per\kilo\gram}. The extremely high predicted 
values resulted from the inclusion of the topographic position index derived from the more detailed DEM, using 
a window size of $15 \times 15$~pixels (\texttt{TPI\_10\_15}) to model the deterministic trend. 
\texttt{TPI\_10\_15} values in the point calibration data range from \num{-7} to \SI{6}{\m}, while in the 
central part of the study area they range from \num{12} to \SI{31}{\m}. Thus, feature-space extrapolation 
explains the extremely high predicted values for SOC. Abrupt changes in predicted SOC are also observed at 
locations with low to moderate SOC (\SIrange{40}{80}{\gram\per\kilo\gram}). This is caused by using the more 
detailed land use map.

Predicted ECEC (\autoref{fig:chap06-ecec-base-pred} and \autoref{fig:chap06-ecec-best-pred}) had a large 
dependency on land use and geologic maps. Several features observed in the prediction maps derive from these 
two covariates. The influence of land use is seen in the northern part, while in the western, central, and 
eastern parts the influence of both covariates create an irregular pattern in the spatial distribution of 
ECEC. It is also in these parts that the largest prediction error standard deviations occur, following the 
spatial pattern of the covariates.

\begin{figure}[!ht]
 \centering
 \begin{minipage}[b]{63mm}
  \subcaption{}
  \label{fig:chap06-clay-base-pred}
  \centering
  \includegraphics[width=63mm]{fig/chap06-FIG7a}
 \end{minipage}
 \begin{minipage}[b]{63mm}
  \subcaption{}
  \label{fig:chap06-clay-best-pred}
  \centering
  \includegraphics[width=63mm]{fig/chap06-FIG7d}
 \end{minipage}
 \begin{minipage}[b]{63mm}
  \subcaption{}
  \label{fig:chap06-soc-base-pred}
  \centering
  \includegraphics[width=63mm]{fig/chap06-FIG7b}
 \end{minipage}
 \begin{minipage}[b]{63mm}
  \subcaption{}
  \label{fig:chap06-soc-best-pred}
  \centering
  \includegraphics[width=63mm]{fig/chap06-FIG7e}
 \end{minipage}
 \begin{minipage}[b]{63mm}
  \subcaption{}
  \label{fig:chap06-ecec-base-pred}
  \centering
  \includegraphics[width=63mm]{fig/chap06-FIG7c}
 \end{minipage}
 \begin{minipage}[b]{63mm}
  \subcaption{}
  \label{fig:chap06-ecec-best-pred}
  \centering
  \includegraphics[width=63mm]{fig/chap06-FIG7f}
 \end{minipage}
 \caption[Predicted values for CLAY, SOC and ECEC.]{Predicted values for CLAY (\si{\gram\per\kilo\gram}) (a, 
b), SOC (\si{\gram\per\kilo\gram}) (c, d), and ECEC (\si{\milli\mole\per\kilo\gram}) (e, f) using the 
\emph{baseline} (left) and \emph{best performing} (right) linear mixed models.}
 \label{fig:chap06-kriging}
\end{figure}

\begin{figure}[!ht]
 \centering
 \begin{minipage}[b]{63mm}
  \subcaption{}
  \label{fig:chap06-clay-base-var}
  \centering
  \includegraphics[width=60mm]{fig/chap06-FIG8a}
 \end{minipage}
 \begin{minipage}[b]{63mm}
  \subcaption{}
  \label{fig:chap06-clay-best-var}
  \centering
  \includegraphics[width=60mm]{fig/chap06-FIG8d}
 \end{minipage}
 \begin{minipage}[b]{63mm}
  \subcaption{}
  \label{fig:chap06-soc-base-var}
  \centering
  \includegraphics[width=60mm]{fig/chap06-FIG8b}
 \end{minipage}
 \begin{minipage}[b]{63mm}
  \subcaption{}
  \label{fig:chap06-soc-best-var}
  \centering
  \includegraphics[width=60mm]{fig/chap06-FIG8e}
 \end{minipage}
 \begin{minipage}[b]{63mm}
  \subcaption{}
  \label{fig:chap06-ecec-base-var}
  \centering
  \includegraphics[width=60mm]{fig/chap06-FIG8c}
 \end{minipage}
 \begin{minipage}[b]{63mm}
  \subcaption{}
  \label{fig:chap06-ecec-best-var}
  \centering
  \includegraphics[width=60mm]{fig/chap06-FIG8f}
 \end{minipage}
 \caption[Prediction error standard deviations for CLAY , SOC and ECEC.]{Prediction error standard deviations 
for CLAY (\si{\gram\per\kilo\gram}) (a, b), SOC (\si{\gram\per\kilo\gram}) (c, d), and ECEC 
(\si{\milli\mole\per\kilo\gram}) (e, f) using the \emph{baseline} (left) and \emph{best performing} (right) 
linear mixed models.}
 \label{fig:chap06-kriging-variance}
\end{figure}

The smallest prediction error standard deviations occur at lower elevations, along the three main streams, and 
close to the water outlet in the southern part of the study area. These areas have the highest density of 
point soil observations used to calibrate the models, and the smallest values for all three soil properties. 
While the first determines the accuracy of the EBLUP, the second influences the final accuracy through the 
back-transformation of predicted values.

\section{DISCUSSION}

Our main goal was to evaluate whether investing in more spatially detailed covariates improves the accuracy of 
soil maps. We saw that calibrating the models with more detailed covariates generally has a small to moderate, 
but positive, impact on the predictions. The magnitude of this benefit depends on the magnitude of the 
increase of the spatial detail of the covariate, on the other covariates included in the model, and on the 
soil 
property. However, there seems to be a limit above which the increase of spatial detail has a negative impact 
on the predictions. In the next two subsections we interpret the results from a pedological perspective and 
assess whether the investment in more detailed covariates is worthwhile or if alternatives to improve 
prediction accuracy should be favoured.

\subsection{Spatio-Temporal Controls of Soil Properties}

CLAY was moderately well predicted using less detailed covariates, with small improvement when using the more 
detailed covariates. CLAY was expected to have a strong correlation with topography and parent material. This 
correlation was already considerable when the less detailed DEM and geologic map were used, and improved only 
marginally with the more detailed version. One sensible explanation is that the effective (actual rather than 
theoretical) spatial detail of the two geologic maps was similar, although they had a four-fold difference in 
the size of the minimum legible delineation (see \citet{HenglEtAl2006a} for a discussion on effective 
scale). For the DEM, many studies have already suggested that its resolution may be of secondary importance 
when calculating DEM derivatives for soil mapping \cite{ZhuEtAl2008, BehrensEtAl2010a, MillerEtAl2015}. The 
influence of land use on CLAY is currently small due to reduction of soil erosion in the first decade of the 
\num{21}st century \cite{MiguelEtAl2012, TenCatenEtAl2012b}. A moderate within-field spatial variation may 
exist due to past erosional processes \cite{MouraBueno2012}, but we lack evidence of how well this source of 
variation was captured in the present-time point soil data.

It is worthwhile to consider the influence of the more detailed soil map on predicting CLAY. Due to its 
production process, the more detailed soil map derives a large amount of spatial detail from the geologic map, 
land use map and DEM -- note that the second-best performing model for CLAY included the more detailed 
geologic map instead of the more detailed soil map (\autoref{fig:chap06-model-series}). However, most of the 
additional spatial detail included in the more detailed soil map was probably based on the spatial variation 
of soil texture, because this is a strongly marked soil feature in the area \cite{MiguelEtAl2012}. Soil 
texture 
is one of the most important soil properties used by soil surveyors in the field to identify mapping units 
\cite{Legros2006}. These findings help explain why in the end the more detailed soil map was the most 
beneficial for CLAY instead of the geologic map.

SOC and ECEC were considerably better predicted when more detailed covariates were used. Our expectation that 
SOC and ECEC would have a strong correlation with land use was confirmed by the fact that this covariate 
explained a large amount of the variance and was highly beneficial for improving the predictions. Although the 
available point soil data are limited to the \num{2004}--\num{2011} period, we believe that land use changes 
in the last \num{30}~years \cite{MiguelEtAl2012, TenCatenEtAl2012b} strongly affected SOC and ECEC. Thus, the 
more detailed land use map is likely to have considerably improved model performance because it is up-to-date 
and, possibly, because it has \num{40}~times more spatial detail than its less detailed version. Despite the 
fact that the two land use maps used in this study were from different time periods, which confounds the 
analysis, the results obtained indicate that a more detailed land use map improves the prediction of SOC. For 
example, the areas used for crop agriculture, which are well known for having lower SOC and ECEC 
\cite{Menezes2008, MouraBueno2012}, are not depicted in the less detailed land use map.

We expected SOC to have a stronger correlation with the DEM than with the geologic map due to its strong 
dependence on erosion, but we observed the contrary. This result may be partially explained by the fact that 
there is a strong relation between geology and topography in the study area \cite{Sartori2009}. Due to its 
production process \cite{MacielFilho1990}, the geologic maps can be interpreted as an aggregated version of a 
DEM. A second sensible explanation is that the effect of erosion on SOC is not that large because erosion was 
considerably reduced in the last decade \cite{MiguelEtAl2012, TenCatenEtAl2012b}. A last possible explanation, 
which integrates the previous two, is the existence of a spatial relation between SOC and CLAY, the last being 
strongly correlated with parent material. These relations help explain why the more detailed DEM was almost as 
beneficial as the more detailed geologic map for SOC predictions. In the case of ECEC, our expectation of a 
strong dependency on a more detailed geologic map for producing more accurate predictions was confirmed.

The observed benefit of the more detailed geologic map and DEM for making more accurate CLAY, SOC, and ECEC 
predictions suggests that these soil properties are spatially related in the study area. We also hypothesize 
that the complexity of current land use makes it difficult to achieve SOC and ECEC models with performances 
comparable to CLAY. One important source of variation in forested areas is its use for animal grazing 
\cite{SamuelRosaEtAl2011a}. This influences nutrient cycling and soil nutrient availability 
\cite{SchramaEtAl2013}. Current remote sensing technology is unable to capture the data needed to proxy the 
environmental conditions created by these processes.

\subsection{Using More Detailed Covariates}

More detailed covariates are usually expected to improve predictions in soil mapping \cite{CavazziEtAl2013, 
MaynardEtAl2014}. However, deciding whether to invest or not in more detailed covariates requires careful 
thinking and depends on case-specific elements. We generally saw improvement in the predictions in our study, 
but the improvement was not large and may not outweigh the costs. Also, the models calibrated with the more 
detailed versions of all covariates were not the best performing models. Using more detailed satellite images 
and land use maps degraded CLAY predictions. Although the more detailed soil map had the largest benefit for 
CLAY, it may be too costly and impractical since its production usually requires having available more 
detailed 
versions of all other covariates. For SOC and ECEC, simply using a more detailed land use map resulted in 
considerably more accurate predictions. However, the superior performance may not outweigh the extra costs 
because producing a more detailed land use map usually requires up-to-date field observations and satellite 
images. Thus, the decision to adopt a more detailed covariate for soil mapping will ultimately depend on a 
trade-off between the increased accuracy and the extra budget required. It may also depend on other potential 
applications of the covariates, but this is not our concern here.

One interesting observation is that if a less detailed covariate yields poor predictions, its more detailed 
version has the potential to produce larger improvement in model performance. However, this is only a 
potential, not a guarantee. For instance, \citet{EldeiryEtAl2008} were not able to increase the $R^2 = 
0.31$ of linear regression models of soil salinity by more than \num{0.07}~points using \num{7.5}~times more 
detailed satellite images. On the other hand, model performance is likely to be hardly improved using more 
detailed covariates if their less detailed version has already produced accurate predictions. This agrees with 
findings by \citet{ThompsonEtAl2001} and \citet{KimEtAl2014}.

We also observed that the predictions can be degraded when using the more detailed version of covariates. In 
our study, this happened with the satellite image (all three soil properties), land use map (CLAY) and soil 
map (SOC and ECEC). A (small) benefit was observed only when these covariates were used along with the more 
detailed version of other covariates. As pointed out above, such a small benefit may not outweigh the increase 
in mapping costs. The trade-off between reducing model performance and being beneficial seems to depend on how 
much more spatial detail a covariate will have and on its correlation with the soil property. For example, the 
land use map was strongly correlated with SOC and ECEC, but not with CLAY, and its more detailed version had 
\num{40}~times more spatial detail. It helped improve SOC and ECEC predictions, but degraded CLAY predictions, 
resulting in only a small improvement when used along with the more detailed satellite image and geologic map.

If the influence of a more detailed covariate depends on the increase of spatial detail, then the priority 
should be to improve the spatial detail of the most beneficial covariate. This requires solid subject area 
knowledge because empirical evidence from the \emph{baseline} model may be insufficient. The most beneficial 
covariate is not necessarily that which explained the largest part of the variance in the \emph{baseline} 
model (see \autoref{tab:chap06-drop}). This occurs because increasing the spatial detail reduces the 
correlation between the covariate and the soil property. And also because there is little room to improve a 
correlation that is already high in the \emph{baseline} model. \citet{CavazziEtAl2013} suggest that the 
more detailed covariate has an excess of detail, a \q{noise} that degrades the predictions. This could explain 
the results for \texttt{sat}: higher resolution images can resolve smaller objects (e.g. individual plants) 
whose spectral behaviours are highly variable, adding noise to the \texttt{sat}-soil property correlation; on 
the other hand, lower resolution images capture collections of objects, and thus their variation is smoothed 
out in the pixel, reducing noise.

According to information theory one should optimize (maximize) the correlation between the point soil data and 
the covariates. This was described elsewhere as matching the \q{phenomenon scale} (the spatial pattern of the 
soil property) with the \q{analysis scale} (the spatial pattern of the covariates) \cite{DunganEtAl2002, 
MillerEtAl2014}. Finding the \q{optimum} requires evaluating the strength of the correlation using covariates 
with different levels of spatial detail \cite{DragutEtAl2009, CavazziEtAl2013, MillerEtAl2015}. Our results 
show that this approach may be too costly and impractical. Since modern soil mapping techniques explore only 
the empirical relation among environmental conditions and soil properties \cite{Grunwald2009}, the \q{optimum}
is a \q{conditional optimum} -- conditional on the point soil data available. It does not necessarily mean 
that the most accurate predictions will be made, but only that there is a level of spatial detail at which the 
correlation between the covariate and the point soil data is at a maximum. We suggest that instead more 
comprehensive approaches should be used to explore the full potential of the available covariates (see 
\citet{BehrensEtAl2010a} and \citet{MillerEtAl2015} for examples).

Finally, one must still judge whether the potential improvement in predictions is sufficient given the extra 
costs involved with using more detailed covariates. If the extra budget is spent on deriving more detailed 
covariates, we suggest that it may be better to substantially improve the detail of a less influential
covariate than marginally increase the detail of the most influential covariate. However, other means to spend 
the extra budget should be considered. For instance, it may be more efficient to concentrate on obtaining more 
soil observations. These may focus on better capturing the short range spatial variation \cite{BrusEtAl2007a} 
or improving the representation of the feature space to avoid undesirable extrapolations 
\cite{MinasnyEtAl2006b}.

\section{CONCLUSIONS}

This study has shown that:

\begin{enumerate}[label = (\Roman*)]
 \item Using more detailed covariates results in only a modest increase in the prediction accuracy of linear 
 prediction models;
 
 \item A more detailed covariate has a greater potential to improve prediction accuracy when the soil property
 is poorly predicted by its less detailed version;
 
 \item The impact on prediction accuracy when using the more detailed version of a less important covariate 
 may depend on which other covariates are included in the model;
 
 \item Choosing whether or not to invest in more detailed covariates depends on the strength of the 
 relationship between the covariates and the soil property being modelled, and on the relative difference 
 between the less detailed and the more detailed versions of the covariates.
\end{enumerate}
 % Spatial point pattern analysis of soil survey sampling locations
\artigotrue
\chapter{OPTIMIZATION OF SAMPLE CONFIGURATIONS FOR SPATIAL TREND ESTIMATION FOR SOIL MAPPING}
\label{chap:chap07}

\def\ptkeys{}

\begin{chapterabstract}{brazilian}{\ptkeys}

\end{chapterabstract}

\def\enkeys{}
  
\begin{chapterabstract}{english}{\enkeys}

\end{chapterabstract}

\formatchapter

\section{INTRODUCTION}
\label{sec:chap07-intro}

\titlenote{This chapter is based on the study \textit{spsann -- optimization of sample patterns using spatial 
simulated annealing}, presented at the EGU General Assembly 2015 \cite{Samuel-RosaEtAl2015a}, and 
\textit{Optimization of sample configurations for spatial trend estimation}, presented at Pedometrics 2015 
\cite{Samuel-RosaEtAl2015d}. Also collaborated in the preparation of this document: Dick J. Brus (Alterra, 
Wageningen University and Research Centre, the Netherlands), Gerard B. M. Heuvelink (ISRIC -- World Soil 
Information), Gustavo M. Vasques (Embrapa Soils, Brazil), and Lúcia Helena Cunha dos Anjos (Universidade 
Federal Rural do Rio de Janeiro, Brazil).}

Modern soil mapping is based on using a model of spatial variation composed of two terms,

\begin{equation}
 Y(\boldsymbol{s}) = m(\boldsymbol{s}) + e(\boldsymbol{s}).
\end{equation}\label{eq:chap07-lmm}

\def\footgerard{\footnote{Gerard Heuvelink shared the same opinion during his Richard Webster Medal speech at 
the conference of the Pedometrics Commission of the IUSS, which took place from 14--18 September 2015, in 
Córdoba, Spain.}}

\nointent The first term in the right-hand size in \autoref{eq:chap07-standard-model} is the spatial trend, 
which corresponds to the spatial variation of the soil property $Y(\boldsymbol{s})$ that is explained 
deterministically using spatially exhaustive covariates; the remaining spatial variation of 
$Y(\boldsymbol{s})$ is explained stochastically with the second term \cite{Cressie1993}. Soil scientists 
devoted all their attention to $m(\boldsymbol{s})$ for more than a century \cite{Jenny1961, Florinsky2012}. 
Post-war technological developments in the fields of mathematics, statistics, and informatics, made many soil 
scientists turn their focus to $e(\boldsymbol{s})$ \cite{WebsterEtAl1990}. Recent developments in remote 
sensing and machine-learning algorithms made those soil scientists shift their attention back to 
$m(\boldsymbol{s})$ \cite{MooreEtAl1993} -- but without forgetting of $e(\boldsymbol{s})$ \cite{OdehEtAl1994} 
--, which now usually explains a considerably large proportion of the variation of $Y(\boldsymbol{s})$ 
compared to $e(\boldsymbol{s})$\footgerard. Besides, it is in $m(\boldsymbol{s})$ where we can incorporate most 
of our pedological knowledge \cite{Lark2012}.

Recent studies have shown that using more detailed covariates or more complex machine-learning algorithms can 
deliver more accurate soil maps, but the increase in prediction performance may be modest 
(\cite{Samuel-RosaEtAl2015}) and largely depends on the calibration data \cite{HeungEtAl2016}. Limited to the 
currently available covariates and machine-learning algorithms, and to the existing pedological knowledge, one 
of the major operational issue that needs to be solved in any soil mapping project is how to design an 
efficient spatial sample to estimate $m(\boldsymbol{s})$. The sampling method most commonly used to solve this 
problem is the \emph{conditioned Latin hypercube sampling} (CLHS). The CLHS was developed by Budiman Minasny 
and Alex McBratney at the University of Sydney in 2005, using an idea borrowed from the Latin hypercube 
sampling \cite{McKayEtAl1979, MinasnyEtAl2006b}. The popularity of the CLHS is due to its non\-/probabilistic 
nature, seen as a link with the sampling strategies used in \q{traditional soil survey}, easiness to 
implement, and the high flexibility which makes the addition of new features simple \cite{MinasnyEtAl2010a, 
RoudierEtAl2012, MulderEtAl2013, CarvalhoJuniorEtAl2014, CliffordEtAl2014}.

The CLHS is a heuristic strategy of creating spatial samples that aim at three objectives: ($\mathcal{O}_1$) 
uniform coverage of the marginal distribution of numeric covariates (continuous and discrete data, e.g. 
elevation, slope, etc.), ($\mathcal{O}_2$) proportional sample sizes for the classes of factor covariates 
(binary, categorical, and ordinal data, e.g. geology, land use, etc.), and ($\mathcal{O}_3$) reproduction of 
the linear correlation of numeric covariates. The main idea was that if a spatial sample reproduces the 
marginal distribution of the numeric and factor covariates, as well as the correlation matrix of the numeric 
covariates, it will approximately cover the multivariate distribution of the covariates -- this should put us 
closer to identifying the \q{true} spatial trend if we are (or assume to be) ignorant about its form.

Some critiques of the CLHS appeared in the literature since it was first published. Most of them focused on 
operational difficulties encountered in the field. For example, \citet{CambuleEtAl2013} argued that the 
CLHS is impractical in poorly-accessible areas, but \citet{RoudierEtAl2012} and 
\citet{MulderEtAl2013} showed that this is just a matter of how the algorithm is implemented. And 
\citet{CliffordEtAl2014} presented an algorithm for selecting an alternative sampling point when a CLHS 
sample point is inaccessible. Only recently soil scientists started paying more attention to the theoretical 
and algorithmic aspects of the CLHS. \citet{MinasnyEtAl2010a} demonstrated that, given an assumed known 
linear spatial trend, the CLHS is suboptimal. \citet{CliffordEtAl2014} questioned the importance of 
meeting the third objective ($\mathcal{O}_3$), as well as the mathematical approach used to find a solution 
for all three objectives jointly (see below). Finally, \citet{Brus2015} proposed an alternative method 
for selecting Latin hypercube samples with known inclusion probabilities so that these samples can also be 
used for design-based inference.

Our objective is to propose conceptual and algorithmic improvements on the CLHS, all of which we describe in 
the next section. We then evaluate if the proposed improvements result in a more accurate representation of 
the feature space and spatial predictions.

\section{PROPOSED IMPROVEMENTS}

\subsection{Defining the Marginal Sampling Strata}

Given a \emph{numeric} covariate, the CLHS uses the sample size $n$ to define the number of marginal 
sampling strata $c$, i.e. $c = n$, and the interpolated sample quantiles to define the breakpoints of the 
$c$ marginal sampling strata. The first objective of the CLHS ($\mathcal{O}_1$) is to have exactly one 
sample point falling in each marginal sampling strata. However, depending on the level of discretization of 
the covariate values, the CLHS may produce replicated breakpoints in the regions with a relatively high 
frequency of covariate values. For example, given a sample size of $n = 5$ and a covariate $\boldsymbol{a}$ 
with (ordered integer) values $\boldsymbol{a} = (1, 1, 1, 1, 2, 2, 3, 3, 4, 5, 8, 9, 9, 9, 9)$, the lower and 
upper boundaries of the marginal sampling strata are $\boldsymbol{a}_{mss} = (1.0, 1.0, 2.6, 4.4, 9.0, 9.0)$. 
Because the marginal sampling strata in which a sample point $b_i$ falls is evaluated using the indicator 
function

\begin{equation*}
 b_{sol_i} = 
 \begin{cases}
  1, & \text{if}\ a_{mss_j} \leq b_i \leq a_{mss_{j + 1}}\ \text{and}\ j = 1 \\ 
  1, & \text{if}\ a_{mss_j} < b_i \leq a_{mss_{j + 1}}\ \text{and}\ j > 1 \\ 
  0, & \text{otherwise}
 \end{cases}
\end{equation*}

\noindent where $i = 1, 2, \ldots, n$, and $j = 1, 2, \ldots, c$, the first and last marginal sampling strata 
of $\boldsymbol{a}$ will be empty, and the respective $n‘ = 2$ sample points will be allocated among the other 
three marginal sampling strata, with the set of allocation solutions $\boldsymbol{b}_{sol} = \{(0, 2, 1, 2, 
0), (0, 1, 2, 2, 0), (0, 2, 2, 1, 0)\}$. Ergo, the CLHS will be unable to find the globally optimum allocation 
solution $\boldsymbol{b}_{sol} = (1, 1, 1, 1, 1)$.

We propose defining the marginal sampling strata using only the unique values of the sample quantiles 
estimated with a discontinuous function \cite{HyndmanEtAl1996}. Accordingly, for our example, 
$\boldsymbol{a}_{mss} = (1, 2, 4, 9)$. The number of sample points that should fall in each marginal sampling 
strata is directly proportional to the number of sampling units (grids cells of a raster image) in that 
stratum of the covariate. For $\boldsymbol{a}$, this is $\boldsymbol{b}_{sol} = (2, 1, 2)$. The direct 
consequence of this modifications is that, given a set of $p$ covariates, each of them will potentially have a 
different number of (quasi-equal-size) marginal sampling strata, i.e. $c_i \leq n$, where $i = 1, 2, \ldots, 
p$. This will ultimately depend on the shape of their empirical frequency distribution, on the level of 
discretization of the covariate values, and on the sample size $n$.

\subsection{Measuring the Association/Correlation Between Covariates}

Two of the objectives of the CLHS ($\mathcal{O}_1$ and $\mathcal{O}_3$) are concerned with \emph{numeric} 
covariates, while only one ($\mathcal{O}_2$) focuses on \emph{factor} covariates. $\mathcal{O}_1$ and 
$\mathcal{O}_2$ are mathematically equivalent -- they aim at the coverage of the marginal distribution of the 
numeric and factor covariates --, and $\mathcal{O}_3$ measures the similarity between the population and 
sample correlation matrices of the numeric covariates as estimated with the Pearson`s $r$. The CLHS ignores 
the association among factor covariates, as well as of those with the numeric covariates. This means that the 
CLHS gives more importance to numeric covariates. Such a bias cannot be corrected by simply attributing 
different \emph{weights} to each objective (see below).

We propose to replace the Pearson`s $r$ with the Cramér`s $v$

\begin{equation}
 v =  \sqrt{\frac{\chi^2 / n}{min(ncol - 1, nrow - 1)}},
\end{equation}\label{eq:chap07-cramer}

\noindent where $nrow$ and $ncol$ are the number of rows and columns of the bivariate contingency table, 
$n$ is the sample size, and $\chi^2$ is the chi-squared statistic

\begin{equation}
 \chi^2 = \sum_{i = 1}^{nrow}\sum_{j=1}^{ncol}\frac{(O_{ij} - E_{ij})^2}{E_{ij}},
\end{equation}\label{eq:chap07-chi-squared}

\noindent where $O_{ij}$ and $E_{ij}$ are the observed and expected frequency, respectively, the marginal 
proportions of $O$ being the maximum likelihood estimates of the marginal proportions of $E$ 
\cite{Cramer1946, Agresti2002}. The Cramér`s $v$ is a measure of association between factor covariates that 
ranges from $0$ to $+1$: the closer to $+1$, the larger the association between two factor covariates. 
Accordingly, the only requirement for using the Cramér`s $v$ -- instead of the Pearson`s $r$ -- is that any 
numeric covariate be transformed into a factor covariate, with the factor levels defined using the marginal 
sampling strata. One could still use the Pearson`s $r$ when all covariates are numeric because computing the 
Cramér`s $v$ is more computationally demanding.

\subsection{Aggregating the Objectives}

Sampling for spatial trend estimation is a \emph{multi-objective combinatorial optimization problem} (MOCOP): 
we have to find a spatial sample that meets a list of objectives among an almost infinite set of possible 
spatial samples. An important step for solving a MOCOP is to define each objective as a function, i.e. an 
\emph{objective function} $f_i$ \cite{Arora2011}. An $f_i$ associates a numerical value with each spatial 
sample as a function only of the values of the $p$ covariates used to describe the spatial domain -- also 
known as \emph{design variables} \cite{Arora2011} -- at the $n$ sample points. The lower the objective 
function value, the closer the spatial sample is to meeting the respective objective. Thus, when solving a 
MOCOP, one aims at minimizing the vector of $k$ objective functions \cite{Arora2011}

\begin{equation}
 \boldsymbol{f}(\boldsymbol{X}) = (f_1(\boldsymbol{X}), f_2(\boldsymbol{X}), \ldots, f_k(\boldsymbol{X})),
\end{equation}

\noindent where $\boldsymbol{X}$ is the design matrix, a $n \times p$ matrix subject to the implicit 
constraints imposed by the finiteness of the spatial domain and discreteness of the $p$ design variables. 
These implicit constraints define the set of values that can be assigned jointly to the design variables, i.e. 
the $p$-dimensional \emph{feasible design space} $\mathcal{S}$, which, in turn, defines the set of numerical 
values that can be returned by the objective functions, i.e. the $k$-dimensional \emph{feasible objective 
space} $\mathcal{Z}$ \cite{MarlerEtAl2004}.

Ideally, there is a traceable unique \emph{point cloud} $\boldsymbol{X}^*$ (i.e. a spatial sample with the 
values of the covariates at its sample points) that minimizes all objective functions simultaneously 
\cite{MarlerEtAl2009}. However, in practice such a unique point cloud seldom exists, and if it exists it is 
hard to find. In most cases there is a large set of optima point clouds that map onto a set of optima points 
on $\mathcal{Z}$ because, for example, multiple point clouds can return the very same objective function value 
\cite{Arora2011}. The set of optima point clouds is commonly defined using the concept of \emph{Pareto 
optimality} \cite{MarlerEtAl2004}: a point cloud $\boldsymbol{X}^*$ in $\mathcal{S}$ is Pareto optimum if and 
only if there is no other point cloud $\boldsymbol{X}$ in $\mathcal{S}$ that decreases the value of at least 
one objective function without increasing the value of another objective function.

A reasonable strategy to find a single optimum solution is to aggregate the objective functions into a single 
\emph{utility function} $U$ \cite{MarlerEtAl2005}. The most common aggregation method is the \emph{weighted 
sum} method, which is used in the CLHS. It employs weights to incorporate the \emph{a priori} preferences of 
the user, their relative values reflecting the importance of each objective function \cite{MarlerEtAl2009}. 
Thus, the MOCOP boils down to minimizing the convex combination of objective functions

\begin{equation}
 U = \sum_{i=1}^{k} w_i f_i(\boldsymbol{X}),
\end{equation}\label{eq:chap07-utility}

\noindent which means that the weights $w_i$ are constrained to $w_i > 0$ and $\sum_{i=1}^{k} w_i = 1$ 
\cite{MarlerEtAl2005, MarlerEtAl2009}. An important requirement of the weighted sum method is that the 
objective functions be scaled to the same approximate range of values so that any potential numerical 
dominance can be eliminated or minimized, and the weights can play the desired role \cite{MarlerEtAl2005, 
MarlerEtAl2009}. 

There are several methods to scale the objective functions \cite{MarlerEtAl2005}. The Fortran source code of 
the CLHS shows that, although not mentioned in the original paper, the CLHS scales $\mathcal{O}_1$ and 
$\mathcal{O}_3$ using the \emph{upper-bound approach}, $f_i'' =f_i(\boldsymbol{X}) / f_i^{max}$, where 
$f^{max}_{\mathcal{O}_1} = n \times p^{num}$ and $f^{max}_{\mathcal{O}_3} = 0.5p^{num^2} + p^{num}$, 
$p^{num}$ being the number of numerical covariates. $\boldsymbol{f}^{max}$ is a rough estimate of the 
single worst solution for $\mathcal{O}_1$ and $\mathcal{O}_3$, called the \emph{nadir point cloud} 
\cite{MarlerEtAl2004}. Thus, this transformation results in a non-dimensional objective function with an upper 
limit around 1, and its use relies on the fact that, by definition, the three objective functions yield 
objective function values of very different orders of magnitude: $\mathcal{O}_1$ > $\mathcal{O}_3$ > 
$\mathcal{O}_2$. This is because $\mathcal{O}_1$ uses the number of sample points per strata (0--n), while 
$\mathcal{O}_3$ uses the linear correlation coefficient (-1--1), and $\mathcal{O}_2$ uses the proportion of 
sample points per strata (0--1).

We believe that the \emph{upper-bound approach} is insufficient for a proper scaling of the objective 
functions because $\boldsymbol{f}^{max}$ usually is unattainable -- i.e. it does not correspond to any point 
cloud in $\mathcal{S}$, and/or is far from the Pareto optimum set \cite{MarlerEtAl2004}. Defining 
$\boldsymbol{f}^{max}$ as the median of the objective functions over multiple spatial samples generated by 
simple random sampling \cite{CliffordEtAl2014} is a suboptimal strategy because it only ensures that the 
objective functions will have similar orders of magnitude at the beginning of the optimization, which might 
have a negligible influence in the definition of $\mathcal{Z}$ \cite{MarlerEtAl2005}. Besides, provided the 
optimization algorithm is well designed, the starting point should not influence the solution of the MOCOP 
(see below).

We propose using a more robust approach, i.e. the \emph{upper-lower bound approach},

\begin{equation}
 f_i'' = \frac{f_i(\boldsymbol{X}) - f_i^{\circ}}{f_i^{max} - f_i^{\circ}}
\end{equation}

\nointent where $f_i''$ is the $i$th non-dimensional, scaled objective function constrained between zero 
and one \cite{MarlerEtAl2005}. Because of the above-mentioned problems regarding the definition of 
$\boldsymbol{f}^{max}$, it is more appropriate to use the \emph{Pareto maximum}, $f_i^{max} = max_{1 \leq j 
\leq k} f_ i(\boldsymbol{X}_j^*)$, where $\boldsymbol{X}_j^*$ is the point cloud that minimizes the $j$th 
objective function \cite{MarlerEtAl2005}. In practice, we find the optimum point cloud for each objective 
function individually, and then calculate the objective function value of every other objective function. The 
Pareto maximum of a given objective function is the largest absolute maximum value obtained for that objective 
function. The same applies for $f_i^{\circ}$, the \emph{utopia point} -- the single best solution for the 
$i$th objective function, which exists in the objective space, but usually is unattainable, i.e. it does not 
correspond to any point cloud in $\mathcal{S}$ \cite{Arora2011} -- which is replaced with the Pareto 
minimum. The drawback of this approach is the extra time needed to optimize the $k$ objective functions 
individually.

\subsection{Resulting Problem Definition}

Given the proposed modifications, the problem of sampling for spatial trend estimation for soil mapping is 
redefined using two objective functions,

\begin{equation}
 \text{CORR} = \sum_{i=1}^{p}\sum_{j=1}^{p}|\varphi_{ij} - v_{ij}|,
\end{equation}\label{eq:chap07-corr}

\noindent where $\varphi_{ij}$ and $v_{ij}$ are the population and sample associations (or correlations in 
case all covariates are numeric) at the $i$th row and $j$th column of the $p$-dimensional population and 
sample association (or correlation) matrices, and

\begin{equation}
 \text{DIST} = \sum_{i=1}^{p}\sum_{j=1}^{c_i} |\pi_{ij} - \gamma_{ij}|,
\end{equation}\label{eq:chap07-dist}

\noindent where $\pi_{ij}$ and $\gamma_{ij}$ are the proportion of sample and population points that fall 
in the $j$th class (or marginal sampling strata) of the $i$th covariate, $c_i$ being the number of classes of 
the $i$th covariate. With these two objective functions, we define an utility function $U$ as in 
\autoref{eq:chap07-utility} aiming at a spatial sample that reproduces an 
\textbf{A}ssociation/\textbf{C}orrelation measure and the marginal \textbf{D}istribution of the 
\textbf{C}ovariates,

\begin{equation}
 \text{ACDC} = w_1\text{CORR} + w_2 \text{DIST},
\end{equation}\label{eq:chap07-acdc}

\noindent with $w_1 = w_2 = 0.5$ in the general setting.

\section{CASE STUDY}

We developed a case study to evaluate the proposed improvements and compare them with the original CLHS. It 
was based on using synthetic data derived from a real-world study case \cite{Samuel-RosaEtAl2015}. The study 
site is a small catchment of about \SI{2000}{\hectare} located on the southern edge of the plateau of the 
Paraná Geologic Province, Rio Grande do Sul, Brazil. The real-world dataset contains $n = 350$ point soil 
observations of the topsoil, and includes several soil properties, but only bulk density data 
(BUDE,~\si{\mega\gram\per\metre\cubic) was used ($n = 282$). The dataset also includes several covariates 
derived from area-class soil maps, digital elevation models, geological maps, land use maps, and satellite 
images. All processing steps used to derive the covariates were described by \citet{Samuel-RosaEtAl2015}.

\subsection{Soil Data Generating Process}

In an ideal world, we would create $\mathcal{R} \geq 100$ spatial samples of $\mathcal{N} \geq 2$ sizes 
with each of the $\mathcal{A} \geq 2$ algorithms that we want to compare. Then we would go to the field, 
sample the soil, and measure a property to construct $\mathcal{D} = \mathcal{R} \times \mathcal{N} \times 
\mathcal{A}$ calibration datasets. The same property would be measured at a fixed set of probabilistically 
selected validation sites. Each calibration dataset would be used to calibrate a model, with which we would 
predict at the validation sites. The $\mathcal{A}$ sampling algorithms would then be compared on how well 
they performed, for each of the $\mathcal{N}$ sizes, using the confidence interval of a prediction error 
statistic over all $\mathcal{R}$ spatial samples. Here, the random selection of  spatial samples would be the 
\emph{source of variation} \cite{deGruijterEtAl1990}.

In the real world\dots Because resources are limited, we decided to create only $\mathcal{R} = 1$ spatial 
sample with $\mathcal{N} = 3$ sizes with each of the $\mathcal{A} = 4$ sampling algorithms that we want to 
compare (CORR, DIST, ACDC, and CLHS). The variation had to come from another source: we chose it to be the 
soil property data. We did so using unconditional sequential Gaussian simulation \cite{Goovaerts2001, 
Pebesma2004}. To start, we defined a theoretical (or super-population) model, our \emph{soil data generating 
process}. To be as close to reality as possible, the soil data generating process was defined empirically 
calibrating a (non)linear mixed model to BUDE. The main calibration steps are as follows \cite{Breiman2001, 
LiawEtAl2002, DiggleEtAl2007, Lark2012}:

\begin{enumerate}
 \item Random regression forest: grow $n_{\text{trees}} = 500$ regression trees with a maximum terminal node 
 size of $n_{\text{node size}} = 5$ points, each tree grown using $n = 282$ calibration points randomly 
 selected with replacement from the set of $n = 282$ point soil observations (about $n_{\text{in-bag}} = 178$ 
 unique point soil observations), and $p_{\text{in-bag}} = 4$ covariates randomly selected at each split out 
 of a set of $p = 12$ covariates selected as in \citet{Samuel-RosaEtAl2015}.
 
 \item Out-of-bag predictions: use each of the $n_{\text{trees}} = 500$ regression trees from step (1) to 
 predict BUDE at the  point soil observations not included (out-of-bag) in the respective calibration dataset 
 (about $n_{\text{out-of-bag}} = 104$ point soil observations), and compute the average of the predicted BUDE 
 at each point soil observation (about 184 predicted values for each out-of-bag point).
 
 \item Linear mixed model: assume that the average of the out-of-bag predictions from step (2) are linearly 
 related to BUDE and present insignificant conditional bias, and use them as a covariate in the fixed effects 
 of a linear mixed model (LMM), the random effects modelled using the Whittle-Matérn model, all parameters 
 being estimated by Gaussian restricted maximum likelihood (REML).
\end{enumerate}

The parameters of the LMM are the coefficients $\beta_0$ and $\beta_1$ of the linear trend, which correct 
any linear bias in the random regression forest out-of-bag predictions \cite{LiawEtAl2002}, and the nugget 
($\tau^2$), sill ($\sigma^2$), and range ($\alpha$) of the Whittle-Matérn model. The shape parameter 
($\nu$) of the Whittle-Matérn model was defined separately, by choosing from a set of discrete values 
$\nu = (0.5, 1.0, 2.0, 4.0, 8.0)$ based on the resulting profile likelihood for $\nu$ and maximized restricted 
log-likelihood, and on the computing time \cite{Stein1999, DiggleEtAl2007}. The fitted LMM 
($\beta_0 = \SI{13.35}{\mega\gram\per\cubic\metre}$, $\beta_1 = 0.91$, 
$\tau^2 = \SI{349.51}{\mega\gram\per\metre\tothe{6}}$, $\sigma^2 = \SI{97.24}{\mega\gram\per\metre\tothe{6}}$, 
$\alpha = \SI{210.99}{\metre}$, $\nu = 2.0$) explained \num{38} and \SI{18}{\percent} of the sample variance of 
BUDE with $m(\boldsymbol{s})$ and $e(\boldsymbol{s})$, respectively (\autoref{fig:chap07-bude-vario}).

\begin{figure}[!ht]
 \centering
 \includegraphics[width=90mm]{fig/chap07-bude-vario}
 \caption[Variogram model representing the stochastic term of the linear mixed model fitted to the soil bulk
 density data.]{Variogram model (red line) representing the stochastic term of the linear mixed model fitted
 to the soil bulk density data. Exponential spacings are used to depict the sample variogram (blue dots) along 
 with  the number of point-pairs in each variogram bin.}
 \label{fig:chap07-bude-vario}
\end{figure}

With the random regression forest and the LMM at hand, we simulated $\mathcal{R} = 1000$ equiprobable 
realizations of BUDE at a fine grid of \num{\sim800000} points covering the entire study area. It is using 
this uncertain reality that we tested our $\mathcal{A} = 4$ sampling algorithms.

\subsection{Sampling and Model Calibration}

We then sampled from each realization using the optimized spatial sample configurations. Because we wanted to 
check the effect of the sample size, each algorithms was run using $$\mathcal{N} = 3$$ sample sizes of $n = 
(100, 200, 400)$, which amounts to $\mathcal{D} = 3000}$ calibration datasets for each algorithm. Each 
calibration dataset was used to calibrate a random regression forest using the same covariates as used in 
simulating the fields.

\section{RESULTS AND DISCUSSION}

\begin{figure}[!ht]
 \centering
 \includegraphics[width=\textwidth]{fig/chap07-energy_corr_dist_acdc_clhs}
 \caption[Objective function values during the optimization of three sample configurations using four sampling 
 algorithms.]{Objective function values during the optimization of sample configurations of size 
 $n = (100, 200, 400$) using sampling algorithms CORR, DIST, ACDC, and CLHS against the number of Markov 
 chains of length $n$.}
 \label{fig:chap07-energy-all}
\end{figure}

\begin{figure}[!ht]
 \centering
 \includegraphics[width=\textwidth]{fig/chap07-energy_acdc_clhs}
 \caption[Region of the feasible objective space explored by pairs of objective functions that compose CLHS 
 and ACDC.]{Region of the feasible objective space $\mathcal{Z}$ explored by pairs of objective functions (x 
 vs  y) that compose CLHS ($\mathcal{O}_1$, $\mathcal{O}_2$, and $\mathcal{O}_3$) and ACDC (CORR and DIST)  
 during the optimization of a sample configuration of size $n = 100$ using $n_{\text{chains}} = 500$ Markov 
 chains of length $n$.}
 \label{fig:chap07-energy-acdc-clhs}
\end{figure}

\begin{figure}[!ht]
 \centering
 \includegraphics[width=\textwidth]{fig/chap07-points_corr_dist_acdc_clhs}
 \caption[Sample configurations optimized using four sampling algorithms.]{Sample configurations of size 
 $n = (100, 200, 400)$ optimized using sampling algorithms CORR, DIST,  ACDC, and CLHS superimposing the
 drainage network.}
 \label{fig:chap07-points}
\end{figure}

\section{FINAL CONSIDERATIONS}


 % Optimization of sample configurations for spatial trend estimation for soil mapping
% \artigotrue
\chapter{SAMPLING FOR SOIL MAPPING IN \emph{TERRA INCOGNITA}}
\label{chap:chap08}

\def\ptkeys{}

\begin{chapterabstract}{brazilian}{\ptkeys}

\end{chapterabstract}

\def\enkeys{}
  
\begin{chapterabstract}{english}{\enkeys}

\end{chapterabstract}

\formatchapter

\section{INTRODUCTION}

\titlenote{This chapter is based on the study \textit{spsann -- optimization of sample patterns using spatial 
simulated annealing}, presented at the EGU General Assembly 2015 \cite{Samuel-RosaEtAl2015a}, and 
\textit{Optimization of sample configurations for variogram estimation}, presented at Pedometrics 2015 
\cite{Samuel-RosaEtAl2015c}. Also collaborated in the preparation of this document: Gerard B. M. Heuvelink 
(ISRIC -- World Soil Information), Dick J. Brus (Alterra, Wageningen University and Research Centre, the 
Netherlands), Gustavo M. Vasques (Embrapa Soils, Brazil), and Lúcia Helena Cunha dos Anjos (Universidade 
Federal Rural do Rio de Janeiro, Brazil).}

The success of soil mapping largely depends on the sampling data because the last are used to 1) estimate the 
spatial trend, 2) estimate the variogram of the residuals, and 3) make spatial predictions by calculating 
conditional distributions. A poor sampling strategy is likely to deliver a poor model and large prediction 
errors, resulting in a waste of financial resources, staff and time \cite{vanGroenigenEtAl1999,  
deGruijterEtAl2006, LanEtAl2010}. This is undesirable because sampling already is the largest contributor to 
the costs of soil mapping \cite{WebsterEtAl1990, vanGroenigenEtAl1999, KempenEtAl2012}.

The focus of this study is on the optimization of spatial sample configurations for soil mapping. We explore a 
scenario in which a) multiple soil properties have to be mapped, b) we are ignorant about the form of the model 
of spatial variation, and c) the operational constraints limit sampling to a single phase. The objective is to 
evaluate the ability of different sampling configuration types and sample sizes to capture the true form of the 
model of spatial variation and make accurate predictions. We also quantify the gain in prediction accuracy by 
combining popular sampling methods in a multi-objective optimization problem. This study addresses a problem 
that many soil scientists involved in soil mapping projects face: how to come up with a spatial sample 
configuration that is effective and robust in situations where we know very little?

\section{PURPOSIVE SAMPLING}

\emph{Purposive sampling} is the non-probability sampling mode by which the sampling locations are selected 
intentionally as to satisfy an \textit{a priori} criterion. This criterion is commonly defined based on the 
model that will be used to infer the structure of spatial variation of a soil property $Y(\boldsymbol{s})$. 
Compared to probability sampling, purposive sampling generally is more efficient for \emph{model-based 
inference} \cite{deGruijterEtAl2006}.

The criterion used to select the sampling locations can be defined based on the chosen statistical model 
\cite{deGruijterEtAl2006, Mueller2007, WebsterEtAl2013}. A set of mathematical and heuristic rules is then 
formalized in the form of a computer algorithm to find the sampling locations that minimize (or maximize) that 
criterion. The more we know about the structure of spatial variation of $Y(\boldsymbol{s})$, the more likely we 
are to obtain the optimum sample configuration given the chosen statistical model.

However, the statistical model is usually unknown before we sample. This is especially common when multiple 
soil properties have to be mapped and the available information is insufficient to decide on the structure of 
the spatial variation. Because we usually want to make the least possible number of assumptions about the model 
structure, the safest solution is to use a space filling design \cite{HenglEtAl2003a, deGruijterEtAl2006, 
Mueller2007, WalvoortEtAl2010}: the locations are selected as to generate a sample that covers the geographic 
and/or feature space(s) as evenly as possible. In areas with very little information on the spatial variation 
of the soil properties of interest, referred to \emph{terra incognita} by \citet{WebsterEtAl2007}, there 
usually are operational constraints that limit the sampling to a single phase. The spatial sample configuration 
has to be optimized to identify the correct model structure, estimate model parameters, and make spatial 
predictions.

\subsection{Sampling for Spatial Trend Estimation}

The spatial trend corresponds to the spatial variation of $Y(\boldsymbol{s})$ that is explained linearly or 
nonlinearly by the covariates. For a linear spatial trend, the sample should cover the extremes of the 
distribution of the covariates \cite{Mueller2007}. For models with interactions and/or higher order terms there 
are the response surface designs \cite{BoxEtAl1951, LeschEtAl1995}. These approaches produce clusters of points 
and ignore the spatial autocorrelation of the residuals \cite{BrusEtAl2007a, Mueller2007}. Optimal sampling 
designs for neural nets, random forests, etc., are yet unknown.

A common solution for spatial trend estimation in \emph{terra incognita} is to use a feature space filling 
sample. \citet{HenglEtAl2003a} sampled along the marginal distribution of the covariates using equal-range 
strata with weights proportional to the frequency distribution. \citet{MinasnyEtAl2007a} sampled equal-variance 
geographic strata created using the variance of the covariates retained in their first principal component.

A more elaborated method, formulated as a multi-objective optimization problem, was developed by 
\citet{MinasnyEtAl2006b} based on the Latin hypercube sampling \cite{McKayEtAl1979}, known as \emph{conditioned 
Latin hypercube sampling} (CLHS). The CLHS is based on sampling along the marginal distribution of the numeric 
and factor covariates using equal-area strata (quantile sampling) and proportionally to the area occupied by 
each level, respectively, and reproducing the linear correlation of the numeric covariates 
\cite{MinasnyEtAl2006b}. The method is very flexible and can be easily extended \cite{MinasnyEtAl2010a, 
RoudierEtAl2012}. Recently, \citet{Samuel-RosaEtAl} proposed conceptual and algorithmic improvements on the 
CLHS. In short, the proposed improvements concern the definition of the marginal sampling strata, the 
measurement of the correlation between covariates, and the aggregation of the objective functions.

\subsection{Sampling for Variogram Estimation}

A variogram model explains the spatially correlated random part of the spatial variation of 
$Y(\boldsymbol{s})$. Several sampling methods exist to identify and/or estimate the variogram and its 
parameters \cite{BrusEtAl1994, deGruijterEtAl2006, Mueller2007, WebsterEtAl2013}. Modern ones focus on maximum 
likelihood estimators \cite{Lark2002, Zimmerman2006, Mueller2007}. Their limitation is that a minimum knowledge 
about the form of the variogram is required. A Bayesian approach was suggested to account for the uncertainty 
of the estimated variogram \cite{DiggleEtAl2006, MarchantEtAl2006, ZhuEtAl2006}. But it is hard to implement 
for multiple variables simultaneously, and the uncertainty is likely to increase with the number of parameters 
that need to be estimated.

Sampling for variogram estimation should concentrate on relevant pairwise distances \cite{MuellerEtAl1999, 
Lark2002}. But how to do that when we are ignorant about the shape of the variogram? \citet{BreslerEtAl1982, 
Russo1984, WarrickEtAl1987} proposed a conservative solution focusing on the method of moments: the points 
should be located as to match a uniform distribution of pairwise distances. Their claim was that the sample 
would be globally optimal for an infinite set of unknown variograms. This has not been proved mathematically 
nor corroborated by empirical evidence. The resulting sample usually is redundant (poorly informative), 
concentrating most of the points in a single large cluster, with a few scattered points -- many of the 
point-pairs are computed using the same subset of points.

\subsection{Sampling for Spatial Interpolation}

Kriging is the best unbiased linear predictor of soil properties \cite{LarkEtAl2006}. The overall prediction 
accuracy depends on spreading the sample points as uniformly as possible throughout the study area. This is 
because for a stationary isotropic random field the kriging variance is a function only of the distance between 
sample points \cite{Cressie1993}. Regular sampling grids are commonly used to obtain a uniform geographic 
coverage, although triangular equilateral grids are more efficient \cite{WebsterEtAl2007}. Regular grids 
usually are inappropriate for irregularly shaped areas \cite{WalvoortEtAl2010}.

The regression-kriging approach for soil mapping \cite{HenglEtAl2007b} lead to the development of sampling 
methods that account for both feature and geographic spaces. \citet{HenglEtAl2003a} proposed sampling 
iteratively in the feature space and keeping the sample configuration with the best geographic coverage. 
\citet{MinasnyEtAl2006b} developed a sampling strategy for spatial trend estimation and claimed that the 
geographic space could be considered as well. \citet{MinasnyEtAl2007a} suggested that a geographic 
stratification based on the variance of the covariates would take into consideration the geographic coverage. 
These methods are suboptimal for spatial interpolation because they essentially operate in the feature space.

Efficient optimization of sample configurations for spatial interpolation depends upon minimizing a 
distance-based metric \cite{RoyleEtAl1998}. One such metric is the Mean Squared Shortest Distance (MSSD) 
between sample and prediction points \cite{BrusEtAl2006}. It is equivalent to finding, for each prediction 
point, the nearest neighbouring sample point. This metric can be minimized using the \textit{k}-means 
clustering algorithm \cite{WalvoortEtAl2010}, which is computationally fast, but sensitive to local optima 
solutions.

\section{PROPOSED INNOVATIONS AND MODIFICATIONS}

We believe that there is room to improve on the existing methods and propose innovations for sampling for soil 
mapping in \emph{terra incognita}. Our proposed innovations and modifications have been implemented in the 
publicly available \Rpackage{spsann} (\url{https://cran.r-project.org/web/packages/spsann}).

\subsection{Sampling for Spatial Trend Estimation}

We consider the method of \cite{MinasnyEtAl2006b} to be the most suited to sample for spatial trend estimation 
for soil mapping in \emph{terra incognita}.

\subsection{Sampling for Variogram Estimation}

We propose that sampling to estimate the variogram model for soil mapping in \emph{terra incognita} should be 
based on placing several small clusters scattered throughout the spatial domain as to maximize the amount of 
information. The most relevant pairwise distances are those that enable an accurate estimate of the behaviour 
of the variogram near the origin. We use exponentially spaced lags defined up to the circumradius $r$ of the 
bounding box of the area. The exponential spacings are created sequentially from the largest to the smallest 
lag by halving the immediately preceding larger lag, resulting in narrower lags in the left side of the 
variogram. It works as follows:

\begin{enumerate}
 \item Find $r$. Use the result to define the upper bound (UB) of the first rightmost lag.
 \item Halve $r$. Use the result to define the lower bound (LB) of the first rightmost lag.
 \item Go to the next lag.
 \item Set the LB of the last lag as the UB of the current lag.
 \item Halve the UB. Use the result to define the LB of the current lag.
 \item Proceed as in 3--5 till the UB and LB of leftmost lag have been defined.
\end{enumerate}


We define seven exponentially spaced lag-distance classes up to half the 
diagonal of the spatial domain. They are created sequentially by halving the immediately preceding larger lag 
\cite{TruongEtAl2013}. Our objective is to place the points as to have each of them contributing to all lags. 
The criterion to be minimized is the sum of differences between the vectors of the pre-specified 
$\boldsymbol{l}^*$ and observed $\boldsymbol{l}$ distributions of unique \textbf{P}oints \textbf{P}er 
\textbf{L}ag

\begin{equation}
 \text{PPL} = \sum_{i = 1}^{n} w_i (l_i^* - l_i),
\end{equation}\label{eq:chap08-ppl}

\noindent where $\boldsymbol{w}$ is a vector of weights for the $n$ lag-distance classes.

\subsection{Sampling for Spatial Interpolation}

The MSSD seems to be the most suited criterion to optimize sample configurations for spatial interpolation for 
soil mapping in \emph{terra incognita}. However, the available algorithms cannot be used to formulate 
multi-objective optimization problems. We suggest using the spatial simulated annealing algorithm instead, 
eliminating the sensitivity to local optima solutions \cite{KirkpatrickEtAl1983, Groenigen1999a}.

\subsection{Sampling for Soil Mapping in \emph{Terra Incognita}}

We propose a heuristic, general-purpose method to design sample configurations for soil mapping in \emph{terra 
incognita}. Like sampling for spatial trend estimation, it is based on solving a MOOP. An utility function is 
defined aggregating the four objective functions described above so that the sample points SPAN the feature, 
variogram and geographic spaces,

\begin{equation}
\text{SPAN} = w_1 \text{CORR} + w_2 \text{DIST} + w_3 \text{PPL} + w_4 \text{MSSD}, 
\end{equation}\label{eq:chap08-span}

with $w_1 = w_2$ and $w_1 + w_2 = w_3 + w_4$ in the \emph{terra incognita} setting.

\section{CASE STUDY}

The study was developed using synthetic data derived from a real-world study case described by 
\citet{Samuel-RosaEtAl2015}. The study site is a small catchment of about \SI{2000}{\hectare} located on the 
southern edge of the plateau of the Paraná Sedimentary Basin, Rio Grande do Sul, Brazil 
(\autoref{fig:chap08-location}). The real-world dataset contains $n = 350$ point soil observations of the 
topsoil, and includes several soil properties, but only two were explored in this study: clay content (CLAY, 
\si{\gram\pre\kilo\gram}) and bulk density (BUDE, \si{\mega\gram\per\cubic\metre}). The dataset also includes 
several covariates derived from area-class soil maps, digital elevation models, geological maps, land use maps, 
and satellite images. All preprocessing steps and methods used to derive the covariates were described by 
\citet{Samuel-RosaEtAl2015}.

\begin{figure}[!ht]
 \centering
 \includegraphics[width = 90mm]{fig/chap08-location}
 \caption{Location of the real-world study area in Santa Maria, southern Brazil.}
 \label{fig:chap08-location}
\end{figure}

\subsection{Soil Data Generating Process}

We assumed the soil properties ($Y$) to be a function of the interplay of environmental conditions defined by 
the climate, organisms, relief, parent material, time, and other unknown players \cite{Jenny1994, 
McBratneyEtAl2003, Florinsky2012}. Because our pedologial knowledge and data available still are limited to 
build such a complex \emph{mechanistic model}, we assumed the soil properties to be the outcome of a spatial 
stochastic process composed of the additive combination of fixed and random effects, i.e. $Y(\boldsymbol{s}) = 
m(\boldsymbol{s}) + e(\boldsymbol{s})$. Here the soil property is a random variable $Y(\boldsymbol{s})$, 
$m(\boldsymbol{s})$ is a deterministic trend, and $e(\boldsymbol{s})$ is a spatially correlated, Gaussian 
distributed random variable, that is stationary in the mean and covariance \cite{HeuvelinkEtAl2001}.




\section{FINAL CONSIDERATIONS}

The main free and open source implementation of the CLHS is the \Rpackage{clhs} \cite{RoudierEtAl2012}. The 
package extended the CLHS considering the use of a cost surface that penalizes sampling locations that are 
difficult to access. Other researchers have also implemented the CLHS as to consider a cost function 
\cite{MulderEtAl2013, CliffordEtAl2014}, but none is available as a free software package.
 % Sampling for soil mapping in terra incognita
\setlength{\baselineskip}{\baselineskip}

%%=============================================================================
%% Apêndices
%%=============================================================================
\appendix
% \include{chap/} % Introdução Geral
% \include{chap/} % Conclusão Geral
% \artigofalse
\chapter{Modelo conceitual de pedogênese}
\label{apen:pedogenesis}

\tocless\section{Apresentação}

A construção de modelos de mapeamento do solo inicia com a definição de um \textit{modelo conceitual
de pedogênese}. Um modelo conceitual de pedogênese constitui uma representação verbal da realidade 
sob estudo que inclui a descrição explícita dos fatores e processos de formação do solo que 
determinam as características do solo e o seu padrão de distribuição espacial. Isso requer a reunião
de toda a informação ambiental disponível e aplicação dos conceitos de relação solo paisagem, 
desenvolvimento do solo em catenas, ou outro modelo teórico de explicação da variação espacial do 
solo.

O presente documento apresenta o modelo conceitual de pedogênese da Área de Estudos em Pedologia 
Quantitativa de Santa Maria (AEPQ-SM), criada no ano de 2008 pelo grupo de pesquisa 
\href{dgp.cnpq.br/dgp/espelhogrupo/9373361709890764}{Gênese, composição e comportamento dos solos do
RS}], coordenado pelos pesquisadores \href{http://lattes.cnpq.br/3735884911693854}{Ricardo Simão 
Diniz Dalmolin} e \href{http://lattes.cnpq.br/6868334304493274}{Fabrício de Araújo Pedron}, e 
sediado no Departamento de Solos da Universidade Federal de Santa Maria 
(\href{http://site.ufsm.br/}{UFSM}). Sua elaboração teve como motivação inicial o desenvolvimento do
projeto de doutorado intitulado \textit{Contribuição à construção de modelos de predição de 
propriedades do solo}. Os resultados obtidos com esse projeto contribuíram para a expansão desse 
documento. A presente versão não é definitiva, tendo como objetivo servir de suporte ao 
desenvolvimento de novos estudos em pedologia quantitativa. Dado que o conhecimento sobre a AEPQ-SM 
continua sendo construído, novas versões do modelo conceitual de pedogênese deverão ser construídas 
por esses estudos.

\tocless\section{Localização}

A AEPQ-SM faz parte da bacia de captação do reservatório do DNOS/CORSAN (Departamento Nacional de 
Obras de Saneamento/Companhia Riograndense de Saneamento), localizada na divisa entre os municípios 
de \href{http://pt.wikipedia.org/wiki/Itaara}{Itaara} (ao norte) e 
\href{http://pt.wikipedia.org/wiki/Santa_Maria_\%28Rio_Grande_do_Sul\%29}{Santa Maria} (ao sul), na 
porção sul da \href{http://pt.wikipedia.org/wiki/Bacia_do_Paran\%C3\%A1}{Bacia Sedimentar do Paraná},
estado do Rio Grande do Sul, Brasil (\autoref{fig:location}).

\begin{figure}[ht]
  \centering
  \includegraphics[width=0.45\textwidth]{figures/location.pdf}
  \caption{Localização da Área de Estudos de Pedologia Quantitativa de Santa 
  Maria (AEPQ-SM), estado do Rio Grande do Sul, Brasil.}
  \label{fig:location}
\end{figure}

A bacia de captação do reservatório do DNOS/CORSAN corresponde à cabeceira da bacia hidrográfica do 
\href{http://pt.wikipedia.org/wiki/Rio_Vacaca\%C3\%AD-Mirim}{rio Vacacaí-Mirim}, tributário do 
\href{http://pt.wikipedia.org/wiki/Rio_Jacu\%C3\%AD}{rio Jacuí} e, consequentemente, do 
\href{http://pt.wikipedia.org/wiki/Lago_Gua\%C3\%ADba}{rio Guaíba} e da 
\href{http://pt.wikipedia.org/wiki/Lagoa_dos_Patos}{Lagoa dos Patos}. A bacia de captação cobre uma 
área de aproximadamente \SI{29}{\square\kilo\metre} e alimenta um reservatório com volume máximo 
de aproximadamente \SI{3800000}{\cubic\metre} em uma área inundada de \SI{0,74}{\square\kilo\metre}.
Esse reservatório contribui com até \SI{30}{\percent} do abastecimento de água da cidade de Santa 
Maria \cite{Dias2003, DillEtAl2004, Miguel2010}.

Apesar de cobrir apenas \SI{60}{\percent} da bacia de captação do reservatório do DNOS/CORSAN, o 
que corresponde à uma área de \SI{\pm18}{\square\kilo\metre}, a area de estudo engloba a maior parte 
da variação presente naquela. As limitações operacionais do grupo de pesquisa na época impuseram a 
necessidade de selecionar uma sub-área da da bacia de captação do reservatório do DNOS/CORSAN 
(\autoref{fig:aepsm}).

\begin{figure}[ht]
  \centering
  \includegraphics[width=0.45\textwidth]{figures/aepsm.pdf}
  \caption{A Área de Estudos de Pedologia Quantitativa de Santa Maria (AEPQ-SM) faz parte da bacia 
  de captação do reservatório do DNOS/CORSAN.}
  \label{fig:aepsm}
\end{figure}

\tocless\section{Clima}

O clima local é classificado como \href{http://pt.wikipedia.org/wiki/Clima_subtropical_\%C3\%BAmido}{Cfa}
(subtropical úmido sem estação seca definida), com temperatura média anual de \SI{19,1}{\celsius}. 
As temperaturas podem alcançar \SI{>40}{\celsius}, no verão, e valores negativos no inverno 
\cite{HeldweinEtAl2009}. A precipitação média anual é de \SI{1708}{\milli\metre} bem distribuídos 
ao longo do ano \cite{Maluf2000}. Predominam os ventos do quadrante leste (frio, úmido e de 
intensidade fraca a moderada), oeste (frio, seco e de intensidade fraca a moderada) e norte (quente,
seco e de intensidade moderada a forte) \cite{HeldweinEtAl2009}.

O padrão predominante das chuvas é o avançado, caracterizado por ter seu pico de maior intensidade 
no início da precipitação \cite{MehlEtAl2001}. As chuvas de maior intensidade ocorrem nos meses do 
final da primavera, verão e início do outono \cite{MouraBueno2012}. Como resultado desse padrão, as
chuvas de inverno são as menos erosivas, mesmo que o conteúdo de água do solo permaneça elevado 
durante todo o período. O padrão de precipitação também é condicionado pelo relevo. Observações 
feitas em três locais da AEPQ-SM durante o ano de 2011, marcado por forte estiagem, mostram variação
na lâmina total precipitada entre \num{1317} e \SI{1411}{\milli\metre} \cite{MouraBueno2012}. 
Assim, o relevo plano a montanhoso, com vales encaixados, parece condicionar a formação de 
diferentes regiões microclimáticas, refletindo no volume e intensidade das chuvas 
\cite{MouraBueno2012}.

O relevo também deve condicionar o fluxo radiativo que atinge as diferentes superfícies. Apesar de 
não haver estudos que demonstrem a efetividade desse fenômeno na área, é reconhecido que grande 
parte da superfície em terrenos de topografia complexa é influenciada pelo efeito de sombreamento, 
sobretudo nas primeiras horas da manhã e no final da tarde \cite{OliphantEtAl2003}. Além disso, a 
declividade do terreno possui forte influência sobre o ângulo de interceptação da radiação solar 
pelas superfícies \cite{Birkeland1999}. Como consequência, deve ocorrer variações na temperatura e 
conteúdo de água no solo nas diferentes superfícies. Os meses de inverno são marcados por ainda 
menor disponibilidade de radiação solar devido à alta frequência de nevoeiros, sobretudo nas partes 
mais baixas, com valores normais de insolação de \SI{5,1}{\hour\per\day} \cite{HeldweinEtAl2009}. 
Além disso, devido à variação de altitude entre \num{139} e \SI{475}{\metre}, deve ocorrer diferença
na temperatura da ordem de aproximadamente \SI{4}{\celsius} entre a parte mais baixa e a parte mais 
alta da AEPQ-SM \cite{HeldweinEtAl2009}.

\tocless\section{Geologia}

A geologia da AEPQ-SM é bastante complexa, sendo constituída por três formações geológicas, além de 
depósitos coluvionares e de aluvião do Quaternário. A literatura sobre o tema é vasta 
\cite{Bortoluzzi1974, Brasil1980, GasparettoEtAl1988, MacielFilho1990, Machado1998, PieriniEtAl2002, 
MarquesEtAl2005, Milani2005, Pinto2005, CPRM2007, Pedron2007, Sartori2009, NascimentoEtAl2010, 
WerlangEtAl2010, PedronEtAl2012}, e uma revisão da mesma é apresentada aqui.

% \begin{figure}[h]
%   \centering
%   \includegraphics[height=7cm]{figures/geo1988}
%   \caption{Mapa da geologia da AEPQ-SM publicado na escala 1:50.000 \cite{GasparettoEtAl1988}.\\Legenda: SG-S - Sequência Superior da Formação Serra Geral, CT - Formação Caturrita, BT - Formação Botucatu, SG-I - Sequência Inferior da Formação Serra Geral.}
%   \label{fig:geo1988}
% \end{figure}

Na base da sequência estratigráfica, em elevações abaixo de \SI{\pm200}{\metre}, está a 
\href{http://pt.wikipedia.org/wiki/Forma\%C3\%A7\%C3\%A3o_Caturrita}{Formação Caturrita}, 
constituída por material sedimentar depositado em ambiente fluvial no Triássico Superior. Sua 
composição é diversa, apresentando seixos de siltito argiloso vermelho na base, seguido por arenito 
avermelhado de granulometria fina à média, composição quartzosa e matriz argilosa, podendo ainda 
conter considerável teor de feldspato, sobreposto por siltito e folhelho também avermelhados. Em 
geral, a granulometria do arenito é mais grosseira e menos argilosa na base da deposição. Dado a sua
origem fluvial, a Formação Caturrita apresenta marcada estratificação cruzada acanalada e tabular. 
A origem fluvial também resulta em significativa variação espacial na granulometria do arenito, 
identificada pelo contraste entre áreas de maior cimentação e coesão, com outras de maior 
condutividade hidráulica. Imediatamente acima da Formação Caturrita encontra-se, ora a Formação 
Botucatu, ora a Sequência Inferior da Formação Serra Geral.

% \begin{figure}[h]
%   \centering
%   \includegraphics[height=7cm]{figures/geo1990}
%   \caption{Mapa da geologia da AEPQ-SM publicado na escala 1:25.000 \cite{MacielFilho1990}.\\Legenda: QD - Depósitos do Quaternário, SG-S - Sequência Superior da Formação Serra Geral, CT - Formação Caturrita, BT - Formação Botucatu, SG-I - Sequência Inferior da Formação Serra Geral.}
%   \label{fig:geo1990}
% \end{figure}

Em elevações entre \num{\pm200} e \SI{\pm350}{\metre} está a Sequência Inferior da 
\href{http://pt.wikipedia.org/wiki/Forma\%C3\%A7\%C3\%A3o_Serra_Geral}{Formação Serra Geral} 
(basaltos-andesitos tholeíticos). As rochas básicas são de coloração cinza-escura e são constituídas
por plagioclásio cálcico clinopiroxênio, magnetita e material intersticial de quartzo e material 
desvitrificado. Em elevações superiores a \SI{\pm350}{\metre} está a Sequência Superior da Formação 
Serra Geral (vitrófilos, riólitos-riodacitos granofíricos). As rochas ácidas apresentam cor 
cinza-clara, estrutura microcristalina e são constituídas por cristais e plagioclásio, 
clinopiroxênios, hornblenda uralítica e magnetita. A origem desse material remonta o Cretáceo, 
quando sucessivos derrames de lavas de origem vulcânica fissural ocorreram durante aproximadamente 
10 milhões de anos em toda a Bacia do Paraná. Esses eventos ocorreram ao mesmo tempo em que iniciava
a separação das plataformas continentais que hoje constituem a América do Sul e África, marcando o 
final da existência do supercontinente Pangea.

O arenito eólico constituinte da 
\href{http://pt.wikipedia.org/wiki/Forma\%C3\%A7\%C3\%A3o_Botucatu}{Formação Botucatu} é encontrado 
tanto assentado sobre a Formação Caturrita, como no interior da Formação Serra Geral (arenito 
\textit{intertrap}). Trata-se de um arenito quartzoso de granulometria fina à média, contendo 
feldspato alterado e cimentado por sílica ou por óxido de ferro, que lhe confere a coloração 
rosa-avermelhada. Sua deposição teve início no Cretáceo Inferior, período em que a Bacia do Paraná 
estava sob influência de clima desértico. Essa condição climática continuou durante todo o período 
em que ocorreram as dezenas de eventos de vulcanismo fissural, fazendo com que os mesmos fossem 
sucedidos por deposições eólicas de duração variável. Como a duração e a quantidade de material 
depositado pelos eventos de vulcanismo fissural era variável, assim como o intervalo de tempo entre 
cada novo evento e a intensidade das deposições de sedimentos eólicos, a espessura das camadas do 
arenito eólico e das rochas vulcânicas é bastante variável. Além disso, devido ao diversos eventos 
de subsidência que ocorreram no eixo central da Bacia do Paraná, com consequente soerguimento de 
suas bordas, as camadas dessas rochas possuem diferentes inclinações ao longo de sua faixa de 
exposição, sendo caracteristicamente ondulada e com suave tendência de inclinação para sudoeste.

As deposições do Quaternário são constituídas por depósitos coluvionares e de aluvião. Em elevações 
entre \num{\pm200} e \SI{\pm300}{\metre} encontram-se depósitos coluviais de material proveniente de
uma ou ambas as Formações Serra Geral (fragmentos de tamanho variado) e Botucatu. Em elevações 
abaixo de \SI{\pm200}{\metre} são mais comuns os depósitos coluviais de uma ou ambas as Formações 
Botucatu e Caturrita. Esses depósitos ocorrem de maneira descontínua nas encostas. Próximo aos 
cursos de água na porção mais baixa da bacia e no entorno do reservatório, encontram-se depósitos 
fluviais recentes, geralmente formados por fragmentos arredondados (seixo) de tamanho variável e/ou 
sedimento arenoso. Em pequenas áreas abaciadas e mal drenadas, os sedimentos apresentam 
granulometria mais fina.

\tocless\section{Geomorfologia}

A AEPQ-SM está situada na porção sul da Bacia Sedimentar do Paraná. Assim, as geoformas atuais são 
resultado dos processos erosivos que ocorreram durante o Terciário e o Quaternário 
\cite{Sartori2009}, após as últimas deposições de lavas vulcânicas e de sedimentos eólicos. Durante
esse período, a esculturação da paisagem foi determinada pelas alternações entre climas úmidos, 
semi-áridos e áridos \cite{Sartori2009}. Atualmente, o clima subtropical úmido favorece a 
instalação e permanência de uma vegetação mais eficiente na redução do processo de dissecação da 
paisagem \cite{Sartori2009, NascimentoEtAl2010}. Isso permite que as superfícies geomórficas 
atinjam maior estabilidade e maturidade.

% \begin{figure}[h]
%   \centering
%   \includegraphics[height=7cm]{figures/dem90}
%   \caption{Modelo digital de elevação da AEPQ-SM com resolução espacial de 90~m (SRTM DEM versão 4) \cite{JarvisEtAl2008}.}
%   \label{fig:dem90}
% \end{figure}

As unidades morfoestruturais que abrangem a área são o Planalto e o Rebordo do Planalto da Bacia do 
Paraná \cite{NascimentoEtAl2010}. O Planalto é marcado por relevo suave-ondulado a ondulado, com 
formas denudacionais de superfícies planas com topos convexos (fluxo hídrico divergente). Nesses 
locais, as vertentes assumem a forma convexa levemente ondulada \cite{NascimentoEtAl2010}, muitas 
vezes encaixadas em falhas e/ou fraturas. Já o Rebordo do Planalto é caracterizado pela ampla 
variação altimétrica, declividade acentuada e escarpas abruptas, apresentando formas denudacionais 
com topos convexos (fluxo hídrico divergente), aguçados e em formas de escarpas. Nesses locais, as 
vertentes assumem forma retilínea com grande desnível \cite{NascimentoEtAl2010}, muitas vezes 
interrompidas por degraus ou patamares, na maioria das vezes encaixadas em falhas e ou fraturas. 
Esses patamares são resultado da ação diferencial dos processos denudacionais sobre a paisagem, 
geralmente condicionados pela resistência do material de origem, seja ela química/mineralógica 
(rochas vulcânicas básicas vs. ácidas), seja ela física/granulométrica (rochas vulcânicas vs. 
sedimentares), ou estrutural (padrão de diaclasamento vertical vs. horizontal das rochas vulcânicas)
\cite{Holtz2003, Pedron2007, StreckEtAl2008}. Entretanto, em algumas situações, os patamares são 
formados a partir de eventos pluviométricos de elevada intensidade e/ou duração que causam 
movimentos de massa, formando depósitos coluviais mesmo nas partes altas do Rebordo do Planalto 
\cite{PinheiroEtAl2004, PaisaniEtAl2010}. Nesses casos, os patamares possuem menor dimensão e maior
declividade. Em outras situações, os patamares são inteiramente recobertos por colúvios provenientes
de áreas à montante, sobretudo dos arenitos da Formação Botucatu, formando vertentes alongadas e com
menor inclinação, que chegam até a margem dos cursos de água. Assim, as encostas que apresentam 
patamares constituem, de modo geral, vertentes mais curtas e inclinadas, muitas vezes escarpadas.

% \begin{figure}[h]
%   \centering
%   \includegraphics[height=7cm]{figures/dem10}
%   \caption{Modelo digital de elevação da AEPQ-SM com interpolado de curvas de nível com 10~m de equidistância \cite{DSG1980, DSG1992, DSG1992a}.}
%   \label{fig:dem10}
% \end{figure}

Apesar da AEPQ-SM ser abrangida pelas unidades morfoestruturais do Planalto e do Rebordo do Planalto
da Bacia do Paraná \cite{NascimentoEtAl2010}, as características geomorfológicas das porções mais 
baixas da paisagem são similares àquelas encontradas na Depressão Periférica. Essas áreas são 
marcadas pelo acúmulo de sedimentos provenientes do Planalto e do Rebordo, formando planícies 
aluviais que se intercalam entre as coxilhas (denominação regional de colinas). Essas áreas são 
caracterizadas pela presença de formas agradacionais de planície fluvial e formas denudacionais de 
topos convexos e de superfícies planas. As últimas correspondem às coxilhas de algumas dezenas de 
metros de altitude, geralmente formadas sobre o substrato da Formação Caturrita 
\cite{GasparettoEtAl1988}, geralmente assentadas na base do Rebordo do Planalto. Essas coxilhas 
constituem divisores de água de pequena amplitude que auxiliam na subdivisão da área em pequenas 
sub-bacias \cite{Marins2004, Sartori2009}. Nesses locais as vertentes costumam ser alongadas e 
assumem a forma predominantemente côncava devido aos processos de deposição sedimentar e erosão 
fluvial, apesar dos mesmos serem muito fracos sob a atual condição climática 
\cite{NascimentoEtAl2010, WerlangEtAl2010}.

A grande heterogeneidade geomorfológica da AEPQ-SM se traduz em uma grande heterogeneidade textural 
do relevo, ou rugosidade, resultante da ação climática ao longo do tempo geológico 
\cite{NascimentoEtAl2010}. Entretanto, em menor escala, a ação antrópica também atua sobre a 
configuração geomorfológica da área. O principal efeito se dá através da erosão da camada 
superficial do solo, cultivado intensivamente sem o uso de práticas conservacionistas ao longo de 
inúmeras décadas \cite{Menezes2008, Sturmer2008, Miguel2010, SamuelRosaEtAl2011a}. Além disso, a 
abertura de caminhos para acesso às áreas de produção nos patamares do Rebordo do Planalto e nos 
topos de morros proporcionou a formação de canais de concentração e escoamento dos fluxos hídricos, 
levando à formação inicial de voçorocas. Obras de maior expressividade, como ferrovias, ruas e 
rodovias pavimentadas, conjuntos habitacionais e construções isoladas, as quais envolvem operações 
de terraplanagem e aterramento, também resultaram em modificações localizadas na geomorfologia da 
área. Por fim, a construção do reservatório possibilitou a formação de uma área de agradação da 
paisagem bastante estável em seu entorno, uma vez que o sedimento removido do Planalto e do 
Rebordo do Planalto não são mais transportados a jusante.

\tocless\section{Hidrografia}

A hidrografia da AEPQ-SM é condicionada pelas condições geomorfológicas, geológicas, pedológicas e 
climáticas, ao mesmo tempo que exerce forte influência sobre a modelagem da paisagem 
\cite{NascimentoEtAl2010}. Como a AEPQ-SM é abrangida pelas unidades morfoestruturais do Planalto 
e do Rebordo do Planalto da Bacia do Paraná, a drenagem apresenta um padrão bem definido, geralmente
retangular, determinado pelas falhas e/ou fraturas \cite{Bortoluzzi1974, GasparettoEtAl1988, 
NascimentoEtAl2010}. A própria formação da bacia de captação do reservatório do DNOS/CORSAN se deve 
a existência de falhas e/ou fraturas \cite{GasparettoEtAl1988}. A principal e maior delas 
localiza-se no eixo central da bacia, onde atualmente está localizado o leito do rio Vacacaí-Mirim. 
Quanto aos tributários do rio Vacacaí-Mirim, a maioria possui o leito assentado sobre outras falhas 
e/ou fraturas de menor dimensão perpendiculares àquela do eixo central da bacia.

Nas áreas mais baixas, cujas características geomorfológicas se assemelham à Depressão Periférica, 
a drenagem apresenta configuração serpenteada, resultado dos processos de deposição sedimentar e 
erosão fluvial \cite{PaivaEtAl2001, SutiliEtAl2009}. Como o relevo é plano a suave-ondulado, e as 
vertentes longas e predominantemente côncavas, o lençol freático fica próximo da superfície do solo.
A variação das condições meteorológicas ao longo do ano fazem com que o lençol freático apresente 
flutuação significativa, mantendo o conteúdo de água do solo elevado durante os meses mais frios 
(menor evapotranspiração) \cite{HeldweinEtAl2009}. Isso também favorece a ocorrência de inundações,
sobretudo nas proximidades dos cursos de água e do reservatório \cite{Goldani2006}.

Muitos cursos de água localizados no Planalto, no Rebordo do Planalto e nas coxilhas assentadas em 
sua base, são sazonais. Em geral, esses cursos de água estão em atividade apenas nos meses mais 
frios do ano, quando a disponibilidade de água no ambiente é maior, ou durante os eventos de 
precipitação de forte intensidade, que ocorrem nos meses de verão \cite{HeldweinEtAl2009, 
MouraBueno2012}. Os cursos de água do Rebordo do Planalto costumam apresentar leito raso e 
pedregoso, muitas vezes assentado sobre rochas da Formação Serra Geral \cite{SutiliEtAl2009}. Já 
os cursos de água localizados nos patamares e coxilhas costumam ser rasos, se assentados sobre as 
rochas da Formação Caturrita, ou profundos, se assentados sobre as rochas da Formação Botucatu ou 
depósitos coluvionares, formando voçorocas. Segundo relatos de alguns dos moradores mais antigos, 
muitas nascentes e pequenos cursos de água já perderam totalmente sua atividade, sobretudo quando 
localizadas no interior ou à jusante de áreas de uso antrópico.

Dado que a declividade e o desnível entre a parte mais baixa e a parte mais alta da AEPQ-SM são 
acentuados, as cheias costumam apresentar velocidade e vazão bastante grandes \cite{PaivaEtAl2001, 
SutiliEtAl2009}. Em média, nos \SI{7}{\kilo\metre} de extensão do rio Vacacaí-Mirim, da nascente até
o reservatório, a declividade média é de \SI{0,03}{\metre\per\metre}. Isso representa um desnível de
\SI{\pm210}{\metre}, o que resulta em um tempo de concentração da bacia estimado em \SI{3}{\hour} 
\cite{PaivaEtAl2001}. Essas características causam erosão severa nas margens dos cursos de água nas
áreas mais baixas (depósitos aluviais), sobretudo nos raios externos das curvas, onde a velocidade 
da água é maior \cite{SutiliEtAl2009}. Em alguns trechos, os cursos de água chegam a ter sua 
largura duplicada em menos de uma década, resultando no aumento da sinuosidade e do nível de fundo 
\cite{PaivaEtAl2001}. Esses eventos comprometem áreas de produção, bem como a estrutura de 
residências e vias públicas localizadas nas margens dos cursos de água. Grande parte do material 
removido das margens dos cursos de água é transportado para dentro do reservatório, que já perdeu 
mais de \SI{30}{\percent} de sua capacidade inicial de armazenamento de água \cite{DillEtAl2004}.

\tocless\section{Uso da terra}

A AEPQ-SM já foi intensamente ocupada em tempos pretéritos. Isso é evidenciado pela grande 
quantidade, extensão e boa distribuição da rede viária em toda a sua área. Além disso, a área é 
cortada por uma estrada férrea e avizinha uma rodovia federal, ambas muito movimentadas. Entretanto,
nas últimas décadas, muitos caminhos internos das propriedades rurais foram desativados, fruto do 
abandono de áreas de produção, a maioria delas localizada nos topos dos morros, patamares do Rebordo
ou no fundo de vales, onde a manutenção dos caminhos é muito onerosa, ou ainda, distantes da sede 
das propriedades. No passado, essas estradas internas possibilitaram o trânsito de pessoas e o 
transporte da produção agrosilvopastoril, a qual era comercializada às margens da estrada férrea e 
nas áreas urbanas do município. Atualmente, o tráfego está concentrado nas estradas principais e 
secundárias, enquanto alguns poucos caminhos internos são utilizados apenas na condução de rebanhos 
ou no acesso a pequenas áreas de produção agrícola. A maioria dos caminhos internos inativos está 
localizada no interior de áreas de floresta natural regenerada, o que dificulta a sua identificação 
pelo uso de imagens aéreas ou orbitais. Em alguns casos, esses caminhos são utilizados em atividades
de turismo ecológico, especialmente para caminhadas a pé, ou atividades esportivas com motocicletas
(trilhas).

% \begin{figure}[h]
%   \centering
%   \includegraphics[height=7cm]{figures/land1980}
%   \caption{Mapa do uso da terra da AEPQ-SM publicado na escala 1:25.000 \cite{DSG1980, DSG1992, DSG1992a}.\\Legenda de acordo com o guia para descrição do solo da FAO \cite{FAO2006}: S - Assentamentos humanos, PF - Silvicultura, H - Pecuária, FS - Floresta nativa.}
%   \label{fig:land1980}
% \end{figure}

O abandono de muitas áreas de produção possibilitou a regeneração da vegetação natural em grande 
parte da AEPQ-SM. Atualmente, a área ocupada por florestas e vegetação secundária (capoeira) 
representa aproximadamente \SI{60}{\percent} da área total \cite{SamuelRosaEtAl2011a}. Isso mostra 
que, assim como toda a região do Rebordo do Planalto da Bacia do Paraná, o uso da terra foi 
desintensificado nas últimas décadas \cite{SEMA/UFSM2001, DillEtAl2004, Poelking2007, Miguel2010, 
SamuelRosaEtAl2011a, Dullius2012, tenCatenEtAl2012}. As áreas florestadas são encontradas nos mais 
diversos estádios de desenvolvimento, com os estratos intermediário e superior concentrando a maior 
parte dos indivíduos. As florestas originais, e secundárias em estágio avançado de desenvolvimento, 
são encontradas apenas em áreas de difícil acesso. Em geral, as áreas florestadas predominam nas 
regiões de maior declividade, com solo raso e pedregoso. Tais condições pedológicas são resultado, 
na maioria dos casos, das condições geológicas e geomorfológicas, mas também podem derivar da 
degradação causada pelo intenso uso da terra durante inúmeras décadas \cite{SamuelRosaEtAl2011a}. 
A ocorrência de fragmentos ao longo dos cursos de água, estradas e próximo de edificações é comum. 
Nas formações mais jovens, é comum encontrar fragmentos de carvão e sinais de uso agrícola no 
passado. As áreas sob vegetação secundária (capoeira) também ocorrem em toda a bacia de captação, 
predominantemente em locais de difícil acesso e acentuada declividade, geralmente interligados por 
caminhos internos, indicando sua utilização agrícola no passado \cite{SamuelRosaEtAl2011a}. Sua 
composição florística varia entre as áreas, predominando espécies de porte herbáceo e arbustivo. O 
solo dessas áreas costuma ser menos pedregoso que nas áreas florestadas, mas apresentam profundidade
semelhante \cite{SamuelRosaEtAl2011a}, indicando que as atividades agrícolas foram encerradas antes
do solo atingir seu nível máximo de degradação. Entretanto, a reserva de nutrientes do solo, assim 
como a sua fertilidade física, foram esgotadas à tal ponto que a vegetação secundária ainda não foi 
capaz de produzir melhorias significativas quando comparado com as condições originais 
\cite{Menezes2008, Zalamena2008}.

% \begin{figure}[h]
%   \centering
%   \includegraphics[height=7cm]{figures/land2009}
%   \caption{Mapa do uso da terra da AEPQ-SM publicado na escala 1:2.000 \cite{SamuelRosaEtAl2011a}.\\Legenda de acordo com o guia para descrição do solo da FAO \cite{FAO2006}: O - Outros usos (água), S - Assentamentos humanos, PF - Silvicultura, AA - Agricultura anual, H - Pecuária, SS - Vegetação secundária (capoeira), FS - Floresta nativa.}
%   \label{fig:land2009}
% \end{figure}

Apesar do abandono de muitas áreas de produção agrícola nas últimas décadas, sobretudo aquelas com 
maior dificuldade de acesso, os caminhos internos utilizados para o escoamento da produção, mesmo 
depois de inativos, continuam associados ao processo de degradação da terra. Isso ocorre porque, na 
grande maioria dos casos, todo o fluxo da água de escorrimento superficial proveniente dos eventos 
de precipitação é concentrado nesses caminhos internos, onde, geralmente, quantidade significativa 
de resíduos vegetais está depositada. Esse resíduo vegetal, mais a fração mineral do solo carregada 
pela enxurrada, costuma chegar com facilidade aos cursos de água. No caso dos caminhos internos 
ainda utilizados na condução de rebanhos, a produção de sedimentos minerais tende a ser ainda maior,
haja vista o impacto mecânico do pisoteio animal sobre a desagregação do solo. Além disso, muitas 
áreas florestadas e sob vegetação secundária ainda são utilizadas para o pastoreio de bovinos e 
equinos \cite{SamuelRosaEtAl2011a}. Isso causa a degradação da camada superficial do solo, 
sobretudo pela sua compactação, dificultando a infiltração da água dos eventos de precipitação, o 
que resulta na perda da serapilheira \cite{ScheneiderEtAl1978}. Além disso, a própria regeneração 
natural é prejudicada, uma vez que plantas em estádio inicial de desenvolvimento e raízes 
superficiais são destruídas \cite{ScheneiderEtAl1978, HackEtAl2005}. O resultado desse processo é 
a perda da qualidade do solo e, consequentemente, da produtividade da floresta \cite{KonigEtAl2002}.

% \begin{figure}[h]
%   \centering
%   \includegraphics[height=7cm]{figures/ndvi30}
%   \caption{Mapa do índice de vegetação por diferença normalizada (Landsat~5~TM) da AEPQ-SM em dezembro de 2010.}
%   \label{fig:ndvi30}
% \end{figure}

Dado que a AEPQ-SM também possui áreas com relevo plano a suave-ondulado e solo profundo, 
\SI{\pm30}{\percent} de sua superfície ainda é utilizada com atividades agrosilvipastoris ao longo 
de todo o ano, principalmente a pecuária extensiva \cite{SamuelRosaEtAl2011a}. As áreas de produção
animal extensiva predominam nas porções norte, sul e centro-oeste, ao longo dos cursos de água e 
estradas. O solo é mais profundo e menos pedregoso do que em áreas de floresta natural e vegetação 
secundária. Além disso, são encontrados fragmentos de carvão e formações de micro relevo devido ao 
uso de implementos agrícolas para o revolvimento do solo na maior parte dos locais, evidenciando seu
uso agrícola no passado. Essas áreas podem ser divididas entre aquelas de campo sujo (pastagens 
naturais mal manejadas, com predomínio de vegetação de porte herbáceo-arbustivo) e campo limpo 
(pastagens naturais e perenes bem manejadas) \cite{SamuelRosaEtAl2011a}. Em geral, as áreas de 
campo sujo estão próximas a áreas de vegetação secundária (capoeira), com relevo mais declivoso e 
solo mais pedregoso e raso do que aquelas de campo limpo, indicando que podem vir a ser abandonadas 
dentro de pouco tempo. Assim como as áreas florestadas e sob vegetação secundária, a dinâmica de uso
dessas áreas ao longo do tempo é bastante complexa, dificultando o estabelecimento de relações 
diretas com a maior parte das características do solo. Entretanto, sob o ponto de vista da reserva 
de nutrientes e matéria orgânica, o solo pode ser considerado pobre, uma vez que a exploração 
pecuária é totalmente extrativista e extensiva.

% \begin{figure}[h]
%   \centering
%   \includegraphics[height=7cm]{figures/ndvi5}
%   \caption{Mapa do índice de vegetação por diferença normalizada (RapidEye) da AEPQ-SM em novembro de 2012.}
%   \label{fig:ndvi5}
% \end{figure}

As atividades agrosilvopastoris que ocupam menor extensão territorial da AEPQ-SM são a agricultura e
a silvicultura. As áreas de lavoura estão dispersas em toda a área, geralmente localizadas em 
terrenos de pequena declividade e com solo medianamente profundo. Entretanto, algumas áreas de 
produção ocorrem em declives superiores a \SI{50}{\percent} e com solo raso e pedregoso 
\cite{SamuelRosaEtAl2011a}. Em qualquer das situações, as condições de degradação do solo costumam 
ser bastante avançadas, exceto por algumas áreas de produção olerícola profissional. Assim como as 
áreas de agricultura, as florestas exóticas (eucalipto) são implantadas em áreas de menor 
declividade e com solo mais profundo, geralmente onde o acesso com máquinas é melhor, sobretudo pela
necessidade de escoamento da produção. A área ocupada por essa atividade possui tendência de 
crescimento, haja vista os novos plantios existentes e o relato de alguns moradores 
\cite{SamuelRosaEtAl2011a}. Em geral, os novos plantios são implantados em áreas de produção 
agropecuária, seja pelo elevado nível de degradação do solo já atingido, seja pela redução da força 
de trabalho das famílias devido ao êxodo rural, ou pela maior lucratividade dessa atividade.

Por fim, as obras de engenharia e assentamentos urbanos são aquelas que ocupam a menor parte da 
AEPQ-SM. No que diz respeito à malha viária, a maior concentração ocorre na porção sul, junto ao 
maior assentamento urbano, localizado no entorno do reservatório. Entretanto, diversas construções 
são encontradas ao longo das estradas que cortam a área, muitas das quais são pertencentes a 
moradores do centro da cidade de Santa Maria e são utilizadas apenas como sítios de final de semana.
O número de sítios de final de semana aumentou significativamente ao longo da última década 
\cite{Goldani2006}, muitos dos quais construídos em locais inapropriados, como margens dos corpos 
de água e áreas de forte declividade. Esse processo de urbanização desordenada, que exigiu a 
realização de obras de corte e aterramento dos terrenos, é um importante contribuinte da carga de 
sedimentos recebida pelo reservatório anualmente \cite{PaivaEtAl2001, DillEtAl2004}. Soma-se a isso
os resíduos domésticos e cloacais despejados nos cursos de água devido à falta de coleta e 
tratamento \cite{Goldani2006}. Quanto às demais obras de engenharia, destacam-se os reservatórios 
de água, a maioria deles de pequena extensão, utilizados para a dessedentação animal. Como os cursos
de água de maior volume estão localizados na porção sul da área, a maior parte dos reservatórios de 
água está na porção norte \cite{SamuelRosaEtAl2011a}. Além disso, a menor permeabilidade do solo e 
do substrato rochoso, bem como da condição topográfica, favorecem essa característica.

\tocless\section{Pedologia}

O solo da área de estudo possui características com forte dependência do material de origem 
\cite{NascimentoEtAl2010} que, conforme descrito acima, é um importante condicionante da 
geomorfologia e da hidrologia. Nas superfícies geomórficas mais estáveis, como no topo do Planalto 
(rochas vulcânicas), nos terraços do Rebordo (rochas vulcânicas e sedimentares) e nas coxilhas de 
relevo suave-ondulado a ondulado (rochas sedimentares), as condições ambientais proporcionam maior 
desenvolvimento do solo em profundidade \cite{Moser1990}. Já nas áreas de relevo mais acidentado 
do Rebordo, onde a taxa de formação do solo deve ser semelhante a taxa de remoção, as condições 
ambientais são pouco favoráveis ao desenvolvimento do solo \cite{Moser1990, DalmolinEtAl2006a, 
Sturmer2008, SamuelRosaEtAl2011a}. Entretanto, além da condição geomorfológica que impede ou limita 
o desenvolvimento do solo nesses locais, a resistência do material parental ao intemperismo também 
é um importante condicionante \cite{Pedron2007}. Já nas planícies aluviais, as características do 
solo são fortemente influenciadas pelo hidromorfismo e deposição sedimentar \cite{Moser1990, 
Miguel2010}.

% \begin{figure}[h]
%   \centering
%   \includegraphics[height=7cm]{figures/solo1989}
%   \caption{Mapa do solo da AEPQ-SM publicado na escala 1:100.000 \cite{AzolinEtAl1988}.\\Legenda de acordo com o sistema brasileiro de taxonomia do solo \cite{SoaresEtAl2005}: TBa-Rd - Terra Bruna Estruturada álica, Re4 - Solo Litólico Eutrófico/Distrófico relevo montanhoso, Re-C-Co - Solo Litólico Eutrófico relevo forte ondulado, Rd1 - Solo Litólico Distrófico/Eutrófico, C1 - Cambissolo Eutrófico.}
%   \label{fig:solo1989}
% \end{figure}

Como a AEPQ-SM possui a maior parte de sua superfície em condições de forte declividade, o solo é, 
predominantemente, pouco desenvolvido, com profundidade inferior à \SI{50}{\centi\metre} até o 
contato lítico e comum ocorrência de pedregosidade e rochosidade \cite{Miguel2010}. Assim, 
predominam as áreas de solo classificado como Neossolo Litólico Distro-Úmbrico típico, Cambissolo 
Háplico Ta Eutrófico típico, Neossolo Litólico Eutro-Úmbrico típico e Neossolo Regolítico 
Distro-Úmbrico típico. Em muitas áreas de maior estabilidade (topos de morros, patamares do Rebordo 
do Planalto e coxilhas), onde as condições para o desenvolvimento pedogenético são mais favoráveis 
\cite{Moser1990}, o solo também apresenta características que indicam seu fraco desenvolvimento 
\cite{MouraBueno2012}. Entretanto, nessas áreas, o solo sofreu severa degradação ao longo das 
inúmeras décadas de cultivo intensivo sem uso de práticas conservacionistas, fazendo com que parte 
significativa de sua camada superficial fosse perdida \cite{SamuelRosaEtAl2011a}. Nos patamares 
constituídos inicialmente por colúvios sedimentares (arenito Botucatu) e vulcânicos (fragmentos de 
tamanhos variáveis), atualmente, devido à forte erosão à que foi submetido, o solo apresenta 
superfície recoberta por fragmentos rochosos que limitam seu uso para atividades agrosilvopastoris 
\cite{MouraBueno2012}. Dado que a ocorrência de ambientes com solo pouco desenvolvido não é função 
restrita da geomorfologia e do material de origem, mas também do uso antrópico, não é possível 
estabelecer uma relação direta e unívoca da textura do solo com os níveis taxonômicos mais altos. 
Essa diferenciação pode ser realizada mais adequadamente a partir da identificação do material 
originário. Assim, os arenitos conferem textura arenosa ao solo, sobretudo aqueles da Formação 
Botucatu, enquanto as rochas vulcânicas conferem textura média ao solo, sobretudo os 
basaltos-andesitos tholeíticos.

% \begin{figure}[h]
%   \centering
%   \includegraphics[height=7cm]{figures/solo2010}
%   \caption{Mapa do solo da AEPQ-SM publicado na escala 1:25.000 \cite{MiguelEtAl2012}.\\Legenda de acordo com o sistema brasileiro de taxonomia do solo \cite{SantosEtAl2006}: PBAC - Argissolo Bruno-Acinzentado, PV - Argissolo Vermelho, C-R - Cambissolo-Neossolo, RY - Neossolo Flúvico, RL - Neossolo Litólico, RL-RR - Neossolo Litólico-Neossolo Regolítico, RR - Neossolo Regolítico, SX - Planossolo Háplico.}
%   \label{fig:solo2010}
% \end{figure}

As áreas com maior desenvolvimento pedogenético em profundidade ocorrem na paisagem menos declivosa 
do Planalto (rochas vulcânicas), algumas coxilhas (rochas sedimentares) e depósitos aluviais, mas 
com menor expressão territorial \cite{Miguel2010}. Quando em áreas de material de origem vulcânica,
sobretudo basaltos-andesitos tholeíticos, encontra-se solo classificado como Argissolo Vermelho 
Alítico típico. Trata-se de solo com horizonte superficial de textura arenosa a média, sobrejacente 
a um horizonte subsuperficial de textura média a argilosa \cite{Miguel2010}. Nas áreas do Planalto 
em que ocorrem riólitos-riodacitos granofíricos, o solo costuma apresentar menor desenvolvimento, 
predominantemente classificado como Neossolo Litólico e Cambissolo Háplico. Em pequenas manchas, o 
solo é classificado como Argissolo Vermelho-Amarelo. O menor desenvolvimento do solo originário de 
riólitos-riodacitos granofíricos deve-se, exatamente, a maior resistência da rocha ao intemperismo, 
uma vez que possui maior teor de sílica, sobretudo na forma de grandes de cristais de quartzo 
cristalizados a baixa temperatura (\SI{<600}{\celsius}) \cite{Pedron2007}.

Quando em áreas de material de origem sedimentar (Formação Caturrita), o solo costuma ser 
classificado como Argissolo Bruno-Acinzentado Alítico abrúptico. Trata-se de solo com horizonte 
superficial de textura arenosa que transiciona de maneira abrupta para um horizonte subsuperficial 
de textura argilosa \cite{Miguel2010}. Essa descontinuidade textural pode ser devida a fatores bem 
distintos. O primeiro deles estaria relacionado às características do próprio material de origem 
que, por ter sido formado em ambiente fluvial durante um período de mudança climática no Triássico 
\cite{PieriniEtAl2002}, apresenta camadas deposicionais com granulometria diferenciada. Assim, a 
presença de camadas de siltitos e folhelhos pode ter contribuído para a formação da descontinuidade 
textural. Outra hipótese trata da contribuição dos arenitos da Formação Botucatu e 
\textit{intertrap} na Formação Serra Geral. Dado que esse material está localizado em posições 
superiores na paisagem e são bastante suscetíveis à erosão, o mesmo pode ter contribuído para a 
formação do horizonte superficial arenoso do solo. Em ambos os casos, a hipótese formulada refere-se
à ocorrência de descontinuidade litológica, com a diferença de que no primeiro caso as litologias 
pertencem à mesma formação geológica. Por fim, a descontinuidade textural observada pode ser 
resultado dos processos pedogenéticos, sobretudo a perda (erosão lateral seletiva) e translocação 
(argiluviação) das partículas mais finas. Até o presente momento, essa é a hipótese mais aceita.

Nas áreas deprimidas da paisagem do Planalto, formando pequenas bacias de acumulação, ou nas áreas 
planas ao longo dos cursos de água da Depressão Periférica, o solo é classificado como Planossolo 
Háplico Alítico típico \cite{Miguel2010}. Trata-se de solo com horizonte superficial de textura 
média sobre horizonte subsuperficial de textura média a argilosa, podendo ou não apresentar 
horizonte intermediário eluvial. Ainda mais próximo dos cursos de água, o solo é classificado como 
Neossolo Flúvico Tb Eutrófico fragmentário \cite{Miguel2010}. Como o material de origem é diverso 
e possui arranjamento espacial discordante, a textura do solo é variável, mas sempre arenosa ou 
média, nunca argilosa, mesmo quando presente em áreas do Planalto ou do Rebordo. Por fim, com 
expressão territorial ainda menor, o solo é classificado como Neossolo Quartzarênico Órtico típico 
em alguns patamares do Rebordo, sejam eles de origem estrutural ou da deposição de colúvios do 
arenito da Formação Botucatu.

Apesar das características do solo da área apontarem para sua fragilidade, sobretudo sua 
granulometria, a geomorfologia possui papel preponderante sobre o potencial de perda de solo por 
erosão laminar \cite{Miguel2010}. Contudo, as áreas de maior declividade estão atualmente ocupadas 
por densa cobertura florestal \cite{SamuelRosaEtAl2011a}, reduzindo a fragilidade do solo. 
Estimativas indicam que apenas uma pequena fração da área deve apresentar perdas de solo por erosão 
laminar acima de valores toleráveis \cite{Miguel2010}. Algumas observações até mesmo indicam que 
essas estimativas estão acima da perda real de solo por erosão laminar na bacia \cite{Branco1998, 
MouraBueno2012}. Isso permite afirmar que os processos de degradação do solo na AEPQ-SM são de 
pequena intensidade e bastante pontuais (áreas urbanizadas). A provável redução da dinâmica de 
alteração das características da paisagem num futuro próximo deverá contribuir para a construção de 
um entendimento mais acurado da relação entre o solo e os demais componentes ambientais.

 % conceptual model of pedogenesis
% \artigofalse
\chapter{Santa Maria dataset}
\label{apen:database}

\tocless\section{Point soil data}

\tocless\subsection{Soil sampling}

% TODO: provide a better description of the purposive sampling strategy used.
The soil database is composed by 350 point observations. They were sampled during soil and land use surveys carried out between 2003 and 2012 \cite{Pedron2005, SamuelRosaEtAl2011a, MiguelEtAl2012}. Sampling locations were selected using expert knowledge (purposive sampling). Soil scientists made soil observations in most common geomorphological features and land uses, and in patches with similar soil taxa, trying to obtain a somehow uniform coverage of the geographic and feature spaces. Resulting sampling density is of about 0.18 observations per hectare. Average separation distance between two neighboring points is 181 m. Minimum and maximum separation distances are, respectively, 18 and 328 m. Standard deviation is 80 m, and 95\% of neighboring point-pairs are separated by more than 49 m.

During soil sampling, the soil scientists defined an area of about 100 m$^2$ around each sampling location. Three soil pits were opened within this area. Soil was collected to a depth of 20 cm, or down to the bedrock when soil depth was smaller than 20 cm. Composite samples were obtained for laboratory analysis. Sample locations were georeferenced using a GNSS navigation receiver with horizontal accuracy of about 8 m. In some situations the GNSS signal was compromised, such as inside forest canopy and deep valleys, resulting in a positional error larger than about 8 m. In such cases georeferencing was performed on the computer screen using Google Earth\textregistered{} satellite images (spatial resolution > 1 m). Positional error of these images was assumed to be smalled than about 8 m from visual assessment.

Sixty (60) validation samples were collected at regular spacings of 100 m from 12 linear transects of 400 m during the years 2012 and 2013. Sampled transects were randomly selected from a set of 180 transects drawn by three experts in Google Earth\textregistered{} (Figure \ref{fig:transects}). They were located in areas where experts were almost certain that soil samples could be collected, and aligned in the direction of maximum expected spatial variance of environmental features. Sampling sites were located in the field using a GNSS navigation receiver with horizontal accuracy of about 8 m. A differential GNSS signal receiver with centimetric horizontal precision was used for georeferencing and collection of altimetric data. A single soil sample was collected to a depth of 20 cm (or down to the bedrock when soil depth was smaller than 20 cm) in a soil pit opened within a radius of 2 m.

% TODO: add a figure with all transects highlighting the selected ones.
% \begin{figure}[!ht]
%   \centering
%   \includegraphics[width=,height=]{fig/transects}
%   \caption{\small Three experts drawn 180 linear transects aligned in the direction of maximum expected spatial variance of environmental features. They avoided locations were it was known that geographic barriers or land owners would impede the access to collect soil samples. Using probability sampling, 12 transects were selected to collect validation observations (red).}
%   \label{fig:transects}
% \end{figure}

\tocless\subsection{Laboratory analysis}

Soil samples were air dried, crushed and passed through a 2 mm-sieve prior to laboratory analysis. One laboratory replicate was used to calculate analytical errors. Only a few soil observations from \citep{Pedron2005, MiguelEtAl2012} lack estimates of analytical errors.

Particle size analysis was performed using sodium hydroxide (NaOH) 1 mol $\text{L}{}^{-1}$ as dispersing agent. The clay fraction (< 0.002 mm) was determined by the pipette method; the sand fraction (0.053 to 2 mm) was determined by wet sieving; and the silt fraction (0.002 to 0.053 mm) was calculated by difference. Soil samples with organic matter content larger than 5\% were submitted to oxidative treatment with hydrogen peroxide ($\text{H}_{2}\text{O}_{2}$) prior to the analysis.

Organic carbon content was determined through wet digestion using 0.067 mol $\text{L}^{-1}$ sulfocromic solution (potassium bichromate - $\text{K}_{2}\text{Cr}_{2}\text{O}_{7}$, and sulphuric acid - $\text{H}_{2}\text{SO}_{4}$) in a digestion block at 150$^{\circ}$C during 30 min. The solution was titrated using 0.1 mol L$^{-1}$ ammonium ferrous sulfate {[}$\text{Fe(NH}_{4}\text{)}_{2}\text{(SO}_{4}\text{)}_{2}\text{.6H}_{2}\text{O}${]} and results were multiplied by 1.11 to correct to the standard method (dry combustion).% TODO: reference to this correction factor.

Effective cation exchange capacity (ECEC) was calculated as the sum of exchangeable bases plus exchangeable acidity. Calcium (Ca) and magnesium (Mg) were quantified by means of atomic absorption spectroscopy, and aluminum (Al) was quantified by titration with 0.025 mol L$^{-1}$ NaOH solution after extraction with 1.0 mol L$^{-1}$ KCl solution. Potassium (K) and sodium (Na) were quantified by means of flame atomic emission spectrometry after extraction with 0.05 mol L$^{-1}$ HCl solution + 0.025 mol L$^{-1}$ $\text{H}_{2}\text{SO}_{4}$ solution (Mehlich-I solution). Because the standard method for determining exchangeable bases relies on the use of barium chloride ($\text{BaCl}_{2}$), a correction factor will be calculated using 60 soil samples from the soil database. Soil samples will be selected through probability sampling using information on clay content, organic carbon content and sum of bases. Validation will be performed using 20 independent soil samples.

\tocless\section{Environmental co-variates}

Environmental co-variates are used to build predictive models of soil properties because they are proxies of soil forming processes. They provide information about topography, vegetation, land use, geology, soil parent material, climate, soil itself and other intimately-associated surface conditions. In the present study, more than 100 environmental co-variates are used. All of them were derived from freely available sources, including area-class soil maps, digital elevation models, geological maps, land use maps, and orbital images.
%TODO: include a description of the GCPs used to validate the environmental covariates and to orthorectify satellite images. Also, present the equations used to calculate validation statistics.

\tocless\subsection{Area-class soil maps}

Most soil maps available use the area-class model of representation. It is a discrete model of spatial variation that divides the survey area into internally more homogeneous polygons sharing sharp and well-defined boundaries \citep{Rossiter2000}. Each polygon receives a class name and is defined as a mapping unit in the map legend. The main objective of the area-class model is to minimize the within-class variance and maximize the between-class variance \cite{WebsterEtAl1990}.

Several area-class soil maps are available for the study area, but only two are used in the present study. The first of them (\texttt{SOIL\_100}) was published at a scale of 1:100,000 \citep{AzolinEtAl1988} (Figure \ref{fig:soil-maps}, left). Existing area-class soil maps and technical reports \cite{Brasil1973, Azolin1977, MacielEtAl1987a, MacielEtAl1987, AbraoEtAl1988}, and sparse field visits were used to elaborate the preliminary legend of the soil map. Aerial photographs (1:60,000) were used to produce the first draft of the map. Field verification of soil polygons was done along the road network (convenience sampling). These observations were used to estimate the composition (occurrence and spatial distribution of soil taxa) of soil mapping units. They were also used to revise the first draft of the map. The final version of the map was prepared using topographic maps originally published at a scale of 1:50,000 and resampled to a scale of 1:100,000. Soil classification followed the criteria adopted by the Brazilian soil science community at that time \citep{Brasil1973, CamargoEtAl1982, Carvalho1982, LemosEtAl1982, OlmosEtAl1982}. Identification of soil classes was performed based on morphological features, analytical data compiled from existing technical reports and analysis of soil profiles collected along the road network. Description of each soil mapping unit includes the estimated area (in hectares) and the approximate taxonomic composition (in percentage). Validation statistics are absent in the survey report.

The second area-class soil map (\texttt{SOIL\_25}) used in the present study is authored by \cite{Miguel2010} and was prepared at a scale of 1:25,000 (Figure \ref{fig:soil-maps}, right). Orbital images produced by Digital Globe\textregistered{} (Quick Bird satellite) freely available for visualization in Google Earth\textregistered{} were used to produce the first draft of the soil map. Existing area-class soil maps and technical reports \citep{Pedron2005, Poelking2007, Sturmer2008} were used to help defining the preliminary map legend. Field observations (auger holes) were done in approximately 350 locations using a purposive sampling approach. These observations helped to identify six representative soil profiles. Soil sampling and description of representative soil profiles, and laboratory analysis of soil samples, followed the standard protocol adopted in Brazil \citep{ClaessenEtAl1997, SantosEtAl2005}. Soil classification was done following the criteria of the Brazilian System of Soil Classification \citep{SantosEtAl2006}. The final version of the map was prepared using orbital images freely available for visualization in Google Earth\textregistered{} and manually-digitalized topographic maps published at a scale of 1:25,000 \citep{DSG1992a, DSG1992}. Description of soil mapping units include only the most common soil taxon, followed by morphological and laboratory data of representative soil profiles. Alike \texttt{SOIL\_100} described above, validation statistics are absent in the survey report of \texttt{SOIL\_25}.

Both area-class soil maps went through different preprocessing routines. The original \texttt{SOIL\_100} is available only in the analogical format, what required its digitalization prior to this study. Georeferencing was carried out using the GDAL Georeferencer plug-in in QGIS \citep{GDAL2013, QGIS2013}. Intersections between all meridians and parallels (a total of nine) were used as control points to adjust a second order polynomial model. Resampling was performed using the cubic resampling method. Soil polygons and their attributes were also manually digitalized in QGIS. Because of the coarseness on the map scale, most geographical markers used to locate validation GCPs could not be identified and positional validation was performed using only four GCPs. Estimated error statistics suggest that there can exist large positional errors in all directions, with an estimated accuracy of RMSE = 114 m and a mean azimuth of 128$^{\circ}$ (Table \ref{tab:soil-geo-val} and Figure \ref{fig:soil-azim}).

\begin{table}[ht]
  \caption{Estimated error statistics (standard deviation between parenthesis) of the validation of area-class soil map \texttt{SOIL\_100} in the geographic space. Validation statistics were estimated using four GCPs located in easily identifiable geographical markers. Estimates were corrected to the size of the population.}
  \label{tab:soil-geo-val}
  \centering
  {\small
  \begin{tabular}{lrrrr}
    \hline
    Statistics           & X coordinate & Y coordinate & Error vector & Azimuth                  \\
    \hline
    Mean, m              & 30   (79)    & -36  (67)    & 105   (43)   & 128$^\circ$ (80$^\circ$) \\ 
    Absolute mean, m     & 58   (63)    & 64   (40)    & -            & -                        \\ 
    Squared mean, m$^2$  & 7241 (11353) & 5712 (6197)  & 12953 (9613) & -                        \\ 
    \hline
  \end{tabular}}
\end{table}

% \begin{figure}[!ht]
%   \centering
%   \includegraphics[width=0.45\textwidth]{azim-soil100}
%   \includegraphics[width=0.45\textwidth]{azim-soil25}
%   \caption{Histogram of the azimuth distribution of the validation of area-class soil maps \texttt{SOIL\_100} and \texttt{SOIL\_25} in the attribute space. Azimuth values were estimated using, respectively, four and ... GCPs located in easily identifiable geographical markers. Estimates were corrected to the size of the population. The graph was produced using R-package \textit{VecStatGraphs2D}.}
%   \label{fig:soil-azim}
% \end{figure}

The original \texttt{SOIL\_25} is available in digital format in the personal database of the author \citep{Miguel2010}. A topology check (Topology Checker plug-in in QGIS) identified that it presents many gaps and overlaps between polygons. This required a topological edition prior to the use of \texttt{SOIL\_25}. There also is a mismatch between the boundary of the survey area used to produce \texttt{SOIL\_25} and the actual boundary of the study area estimated using \texttt{ELEV\_10} (Section \ref{sec:dem}). This occurred because the database used to produce \texttt{SOIL\_25} included Google Earth imagery\textregistered{} and topographic maps, which are data sources that differ considerably in their positional accuracy (Sections \ref{sec:dem} and \ref{sec:land}). To avoid data losses, all boundary gaps were manually filled using the closest mapping unit. Boundaries of soil polygons were defined based on land use (\texttt{LU2009}, Section \ref{sec:land}) and topographic data (contour lines, Section \ref{sec:dem}) as it was done for the original map \citep{Miguel2010}. New delineations were checked and approved without modifications by the author of the original map. Because \texttt{SOIL\_25} includes very few geographical markers, its positional validation was not possible with the available GCPs. However, the positional accuracy (RMSE) is expected to vary between 8 m and 114 m across the map as a result of the different errors present in the data sources used in its production.

Both \texttt{SOIL\_100} and \texttt{SOIL\_25} were imported into GRASS GIS, cropped to the bounding box of the study area, and geometrically corrected to match the prediction grid (5 meters pixel size). Registration and geocoding was performed using the nearest neighbor resampling method. Each category was named with the code of respective mapping units in the original maps. Prior to validation in the attribute space, class codes of \texttt{SOIL\_100} were changed to match soil taxa codes of the current Brazilian System of Soil Classification using a standard correlation table \citep{SantosEtAl2006}.

Table \ref{tab:soil-attr-val} shows that the overall purity of both soil maps is not significantly different. The main reason for this is that validation was performed considering only the second level of the Brazilian System of Soil Classification. It is expected that \texttt{SOIL\_25} would outperform \texttt{SOIL\_100} if validation data included soil classification up to the fourth level of the Brazilian System of Soil Classification. Estimated overall purity values are also very low (< 35\%). The main reason can be the fact that very few soil profiles were described and sampled to produce both maps. There also are two minor potential sources of error. First, because \texttt{SOIL\_100} does not include analytical soil data in the survey report, all soil taxa had to be translated to the current Brazilian System of Soil Classification based only on a standard correlation table \citep{SantosEtAl2006} and expert knowledge. Second, soil taxa described at the validation points was obtained analyzing only morphological soil properties and the basis and concepts of the Brazilian System of Soil Classification (expert knowledge).

\begin{table}[ht]
\caption{Estimated error statistics of the validation of area-class soil maps \texttt{SOIL\_100} and \texttt{SOIL\_25} in the attribute space. Validation statistics were estimated using 60 validation points located in 12 linear transects (clustered samples).}
\label{tab:soil-attr-val}
\centering
{\small
\begin{tabular}{lrrr}
\hline
Soil map              & LCB95Pct & Estimate & UCB95Pct \\
\hline
\texttt{SOIL\_100}    & 21.69    & 31.67    & 41.65    \\
\texttt{SOIL\_25}     & 20.81    & 30.00    & 39.19    \\
\hline
\end{tabular}}
\end{table}

% TODO: figure with both area-class soil maps
% \begin{figure}[!ht]
%   \centering
%   \includegraphics[width=0.3\textwidth]{fig/soil-100}
%   \includegraphics[width=0.3\textwidth]{fig/soil-25}
%   \caption{Area-class soil maps used as sources of environmental co-variates. On the left, the area-class soil map produced by \cite{AzolinEtAl1988} and published at a scale of 1:100,000 (\texttt{SOIL\_100}). On the right, the area-class soil map produced by \cite{Miguel2010} at a scale of 1:25,000 (\texttt{SOIL\_25}).}
%   \label{fig:soil-maps}
% \end{figure}

The main advantage of \texttt{SOIL\_25} in relation to \texttt{SOIL\_100} is the level of detail. While \texttt{SOIL\_100} has only five classes covering the study area, \texttt{SOIL\_25} has eight classes. This enabled the derivation of six environmental covariates from \texttt{SOIL\_100} and ten environmental covariates from \texttt{SOIL\_25}. Environmental covariates derived from \texttt{SOIL\_100} are the following:

\begin{description}
  \item[\texttt{SOIL\_100a}] This covariate separates map unit Rd1 from other map units. It is composed mainly by shallow soils with low to high base saturation (Solo Litólico distrófico/eutrófico; Neossolo Litólico distrófico/eutrófico; Distric/Eutric Leptosol) located in slopping terrain;
  
  \item[\texttt{SOIL\_100b}] This covariate separates map unit Re4 from other map units. It is also composed mainly by shallow soils with low to high base saturation (Solo Litólico eutrófico/distrófico relevo montanhoso; Neossolo Litólico distrófico/eutrófico; Distric/Eutric Leptosol), the difference being that it covers mountainous terrain;
  
  \item[\texttt{SOIL\_100c}] This covariate separates map unit Re-C-Co from other map units. It is a map unit composed by shallow soils with high base saturation located in strongly sloping terrain (Solo Litólico eutrófico relevo forte ondulado; Neossolo Litólico Eutrófico; Eutric Leptosol), low weathered soils (Cambissolo eutrófico; Cambissolo Háplico eutrófico; Eutric Cambisol), and colluvial deposits;
  
  \item[\texttt{SOIL\_100d}] This covariate separates map unit TBa-Rd from other map units. It is a map unit composed by deep, well-structured, low base saturation soils (Terra Bruna Estruturada álica; Nitossolo; Nitisol), and shallow soils (Solo Litólico; Neossolo Litólico; Leptosol);
  
  \item[\texttt{SOIL\_100e}] This covariate was produced combining map units Rd1 and Re4. Thus, it includes mapping units composed by shallow soils in both sloping and mountainous terrain;
  
  \item[\texttt{SOIL\_100f}] This covariate was produced combining map units TBa-Rd and C1. The last is composed by low weathered soils developed in lower landscape positions, close to drainage channels (Cambissolo eutrófico; Cambissolo Eutrófico; Eutric Cambisol). Thus, this covariate includes the best soil types for agricultural practices among those identified in the survey.
\end{description}

% TODO: figure with covariates derived from SOIL_100
% \begin{figure}[!ht]
%   \centering
%   \includegraphics[width=0.3\textwidth]{fig/soil-100a}
%   \includegraphics[width=0.3\textwidth]{fig/soil-100b}
%   \includegraphics[width=0.3\textwidth]{fig/soil-100c}
%   \includegraphics[width=0.3\textwidth]{fig/soil-100d}
%   \includegraphics[width=0.3\textwidth]{fig/soil-100e}
%   \includegraphics[width=0.3\textwidth]{fig/soil-100f}
%   \caption{Environmental covariates derived from the area-class soil map produced by \cite{AzolinEtAl1988} and published at a scale of 1:100,000 (\texttt{SOIL\_100}).}
%   \label{fig:soil100-covars}
% \end{figure}

Environmental covariates derived from \texttt{SOIL\_25} are the following:

\begin{description}
  \item[\texttt{SOIL\_25a}] This covariate separates map unit PBAC from other map units. It is composed mainly by moderately deep soils derived from sedimentary rocks, with abrupt textural change and low base saturation (Argissolo Bruno-Acinzentado; Alisol);

  \item[\texttt{SOIL\_25b}] This covariate separates map unit PV from other map units. It is composed mainly by deep soils derived from igneous rocks, with moderate textural gradient, and low base saturation (Argissolo Vermelho; Acrisol);
 
  \item[\texttt{SOIL\_25c}] This covariate separates map unit C-R from other map units. It is composed mainly by low weathered soils (Cambissolo; Cambisol) and shallow soils with low to high base saturation (Neossolo Litólico/Regolítico eutrófico/distrófico; Eutric/Distric Leptosol/Regosol);
 
  \item[\texttt{SOIL\_25d}] This covariate separates map unit RL from other map units. It is composed mainly by shallow soils with low to high base saturation (Neossolo Litólico eutrófico/distrófico; Eutric/Distric Leptosol);
 
  \item[\texttt{SOIL\_25e}] This covariate separates map unit RL-RR from other map units. It is composed mainly by shallow soils (Neossolo Litólico + Neossolo Regolítico; Leptosol + Regosol) with low to high base saturation;
 
  \item[\texttt{SOIL\_25f}] This covariate separates map unit RR from other map units. It is composed mainly by shallow soils (Neossolo Regolítico; Regosol), with low base saturation, developed on sedimentary rocks;
 
  \item[\texttt{SOIL\_25g}] This covariate separates map unit RY from other map units. It is composed mainly by soils developed from fluvial deposits (Neossolo Flúvico; Fluvisol);
 
  \item[\texttt{SOIL\_25h}] This covariate was produced combining map units PBAC, PV, and SX. The last is composed mainly by moderately deep soils derived from sedimentary rocks, with abrupt textural change, low base saturation, and which are saturated with water for long periods of the year (Planossolo Háplico; Planosol). Thus, this covariate includes the best soil types for agricultural practices among those identified in the survey;
 
  \item[\texttt{SOIL\_25i}] This covariate was produced combining map units RL, RL-RR, and RR. Thus, this covariate includes all three map units composed mainly by shallow soils (Neossolo Litólico and Neossolo Regolítico; Leptosol and Regosol);
  
  \item[\texttt{SOIL\_25j}] This covariate was produced combining map units PV, RL, RL-RR, and C-R. Thus, it includes all four map units composed mainly by soils derived from igneous rocks.
\end{description}

% TODO: figure with covariates derived from SOIL_25
% \begin{figure}[!ht]
%   \centering
%   \includegraphics[width=0.3\textwidth]{fig/soil-25a}
%   \includegraphics[width=0.3\textwidth]{fig/soil-25b}
%   \includegraphics[width=0.3\textwidth]{fig/soil-25c}
%   \includegraphics[width=0.3\textwidth]{fig/soil-25d}
%   \includegraphics[width=0.3\textwidth]{fig/soil-25e}
%   \includegraphics[width=0.3\textwidth]{fig/soil-25f}
%   \includegraphics[width=0.3\textwidth]{fig/soil-25g}
%   \includegraphics[width=0.3\textwidth]{fig/soil-25h}
%   \includegraphics[width=0.3\textwidth]{fig/soil-25i}
%   \includegraphics[width=0.3\textwidth]{fig/soil-25j}
%   \caption{Environmental covariates derived from the area-class soil map produced by \cite{Miguel2010} at a scale of 1:25,000 (\texttt{SOIL\_25}).}
%   \label{fig:soil25-covars}
% \end{figure}

\tocless\subsection{Digital elevation models}\label{sec:dem}

Digital elevation models (DEMs) are one of the main sources of environmental co-variates used to build predictive models of soil properties. This is due to their extensive availability and the usually strong correlation between DEM derivatives and soil properties \cite{BishopEtAl2006, Grunwald2009}.

Three DEMs are used in the present study as sources of environmental co-variates. The first DEM (\texttt{ELEV\_10}) is the result of the interpolation of the contour lines of the most recent topographic maps produced by the Brazilian Army (scale of 1:25,000) \citep{DSG1980, DSG1992, DSG1992a}. Because all three topographic maps needed to cover the study area are available only in the analogical format, their digitalization was necessary. Georeferencing was carried out using the GDAL Georeferencer plug-in in QGIS \citep{GDAL2013, QGIS2013}. Intersections between all meridians and parallels (about 160 per topographic map) were used as control points to adjust a third order polynomial model. Resampling was performed using the cubic resampling method. All contour lines, peaks, lakes and rivers, and their respective attributes within a distance of 1000 m from the boundary of the study area were also manually digitalized and stored in the vector format. After digitalization, the original coordinate reference system (EPSG:31982 - SIRGAS 2000 / UTM zone 22S) of all vector files was transformed to WGS 1984 / UTM zone 22S (EPSG:32722) using the R-package \textit{rgdal} \citep{BivandEtAl2013a}.

The positional validation of topographic maps was performed using 14 GCPs located at easily identifiable geographical markers. According to Brazilian legislation, the positional accuracy of these topographic maps is expected to be of, at least, 15 m \citep{Brasil1984}. Estimated validation statistics show that the observed positional error (RMSE = 65 m) is larger than established by current regulations (Table \ref{tab:topomap-geo-val}). The mean error vector (module) is larger than 60 m with an azimuth of 63 $^{\circ}$ (Figure \ref{fig:topomap-azim}). Both x and y coordinates are positively biased, but the largest error occurs in the x coordinate (50 m). Similar mean and mean absolute errors suggest that there is a systematic positional error. An affine transformation was employed using the R-package \textit{vec2dtransf} \citep{Carrillo2012} to eliminate this systematic error. Model parameters were adjusted using the same set of GCP's used for the validation in the geographic space.

\begin{table}[ht]
  \caption{Estimated error statistics (standard deviation between parenthesis) of the validation of topographic maps (scale of 1:25,000) in the geographic space. Validation statistics were estimated using 14 GCPs located in easily identifiable geographical markers. Estimates were corrected to the size of the population.}
  \label{tab:topomap-geo-val}
  \centering
  {\small
  \begin{tabular}{lrrrr}
    \hline
    Statistics          & X coordinate & Y coordinate & Error vector & Azimuth                 \\
    \hline
    Mean, m             & 50   (25)    & 27   (22)    & 63   (19)    & 63$^\circ$ (30$^\circ$) \\ 
    Absolute mean, m    & 50   (25)    & 32   (13)    & -            & -                       \\ 
    Squared mean, m$^2$ & 3088 (3034)  & 1180 (820)   & 4268 (2825)  & -                       \\ 
    \hline
  \end{tabular}}
\end{table}

% \begin{figure}[!ht]
%   \centering
%   \includegraphics[width=0.5\textwidth]{azim-car25}
%   \caption{Histogram of the azimuth distribution of the validation of topographic maps in the attribute space. Azimuth values were estimated using 14 GCPs located in easily identifiable geographical markers. Estimates were corrected to the size of the population. The graph was produced using R-package \textit{VecStatGraphs2D}.}
%   \label{fig:topomap-azim}
% \end{figure}

Interpolation of the raster surface with 5-m pixel size was performed using the function \texttt{Topo to Raster} in ArcGIS\textregistered{} software by ESRI, which includes an interpolation method based on the ANUDEM program developed by \cite{Hutchinson1989}. Vector files of contour lines (multiline), drainage network (multiline), lakes (polygons) and peaks (points) were used to generate an hydrologically correct DEM, that is, a DEM without spurious depressions and giving an accurate representation of the real hydrology. Next, the interpolated DEM was imported into GRASS GIS \citep{GRASS2012}, where a neighborhood average filter was used to remove stair-like artifacts. A window of 7 x 7 pixels was used because it removed a significant amount of the artifacts and did not affect the derived boundary of the study area (see more bellow).

The vertical datum of the DEM was transformed from the local datum to a global datum. The geoidal models MAPGEO 2010 \citep{IBGE2010a} and EGM 1996 \citep{LemoineEtAl1998} were used to calculate the geoidal undulation for the local and global datums, respectively. MAPGEO 2010 is optimized to estimate geoidal undulations in the Brazilian territory, while EGM 1996 is a gravitational model of the Earth and is used as the vertical datum for SRTM products. The following equation was used:

\begin{center}
  \label{eq:geoidal}
  \begin{equation}
    h = H + N,
  \end{equation}
\end{center}

\noindent where $h$ is the ellipsoidal height (height above the reference ellipsoid that approximates the surface of the planet), $H$ is the orthometric height (height above the imaginary surface called geoid and commonly referred as mean sea level), and $N$ is the geoidal undulation. Ellipsoidal heights estimated by MAPGEO 2010 are referenced to the world ellipsoid of 1980, while EGM 1996 estimates ellipsoidal heights referenced to the world ellipsoid of 1984. Because the difference between both ellipsoids is of the order of millimeters, it can be assumed that both models estimate the same ellipsoidal height. Therefore, if $h_{EGM 1996} = h_{\text{MAPGEO 2010}}$, then orthometric heights referenced to the local vertical datum can be transformed to the global vertical datum using the following equation:

\begin{center}
  \begin{equation}
    H_{\text{EGM 1996}} = H_{\text{MAPGEO 2010}} + N_{\text{MAPGEO 2010}} - N_{\text{EGM 1996}}.
  \end{equation}
\end{center}

The difference in the geoidal undulation estimated by both models is of about one meter in the entire study area. Thus, transforming the vertical datum was done adding one meter to the raster surface interpolated from contour lines, yielding the first DEM used in this study (\texttt{ELEV\_10}).

The second DEM (\texttt{ELEV\_90}) used in this study is the well known SRTM DEM (3 arc-seconds $\approx$ 90 m spatial resolution) produced by NASA’s Jet Propulsion Laboratory in collaboration with the National Geospatial-Intelligence Agency \citep{RodriguezEtAl2006}. The SRTM DEM version used here is the \textit{hole-filled SRTM version 4}, prepared by \href{http://www.cgiar.org/}{CGIAR} using the same hydrologically correct interpolation method that was used above to produce \texttt{ELEV\_10} \citep{ReuterEtAl2007, Jarvis2008}. However, the only data source used was the original SRTM DEM converted to point data.

Prior to processing, the SRTM DEM was cropped to the extent of the study area and the coordinate reference system was transformed from WGS 1984 (EPSG:4326) to WGS 1984 / UTM zone 22S (EPSG:32722) using cubic resampling in GDAL (module \texttt{gdalwarp}). This resampling method was used because it is efficient in minimizing the double-oblique stripping present in SRTM products \citep{Samuel-RosaEtAl2013c}. Next, the DEM was resampled to 15 m (GRASS module \texttt{r.resamp.interp}) using cubic resampling. Sinks produced during the datum transformation were filled using the GRASS module \texttt{r.fill.dir}. Vertical datum transformation was not necessary because elevation values of the SRTM DEM already are referenced to the global geoidal model EGM 1996 (orthometric heights).

The third DEM (\texttt{ELEV\_30}) used in this study was produced by the Brazilian National Institute for Space Research (\href{http://www.inpe.br/}{INPE}). This DEM is the result of refining the original SRTM DEM to 1 arc-second spatial resolution ($\approx$30 m) using ordinary kriging with a Gaussian model of spatial covariance \citep{ValerianoEtAl2012}. Different from \texttt{ELEV\_90}, \texttt{ELEV\_30} was not used to calculate DEM derivatives. Instead it was used in the orthorectification and topographic correction of satellite images (\ref{sec:sat}).

Eight tiles were downloaded from the \href{http://www.dsr.inpe.br/topodata/}{TOPODATA} website, imported into QGIS and mosaicked using GDAL module \texttt{gdal\_translate}. The coordinate reference system was transformed from WGS 1984 (EPSG:4326) to WGS 1984 / UTM zone 22S (EPSG:32722) using cubic resampling (GDAL module \texttt{gdalwarp}). Again, this resampling method was used because it is efficient in minimizing the double-oblique stripping present in SRTM products \citep{Samuel-RosaEtAl2013c}. Sinks produced during the datum transformation were filled using GRASS module \texttt{r.fill.dir} implemented in the SEXTANTE library \citep{SEXTANTE2012}.

Because orbital satellites use the WGS 1984 ellipsoid as vertical datum, orthorectification of satellite images has to be done using a DEM with ellipsoidal heights. Conversion from orthometric heights was performed using Equation \ref{eq:geoidal}, with geoidal undulation calculated with the gravitational model EGM 1996. The original DEM with orthometric heights was cropped to the boundary of the study area and resampled to five meters using GRASS GIS module \texttt{r.resamp.interp} with the bicubic resampling method. This DEM was used only to estimated error statistics for the validation in the attribute space.

Table \ref{tab:dem-attr-val} shows that the three DEMs present similar accuracy estimates in the attribute space (RMSE $\approx$ 19 m). In the case of the ELEV\_10, which was derived from contour lines published at a scale of 1:25,000, the estimated accuracy does not meet current Brazilian legislation, which states that the accuracy should be of, at least, 5 m (1/2 of the distance between contour lines) \citep{Brasil1984}.
 
\begin{table}[ht]
  \caption{Estimated error statistics (standard deviation between parenthesis) of the validation of digital elevation models \texttt{ELEV\_90}, \texttt{ELEV\_30} and \texttt{ELEV\_10} in the attribute space. Validation statistics were estimated using 60 validation points located in 12 linear transects (clustered samples).}
  \label{tab:dem-attr-val}
  \centering
  {\small
  \begin{tabular}{lrrrrrr}
    \hline
    Statistics           & \texttt{ELEV\_90} & \texttt{ELEV\_30} & \texttt{ELEV\_10} \\
    \hline
    Mean, m              & -15 (10)          & -17 (9)           & -16 (10)          \\ 
    Absolute mean, m     & 15  (10)          & 17  (9)           & 16  (10)          \\ 
    Squared mean, m$^2$  & 350 (428)         & 361 (406)         & 374 (431)         \\ 
    \hline
  \end{tabular}}
\end{table}

Figure \ref{fig:cdf-elev} shows that estimated validation statistics have different cumulative distribution functions (CDF). The estimates are more uniformly distributed along the interval of values for \texttt{ELEV\_10} than for \texttt{ELEV\_90} and \texttt{ELEV\_30}. While \texttt{ELEV\_10} has a 50\% probability that absolute errors are bellow 15 m, \texttt{ELEV\_90} has a 70\% probability that absolute errors are bellow 15 m. This suggests that the accuracy of \texttt{ELEV\_90} is very consistent across the study area, with a few extreme values, while the accuracy of \texttt{ELEV\_10} have a stronger spatial variation. For \texttt{ELEV\_30}, the interpolation method used to refine the original SRTM DEM to 30 m \citep{ValerianoEtAl2012} seems to have produced a spatial redistribution of the errors.

% \begin{figure}[!ht]
%   \centering
%   \includegraphics[width=0.9\textwidth]{fig/cdf-ELEV-90} 
%   \includegraphics[width=0.9\textwidth]{fig/cdf-ELEV-30}
%   \includegraphics[width=0.9\textwidth]{fig/cdf-ELEV-10}
%   \caption{Cumulative distribution functions of mean error, mean absolute error, and squared error of elevation values estimates by digital elevation models \texttt{ELEV\_90}, \texttt{ELEV\_30}, and \texttt{ELEV\_10}.}
%   \label{fig:cdf-elev}
% \end{figure}

Despite the similar accuracy estimates in the feature space, \texttt{ELEV\_10} is used in this study as a more accurate DEM than \texttt{ELEV\_90}. The main argument is that \texttt{ELEV\_10} provides a better hydrological representation of the study area because it was produced using information about the drainage network and location of lakes and natural depressions. This is evidenced by the shape of the boundaries derived from each DEM using the GRASS modules \texttt{r.watershed} and \texttt{r.water.outlet} (Figure \ref{fig:elev-maps}). The boundary derived from \texttt{ELEV\_90} is clearly unable to capture all hydrological features of the study area. Therefore, the boundary derived using \texttt{ELEV\_10} is used throughout this study with the addition of a 30-m buffer, which is the estimated uncertainty (RMSE = 29.55 m) of the affine transformation used to correct the systematic error identified in topographic maps. The water outlet point used to derive the boundary is located on the bridge that crosses the main drainage channel (-29.65868$^\circ$, -53.78969$^\circ$).

% TODO: figure with both digital elevation models, including the real drainage network and the boundary of the study area.
% \begin{figure}[!ht]
%   \centering
%   \includegraphics[width=0.3\textwidth]{fig/elev-90}
%   \includegraphics[width=0.3\textwidth]{fig/elev-10}
%   \caption{Digital elevations models used as sources of environmental co-variates. On the left, the SRTM digital elevation models prepared by CGIAR and published at a resolution of about 90 m (\texttt{ELEV\_90}). On the right, the digital elevation models produced interpolating contour lines manually digitalized from topographic maps published at a scale of 1:25,000 (\texttt{ELEV\_10}).}
%   \label{fig:elev-maps}
% \end{figure}

Eight terrain attributes were derived from each of \texttt{ELEV\_90} and \texttt{ELEV\_10}, the first of them being the elevation (\texttt{ELEV}). The others are slope, aspect, northernness, flow accumulation, topographic wetness index, stream power index, and topographic position index.

Slope (\texttt{SLP}) and aspect (\texttt{ASP}) were calculated using GRASS module \texttt{r.param.scale}. This module calculates terrain attributes fitting a bivariate quadratic polynomial using least squares \citep{Wood1996}. It allows using different window sizes to fit the bivariate quadratic polynomial, thus including the effect of scale in the calculation of terrain attributes. In the present study, seven window sizes were used (3, 7, 15, 31, 63, 127, and 255) and the results for calculated slope can be seen in Figure \ref{fig:slope}. Larger window sizes result in a smoothed version of the terrain attribute, while smaller windows sizes result in raster maps with more (small-scale) details. Several flat surfaces (slope equal to 0$^\circ$) were produced in the slope raster maps calculated using \texttt{ELEV\_90} as a result of resampling the original DEM from 90 to 5 m. A value of 0.1$^\circ$ was added to the rasters to remove these flat surfaces.

% \begin{figure}[!ht]
%   \centering
%   \includegraphics[width=0.25\textwidth]{fig/SLP_10_3}
%   \includegraphics[width=0.25\textwidth]{fig/SLP_10_7}
%   \includegraphics[width=0.25\textwidth]{fig/SLP_10_15}
%   \includegraphics[width=0.25\textwidth]{fig/SLP_10_31}
%   \includegraphics[width=0.25\textwidth]{fig/SLP_10_63}
%   \includegraphics[width=0.25\textwidth]{fig/SLP_10_127}
%   \includegraphics[width=0.25\textwidth]{fig/SLP_10_255}
%   \caption{Slope \texttt{SLP}} raster maps derived from \texttt{ELEV\_10} using seven window sizes (3, 7, 15, 31, 63, 127, 255) to include the effect of scale in the derived terrain attributes.}
%   \label{fig:slope}
% \end{figure}

Aspect values were also corrected before use. The first correction refers to the fact that \texttt{r.param.scale} stores aspect values in the range 0 to +180 from West to North to East, and 0 to -180 degree from West to South to East, when the standard procedure is to work with aspect values ranging from 0 to 360$^\circ$ clockwise. This correction was done using expressions \texttt{if(asp < 0, aspect + 360, aspect)} and \texttt{if(aspect < 90, aspect + 270, aspect - 90)} in \texttt{r.mapcalc}. Mathematically,

\begin{equation}
  \texttt{ASP}_{beta} =
  \begin{cases}
    aspect + 360^\circ & \text{if}\;\; aspect < 0^\circ, \\
    aspect             & \text{else},
  \end{cases}
\end{equation}

\noindent and

\begin{equation}
  \texttt{ASP} =
  \begin{cases}
    \texttt{ASP}_{beta} + 270^\circ & \text{if}\;\; \texttt{ASP}_{beta} < 90^\circ, \\
    \texttt{ASP}_{beta} - 90^\circ  & \text{else}.
  \end{cases}
\end{equation}

\noindent The second correction involved linearizing aspect values. This is necessary because aspect is a circular variable, that is, the beginning (0$^\circ$) and the end (360$^\circ$) of the measurement scale have the same physical meaning. Aspect values were transformed to northernness (\texttt{NOR}), a measure of the degree of exposition of a given surface to the North, a linear variable, using the equation

\begin{equation}
  \texttt{NOR}_i = abs(180^\circ - \texttt{ASP}_i),
\end{equation}\label{eq:NOR}

\noindent where $i$ is the window size used to calculate \texttt{ASP}, with $i$ = 3, 7, 15, 31, 63, 127, and 255.   

Flow accumulation (\texttt{ACC}), also known as catchment area and contributing area, was calculated using GRASS module \texttt{r.watershed}. The resulting raster map was multiplied by the square of the cell size (5 m). This raster map was used to calculate the topographic wetness index (\texttt{TWI}) and the stream power index (\texttt{SPI}) using the following equations:

\begin{equation}
  A = \dfrac{\texttt{ACC}}{\textit{cell}},
\end{equation}\label{eq:sACC}

\begin{equation}
  \texttt{TWI}_i = log \dfrac{A}{tan(\texttt{SLP}_i)},
\end{equation}\label{eq:TWI}

\noindent and

\begin{equation}
  \texttt{SPI}_i = log(A \times tan(\texttt{SLP}_i)),
\end{equation}\label{eq:SPI}

\noindent where $A$ is the specific catchment area, \textit{cell} is the cell size (5 m), and $i$ is the window size used to calculate \texttt{SLP}, with $i$ = 3, 7, 15, 31, 63, 127, and 255.

The topographic position index \texttt{TPI} was calculated in SAGA GIS using the library \texttt{ta\_morphometry}. Different values of maximum radius were used to include the effect of scale, all of them related to the window sizes used to calculate previous terrain attributes. A minimum radius value of three meters was used in all calculations.

\tocless\subsection{Geological maps} 

Data on geology and soil parent material data comes from most recent geological maps published in the scales of 1:25,000 \citep{MacielFilho1990} and 1:50,000 \citep{GasparettoEtAl1988}.

Both geological maps were produced based on the most recent topographic maps produced by the Brazilian Army at the scales of 1:50,000 and 1:25,000. Alike topographic maps, geological maps were also available only in the analogical format, and were hand digitalized and georeferenced in QGIS. Intersections between all meridians and parallels (a total of 16) were used as control points to adjust a second order polynomial model. Resampling was performed using the cubic resampling method. After manual digitalization of geological formations, the original coordinate reference system (EPSG:31982 - SIRGAS 2000 / UTM zone 22S) of all vector files was transformed to WGS 1984 / UTM zone 22S (EPSG:32722) using the R-package \textit{rgdal} \citep{BivandEtAl2013a}.

The positional validation of geological maps was performed using 8 (\texttt{GEO\_50}) and 5 (\texttt{GEO\_25}) GCPs located at easily identifiable geographical markers. Table \ref{tab:geology-geo-val} shows that the positional accuracy of both geological maps does not meet the current regulations of the Brazilian legislation. Estimated RMSE is 147 m and 69 m, respectively, for \texttt{GEO\_50} and \texttt{GEO\_25}, when the maximum RMSE accepted is 30 m and 15 m. For \texttt{GEO\_50}, the lowest accuracy is found in the y coordinate, while for \texttt{GEO\_25}, the x coordinate is the less accurate. Figure \ref{fig:geology-azim} suggest that the low positional accuracy  of both geological maps is due to a systematic error. This systematic error probably was propagated from the topographic maps used to produce the geological maps. Therefore, the same strategy (affine transformation ) used to remove the systematic positional error of the topographic map above was employed on geological maps. Due to the lack of GCPs, model parameters were adjusted using the same set of GCP's used for the validation in the geographic space. The estimated uncertainty of the affine transformation is RMSE = 86 m and RMSE = 22 m, respectively, for \texttt{GEO\_50} and \texttt{GEO\_25}.

\begin{table}[ht]
  \caption{Estimated error statistics (standard deviation between parenthesis) of the validation of geological maps GEO\_50 and GEO\_25 in the geographic space. Validation statistics were estimated using, respectively, eight and five ground control points located in easily identifiable geographical markers (purposive sampling). Estimates were corrected to the size of the population.}
  \label{tab:geology-geo-val}
  \centering
  {\small
  \begin{tabular}{lrrrr}
    \hline
    Statistics           & X coordinate & Y coordinate  & Error vector  & Azimuth                  \\
    \hline
    \multicolumn{5}{l}{\texttt{GEO\_50} (n = 8)}                                                   \\
    \hline
    Mean, m              & 10   (58)    & -102  (87)    & 140   (44)    & 169$^\circ$ (47$^\circ$) \\ 
    Absolute mean, m     & 43   (40)    & 125   (50)    & -             & -                        \\ 
    Squared mean, m$^2$  & 3431 (5914)  & 18067 (13243) & 21498 (12316) & -                        \\
    \hline
    \multicolumn{5}{l}{\texttt{GEO\_25} (n = 5)}                                                   \\
    \hline
    Mean, m              & 51    (29)   & 29    (22)    & 67    (16)    & 58$^\circ$  (30$^\circ$) \\ 
    Absolute mean, m     & 51    (29)   & 29    (22)    & -             & -                        \\ 
    Squared mean, m$^2$  & 3457  (2976) & 1312  (1612)  & 4769  (2306)  & -                        \\
    \hline
  \end{tabular}}
\end{table}

% \begin{figure}[ht]
%   \centering
%   \includegraphics[width=0.45\textwidth]{fig/azim-geo50}
%   \includegraphics[width=0.45\textwidth]{fig/azim-geo25}
%   \caption{Histogram of the azimuth distribution of the validation of geological maps \texttt{GEO\_50} (left) and \texttt{GEO\_25} (right) in the attribute space. Azimuth values were estimated using, respectively, eight and five GCPs located in easily identifiable geographical markers. Estimates were corrected to the size of the population. The graph was produced using R-package \textit{VecStatGraphs2D}.}
%   \label{fig:geology-azim}
% \end{figure}

\begin{table}[ht]
  \caption{Estimated error statistics of the validation of geological maps \texttt{GEO\_50} and \texttt{GEO\_25} in the attribute space. Validation statistics were estimated using 60 validation points located in 12 linear transects (clustered samples).}
  \label{tab:geology-attr-val}
  \centering
  \begin{tabular}{lrrr}
    \hline
    Geological map        & LCB95Pct & Estimate & UCB95Pct \\
    \hline
    \texttt{GEO\_50}      & 76.88    & 83.33    & 89.78    \\
    \texttt{GEO\_25}      & 70.10    & 76.67    & 83.24    \\
    \hline
  \end{tabular}
\end{table}

Three environmental covariates were derived from \texttt{GEO\_50}:

\begin{description}
  \item[\texttt{GEO\_50a}] This covariate includes the Inferior Sequence of the Serra Geral Formation, which is composed mainly by basic igneous rocks (tholeiitic basalt and andesite). It is likely to be related with high clay content and ECEC;
 
  \item[\texttt{GEO\_50b}] This covariate includes the Superior Sequence of the Serra Geral Formation, which is also composed mainly by acid  igneous rocks (granophyric rhyolite and rhyodacite). It is likely to be related with moderate to high CLAY and ECEC;
 
  \item[\texttt{GEO\_50c}] This covariate includes the Botucatu Formation, which is composed mainly by aeolian sandstones. It is likely to ne related with low CLAY and ECEC;
\end{description}

Four environmental covariates were derived from \texttt{GEO\_25}, five of them having the same meaning of those derived from \texttt{GEO\_50}:

\begin{description}
  \item[\texttt{GEO\_25a}] This covariate includes the Inferior Sequence of the Serra Geral Formation;
 
  \item[\texttt{GEO\_25b}] This covariate includes the Superior Sequence of the Serra Geral Formation;
 
  \item[\texttt{GEO\_25c}] This covariate includes the Botucatu Formation;
 
  \item[\texttt{GEO\_25d}] This covariate includes all Quaternary deposits of fluvial, alluvial, and colluvial origin. It can help explaining the low clay content of soils supposedly derived from igneous rocks and vice-versa.
\end{description}

\tocless\subsection{Land use maps}\label{sec:land}

A land use map for the year of 1980 comes from the most recent topographic map produced by the Brazilian Army (scale of 1:25,000). A second land use map is available for the years of 2008 and 2009 and was published at a scale of 1:30,000 \citep{SamuelRosaEtAl2011a}.

\begin{table}[ht]
\caption{Estimated error statistics (standard deviation between parenthesis) of the validation of Google Earth imagery in the geographic space. Validation statistics were estimated using 14 ground control points located in easily identifiable geographical markers (purposive sampling). Estimates were corrected to the size of the population.}
\label{tab:google-geo-val}
\centering
{\small
\begin{tabular}{lrrrr}
\hline
Statistics           & X coordinate & Y coordinate & Error vector  & Azimuth                   \\
\hline
Mean, m              & -1 (4)       & 3  (7)       & 6  (6)        & 184$^\circ$ (125$^\circ$) \\ 
Absolute mean, m     & 3  (2)       & 5  (6)       & -             & -                         \\ 
Squared mean, m$^2$  & 14 (22)      & 57 (132)     & 71 (153)      & -                         \\ 
\hline
\end{tabular}}
\end{table}

\begin{table}[ht]
\caption{Estimated error statistics of the validation of land use maps LU1980 and LU2009 in the attribute space. Validation statistics were estimated using 60 validation points located in 12 linear transects (clustered samples).}
\label{tab:land-attr-val}
\centering
{\small
\begin{tabular}{lrrr}
\hline
Soil map     & LCB95Pct & Estimate & UCB95Pct \\
\hline
LU1980       & 58.52    & 66.67    & 74.82    \\
LU2009       & 61.16    & 70.00    & 78.84    \\
\hline
\end{tabular}}
\end{table}

Alike area-class soil maps and geological maps, environment covariates derived from both land use maps are indicator variables of individual or grouped land use classes. Two indicator variables were derived from \texttt{LU1980}, and five from \texttt{LU2009}.

\begin{description}
  \item[\texttt{LU1980a}] This covariate includes the areas occupied with native forest, which can be related with high soil organic carbon content and ECEC;
  
  \item[\texttt{LU1980b}] This covariate includes the areas used with animal husbandry, which is expected to have a fertility status lower than native forests;
\end{description}


\begin{description}
  \item[\texttt{LU2009a}] This covariate depicts the areas covered with native forest, which can be related with high soil organic carbon content and ECEC;
  
  \item[\texttt{LU2009b}] This covariate includes the areas covered with shrubland, which is expected to have soil organic carbon content and ECEC level above those found in areas used with crop agriculture, but lower than in native forests;
  
  \item[\texttt{LU2009c}] This covariate includes the areas used with animal husbandry, where the soil can have characteristics similar to shrublands and crop agriculture, depending on the management practices;
  
  \item[\texttt{LU2009d}] This covariate includes the areas used for crop agriculture, which are expected to have the lowest fertility due to the usually poor management practices employed;
  
  \item[\texttt{LUdiff}] This covariate depicts the areas in which there was a change in land use between 1980 and 2009. It can be useful to explain, for example, low organic carbon content in forest soils due to previous use with crop agriculture or animal husbandry.
\end{description}

\tocless\subsection{Orbital images}\label{sec:sat}

Satellite images is one of the sources of environmental covariates most used to build predictive models of soil properties. This is due to their extensive spatial and temporal availability, easiness of acquisition, and generally moderate to strong correlation between derived environmental covariates and soil properties \cite{BishopEtAl2006, Grunwald2009}.

In the present study, two sources of satellite images are used. The first is the longest-operating Earth observation satellite Landsat-5 Thematic Mapper, launched on 1 March 1984. The satellite image used was acquired on 26 December of 2010 and is available at the database of the \href{http://www.dgi.inpe.br/CDSR/}{Division of Image Generation Divisão} of the National Institute for Space Research (INPE). The image contains seven spectral bands \ref{tab:satellites}, including a thermal band (which was not used in this study), with eight bits radiometric resolution (digital numbers from 0 to 255) and approximately 30 meters spatial resolution. Orthorectification was performed using Geomatica\textregistered{} OrthoEngine\textregistered{} with the Landsat rigorous model (Toutin's Model). A set of 28 GCPs were collected manually in Google Earth\textregistered{} due to the absence of field GCPs and the high accuracy of Google Earth\textregistered{} imagery in the region covered by the image (Table \ref{tab:google-geo-val}). GCPs were located at easily identifiable geographical markers (road intersection, bridges, etc), evenly distributed throughout the image and covering a variety of elevations, following standard recommendations \citep{PCIGeomatics2007} (Figure \ref{fig:ortho-gcps}). The DEM used is \texttt{ELEV\_30} described in Section \ref{sec:dem} above with the vertical datum corrected with the EGM 1996 geoidal model. Resampling was done using the nearest neighbor method to avoid changes in the digital numbers.

% TODO: figure with GCPs used to ortorectify orbital images. Show the bounding box of the image and the boundary of the study area.
% \begin{figure}
%   \centering
%   \includegraphics[width=\textwidth]{fig/ortho-gcps}
%   \caption{Ground control points used to orthorectify orbital the image produced by Landsat-5 Thematic Mapper.}
%   \label{fig:ortho-gcps}
% \end{figure}

After orthorectification, all bands were imported into GRASS GIS, where all other necessary corrections were performed. Radiometric correction (conversion from digital numbers to top-of-atmosphere reflectance) was performed using the module \texttt{i.landsat.toar}. Atmospheric correction (removal of the effects of the atmosphere on the reflectance values) was performed using the 6S atmospheric model \citep{VermoteEtAl1997} using the module \texttt{i.atcorr}. The correction was performed using the tropical atmospheric model, the continental aerosols model, an image-based visibility estimate of 20 km, and a fixed elevation of 300 meters. Afterwards, all bands were cropped to the bounding box of the study area and geometrically corrected to match the prediction grid (5 meters pixel size). Registration and geocoding was performed using the nearest neighbor resampling method. Topographic correction (removal of the effects of the topography - illumination - on the reflectance values) was performed using the module \texttt{i.topo.corr} with \texttt{ELEV\_30} geometrically corrected to match the prediction grid.

The second source of orbital images is the RapidEye constellation of five satellites, launched in August 2008. It is available through the Brazilian Ministry of the Environment \citep{Brasil2012}, who has a full coverage of the Brazilian territory with images from the RapidEye satellite constellation for the years of 2011 and 2012. The orbital image used (tile number 2225403) was acquired on 16 November of 2012 (second coverage). It contains five spectral bands \ref{tab:satellites}, featuring among them the so called red edge band, located between the red and the near-infrared bands. This spectral band is the main feature distinguishing RapidEye images from most other sources of orbital images, considered to provide additional information about the vegetation \citep{WeicheltEtAl2013}. The orbital image has 16 bits radiometric resolution and 6.5 meters spatial resolution, and was orthorrectified in the source to 5 meters spatial resolution using the hole-filled SRTM version 4 \citep{RapidEye2013}.

Atmospheric correction was performed using the 6S atmospheric model \citep{VermoteEtAl1997} using the Fortran code developed by Dr. \href{http://lattes.cnpq.br/3818721407909667}{Mauro Antonio Homem Antunes}, from the Rural University of Rio de Janeiro. The GRASS GIS module \texttt{i.atcorr} was not used because a \href{http://osgeo-org.1560.x6.nabble.com/i-atcorr-returns-nan-for-Landsat-5-TM-bands-1-and-2-tt5106456.html#a5118122}{bug} was found when trying to correct images from the RapidEye satellite constellation. The correction was performed using the tropical atmospheric model, the continental aerosols model, an image-based visibility estimate of 20 km, and a fixed elevation of 300 meters. Afterwards, all bands were cropped to the bounding box of the study area and geometrically corrected to match the prediction grid (5 meters pixel size). Registration and geocoding was performed using the nearest neighbor resampling method. Topographic correction was performed using the module \texttt{i.topo.corr} with \texttt{ELEV\_30} geometrically corrected to match the prediction grid (Section \ref{sec:dem}).

\begin{table}[ht]
  \caption{Comparison between satellite images produced by Landsat 5 TM and RapidEye constellation used in the present study and derived environmental covariates.}
  \label{tab:satellites}
  \centering
  {\small
  \begin{tabular}{llllll}
    \hline
    \multicolumn{3}{l}{Landsat 5 TM}                         & \multicolumn{3}{l}{RapidEye}                           \\
    Band                 & Interval, nm & Covariate          & Band               & Interval, nm & Covariate          \\
    \hline
    Band 1 Visible       & 450 - 520    & \texttt{BLUE\_30}  & Blue band          & 440-510      & \texttt{BLUE\_5}   \\
    Band 2 Visible       & 520 - 600    & \texttt{GREEN\_30} & Green band         & 520-590      & \texttt{GREEN\_5}  \\
    Band 3 Visible       & 630 - 690    & \texttt{RED\_30}   & Red band           & 630-685      & \texttt{RED\_5}    \\
    -                    & -            & -                  & Red edge band      & 690-730      & \texttt{EDGE\_5}   \\
    Band 4 Near-Infrared & 760 - 900    & \texttt{NIR\_30a}  & Near-infrared band & 760-850      & \texttt{NIR\_5}    \\
    Band 5 Near-Infrared & 1550 - 1750  & \texttt{NIR\_30b}  & -                  & -            & -                  \\
    Band 7 Mid-Infrared  & 2080 - 2350  & \texttt{MIR\_30}   & -                  & -            & -                  \\
    \hline
  \end{tabular}}
\end{table}

\begin{table}[ht]
  \caption{Estimated error statistics (standard deviation between parenthesis) of the horizontal validation of orbital images produced by Landsat 5 TM and RapidEye constellation. Validation statistics were estimated using 14 GCPs located in easily identifiable geographical markers. Estimates were corrected to the size of the population.}
  \label{tab:satellite-geo-val}
  \centering
  {\small
  \begin{tabular}{lrrrr}
    \hline
    Statistics           & X coordinate & Y coordinate  & Error vector  & Azimuth              \\
    \hline
    \multicolumn{5}{l}{Landsat 5 TM}                                                           \\
    \hline
    Mean, m              & 31   (23)   & -11  (33)   & 45   (26)   & 136$^\circ$ (89$^\circ$)  \\ 
    Absolute mean, m     & 33   (21)   & 25   (25)   & -           & -                         \\ 
    Squared mean, m$^2$  & 1494 (1436) & 1223 (2082) & 2717 (2706) & -                         \\ 
    \hline
    \multicolumn{5}{l}{RapidEye}                                                               \\
    \hline
    Mean, m              & -25  (7)     & -25 (10)   & 36   (8)     & 226$^\circ$ (12$^\circ$) \\ 
    Absolute mean, m     & 25   (7)     & 25  (10)   & -            & -                        \\ 
    Squared mean, m$^2$  & 680  (347)   & 708 (692)  & 1388 (703)   & -                        \\ 
    \hline
  \end{tabular}}
\end{table}

In the present study, the orbital image produced by the RapidEye constellation is considered to be of higher quality than the orbital image produced by the satellite Landsat 5 TM. This is mainly due to its finer resolution and thus larger amount of detail. The two-years difference in the acquisition time between the two satellite images is believed to have only a minor effect on the results since land use changes were not significant in the period and soil observations cover the period from 2008 to 2013.

Each band of the orbital images was used to derive an environmental covariate, totaling six from Landsat 5 TM and five from RapidEye (Table \ref{tab:satellites}). Individual bands were also used to calculate two vegetation indexes: the normalized difference vegetation index (NDVI) and the soil-adjusted vegetation index (SAVI). For Landsat images, NDVI and SAVI were calculated using equations

\begin{equation}
  \texttt{NDVI\_30} = \frac{\texttt{NIR\_30a} - \texttt{RED\_30}}{\texttt{NIR\_30a} + \texttt{RED\_30}}
\end{equation}\label{eq:NDVI30}

\noindent and

\begin{equation}
  \texttt{SAVI\_30} = (1.0 + 0.5) \times \frac{\texttt{NIR\_30a} - \texttt{RED\_30}}{\texttt{NIR\_30a} + \texttt{RED\_30} + 0.5}
\end{equation}\label{eq:SAVI30}

\noindent where \texttt{NIR\_30a} is the first near-infrared band (750 - 900 nm) and \texttt{RED\_30} is the red band (630 - 690 nm). For RapidEye image, NDVI and SAVI were calculated using the standard equations

\begin{equation}
  \texttt{NDVI\_5a} = \frac{\texttt{NIR\_5} - \texttt{RED\_5}}{\texttt{NIR\_5} + \texttt{RED\_5}}
\end{equation}\label{eq:NDVI5a}

\noindent and

\begin{equation}
  \texttt{SAVI\_5a} = (1.0 + 0.5) \times \frac{\texttt{NIR\_5} - \texttt{RED\_5}}{\texttt{NIR\_5} + \texttt{RED\_5} + 0.5}
\end{equation}\label{eq:SAVI5a}

\noindent with the red (630 - 685 nm) (\texttt{RED\_5}) and near-infrared (760 - 850 nm) (\texttt{NIR\_5}), and also using the red-edge band (690 - 730 nm) (\texttt{EDGE\_5}) instead of the near-infrared band as follows:

\begin{equation}
  \texttt{NDVI\_5b} = \frac{\texttt{EDGE\_5} - \texttt{RED\_5}}{\texttt{EDGE\_5} + \texttt{RED\_5}}
\end{equation}\label{eq:NDVI5a}

\noindent and

\begin{equation}
  \texttt{SAVI\_5b} = (1.0 + 0.5) \times \frac{\texttt{EDGE\_5} - \texttt{RED\_5}}{\texttt{EDGE\_5} + \texttt{RED\_5} + 0.5}
\end{equation}\label{eq:SAVI5a}

The final number of environmental covariates derived from orbital images is eight, for the Landsat 5 TM, and nine, for the RapidEye constellation.


% End!
 % The Santa Maria dataset -- soil data
% % ATTENTION: lines starting with “%”  will not be included in the document!

\artigofalse
\chapter{The Santa Maria dataset -- covariate data}
\label{apen:covar-data}
\usepackage[utf8]{inputenc}


\tocless\section{Introduction}
\label{sec:covar-data-intro}

The Santa Maria dataset is a data set comprising spatially exhaustive covariate data produced in the 1980's, 
1990's, and 2000's covering the catchment of the DNOS/CORSAN reservoir, located in the southern border of the 
plateau of the Paraná Sedimentary Basin, in the city of Santa Maria, state of Rio Grande do Sul, Brazil. Some 
of these covariate data cover only part of the catchment, mainly the northern sector, which has an area of 
\SI{\pm2000}{\hectare}, corresponding to \SI{\pm60}{\percent} of the entire catchment. These covariate data 
are the outcome of projects that aimed at modelling various environmental features and were carried out as 
part of local (soil, geology, land use), regional (terrain, land use), and global (terrain, land use) 
initiatives.

This document presents a description of the covariate data contained in the Santa Maria dataset, including the 
procedures for their production, as well as the processing methods employed. The original sources of the 
covariate data are freely available at the producers databases or in public libraries.

% TODO: include a description of the GCPs used to validate the environmental covariates and to 
% orthorectify satellite images. Also, present the equations used to calculate validation statistics.

\tocless\section{Ground control points}
\label{sec:covar-data-gcp}

All sources of covariates were validated prior to their use. Horizontal (positional) validation was performed 
using a set of $n = 14$ validation points, here called ground control points (GCP), spread throughout and 
beyond the limits of the study area (\autoref{fig:covar-data-field-gcps}). The location of the GCPs was 
defined based on the existence of easily identifiable geographical markers autocross the sources of 
covariates, including road intersection, fence corners, and property entrances.

\begin{figure}[!ht]
 \centering
\begin{Schunk}
\begin{Soutput}
         used (Mb) gc trigger (Mb) max used (Mb)
Ncells 586009 31.3     940480 50.3   885505 47.3
Vcells 760374  5.9    1650153 12.6  1308379 10.0
\end{Soutput}
\end{Schunk}
\includegraphics{fig/apene-covar-data-field-gcps}
\caption{Spatial distribution of the ground control points ($n = 14$, red) used to validate the sources of the
covariates included in the Santa Maria dataset.}
\label{fig:covar-data-field-gcps}
\end{figure}





This function estimates the difference, absolute difference, and squared difference on x, y and z coordinates 
of two sets of ground control points (GCP). It also estimates the module (difference vector), its square and 
azimuth. The result is a data frame ready to be used to define a object of class spsurvey.object.

Horizontal (positional) validation was performed comparing the position of measured point data (ground control 
point, GCP) with the position of predicted point data. Horizontal displacement (error) is measured in both x- 
and y-coordinates, and is used to calculate the error vector (module) and its azimuth.

Vertical validation exercises are interested in comparing the measured value of a variable at a given location 
with that predicted by some model. In this case, error statistics are calculated only for the the vertical 
displacement (error) in the ‘z’ coordinate. Both objects measured and predicted used with function gcpDiff() 
must be of class SpatialPointsDataFrame. They also must have a column named ‘siteID’ giving the identification 
of evary case. Again, matching of case IDs is mandatory. However, both objects must have a column named ‘z’ 
which contains the values of the ‘z’ coordinate. Other columns are discarded.


% 
% \begin{itemize}
% \item Describe how the ground control points (GCPs) were selected;
% \item Present a figure with the spatial distribution of the GCPs;
% \item Describe how the location of the GCPs was described - we could 
%       add a few images to use as example;
% \item Describe how the data was processed.
% \item Include link to the dataset;
% \end{itemize}
% 
% \begin{description}
% \item [GCP 01] Em Santa Maria, no lado direito da barragem de concreto do reservatório do 
% DNOS/CORSAN, seis metros antes de chegar à ponte sobre o vertedouro, há três metros de distância do 
% centro da estrada que desce da rodovia federal BR 158. Coletado em 03 de janeiro de 2013 por 
% Alessandro Samuel Rosa e Jean Michel Moura Bueno.
% 
% \item [GCP 02] Na entrada da Estrada do Perau, Rua Gralha Azul, em Itaara, no centro do canteiro, 
% entre o outdoor e a árvore (Cedrella fissilis), junto à rodovia federal BR 158.
% 
% \item [GCP 03] Na entrada de Itaara, próximo ao equipamento estático de fiscalização eletrônica de 
% velocidade, na rodovia federal BR 158, à 520 metros do Museu de Ufologia, em frente à Fruteira da 
% Esquina. Do outro lado da rodovia há uma torre, possivelmente de telefonia celular, e entrada para a 
% mina de extração de brita DallaPasqua, localizada à 4 km do local.
% 
% \item [GCP 04] Em Santa Maria, na entrada da estrada de acesso ao cemitério do Bairro Campestre do 
% Menino Deus, do lado direito da Estrada do Perau (subindo), alinhado (erro de 50 cm) com a frente 
% das casas, há 2,2 metros do muro, há 6,5 metros do meio-fio da Esrada do Perau, há 50 metros da 
% ponte sobre o Rio Vacacaí-Mirim.
% 
% \item [GCP 05] Na entrada do Rancho do Amaral, junto à porteira, no lado direito, fora da estrada, 
% distante um metro de uma palmeira e um metro do muro.
% 
% \item [GCP 06] Em Itaara, na Avenida Etelvina, na beira da estrada, aproximadamente 2,5 metros do 
% centro da estrada, alinhado (acurácia de 20 cm) com a cerca que separa a floresta nativa do pomar de 
% citrus localizado do outro lado da estrada. Medir a distância em relação à entrada na rodovia 
% federal BR 158.
% 
% \item [GCP 07] Em Itaara, no lado esquerdo da entrada da estrada que da acesso à mina de extração de 
% brita DallaPasqua. Localidade de Estação Pinhal. Obs,: sob a rede de transmissão de eletricidade → 
% verificar efeito.
% 
% \item [GCP 08] No 10º Distrito de Santo Antão, na entrada da estrada para a Central de Tratamento de 
% Resíduos da Caturita, em frente à Escola Municipal de Ensino Fundamental Intendente Manoel Ribas.
% 
% \item [GCP 09] Em São Martinho da Serra, na localidade de Água Negra, na bifurcação da estrada que 
% vem de Santa Maria e que dá acesso à localidade de Campinas, junto à parada de ônibus, no canteiro 
% no meio da bifurcação, à 40 metros distante do Piquete Laçador Jorge R. da Silva, em frente ao 
% Mercado do Ronaldo. Obs.: coletado em 15 de janeiro de 2013.
% 
% \item [GCP 10] Na estrada de Santa Maria para São Martinho da Serra, em uma curva, no lado externo, 
% próximo a duas pequenas árvores, alinhado (acurácia de 1 m) com cerca que marca a divisa entre duas 
% propriedades com campo nativo. Determina a distância, em km, em relação a ponto de referência em 
% Santa Maria. Em frente à propriedade com duas casas, uma delas com dois andares, e quatro pequenos 
% lagos nos fundos. Depois da capela Santo Antão.
% 
% \item [GCP 11] Em Itaara, na entrada da estrada que dá acesso à Brita Pinhal, junto à rodovia 
% federal BR 185, ao lado do corte no terreno expondo a rocha de arenito da Formação Botucatu, 
% distante 5 metros, aproximadamente 15 metros distante do poste da linha de transmissão de 
% eletricidade da companhia AES, na entrada para a localidade de Rincão dos Minello.
% 
% \item [GCP 12] Em Itaara, em frente ao lago da SOCEPE, na entrada da cidade, junto à rodovia federal 
% BR 158, próximo ao Bar e Armazém Ricardo, deslocado em 1 metro para dentro do passeio em relação ao 
% alinhamento dos postes da rede elétrica.
% 
% \item [GCP 13] Em Itaara, na Vila Etelvina, alinhado com a cerca (acurácia de 30 cm) que divide duas 
% terras, à esquerda coberta com floresta nativa/exótica, à direita ocupada com lavoura de culturas 
% anuais. O ponto está locado no lado oposto. Determinar a distância até algum ponto de referência. 
% Plantação de videiras logo acima, no divisor de águas.
% 
% \item [GCP 14] Em Itaara, na estrada que sobe para a propriedade do Sr. Antoninho Luccas, logo após 
% o término da subida íngreme com calçamento apenas nos trilhos, no final da floresta e início do 
% campo nativo, alinhado (acurácia de 20 cm) com a cerca dos dois lados da estrada. Locado 1 metro 
% distante do moirão da cerca do lado esquerdo subindo, no interior da estrada.
% \end{description}
% 
\tocless\section{Area-class soil maps}
\label{sec:covar-data-soil-maps}

There are several area-class soil maps available for the study area, but only two are included in the Santa 
Maria dataset. The first of them (\soilOld{}) was published at a \scale{100000} \cite{AzolinEtAl1988}. 
Existing area-class soil maps and technical reports \cite{Brasil1973, Azolin1977, MacielEtAl1987a, 
MacielEtAl1987, AbraoEtAl1988}, and sparse field observations were used to elaborate the preliminary legend of 
the soil map. Aerial photographs (\scale{60000}) were used to produce the first draft of the soil map. Field 
checks of soil polygons was done along the road network (i.e. convenience sampling). These observations were 
used to estimate the composition (occurrence and spatial distribution of soil taxa) of soil mapping units. They 
were also used to review the first draft of the soil map. The final version of \soilOld{} was prepared using 
topographic maps originally published at a \scale{50000} and resampled to a \scale{100000}. Soil classification 
followed the criteria adopted by the Brazilian soil science community at that time \cite{Brasil1973, 
CamargoEtAl1982, Carvalho1982, LemosEtAl1982, OlmosEtAl1982}. Identification of soil taxa was performed based 
on morphological features, analytical data compiled from existing technical reports, and analysis of soil 
samples collected from soil profiles observed along the road network. Description of each soil mapping unit 
includes the estimated area (\si{\hectare}) and the approximate taxonomic composition (\si{\percent}). 
Validation statistics are absent in the survey report.

The second area-class soil map (\soilNew{}) included in the Santa Maria dataset \cite{Miguel2010} was prepared 
at a \scale{25000}. Orbital images produced by Digital Globe\textregistered{} (Quick Bird satellite), freely 
available for visualization in Google Earth\textregistered{}, were used to produce the first draft of the soil 
map. Existing area-class soil maps and technical reports \cite{Pedron2005, Poelking2007, Sturmer2008} were used 
to help defining the preliminary soil map legend. Punctual field observations (auger holes) were made in more 
than \num{350} locations using a purposive sampling approach. These observations helped to identify six 
modal (representative) soil profiles. Soil sampling and description of modal soil profiles, and laboratory 
analyses of soil samples, followed the standard protocols adopted in Brazil \cite{ClaessenEtAl1997, 
SantosEtAl2005}. Soil classification was done following the criteria of the Brazilian System of Soil 
Classification \cite{SantosEtAl2006}. The final version of the map was prepared using orbital images freely 
available for visualization in Google Earth\textregistered{} and manually-digitalized topographic maps 
published at a \scale{25000} \cite{DSG1992a, DSG1992}. Description of soil mapping units includes only the 
most common soil taxon, followed by morphological and laboratory data of modal soil profiles. Alike \soilOld{} 
described above, validation statistics are absent in the survey report of \soilNew{}.

Both area-class soil maps went through different preprocessing routines. The original \soilOld{} is available 
only in the analogical format, what required its digitalization. Georeferencing was carried out using the GDAL 
Georeferencer plug-in in QGIS \cite{GDAL2013, QGIS2013}. Intersections between all meridians and parallels (a 
total of nine) were used as control points to adjust a second order polynomial model. Resampling was performed 
using the cubic resampling method. Soil polygons and their attributes were also manually digitalized in QGIS. 
Because of the coarseness on the cartographic map scale, most geographical markers used to locate validation 
GCPs could not be identified and positional validation was performed using only four GCPs. Estimated error 
statistics suggest that there are large positional errors in all directions, with an $\text{RMSE} = 
\SI{114}{\m}$ and a mean azimuth of \SI{128}{\degree} (\autoref{tab:covar-data-soil-geo-val}).

Error statistics include the mean error (ME, \si{\m}), mean absolute error (MAE, \si{m}), and mean squared 
error (MSE, \si{\m\square}), and their respective standard deviation, in the x- and y-coordinates. 

\begin{table}[ht]
 \caption{Validation of the area-class soil map \soilOld{} in the geographic space using $n = 4$ ground 
 control points. }
 \label{tab:covar-data-soil-geo-val}
 \centering
 {\small
 \begin{tabular}{lrrrr}
  \hline
  Statistics           & X coordinate & Y coordinate & Error vector & Azimuth                  \\
  \hline
  Mean, m              & 30   (79)    & -36  (67)    & 105   (43)   & 128$^\circ$ (80$^\circ$) \\ 
  Absolute mean, m     & 58   (63)    & 64   (40)    & -            & -                        \\ 
  Squared mean, m$^2$  & 7241 (11353) & 5712 (6197)  & 12953 (9613) & -                        \\ 
  \hline
 \end{tabular}}
\end{table}

% \begin{figure}[!ht]
%   \centering
%   \includegraphics[width=0.45\textwidth]{azim-soil100}
%   \includegraphics[width=0.45\textwidth]{azim-soil25}
%   \caption{Histogram of the azimuth distribution of the validation of area-class soil maps \texttt{SOIL\_100} 
% and \texttt{SOIL\_25} in the attribute space. Azimuth values were estimated using, respectively, four and ... 
% GCPs located in easily identifiable geographical markers. Estimates were corrected to the size of the 
% population. The graph was produced using R-package \textit{VecStatGraphs2D}.}
%   \label{fig:soil-azim}
% \end{figure}

The original \texttt{SOIL\_25} is available in digital format in the personal database of the author 
\cite{Miguel2010}. A topology check (Topology Checker plug-in in QGIS 2.0.1) identified that there
were many gaps and overlaps between polygons. This required a topological edition prior to the 
use of \texttt{SOIL\_25}. There also was a mismatch between the boundary of \texttt{SOIL\_25} and 
the actual boundary of the catchment of the DNOS/CORSAN reservoir as estimated using 
\texttt{ELEV\_10} (\autoref{sec:covar-data-dem}). This occurred because the database used to 
produce \texttt{SOIL\_25} included Google Earth imagery\textregistered{} and topographic maps, which 
are data sources that differ considerably in their positional accuracy 
(\autoref{sec:covar-data-dem} and \autoref{sec:covar-data-land-use}). To avoid data losses, all 
boundary gaps were manually filled using the closest mapping unit. Boundaries of soil polygons were 
defined based on land use (\texttt{LU2009}, \autoref{sec:covar-data-land-use}) and topographic data 
(contour lines, \autoref{sec:covar-data-dem}) as it was done for the original map \cite{Miguel2010}. 
New delineations were checked and approved without modifications by the author of the original map. 
Because \texttt{SOIL\_25} includes very few geographical markers, its positional validation was not 
possible with the available GCPs. However, the RMSE is expected to vary between 
\SIrange{8}{114}{\metre} across the soil map as a result of the different errors present in the 
data sources used in its production.

Both \texttt{SOIL\_100} and \texttt{SOIL\_25} were imported into GRASS GIS 6.4, cropped to the 
bounding box of the catchment of the DNOS/CORSAN reservoir, and geometrically corrected to match the 
prediction grid (\SI{5}{\metre} grid size). Registration and geocoding was performed using the 
nearest neighbour resampling method. Each category was named with the code of the respective 
mapping unit in the original map. Prior to validation in the attribute space, class codes of 
\texttt{SOIL\_100} were changed to match soil taxa codes of the current Brazilian System of Soil 
Classification using a standard correlation table \cite{SantosEtAl2006}.

Table \ref{tab:covar-data-soil-attr-val} shows that the overall purity of both soil maps is not 
significantly different. The main reason for this is that validation was performed considering only 
the second level of the Brazilian System of Soil Classification. It is expected that 
\texttt{SOIL\_25} would outperform \texttt{SOIL\_100} if validation data included soil 
classification up to the fourth level of the Brazilian System of Soil Classification. Estimated 
overall purity values are also very low (\SI{<35}{\percent}). The main reason can be the fact that 
very few soil profiles were described and sampled to produce both maps. There also are two minor 
potential sources of error. First, because \texttt{SOIL\_100} does not include analytical soil data 
in the survey report, all soil taxa had to be translated to the current Brazilian System of Soil 
Classification based only on a standard correlation table \cite{SantosEtAl2006} and expert 
knowledge. Second, soil taxa described at the validation points was obtained analysing only 
morphological soil properties and the basis and concepts of the Brazilian System of Soil 
Classification.

\begin{table}[ht]
 \caption{Estimated error statistics of the validation of area-class soil maps \texttt{SOIL\_100}
 and \texttt{SOIL\_25} in the attribute space. Validation statistics were estimated using $n = 60$ 
 observation locations placed along $m = 12$ linear transects (clustered samples).}
 \label{tab:covar-data-soil-attr-val}
 \centering
 {\small
 \begin{tabular}{lrrr}
  \hline
  Soil map              & LCB95Pct & Estimate & UCB95Pct \\
  \hline
  \texttt{SOIL\_100}    & 21.69    & 31.67    & 41.65    \\
  \texttt{SOIL\_25}     & 20.81    & 30.00    & 39.19    \\
  \hline
 \end{tabular}}
\end{table}

% TODO: figure with both area-class soil maps
% \begin{figure}[!ht]
%   \centering
%   \includegraphics[width=0.3\textwidth]{fig/soil-100}
%   \includegraphics[width=0.3\textwidth]{fig/soil-25}
%   \caption{Area-class soil maps used as sources of environmental co-variates. On the left, the area-class 
% soil map produced by \cite{AzolinEtAl1988} and published at a scale of 1:100,000 (\texttt{SOIL\_100}). On the 
% right, the area-class soil map produced by \cite{Miguel2010} at a scale of 1:25,000 (\texttt{SOIL\_25}).}
%   \label{fig:soil-maps}
% \end{figure}

The main advantage of \texttt{SOIL\_25} in relation to \texttt{SOIL\_100} is the level of detail. 
While \texttt{SOIL\_100} has only five mapping units covering catchment of the DNOS/CORSAN 
reservoir, \texttt{SOIL\_25} has seven mapping units. This enabled the derivation of six 
covariates from \texttt{SOIL\_100} and ten covariates from \texttt{SOIL\_25}. Covariates derived 
from \texttt{SOIL\_100} are the following:

\begin{description}
%  \item[\texttt{SOIL\_100a}] This covariate separates map unit Rd1 from other map units. It is 
%  composed mainly by shallow soils with low to high base saturation (Solo Litólico 
%  distrófico/eutrófico; Neossolo Litólico distrófico/eutrófico; Distric/Eutric Leptosol) located in 
%  slopping terrain;
  
 \item[\texttt{SOIL\_100b}] Shallow soils with low to high base saturation covering mountainous 
 terrain (Solo Litólico Eutrófico/Distrófico relevo montanhoso; Neossolo Litólico 
 Distrófico/Eutrófico; Distric/Eutric Leptosol);
  
 \item[\texttt{SOIL\_100c}] Shallow soils with high base saturation located in strongly sloping 
 terrain (Solo Litólico Eutrófico relevo forte ondulado; Neossolo Litólico Eutrófico; Eutric 
 Leptosol), low weathered soils (Cambissolo Eutrófico; Cambissolo Háplico Eutrófico; Eutric 
 Cambisol), and colluvial deposits;
  
 \item[\texttt{SOIL\_100d}] Deep, well-structured, low base saturation soils (Terra Bruna 
 Estruturada álica; Nitossolo; Nitisol), and shallow soils (Solo Litólico; Neossolo Litólico; 
 Leptosol);
  
 \item[\texttt{SOIL\_100e}] \textit{Rd1} is composed mainly by shallow soils with low to high base 
 saturation (Solo Litólico Distrófico/Eutrófico; Neossolo Litólico Distrófico/Eutrófico; 
 Distric/Eutric Leptosol) located in slopping terrain. This dummy predictor variable is composed by 
 shallow soils in both sloping and mountainous terrain;
  
 \item[\texttt{SOIL\_100f}] \textit{C1} is composed by low weathered soils developed in lower 
 landscape positions, close to drainage channels (Cambissolo Eutrófico; Cambissolo Eutrófico; 
 Eutric Cambisol). This dummy predictor variable includes the best soil mapping units for crop 
 agriculture among those identified in the soil survey.
\end{description}

% TODO: figure with covariates derived from SOIL_100
% \begin{figure}[!ht]
%   \centering
%   \includegraphics[width=0.3\textwidth]{fig/soil-100a}
%   \includegraphics[width=0.3\textwidth]{fig/soil-100b}
%   \includegraphics[width=0.3\textwidth]{fig/soil-100c}
%   \includegraphics[width=0.3\textwidth]{fig/soil-100d}
%   \includegraphics[width=0.3\textwidth]{fig/soil-100e}
%   \includegraphics[width=0.3\textwidth]{fig/soil-100f}
%   \caption{Environmental covariates derived from the area-class soil map produced by \cite{AzolinEtAl1988} 
% and published at a scale of 1:100,000 (\texttt{SOIL\_100}).}
%   \label{fig:soil100-covars}
% \end{figure}

Covariates derived from \texttt{SOIL\_25} are the following:

\begin{description}
 \item[\texttt{SOIL\_25a}] Moderately deep soils derived from sedimentary rocks, with abrupt 
 textural change and low base saturation (Argissolo Bruno-Acinzentado; Alisol);

 \item[\texttt{SOIL\_25b}] Deep soils derived from igneous rocks, with moderate textural gradient, 
 and low base saturation (Argissolo Vermelho; Acrisol);
 
 \item[\texttt{SOIL\_25c}] Low weathered soils (Cambissolo; Cambisol) and shallow soils with low to 
 high base saturation (Neossolo Litólico/Regolítico Eutrófico/Distrófico; Eutric/Distric 
 Leptosol/Regosol);
 
 \item[\texttt{SOIL\_25d}] Shallow soils with low to high base saturation (Neossolo Litólico 
 Eutrófico/Distrófico; Eutric/Distric Leptosol);
 
%  \item[\texttt{SOIL\_25e}] This covariate separates map unit RL-RR from other map units. It is 
%  composed mainly by shallow soils (Neossolo Litólico + Neossolo Regolítico; Leptosol + Regosol) 
%  with low to high base saturation;
 
%  \item[\texttt{SOIL\_25f}] This covariate separates map unit RR from other map units. It is composed 
%  mainly by shallow soils (Neossolo Regolítico; Regosol), with low base saturation, developed on 
%  sedimentary rocks;
 
%  \item[\texttt{SOIL\_25g}] This covariate separates map unit RY from other map units. It is composed 
%  mainly by soils developed from fluvial deposits (Neossolo Flúvico; Fluvisol);
 
 \item[\texttt{SOIL\_25h}] \textit{SX} is composed by moderately deep soils derived from sedimentary 
 rocks, with abrupt textural change, low base saturation, and which are saturated with water for 
 long periods of the year (Planossolo Háplico; Planosol). This dummy predictor variable includes 
 the best soil mapping units for crop agriculture among those identified in the soil survey;
 
 \item[\texttt{SOIL\_25i}] This dummy predictor variable includes all three mapping units composed 
 mainly by shallow soils (Neossolo Litólico and Neossolo Regolítico; Leptosol and Regosol);
  
 \item[\texttt{SOIL\_25j}] This dummy predictor variable includes all four mapping units composed 
 mainly by soils derived from igneous rocks.
\end{description}

% TODO: figure with covariates derived from SOIL_25
% \begin{figure}[!ht]
%   \centering
%   \includegraphics[width=0.3\textwidth]{fig/soil-25a}
%   \includegraphics[width=0.3\textwidth]{fig/soil-25b}
%   \includegraphics[width=0.3\textwidth]{fig/soil-25c}
%   \includegraphics[width=0.3\textwidth]{fig/soil-25d}
%   \includegraphics[width=0.3\textwidth]{fig/soil-25e}
%   \includegraphics[width=0.3\textwidth]{fig/soil-25f}
%   \includegraphics[width=0.3\textwidth]{fig/soil-25g}
%   \includegraphics[width=0.3\textwidth]{fig/soil-25h}
%   \includegraphics[width=0.3\textwidth]{fig/soil-25i}
%   \includegraphics[width=0.3\textwidth]{fig/soil-25j}
%   \caption{Environmental covariates derived from the area-class soil map produced by \cite{Miguel2010} at a 
% scale of 1:25,000 (\texttt{SOIL\_25}).}
%   \label{fig:soil25-covars}
% \end{figure}

\tocless\section{Digital elevation models}
\label{sec:covar-data-dem}

Three DEMs are include in the Santa Maria dataset as sources of covariates. The first DEM 
(\texttt{ELEV\_10}) is the result of the interpolation of the contour lines of the most recent 
topographic maps produced by the Brazilian Army (\scale{25000}) \cite{DSG1980, DSG1992, DSG1992a}. 
Because all three topographic maps needed to cover the study area are available only in the 
analogical format, their digitalization was necessary. Georeferencing was carried out using the GDAL 
Georeferencer plug-in in QGIS \cite{GDAL2013, QGIS2013}. Intersections between all meridians and 
parallels (about \num{160} per topographic map) were used as control points to adjust a third order 
polynomial model. Resampling was performed using the cubic resampling method. All contour lines, 
peaks, lakes and rivers, and their respective attributes within a distance of \SI{1000}{\metre} 
from the boundary of the study area were also manually digitalized and stored in the vector format. 
After digitalization, the original coordinate reference system (EPSG:31982 -- SIRGAS2000 / UTM 
zone 22S) of all vector files was transformed to WGS1984 / UTM zone 22S (EPSG:32722) using the 
\Rpackage{rgdal} \cite{BivandEtAl2013a}.

The positional validation of topographic maps was performed using \num{14} GCPs located at easily 
identifiable geographical markers. According to Brazilian legislation, the positional accuracy of 
these topographic maps is expected to be of, at least, \SI{15}{\metre} \cite{Brasil1984}. Estimated 
validation statistics show that the observed $\text{RMSE} = \SI{65}{\m}$ is larger than 
established by current regulations (\autoref{tab:covar-data-topomap-geo-val}). The mean error 
vector (module) is larger than \SI{60}{\metre} with an azimuth of \SI{63}{\degree}. Both x- and 
y-coordinates are positively biased, but the largest error occurs in the x-coordinate 
(\SI{50}{\metre}). Similar mean and mean absolute errors suggest that there is a systematic 
positional error. An affine transformation was employed using the \Rpackage{vec2dtransf} 
\cite{Carrillo2012} to eliminate this systematic error. Model parameters were adjusted using the 
same set of GCPs used for the validation in the geographic space.

\begin{table}[ht]
 \caption{Estimated error statistics (standard deviation between parenthesis) of the validation of 
 topographic maps (\scale{25000}) in the geographic space. Validation statistics were 
 estimated using \num{14} GCPs located in easily identifiable geographical markers.}
 \label{tab:tab:covar-data-topomap-geo-val}
 \centering
 {\small
 \begin{tabular}{lrrrr}
  \hline
  Statistics & X coordinate & Y coordinate & Error vector & Azimuth \\
  \hline
  Mean, \si{\metre} & 50 (25) & 27 (22) & 63 (19) & \SI{63}{\degree} (\SI{30}{\degree}) \\ 
  Absolute mean, \si{\metre} & 50 (25) & 32 (13) & - & - \\ 
  Squared mean, \si{\metre\squared} & 3088 (3034) & 1180 (820) & 4268 (2825) & - \\ 
  \hline
 \end{tabular}}
\end{table}

% \begin{figure}[!ht]
%   \centering
%   \includegraphics[width=0.5\textwidth]{azim-car25}
%   \caption{Histogram of the azimuth distribution of the validation of topographic maps in the attribute 
% space. Azimuth values were estimated using 14 GCPs located in easily identifiable geographical markers. 
% Estimates were corrected to the size of the population. The graph was produced using R-package 
% \textit{VecStatGraphs2D}.}
%   \label{fig:topomap-azim}
% \end{figure}

Interpolation of the raster surface with \SI{5}{\metre} pixel size was performed using the function 
\texttt{Topo to Raster} in ArcGIS\textregistered{} software by ESRI, which includes an interpolation 
method based on the ANUDEM program developed by \citeonline{Hutchinson1989}. Vector files of contour 
lines (\texttt{multiline}), drainage network (\texttt{multiline}), lakes (\texttt{polygons}) and 
peaks (\texttt{points}) were used to generate an hydrologically correct DEM, that is, a DEM without 
spurious depressions and giving an accurate representation of the real hydrology. Next, the 
interpolated DEM was imported into GRASS GIS \cite{GRASS2012}, where a neighbourhood average filter 
was used to remove stair-like artefacts. A window of $7 \times 7$ pixels was used because it 
removed a significant amount of the artefacts and did not affect the derived boundary of the study 
area (see more bellow).

The vertical datum of the DEM was transformed from the local datum to a global datum. The geoidal 
models MAPGEO2010 \cite{IBGE2010a} and EGM1996 \cite{LemoineEtAl1998} were used to calculate the 
geoidal undulation for the local and global datums, respectively. MAPGEO2010 is optimized to 
estimate geoidal undulations in the Brazilian territory, while EGM1996 is a gravitational model of 
the Earth and is used as the vertical datum for SRTM products. The following equation was used:

\begin{equation}
 h = H + N,
\end{equation}

\noindent where $h$ is the ellipsoidal height (height above the reference ellipsoid that 
approximates the surface of the planet), $H$ is the orthometric height (height above the imaginary 
surface called geoid and commonly referred as mean sea level), and $N$ is the geoidal undulation. 
Ellipsoidal heights estimated by MAPGEO2010 are referenced to the world ellipsoid of 1980, while 
EGM1996 estimates ellipsoidal heights referenced to the world ellipsoid of 1984. Because the 
difference between both ellipsoids is of the order of millimetres, it can be assumed that both 
models estimate the same ellipsoidal height. Therefore, if 
$h_{\text{EGM1996}} = h_{\text{MAPGEO2010}}$, then orthometric heights referenced to the local 
vertical datum can be transformed to the global vertical datum using the following equation:

\begin{equation}
 H_{\text{EGM1996}} = H_{\text{MAPGEO2010}} + N_{\text{MAPGEO2010}} - N_{\text{EGM1996}}.
\end{equation}\label{eqn:geoidal}

The difference in the geoidal undulation estimated by both models is of about one meter in the 
entire study area. Thus, transforming the vertical datum was done adding one meter to the raster 
surface interpolated from contour lines, yielding the first DEM included in the Santa Maria dataset
(\texttt{ELEV\_10}).

The second DEM (\texttt{ELEV\_90}) used in this study is the well known SRTM DEM 
(\SI{3}{\arcsecond} $\approx$ \SI{90}{\metre} spatial resolution) produced by NASA’s Jet 
Propulsion Laboratory in collaboration with the National Geospatial-Intelligence Agency 
\cite{RodriguezEtAl2006}. The SRTM DEM version used here is the \emph{hole-filled SRTM version 
\num{4}}, prepared by \href{http://www.cgiar.org/}{CGIAR} using the same hydrologically correct 
interpolation method that was used above to produce \texttt{ELEV\_10} \cite{ReuterEtAl2007, 
JarvisEtAl2008}. However, the only data source used was the original SRTM DEM converted to point 
data.

Prior to processing, the SRTM DEM was cropped to the extent of the study area and the coordinate 
reference system was transformed from WGS1984 (EPSG:4326) to WGS1984 / UTM zone 22S (EPSG:32722) 
using cubic resampling in GDAL (module \texttt{gdalwarp}). This resampling method was used because 
it is efficient in minimizing the double-oblique stripping present in SRTM products 
\cite{Samuel-RosaEtAl2013c}. Next, the DEM was resampled to \SI{15}{\metre} 
(\grass{r.resamp.interp}) using cubic resampling. Sinks produced during the datum transformation 
were filled using the \grass{r.fill.dir}. Vertical datum transformation was not necessary because 
elevation values of the SRTM DEM already are referenced to the global geoidal model EGM1996 
(orthometric heights).

The third DEM (\texttt{ELEV\_30}) used in this study was produced by the Brazilian National 
Institute for Space Research (\href{http://www.inpe.br/}{INPE}). This DEM is the result of refining 
the original SRTM DEM to \SI{1}{\arcsecond} spatial resolution (\SI{\pm30}{\metre}) using ordinary 
kriging with a Gaussian model of spatial covariance \cite{ValerianoEtAl2012}. Different from 
\texttt{ELEV\_90}, \texttt{ELEV\_30} was not used to calculate DEM derivatives. Instead it was used 
in the orthorectification and topographic correction of satellite images (\autoref{sec:covar-data-sat-image}).

Eight tiles were downloaded from the \href{http://www.dsr.inpe.br/topodata/}{TOPODATA} website, 
imported into QGIS and mosaicked using GDAL module \texttt{gdal\_translate}. The coordinate 
reference system was transformed from WGS1984 (EPSG:4326) to WGS1984 / UTM zone 22S (EPSG:32722) 
using cubic resampling (GDAL module \texttt{gdalwarp}). Again, this resampling method was used 
because it is efficient in minimizing the double-oblique stripping present in SRTM products 
\cite{Samuel-RosaEtAl2013c}. Sinks produced during the datum transformation were filled using 
\grass{r.fill.dir} implemented in the SEXTANTE library \cite{SEXTANTE2012}.

Because orbital satellites use the WGS1984 ellipsoid as vertical datum, orthorectification of 
satellite images has to be done using a DEM with ellipsoidal heights. Conversion from orthometric 
heights was performed using \autoref{eqn:geoidal}, with geoidal undulation calculated with 
the gravitational model EGM1996. The original DEM with orthometric heights was cropped to the 
boundary of the study area and resampled to five meters using \grass{r.resamp.interp} with the 
bicubic resampling method. This DEM was used only to estimated error statistics for the validation 
in the attribute space.

\autoref{tab:covar-data-dem-attr-val} shows that the three DEMs present similar accuracy 
estimates in the attribute space ($\text{RMSE} \approx \SI{19}{\m}$). In the case of the ELEV\_10, which 
was derived from contour lines published at a \scale{25000}, the estimated accuracy does not 
meet current Brazilian legislation, which states that the accuracy should be of, at least, 
\SI{5}{\metre} (\num{1/2} of the distance between contour lines) \cite{Brasil1984}.
 
\begin{table}[ht]
 \caption{Estimated error statistics (standard deviation between parenthesis) of the validation of 
 digital elevation models \texttt{ELEV\_90}, \texttt{ELEV\_30} and \texttt{ELEV\_10} in the 
 attribute space. Validation statistics were estimated using $n = 60$ validation points located 
 along $m = 12$ linear transects (clustered samples).}
 \label{tab:covar-data-dem-attr-val}
 \centering
 {\small
 \begin{tabular}{lrrrrrr}
  \hline
  Statistics & \texttt{ELEV\_90} & \texttt{ELEV\_30} & \texttt{ELEV\_10} \\
  \hline
  Mean, \si{\metre} & -15 (10) & -17 (9) & -16 (10) \\ 
  Absolute mean, \si{\metre} & 15 (10) & 17 (9) & 16 (10) \\ 
  Squared mean, \si{\metre\square} & 350 (428) & 361 (406) & 374 (431) \\ 
  \hline
 \end{tabular}}
\end{table}

% Figure \ref{fig:cdf-elev} shows that estimated validation statistics have different cumulative 
% distribution functions (CDF). The estimates are more uniformly distributed along the interval of 
% values for \texttt{ELEV\_10} than for \texttt{ELEV\_90} and \texttt{ELEV\_30}. While 
% \texttt{ELEV\_10} has a 50\% probability that absolute errors are bellow 15 m, \texttt{ELEV\_90} has 
% a 70\% probability that absolute errors are bellow 15 m. This suggests that the accuracy of 
% \texttt{ELEV\_90} is very consistent across the study area, with a few extreme values, while the 
% accuracy of \texttt{ELEV\_10} have a stronger spatial variation. For \texttt{ELEV\_30}, the 
% interpolation method used to refine the original SRTM DEM to 30 m \cite{ValerianoEtAl2012} seems to 
% have produced a spatial redistribution of the errors.

% \begin{figure}[!ht]
%   \centering
%   \includegraphics[width=0.9\textwidth]{fig/cdf-ELEV-90} 
%   \includegraphics[width=0.9\textwidth]{fig/cdf-ELEV-30}
%   \includegraphics[width=0.9\textwidth]{fig/cdf-ELEV-10}
%   \caption{Cumulative distribution functions of mean error, mean absolute error, and squared error of elevation values estimates by digital elevation models \texttt{ELEV\_90}, \texttt{ELEV\_30}, and \texttt{ELEV\_10}.}
%   \label{fig:cdf-elev}
% \end{figure}

Despite the similar accuracy in the feature space, \texttt{ELEV\_10} is used in this study 
because it provides a better hydrological representation of the study area because it was 
produced using information about the drainage network and location of lakes and natural depressions. 
This is evidenced by the shape of the boundaries derived from each DEM using the \grass{r.watershed} 
and \texttt{r.water.outlet} (\autoref{fig:covar-data-elev-maps}). The boundary derived from 
\texttt{ELEV\_90} is clearly unable to capture all hydrological features of the study area. 
Therefore, the boundary derived using \texttt{ELEV\_10} is used throughout this study with the 
addition of a \SI{30}{\metre} buffer, which is the estimated uncertainty 
($\text{RMSE} = \SI{29.55}{\metre}$) of the affine transformation used to correct the systematic error 
identified in topographic maps. The water outlet point used to derive the boundary is located on the 
bridge that crosses the main drainage channel (\ang{-29.65868}, \ang{-53.78969}).

% TODO: figure with both digital elevation models, including the real drainage network and the boundary of the study area.
% \begin{figure}[!ht]
%   \centering
%   \includegraphics[width=0.3\textwidth]{fig/elev-90}
%   \includegraphics[width=0.3\textwidth]{fig/elev-10}
%   \caption{Digital elevations models used as sources of environmental co-variates. On the left, the SRTM 
% digital elevation models prepared by CGIAR and published at a resolution of about 90 m (\texttt{ELEV\_90}). 
% On the right, the digital elevation models produced interpolating contour lines manually digitalized from 
% topographic maps published at a scale of 1:25,000 (\texttt{ELEV\_10}).}
%   \label{fig:elev-maps}
% \end{figure}

Eight terrain attributes were derived from each of \texttt{ELEV\_90} and \texttt{ELEV\_10}, the 
first of them being the elevation (\texttt{ELEV}). The others are slope, aspect, northernness, flow 
accumulation, topographic wetness index, stream power index, and topographic position index.

Slope (\texttt{SLP}) and aspect (\texttt{ASP}) were calculated using \grass{r.param.scale}. This 
module calculates terrain attributes fitting a bivariate quadratic polynomial using least squares 
\cite{Wood1996}. It allows using different window sizes to fit the bivariate quadratic polynomial, 
thus including the effect of scale in the calculation of terrain attributes. In the present study, 
seven window sizes were used (\numlist{3;7;15;31;63;127;255}) and the results for calculated slope 
can be seen in \autoref{fig:covar-data-slope}. Larger window sizes result in a smoothed version of 
the terrain attribute, while smaller windows sizes result in raster maps with more (small-scale) 
details. Several flat surfaces (slope equal to \ang{0}) were produced in the slope raster 
maps calculated using \texttt{ELEV\_90} as a result of resampling the original DEM from \num{90} to 
\SI{5}{\metre}. A value of \ang{0.1} was added to the rasters to remove these flat surfaces.

% \begin{figure}[!ht]
%  \centering
%  % TODO: Include R code to produce figures on the go.
%  \caption{\label{fig:covar-data-slope}Slope \texttt{SLP}} raster maps derived from \texttt{ELEV\_10} 
%  using seven window sizes  (\numlist{3;7;15;31;63;127;255}) to include the effect of scale in the 
%  derived terrain attributes.}
% \end{figure}

Aspect values were also corrected before use. The first correction refers to the fact that 
\grass{r.param.scale} stores aspect values in the range \SIrange{0}{+180}{\degree} from West to 
North to East, and \SIrange{0}{-180}{\degree} from West to South to East, when the standard 
procedure is to work with aspect values ranging from \SIrange{0}{360}{\degree} clockwise. This 
correction was done using the following expressions in in \grass{r.mapcalc}:

\begin{verbatim}
 if(asp < 0, aspect + 360, aspect)
 if(aspect < 90, aspect + 270, aspect - 90)
\end{verbatim}

\noindent Mathematically,

\begin{equation}
 \texttt{ASP}_{beta} =
 \begin{cases}
  aspect + \ang{360} & \text{if}\;\; aspect < \ang{0}, \\
  aspect             & \text{else},
 \end{cases}
\end{equation}

\noindent and

\begin{equation}
 \texttt{ASP} =
 \begin{cases}
  \texttt{ASP}_{beta} + \ang{270} & \text{if}\;\; \texttt{ASP}_{beta} < \ang{90}, \\
  \texttt{ASP}_{beta} - \ang{90}  & \text{else}.
 \end{cases}
\end{equation}

\noindent The second correction involved linearizing aspect values. This is necessary because 
aspect is a circular variable, that is, the begging (\ang{0}) and end (\ang{360}) of the
measurement scale have the same physical meaning. Aspect values were transformed to northernness 
(\texttt{NOR}), a measure of the degree of exposition of a given surface to the North, a linear 
variable, using the equation

\begin{equation}
 \texttt{NOR}_i = abs(\ang{180} - \texttt{ASP}_i),
\end{equation}\label{eq:NOR}

\noindent where $i$ is the window size used to calculate \texttt{ASP}, with 
$i = \numlist{3;7;15;31;63;127;255}$.   

Flow accumulation (\texttt{ACC}), also known as catchment area and contributing area, was calculated 
using \grass{r.watershed}. The resulting raster map was multiplied by the square of the cell size 
(\SI{5}{\metre}). This raster map was used to calculate the topographic wetness index (\texttt{TWI}) 
and the stream power index (\texttt{SPI}) using the following equations:

\begin{equation}
 A = \dfrac{\texttt{ACC}}{\textit{cell}},
\end{equation}\label{eq:sACC}

\begin{equation}
 \texttt{TWI}_i = log \dfrac{A}{tan(\texttt{SLP}_i)},
\end{equation}\label{eq:TWI}

\noindent and

\begin{equation}
 \texttt{SPI}_i = log(A \times tan(\texttt{SLP}_i)),
\end{equation}\label{eq:SPI}

\noindent where $A$ is the specific catchment area, \textit{cell} is the cell size 
(\SI{5}{\metre}), and $i$ is the window size used to calculate \texttt{SLP}, with 
$i = \numlist{3;7;15;31;63;127;255}$.

The topographic position index \texttt{TPI} was calculated in \saga{ta\_morphometry}. Different 
values of maximum radius were used to include the effect of scale, all of them related to the window 
sizes used to calculate previous terrain attributes. A minimum radius value of three meters was used 
in all calculations.

\tocless\section{Geological maps}
\label{sec:covar-data-geo-maps}

Data on geology and soil parent material data comes from most recent geological maps published in 
the \scales{25000}{50000} \cite{MacielFilho1990, GasparettoEtAl1988}.

Both geological maps were produced based on the most recent topographic maps produced by the 
Brazilian Army at the \scales{50000}{25000}. Alike topographic maps, geological maps were also 
available only in the analogical format, and were hand digitalized and georeferenced in QGIS. 
Intersections between all meridians and parallels (a total of 16) were used as control points to 
adjust a second order polynomial model. Resampling was performed using the cubic resampling method. 
After manual digitalization of geological formations, the original coordinate reference system 
(EPSG:31982 - SIRGAS2000 / UTM zone 22S) of all vector files was transformed to WGS1984 / UTM zone 
22S (EPSG:32722) using the \Rpackage{rgdal} \cite{BivandEtAl2013a}.

The positional validation of geological maps was performed using eight (\texttt{GEO\_50}) and five 
(\texttt{GEO\_25}) GCPs located at easily identifiable geographical markers. 
\autoref{tab:covar-data-geology-geo-val} shows that the positional accuracy of both geological maps 
does not meet the current regulations of the Brazilian legislation. Estimated RMSE are \SI{147}{\m} 
and \SI{69}{\m} for \texttt{GEO\_50} and \texttt{GEO\_25}, respectively, when the maximum RMSE 
accepted is \SI{30}{\m} and \SI{15}{\m}. For \texttt{GEO\_50}, the lowest accuracy is found in the 
y-coordinate, while for \texttt{GEO\_25}, the x-coordinate is the less accurate. 
\autoref{fig:covar-data-geology-azim} suggests that the low positional accuracy  of both geological 
maps is due to a systematic error. This systematic error probably was propagated from the 
topographic maps used to produce the geological maps. Therefore, the same strategy (affine 
transformation ) used to remove the systematic positional error of the topographic map above was 
employed on geological maps. Due to the lack of GCPs, model parameters were adjusted using the same 
set of GCPs used for the validation in the geographic space. The estimated uncertainty of the 
affine transformation is $RMSE = \SI{86}{\m}$ and $RMSE = \SI{22}{\m}$ for 
\texttt{GEO\_50} and \texttt{GEO\_25}, respectively.

\begin{table}[ht]
 \caption{Estimated error statistics (standard deviation between parenthesis) of the validation of 
 geological maps GEO\_50 and GEO\_25 in the geographic space. Validation statistics were estimated 
 using, respectively, eight and five ground control points located in easily identifiable 
 geographical markers (purposive sampling).}
 \label{tab:covar-data-geology-geo-val}
 \centering
 {\small
 \begin{tabular}{lrrrr}
  \hline
  Statistics & X coordinate & Y coordinate & Error vector & Azimuth \\
  \hline
  \multicolumn{5}{l}{\texttt{GEO\_50} ($n = 8$)}\\
  \hline
  Mean, \si{\m} & 10 (58) & -102 (87) & 140 (44) & \ang{169} (\ang{47}) \\ 
  Absolute mean, \si{\m}  & 43   (40) & 125 (50) & -         & -          \\ 
  Squared mean, \si{\m\square} & 3431 (5914)  & 18067 (13243) & 21498 (12316) & - \\
  \hline
  \multicolumn{5}{l}{\texttt{GEO\_25} ($n = 5$)} \\
  \hline
  Mean, \si{\m} & 51 (29) & 29 (22) & 67 (16) & \ang{58} (\ang{30}) \\ 
  Absolute mean, \si{\m} & 51 (29) & 29 (22) & -  & - \\ 
  Squared mean, \si{\m\square} & 3457 (2976) & 1312 (1612) & 4769 (2306) & - \\
  \hline
 \end{tabular}}
\end{table}

% \begin{figure}[ht]
%  \centering
%  TODO: Include R code to produce figures on the go.
%  \caption{Histogram of the azimuth distribution of the validation of geological maps 
%  \texttt{GEO\_50} (left) and \texttt{GEO\_25} (right) in the attribute space. Azimuth values were 
%  estimated using, respectively, eight and five GCPs located in easily identifiable geographical 
%  markers. The graph was produced using \Rpackage{VecStatGraphs2D}.}
%  \label{fig:covar-data-geology-azim}
% \end{figure}

\begin{table}[ht]
 \caption{Estimated error statistics of the validation of geological maps \texttt{GEO\_50} and 
 \texttt{GEO\_25} in the attribute space. Validation statistics were estimated using $n = 60$ 
 validation points located along $m = 12$ linear transects (clustered samples).}
 \label{tab:covar-data-geology-attr-val}
 \centering
 \begin{tabular}{lrrr}
  \hline
  Geological map        & LCB95Pct & Estimate & UCB95Pct \\
  \hline
  \texttt{GEO\_50}      & 76.88    & 83.33    & 89.78    \\
  \texttt{GEO\_25}      & 70.10    & 76.67    & 83.24    \\
  \hline
 \end{tabular}
\end{table}

Three covariates were derived from \texttt{GEO\_50}:

\begin{description}
 \item[\texttt{GEO\_50a}] Inferior Sequence of the Serra Geral Formation. Composed mainly by basic 
 igneous rocks (tholeiitic basalt and andesite). It is likely to be related with high CLAY and 
 ECEC;
 
 \item[\texttt{GEO\_50b}] Superior Sequence of the Serra Geral Formation. Composed mainly by acid 
 igneous rocks (granophyric rhyolite and rhyodacite). It is likely to be related with moderate to 
 high CLAY and ECEC;
 
 \item[\texttt{GEO\_50c}] Botucatu Formation. Composed mainly by aeolian sandstones. It is likely 
 to be related with low CLAY and ECEC;
\end{description}

Four covariates were derived from \texttt{GEO\_25}, the first three of them having the same meaning 
of those derived from \texttt{GEO\_50}:

\begin{description}
 \item[\texttt{GEO\_25a}] Inferior Sequence of the Serra Geral Formation;
 
 \item[\texttt{GEO\_25b}] Superior Sequence of the Serra Geral Formation;
 
 \item[\texttt{GEO\_25c}] Botucatu Formation;
 
 \item[\texttt{GEO\_25d}] Quaternary deposits of fluvial, alluvial, and colluvial 
 origin. It can help explaining the low CLAY of soils supposedly derived from igneous rocks.
\end{description}

\tocless\section{Land use maps}
\label{sec:covar-data-land-use}

The land use map for the year of \num{1980} was produced by manually digitizing land use data included in the 
most recent topographic map produced by the Brazilian Army (\scale{25000}), and that were used to produce the 
more detailed pedological and geologic maps. The most up-to-date land use map was prepared using high 
resolution orbital images (Quick Bird satellite). It covers the years of \num{2008} and \num{2009}, and was 
prepared at a \scale{2000} \cite{SamuelRosaEtAl2011a}.

Land use maps were registered and geocoded with the prediction grid using the nearest neighbour resampling 
method. This method was used to avoid changing raster values. Systematic positional errors 
\cite{Samuel-RosaEtAl2014} were corrected using affine transformation (\Rpackage{vec2dtransf} 
\cite{Carrillo2012}).

\begin{table}[ht]
 \caption{Estimated error statistics (standard deviation between parenthesis) of the validation of Google 
 Earth\textregistered imagery in the geographic space. Validation statistics were estimated using \num{14} 
 ground
 control points located in easily identifiable geographical markers (purposive sampling).}
 \label{tab:covar-data-google-geo-val}
 \centering
 {\small
 \begin{tabular}{lrrrr}
  \hline
  Statistics           & X coordinate & Y coordinate & Error vector  & Azimuth \\
  \hline
  Mean, \si{\m} & -1 (4) & 3 (7) & 6 (6) & \ang{184} (\ang{125}) \\ 
  Absolute mean, \si{\m} & 3 (2) & 5 (6) & - & - \\ 
  Squared mean, \si{\m\square} & 14 (22) & 57 (132) & 71 (153) & - \\ 
  \hline
 \end{tabular}}
\end{table}

\begin{table}[ht]
 \caption{Estimated error statistics of the validation of land use maps \texttt{LU1980} and  \texttt{LU2009} in 
 the attribute space. Validation statistics were estimated using $n = 60$  validation points located in 
 $m = 12$ linear transects (clustered samples).}
 \label{tab:covar-data-land-attr-val}
 \centering
 {\small
 \begin{tabular}{lrrr}
  \hline
  Land use map & LCB95Pct & Estimate & UCB95Pct \\
  \hline
  \texttt{LU1980} & 58.52    & 66.67    & 74.82    \\
  \texttt{LU2009} & 61.16    & 70.00    & 78.84    \\
  \hline
 \end{tabular}}
\end{table}

Two indicator variables were derived from \texttt{LU1980}, with plantation forests (PF) and human settlements 
(S) being grouped together due to their small importance in terms of covered area (PF) and for containing any soil 
observation (S). These are:

\begin{description}
 \item[\texttt{LU1980a}] Native forest (FS), which is likely to have soils with higher fertility.
  
 \item[\texttt{LU1980b}] Animal husbandry (H), which is likely to have a soil fertility status lower than native 
 forests and is the second most important land use in the study area.
\end{description}

Five indicator covariates were derived from \texttt{LU2009}, with plantation forests (PF), human settlements 
(S), and other land uses (O), which comprise natural and artificial water bodies, being grouped together due to their
small importance in terms of covered area (PF) and for containing any soil observation (S and O). These are:

\begin{description}
 \item[\texttt{LU2009a}] Native forest (FS).
 
 \item[\texttt{LU2009b}] Shrubland (SS), which is likely to have SOC and ECEC level above  those found in areas used 
 with annual crop agriculture and animal husbandry, but lower than in native forests.
 
 \item[\texttt{LU2009c}] Animal husbandry (H).
  
 \item[\texttt{LU2009d}] Annual crop agriculture (AA), which is likely to have the lowest soil fertility due to 
 the usually poor management practices employed.
 
 \item[\texttt{LUdiff}] Land use difference between \num{1980} and \num{2009}. It can be useful to explain, for 
 example, low \texttt{ORCA} in forest soils due to previous use with crop agriculture or animal husbandry.
\end{description}

\tocless\section{Orbital images}
\label{sec:covar-data-sat-image}

Two sources of satellite images were used. The first is the longest-operating Earth observation satellite 
Landsat-5 Thematic Mapper, launched on \num{1} March \num{1984}. The satellite image used was acquired on 
\num{26} December \num{2010} and is available at the database of the Division of Image Generation of the 
National Institute for Space Research (\inpedgi). The image contains seven spectral bands 
\autoref{tab:covar-data-satellites}, including a thermal band (which was not used in this study), with eight 
bits radiometric resolution (digital numbers from \numrange{0}{255}) and approximately \SI{30}{\m} spatial 
resolution. Orthorectification was performed using Geomatica\textregistered{} OrthoEngine\textregistered{} with 
the Landsat rigorous model (Toutin's Model). A set of \num{28} GCPs were manually collected in Google 
Earth\textregistered{} due to the absence of field GCPs and the high accuracy of Google Earth\textregistered{} 
imagery in the region covered by the image (\autoref{tab:covar-data-google-geo-val}). GCPs were located at 
easily identifiable geographical markers (road intersection, bridges), evenly distributed throughout the image 
and covering a variety of elevations, following standard recommendations \cite{PCIGeomatics2007} 
(\autoref{fig:covar-data-ortho-gcps}). The DEM used is \texttt{ELEV\_30} described in 
\autoref{sec:covar-data-dem} above with the vertical datum corrected with the EGM1996 geoidal model. Resampling 
was done using the nearest neighbour method to avoid changes in the digital numbers.

% TODO: figure with GCPs used to ortorectify orbital images. Show the bounding box of the image and the 
% boundary 
% of the study area.
% \begin{figure}
%  \centering
%  \includegraphics[width=\textwidth]{fig/ortho-gcps}
%  \caption{Ground control points used to orthorectify orbital the image produced by Landsat-5 Thematic 
% Mapper.}
%  \label{fig:covar-data-ortho-gcps}
% \end{figure}

After orthorectification, all bands were imported into GRASS GIS, where all other necessary corrections were 
performed. Radiometric correction (conversion from digital numbers to top-of-atmosphere reflectance) was 
performed using \grass{i.landsat.toar}. Atmospheric correction (removal of the effects of the atmosphere on 
the reflectance values) was performed using the 6S atmospheric model \cite{VermoteEtAl1997} using 
\grass{i.atcorr}. The correction was performed using the tropical atmospheric model, the continental aerosols 
model, an image-based visibility estimate of \SI{20}{\km}, and a fixed elevation of \SI{300}{\m}. Afterwards, 
all bands were cropped to the bounding box of the study area and geometrically corrected to match the 
prediction grid (\SI{5}{\m} grid size). Registration and geocoding was performed using the nearest neighbour 
resampling method. Topographic correction (removal of the effects of the topography -- illumination -- on the 
reflectance values) was performed using \grass{i.topo.corr} with \texttt{ELEV\_30} geometrically corrected to 
match the prediction grid.

The second source of orbital images is the RapidEye constellation of five satellites, launched in August 
\num{2008}. It is available through the Brazilian Ministry of the Environment \cite{Brasil2012}, who has a 
full coverage of the Brazilian territory with images from the RapidEye satellite constellation for the years of 
\num{2011} and \num{2012}. The orbital image used (tile number \num{2225403}) was acquired on \num{16} November 
\num{2012} (second coverage). It contains five spectral bands \ref{tab:covar-data-satellites}, featuring among 
them the so called red edge band, located between the red and the near-infrared bands. This spectral band is 
the main feature distinguishing RapidEye images from most other sources of orbital images, considered to 
provide additional information about the vegetation \cite{WeicheltEtAl2013}. The orbital image has 
\SI{16}{\bit} radiometric resolution and \SI{6.5}{\m} spatial resolution, and was orthorrectified in the source 
to \SI{5}{\m} spatial resolution using the hole-filled SRTM version \num{4} \cite{RapidEye2013}.

Atmospheric correction was performed using the 6S atmospheric model \cite{VermoteEtAl1997} using the Fortran 
code developed by Dr. \href{http://lattes.cnpq.br/3818721407909667}{Mauro Antonio Homem Antunes}, from the 
Rural University of Rio de Janeiro. The \grass{i.atcorr} was not used because a \atcorrbug{} was found when 
trying to correct images from the RapidEye satellite constellation. The correction was performed using the 
tropical atmospheric model, the continental aerosols model, an image-based visibility estimate of \SI{20}{\km}, 
and a fixed elevation of \SI{300}{\m}. Afterwards, all bands were cropped to the bounding box of the study area 
and geometrically corrected to match the prediction grid (\SI{5}{\m} grid size). Registration and geocoding was 
performed using the nearest neighbour resampling method. Topographic correction was performed using 
\grass{i.topo.corr} with \texttt{ELEV\_30} geometrically corrected to match the prediction grid 
(\autoref{sec:covar-data-dem}).

\begin{table}[ht]
 \caption{Comparison between satellite images produced by Landsat 5 TM and RapidEye constellation used in the 
 present study and derived covariates.}
 \label{tab:covar-data-satellites}
 \centering
 {\small
 \begin{tabular}{llllll}
  \hline
  \multicolumn{3}{l}{Landsat 5 TM}                         & \multicolumn{3}{l}{RapidEye} \\
  Band & Interval, \si{nm} & Covariate & Band & Interval, \si{\nm} & Covariate \\
  \hline
  Band 1 Visible &\numrange{450}{520} &\covar{BLUE\_30}  &Blue band  &\numrange{440}{510} &\covar{BLUE\_5}\\
  Band 2 Visible &\numrange{520}{600} &\covar{GREEN\_30} &Green band &\numrange{520}{590} &\covar{GREEN\_5}\\
  Band 3 Visible &\numrange{630}{690} &\covar{RED\_30}   &Red band   &\numrange{630}{685} &\covar{RED\_5}\\
  -              &-            & -                   & Red edge band &\numrange{690}{730} &\covar{EDGE\_5}   \\
  Band 4 Near-Infrared &\numrange{760}{900} &\covar{NIR\_30a} & Near-infrared band &\numrange{760}{850}& 
  \covar{NIR\_5}\\
  Band 5 Near-Infrared &\numrange{1550}{1750} &\covar{NIR\_30b} & -                  & -            & -         
  
       \\
  Band 7 Mid-Infrared  &\numrange{2080}{2350} &\covar{MIR\_30}  & -                  & -            & -         
 
        \\
  \hline
 \end{tabular}}
\end{table}

\begin{table}[ht]
 \caption{Estimated error statistics (standard deviation between parenthesis) of the horizontal validation of 
 orbital images produced by Landsat 5 TM and RapidEye constellation. Validation statistics were estimated 
 using $n = 14$ GCPs located in easily identifiable geographical markers.}
 \label{tab:covar-data-satellite-geo-val}
 \centering
 {\small
 \begin{tabular}{lrrrr}
  \hline
  Statistics           & X coordinate & Y coordinate  & Error vector  & Azimuth              \\
  \hline
  \multicolumn{5}{l}{Landsat 5 TM}                                                           \\
  \hline
  Mean, \si{\m} & 31   (23)   & -11  (33)   & 45   (26)   & \ang{136} (\ang{89}) \\ 
  Absolute mean, \si{\m}     & 33   (21)   & 25   (25)   & -           & -                         \\ 
  Squared mean, \si{\m\square}  & 1494 (1436) & 1223 (2082) & 2717 (2706) & -                         \\ 
  \hline
  \multicolumn{5}{l}{RapidEye}                                                               \\
  \hline
  Mean, \si{\m}              & -25  (7)     & -25 (10)   & 36   (8)     & \ang{226} (\ang{12}) \\ 
  Absolute mean, \si{\m}& 25   (7)     & 25  (10)   & -            & -                        \\ 
  Squared mean, \si{\m\square}  & 680  (347)   & 708 (692)  & 1388 (703)   & -                        \\ 
  \hline
 \end{tabular}}
\end{table}

In the present study, the orbital image produced by the RapidEye constellation is considered to be of higher 
quality than the orbital image produced by the satellite Landsat 5 TM. This is mainly due to its finer 
resolution and thus larger amount of detail. The two-years difference in the acquisition time between the two 
satellite images is believed to have only a minor effect on the results since land use changes were not 
significant in the period and soil observations cover the period from \num{2008} to \num{2013}.

Each band of the orbital images was used to derive an environmental covariate, totalling six from Landsat 5 TM 
and five from RapidEye (Table \ref{tab:covar-data-satellites}). Individual bands were also used to calculate 
two vegetation indexes: the normalized difference vegetation index (NDVI) and the soil-adjusted vegetation 
index (SAVI). For Landsat images, NDVI and SAVI were calculated using equations

\begin{equation}
  \covar{NDVI\_30} = \frac{\covar{NIR\_30a} - \covar{RED\_30}}{\covar{NIR\_30a} + \covar{RED\_30}}
\end{equation}\label{eq:NDVI30}

\noindent and

\begin{equation}
  \covar{SAVI\_30} = (1.0 + 0.5) \times \frac{\covar{NIR\_30a} - \covar{RED\_30}}{\covar{NIR\_30a} + 
  \covar{RED\_30} + 0.5}
\end{equation}\label{eq:SAVI30}

\noindent where \covar{NIR\_30a} is the first near-infrared band (\SIrange{750}{900}{\nm}) and \covar{RED\_30} 
is the red band (\SIrange{630}{690}{\nm}). For RapidEye image, NDVI and SAVI were calculated using the 
standard equations

\begin{equation}
  \covar{NDVI\_5a} = \frac{\covar{NIR\_5} - \covar{RED\_5}}{\covar{NIR\_5} + \covar{RED\_5}}
\end{equation}\label{eq:NDVI5a}

\noindent and

\begin{equation}
  \covar{SAVI\_5a} = (1.0 + 0.5) \times \frac{\covar{NIR\_5} - \covar{RED\_5}}{\covar{NIR\_5} + \covar{RED\_5} 
  + 0.5}
\end{equation}\label{eq:SAVI5a}

\noindent with the red (\SIrange{630}{685}{\nm}) (\covar{RED\_5}) and near-infrared (\SIrange{760}{850}{\nm}) 
(\covar{NIR\_5}), and also using the red-edge band (\SIrange{690}{730}{\nm}) (\covar{EDGE\_5}) instead of the 
near-infrared band as follows:

\begin{equation}
  \covar{NDVI\_5b} = \frac{\covar{EDGE\_5} - \covar{RED\_5}}{\covar{EDGE\_5} + \covar{RED\_5}}
\end{equation}\label{eq:NDVI5a}

\noindent and

\begin{equation}
  \texttt{SAVI\_5b} = (1.0 + 0.5) \times \frac{\covar{EDGE\_5} - \covar{RED\_5}}{\covar{EDGE\_5} + 
  \covar{RED\_5} + 0.5}
\end{equation}\label{eq:SAVI5a}

The final number of covariates derived from orbital images is eight, for the Landsat 5 TM, and nine, for 
the RapidEye constellation.
 % The Santa Maria dataset -- covariate data
\artigofalse
\chapter{R-PACKAGE SPSANN: OPTIMIZATION OF SAMPLE CONFIGURATIONS USING SPATIAL SIMULATED ANNEALING}
\label{apen:spsann}

\includepdf[pages=-,pagecommand={}]{chap/spsann.pdf}

% \section{Objective functions}
% 
% \subsection{Spatial trend identification and estimation}
% 
% \subsubsection{DIST}
% 
% Reproducing the marginal distribution of the numeric covariates depends upon
% the definition of marginal sampling strata. These marginal sampling strata 
% are also used to define the factor levels of all numeric covariates that  
% are passed together with factor covariates. Two types of marginal sampling 
% strata can be used: \textit{equal-area} and \textit{equal-range}.
% 
% \textit{Equal-area} marginal sampling strata are defined using the sample 
% quantiles estimated with the \texttt{quantile}-function of the 
% \textbf{stats}-package using a discontinuous function (\texttt{type = 3}). This 
% is to avoid creating breakpoints that do not occur in the population of 
% existing covariate values.
% 
% Depending on the level of discretization of the covariate values, 
% the \texttt{quantile}-function produces repeated breakpoints. A breakpoint 
% will be repeated if that value has a relatively high frequency in the 
% population of covariate values. The number of repeated breakpoints increases 
% with the number of marginal sampling strata. Repeated breakpoints result in
% empty marginal sampling strata. To avoid this, only the unique breakpoints 
% are used.
% 
% \textit{Equal-range} marginal sampling strata are defined by breaking the range
% of covariate values into pieces of equal size. Depending on the level of 
% discretization of the covariate values, this method creates breakpoints that
% do not occur in the population of existing covariate values. Such breakpoints
% are replaced by the nearest existing covariate value identified using 
% Euclidean distances.
% 
% Like the equal-area method, the equal-range method can produce empty marginal
% sampling strata. The solution used here is to merge any empty marginal 
% sampling strata with the closest non-empty marginal sampling strata. This is
% identified using Euclidean distances as well.
% 
% The approaches used to define the marginal sampling strata result in each 
% numeric covariate having a different number of marginal sampling strata, 
% some of them with different area/size. Because the goal is to have a sample 
% that reproduces the marginal distribution of the covariate, each marginal 
% sampling strata will have a different number of sample points. The wanted 
% distribution of the number of sample points per marginal strata is estimated 
% empirically as the proportion of points in the population of existing 
% covariate values that fall in each marginal sampling strata.
% 
% \subsubsection{CORR}
% 
% The \textit{correlation} between two numeric covariates is measured using the 
% Pearson's \textit{r}, a descriptive statistic that ranges from $-1$ to $+1$. 
% This statistic is also known as the linear correlation coefficient.
% 
% When the set of covariates includes factor covariates, all numeric covariates 
% are transformed into factor covariates. The factor levels are defined 
% using the marginal sampling strata created from one of the two methods 
% available (equal-area or equal-range strata).
% 
% The \textit{association} between two factor covariates is measured using the 
% Cramér's \textit{v}, a descriptive statistic that ranges from $0$ to $+1$. The 
% closer to $+1$ the Cramér's \textit{v} is, the stronger the association between 
% two factor covariates. The main weakness of using the Cramér's \textit{v} is 
% that, while the Pearson's \textit{r} shows the degree and direction of the 
% association between two covariates (negative or positive), the Cramér's 
% \textit{v} only measures the degree of association (weak or strong).
% 
% \subsection{Variogram identification and estimation}
% 
% PPL: points and pairs; minimum and distribution
% 
% \subsection{Spatial interpolation}
% 
% MKV and MSSD
% 
% \subsection{Multi-objective optimization}
% 
% ACDC: CORR and DIST;
% CLHS;
% SPAN: CORR, DIST, PPL, and MSSD;
% 
% A method of solving a multi-objective optimization problem is to aggregate 
% the objective functions into a single \textit{utility function}. In the
% \textbf{spsann}-package, the aggregation is performed using the \textit{weighted 
% sum method}, which incorporates in the weights the preferences of the user 
% regarding the relative importance of each objective function.
% 
% The weighted sum method is affected by the relative magnitude of the 
% different function values. The objective functions implemented in the
% \textbf{spsann}-package have different units and orders of magnitude. The 
% consequence is that the objective function with the largest values will have 
% a numerical dominance in the optimization. In other words, the weights will 
% not express the true preferences of the user, and the meaning of the utility 
% function becomes unclear.
% 
% A solution to avoid the numerical dominance is to transform the objective
% functions so that they are constrained to the same approximate range of 
% values. Several function-transformation methods can be used and the 
% \textbf{spsann}-package offers a few of them. The \textit{upper-lower-bound 
% approach} requires the user to inform the maximum (nadir point) and minimum 
% (utopia point) absolute function values. The resulting function values will 
% always range between 0 and 1.
% 
% Using the \textit{upper-bound approach} requires the user to inform only the
% nadir point, while the utopia point is set to zero. The upper-bound approach
% for transformation aims at equalizing only the upper bounds of the objective 
% functions. The resulting function values will always be smaller than or equal
% to 1.
% 
% Sometimes, the absolute maximum and minimum values of an objective function 
% can be calculated exactly. This seems not to be the case of the objective 
% functions implemented in the \textbf{spsann}-package. If the user is 
% uncomfortable with informing the nadir and utopia points, there is the option
% for using \textit{numerical simulations}. It consists of computing the 
% function value for many random sample configurations. The mean function 
% value is used to set the nadir point, while the the utopia point is set to
% zero. This approach is similar to the upper-bound approach, but the function
% values will have the same orders of magnitude only at the starting point of 
% the optimization. Function values larger than one are likely to occur during 
% the optimization. We recommend the user to avoid this approach whenever 
% possible because the effect of the starting point on the optimization as a 
% whole usually is insignificant or arbitrary.
% 
% The \textit{upper-lower-bound approach} with the \textit{Pareto maximum and 
% minimum values} is the most elegant solution to transform the objective 
% functions. However, it is the most time consuming. It works as follows:
% 
% \enumerate{
%   \item Optimize a sample configuration with respect to each objective
%   function that composes the MOOP;
%   \item Compute the function value of every objective function that composes 
%   the MOOP for every optimized sample configuration;
%   \item Record the maximum and minimum absolute function values computed for 
%   each objective function that composes the MOOP -- these are the Pareto
%   maximum and minimum.
% }
% 
% For example, consider that a MOOP is composed of two objective functions: A 
% and B. The minimum absolute value for function A is obtained when the sample
% configuration is optimized with respect to function A. This is the Pareto
% minimum of function A. Consequently, the maximum absolute value for function
% A is obtained when the sample configuration is optimized regarding function
% B. This is the Pareto maximum of function A. The same logic applies for 
% function B.
% 
% \section{Generation mechanism}
% 
% The \textit{generation mechanism} corresponds to the set of formal rules used 
% to randomly perturb the sample configuration to create a new solution out of the
% current one. This is done by adding random noise to the coordinates of one of 
% the sample points, a process known as \textit{jittering}.
% 
% Before we jitter a given sample point, we have to define the maximum quantity 
% of random noise that can be added to its coordinates, i.e. the area within 
% which it can be moved around. In principle, this area corresponds to a rectangle
% centred at the sample point that ignores the presence of non-sampling areas 
% (e.g. buildings and water bodies) and the finiteness of the sampling region. We 
% call this the \textit{neighbourhood}.
% 
% Once we know the size of the neighbourhood, we have to decide upon how much 
% noise will be added to the coordinates of our sampling point, i.e. to choose a
% candidate location in the neighbourhood. This can be done in two different ways.
% We can use an \textit{infinite} set of candidate locations, that is, any 
% location in the neighbourhood can be selected as the candidate location for our 
% sample point. After a candidate location is selected, we check if it falls 
% within the sampling region but does not fall within a non-sampling area. These 
% checks usually are computationally demanding, the reason why this method is not 
% implemented in the \textbf{spsann}-package.
% 
% A more efficient way of selecting a candidate location is to first identify the
% set of \textit{effective} candidate locations for our sample point in the 
% neighbourhood. This can be done using a \textit{finite} set of candidate 
% locations. A finite set of candidate locations is created by discretizing the 
% sampling region beforehand, that is, creating a fine grid of points that serve 
% as candidate locations during the entire search for the optimum sample 
% configuration. This is the least computationally demanding jittering method 
% because, by definition, the candidate location will always fall within the 
% sampling region and out of non-sampling areas.
% 
% Using a finite set of candidate locations has two main disadvantages. First, not
% all locations in the sampling region can enter the sample. The sample points are
% limited to a finite set of regularly spaced candidate locations which is not 
% guaranteed to include the \textit{true} global optimum sample configuration. 
% Second, when a sample point is jittered, it may be that the selected candidate 
% location already is occupied by a sample point. When this happens, another 
% candidate location has to be sought in the neighbourhood because we cannot have 
% more than one sample point in the same location. In the worst case, most (or 
% all) candidate locations in the neighbourhood are already occupied by a sample 
% point -- in general, the more points there are in the sample (or the smaller 
% the size of the neighbourhood (see bellow), the more likely it is that the 
% selected candidate location already is occupied by a sample point. If a 
% candidate location is not found, our sample point is kept in its original 
% location.
% 
% The \textbf{spsann}-package uses a more elegant method based on using a finite
% set of candidate locations coupled with a form of \textit{two-stage random 
% sampling} as implemented in the \texttt{spsample}-function of the 
% \textbf{spcosa}-package \citep{WalvoortEtAl2010}. The fine grid of points that 
% cover the sampling region can be understood as being the centre nodes of a 
% finite set of grid cells (or pixels of a raster image). In the first stage, one 
% of the candidate 'grid cells' is selected with replacement in the neighbourhood,
% i.e. independently of already being occupied by another sample point. The 
% candidate location for our sample point is selected in the second stage within 
% that 'grid cell' by simple random sampling. This method guarantees that a sample
%  point can be placed at \textit{almost} any location within the sampling region.
% It also discards the need to worry if the candidate location already is occupied
% by a sample point, possibly speeding up the computations.
% 
% In order to increase its computational efficiency, the \textbf{spsann}-package 
% uses a decrement function to reduce the size of the neighbourhood as the search
% for the optimum sample configuration evolves. The reason for this is that, as 
% the search evolves and approaches its end, it is likely that moving a sample 
% point over a short distance contributes more to finding the global optimum than 
% moving it over larger distances \citep{GroenigenEtAl1998}. The decrement 
% function determines that the size of the neighbourhood is reduced linearly at 
% the end of each chain $k_i$,
% 
% \begin{equation}
%   x_{max} = x_{max\,0} - k_i / k * x_{max\,0} - x_{min} + x_{dim}
% 
%   y_{max} = y_{max\,0} - k_i / k * y_{max\,0} - y_{min} + y_{dim}
% \end{equation}
% 
% where $x_{max}$ and $y_{max}$ are the dimensions of the neighbourhood in the 
% next chain, i.e. the maximum allowed shifts in the x- and y-coordinates, 
% $x_{max\,0}$ and $y_{max\,0}$ are the dimensions of the neighbourhood in the 
% first chain, $x_{min}$ and $y_{min}$ are the minimum required shifts in the x- 
% and y-coordinates, $x_{dim}$ and $y_{dim}$ are the grid spacings in the x- and 
% y-coordinates, i.e. the grid cell size, and $k$ is the total number of chains.
% The default settings stablish that the size of the neighbourhood in the first
% chain is equal to half the maximum distance in the x- and y-coordinates of the
% entire sampling region, and that the minimum jitter is equal to zero, i.e. that
% the grid cell were the sample point is located can be selected as well. With 
% these settings, at the end of the search, the neighbourhood will be constrained 
% to the set of nine grid cells composed of that in which the sample point falls
% and its eight surrounding grid cells.
% 
% \section{Annealing schedule}
% 
% The \textit{annealing schedule} corresponds to a set of formal rules that 
% determine how the probability of accepting inferior sample configurations is 
% decreased as the search for the globally optimum sample configuration evolves.
% 
 % R-package spsann
\artigofalse
\chapter{R-PACKAGE PEDOMETRICS: PEDOMETRIC TOOLS AND TECHNIQUES}
\label{apen:pedometrics}

\includepdf[pages=-,pagecommand={}]{chap/pedometrics.pdf}
 % R-package pedometrics

%%=============================================================================
%% Referências
%%=============================================================================
\bibliography{ref/biblio}\label{chap:references}

%%=============================================================================
%% Anexos
%%=============================================================================
%\annex
%\include{capitulos/anexoa}

\end{document}
